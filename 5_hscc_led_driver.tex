\chapter{H-SCC LED driver}
\label{ch:hscc_led_driver}

\begin{figure}[!h]
\centering

\ctikzset{bipoles/length=0.75cm}
\begin{circuitikz} [american,scale=0.65]
 
 \draw (0,0) rectangle (3,6);
 
 \draw (-1,1) to[short,o-] (0,1)
       (-1,5) to[short,o-] (0,5)
       (-1,5) to[open,v=$24V$] (-1,1)
       (-2.75,3) node[rotate=90] {e-Merge Alliance}
       (1.5,3) node[rotate=90,text width=2cm,align=center] {H-SCC LED DRIVER};
       
 \draw (3,5) -- (10,5)  to[leD*] (10,2) node[ground]{}
       (7,5) node[anchor=south,text width=4cm] {$11-13V$ $10-12W$ Dimmable LED string    };

 \draw (3,2.5) -- (4,2.5)
       (4,1.5) rectangle (8,3.5)
       (6,2.5) node[text width=2cm,align=center] {$4.5V 100mW$ Low-Voltage Circuits}
       (6,1.5) -- ++(0,-0.5)node[ground]{};
 
      
 
 
 
\end{circuitikz}

\caption[H-SCC LED driver block diagram]{H-SCC LED driver block diagram.}
\label{fig:bd_emerge_drv}
\end{figure}


An experimental converter was built with the goal to validate the performances of a H-SCC as a LED driver. The LED driver, described in the block diagram of Figure~\ref{fig:bd_emerge_drv}, was built using discrete components following the specifications of Table~\ref{tab:dsg_param_drv}.

\begin{table}[!h]
 \caption{LED driver design specifications}\label{tab:dsg_param_drv}
 \centering
 \renewcommand{\arraystretch}{1.2}
 \begin{tabular}{l | cl}
  Items & Value & Unit \\
  \midrule
  $v_{src}$ & 24 & V \\
  \hline
  $v_{LED}$ voltage & 11-13 & V \\
  $v_{LED}$ power & 12 & W \\
  $i_{LED}$ max & 1 & A \\
  $\Delta i_{LED}$ & $\pm$ 10 & \% \\
  \hline
  $v_{aux}$ voltage & 4.5 & V \\
  $v_{aux}$ power & 100m & W \\
  \hline
  $\eta$  & 85 & \% \\
  $f_{sw}$ & 2.77 & MHz\\
\end{tabular}
\end{table}

The driver was designed to be compliment the 24Vdc e-Merge Alliance standard used in track lighting systems, and featured two outputs. The main output supplies a \emph{LUXENON Altion} LED with a maximum current of 1A with a forward voltage around 12V, thus providing 12W at full load. The secondary output is designed for low-voltage and low-power to supply other auxiliary electronic circuits. The converter efficiency was fixed to be higher than 85\%, and for sake of simplicity, the switching frequency was fixed to be 2.77MHz, taking advantage of a 3dB higher tolerance of the conduced EMI standard X.

\section{Design}
The LED driver is composed by two main subsystems, the power train and the close-controller. Therefore the design process is accordingly  divided in three main parts: Power train design, small-signal analysis and close-loop controller design.

\section{Model-centric design}
\begin{figure}[!h]
\ctikzset { bipoles/length=1cm}
\centering
    \begin{circuitikz}[american voltages,scale=0.6]

    \draw
            %Input Supply
            (0,0)  to[V=$v_{src}$]
            %Draw Switches
            (0,10.5)  --
            (5,10.5)  to[gswitch=$s_1$] %node[anchor=west] {$n_1$}%S1
            (5,9)     to[gswitch=$s_2$] %node[anchor=east] {$n_2$}%S1
            (5,7.5)   to[gswitch=$s_3$] %node[anchor=east] {$n_3$}%S1
            (5,6)     to[gswitch=$s_4$] %node[anchor=west] {$n_4$}%S2
            (5,4.5)   to[gswitch=$s_5$] %node[anchor=west] {$n_5$}%S3
            (5,3) --
            %left branch
            (3,3)    to[gswitch=$s_6$]
            (3,1.5)  to[gswitch=$s_7$]
            (3,0);

    \draw   %right branch
            (5,3) --
            (7,3)   to[gswitch,l_=$s_9$]
            (7,1.5) to[gswitch,l_=$s_8$]
            (7,0) -- (0,0);



    \draw %Capacitor C1
           (8.25,6)
            to[pC,l_=$c_1$] (8.252,9) --
           (5,9);

    \draw %Capacitor C2
           (2,4.5)  to[pC,l_=$c_2$] (2,7.5) --
           (5,7.5);

    \draw %Capacitor C3
           (7,1.5) -| (8.25,3)
            to[pC,l_=$c_3$] (8.25,6) --
           (5,6);

    \draw %Capacitor C4
           (3,1.5) --
           (2,1.5) -- (2,3) to[pC,l_=$c_4$](2,4.5) --
           (5,4.5);


    \draw  %LC output filter &  Output LED string
            (8.25,6) -- (10.25,6)node[anchor=south] {$v_x$} to[cute inductor,l=$l_o$] ++(2,0) -- (14,6)
            (13.5,0) to[pC,l=$c_{o}$] (13.5,6) -|
            (16,6) -| (16,5.5)  to[leD*] (16,4) to[leD*] (16,2.5) to[leD*] (16,1)   |- (7,0) ;

    %Vout label
    \draw (16.75,7) to[open,v^=$v_{out}$] (16.75,-0.5);

    \draw %Capacitor C3
           (5,0) to[pC,l_=$c_5$] (5,3) ;% node[anchor=south east] {$n_{dc}$};

     %\draw (7,4) to[short,-o] (10,4) node[anchor=west] {};
     %\draw (7,0) to[short,-o] (12,0) node[anchor=west] {};


     \draw (7,3) --([hs]8.25,4 |- 7,3) arc(180:0:\radius) to[short,-o] (10,3) to[open,v^=$v_{aux}$] (10,0) ;


     \end{circuitikz}
 \caption[5:1 H$^3$-Dickosn 12W LED driver]{ 5:1 H$^3$-Dickson LED driver for 24V e-merge track lighting application. The driver has two outputs: A $12V$, $12W$  LED string, and $4V$, $200mW$  to supply low power auxiliary loads. }
 \label{fig:5_1_hscc_emerge_II}
\end{figure}

Figure~\ref{fig:5_1_hscc_emerge_II} shows the chosen topology for the LED driver, a 5:1 H$^3$-Dickson driver. The chosen topology satisfies the requirements for the output voltages.

The \emph{pwm}-node $v_x$ has a conversion ration of $m_3 = \frac{2+D}{5}$, thus providing an output voltage range between $9.6V$ and $14.4V$ considering the full range of the duty cycle $D$. This dynamic range of regulation is within the  extreme variations of the LED forward voltage, which are defined in the datasheet between 13.2V at 1A with a case temperature of $-40^\circ C$  and 11V at 350mA with a case temperature of $130^\circ C$, thus guaranteeing the operability of the converter a large range of current and temperatures. The converter was design for worst case of 13.2V output voltage and 1A output current.

The \emph{dc}-node has a fixed voltage conversion of $m_{dc} = \frac{1}{5}$, providing a maximum output at $v_{aux}$ of 4.8V.

With the topology already selected, the next step is to size the different components, capacitors and switches. A SCC is by nature lossy, therefore the efficiency of the converter is strongly related to the selection of the right values for the components. That is why, it is essential to have an accurate model in design process of the converter. Indeed, using the algebraic expressions of the model, both, capacitors and switches can be determined as a result of an optimization process. % However in the presented converter owing to the fact that it was implemented with discrete components, the flexibility in the choice of the used components was restricted to the commercial available values.
The converter was design at full load 1A and with the worst case output voltage 13.2V, thus requiring $D=75\%$ to provide this output voltage. At the same time, given the fact the model does not quantifies the switching losses, the switch capacitor stage was designed for a target efficiency of $\eta_trg =90\%$ instead of the $85\%$ given in the specification, allowing a $5\%$ overhead for other sources of losses, mainly switching losses.

As described  in the char flow of figure~\ref{fig:design_flow_drv}, the values for the capacitors and the switches $on$-channel resistance are determined by the equivalent output resistance of both switching limits. Based on the converter efficiency, the target values are defined in the following steps:
\begin{enumerate}
  \item Using~\eqref{eq:r_scc_trg} a target output resistance of the converter is computed, hence
      \begin{equation}
        r_{scc,trg}= \frac{12W (1-0.9)} {1A^2} = 1.2\Omega .
      \end{equation}

  \item The individual contribution of the two switching limits is determined depending on the operation of the $r_{scc}$ curve, in this case, the elbow of the curve where
       \begin{equation}
        r_{ssl} = r_{fsl},
      \end{equation}
      hence both limits have a the same target output resistance of \begin{equation}
        r_{ssl} = r_{fsl} = \frac{1.2}{\sqrt{2}}=845m\Omega.
      \end{equation}

\end{enumerate}


After this point the design process bifurcates, the path on the left describes the procedure to size the capacitors, and the path on the right the procedure to size the transistors. The values for the capacitors are determined in the SSL region, using the $r_{ssl}$ equation of the model:
\begin{equation}
r_{ssl}=\frac{P_{ssl}}{{i_o}^2}=\frac{P_{ssl}}{{(f_{sw} {q_{out}})}^2}=\frac{1}{2 f_{sw}}\sum_{i=1}^{caps.}\sum_{j=1}^{phases}\frac{1}{c_i}{\left[a_{i\
}^j-{D^j} {b_i^j}\right]}^2.
\label{eq:r_ssl_dsg}
\end{equation}
The values for $on$-channel resistance of the switches are determined in the FSL region, using the $r_{fsl}$ equation of the model:
\begin{equation}
r_{fsl}=\sum_{i=1}^{elm.}\sum_{j=1}^{phases}\frac{r_i}{D^j}{ar_i^j}^2.
\label{eq:r_fsl_dsg}
\end{equation}
Actually based on the two asymptotical limits~\citeauthor{Seeman:EECS-2009-78} in his PhD dissertation~\cite{Seeman:EECS-2009-78} describes a methodology to optimise capacitor and switch areas, by minimizing the both expression. The results give capacitor value  and switch-area breakdown. Appendix~\ref{ap:optimitzation} revisits the optimization procedure using the new modeling methodology. The mathematical details are given, along with new insights result of applying the new proposed modeling for SCCs.


\begin{landscape}
    \thispagestyle{empty}
    \begin{figure}[!h]
    \centering
    \newcommand\sHeigh{0.75cm}
\newcommand\sWidth{2.5cm}
\newcommand\dWidth{8cm}

\newcommand\dWidthB{6cm}

\newcommand\xShift{4cm}
\newcommand\xShiftB{3cm}


\usetikzlibrary{shapes.geometric, arrows}

\tikzstyle{startstop} = [rectangle, rounded corners, minimum width=\sWidth, minimum height=\sHeigh,text width=\sWidth,text centered, draw=black, fill=black!20,font=\footnotesize]
\tikzstyle{process} = [rectangle, minimum width=\sWidth, minimum height=\sHeigh, text width=\sWidth,text centered, draw=black, fill=gray!20,font=\footnotesize]
\tikzstyle{decision} = [diamond, minimum width=\sWidth, minimum height=\sHeigh, text width=\sWidth,text centered, draw=black, fill=gray!20,font=\footnotesize]

\tikzstyle{develop} = [rectangle, rounded corners, minimum width=\dWidth, minimum height=\sHeigh,text width=\dWidth, draw=black,thick,dashed,font=\footnotesize]

\tikzstyle{developB} = [rectangle, rounded corners, minimum width=\dWidthB, minimum height=\sHeigh,text width=\dWidthB, draw=black,thick,dashed,font=\footnotesize]

\tikzstyle{arrow} = [thick,->,>=stealth]
\tikzstyle{connect} = [thick,dashed]



\begin{tikzpicture} [node distance = 1.75cm]
\node (start) [startstop, align = left] {
      $P_{o} ~~= 12W$ \\
      $i_o ~~~= 1A$\\
      %\hrule
      $f_{sw}   ~=  2.77MHz$ \\
      $\eta_{trg}=  90\% $ \\
      };

\node (rscc) [process, below of = start,yshift=-0cm] { $r_{scc,trg} = 1.2\Omega$};
\node (op_point) [process, below of =rscc,yshift=-.5cm] {O.P. in $r_{scc}$ curve?};
\node (rssl) [process,below of = op_point, left of = op_point, xshift = -.25cm] {$r_{ssl,trg} = 845m\Omega$};
\node (opssl) [process,below of = rssl,yshift = 0cm] { $ssl$ optimizer};
\node (sslend) [startstop,below of = opssl, right of = opssl ,yshift=-1.5cm,minimum width = 4cm, text width = 4cm, align=left] {$\mathbf{x} ~=[0.3~0.4~0.28~0.01~ 0.01]$
\\$c_T = 647nF$};
\node (aux1) [below of = opssl,yshift=-0.35cm] {};

\node (rfsl) [process,below of = op_point, right of = op_point, xshift = .25cm] {$r_{fsl,trg} = 845m\Omega$};
\node (opfsl) [process, below of = rfsl,yshift = -0cm] { solve $fsl$ };
\node (fslend) [startstop, below of = opfsl, yshift=-0.35cm] {$r_{on} = 240m\Omega$};



\draw [arrow] (start) --   node (x1){}  (rscc);
\draw [arrow] (rscc) --    node (x2){}  (op_point);

\draw [arrow] (op_point)-| node[yshift=-0.5cm] (x3a){} (rssl);
\draw [arrow] (rssl) --    node (x4a){} (opssl);
\draw [arrow] (opssl) |- (aux1) -|   node (x5a){} (sslend);

\draw [arrow] (op_point)-| node[yshift=-0.5cm] (x3b){} (rfsl);
\draw [arrow] (rfsl) --    node (x4b){} (opfsl);
\draw [arrow] (opfsl) --   node (x5b){} (fslend);

\node (step1) [develop,left of = x1 , xshift = -\xShift ] {With the fixed specs the target $r_{scc}$ is given by,
    \begin{equation}
        r_{scc} = \frac{P_o (1-\eta_{trg})}{i_o ^2} \label{eq:r_scc_trg}
    \end{equation} };
\draw [connect] (step1) -- (x1);

\node (step2) [develop,left of = x2 , xshift = -\xShift]
{The converter is designed to operate in the elbow of the $r_{scc}$ curve, fixing  $r_{ssl} = r_{fsl}$, thus
    \begin{equation}
        r_{scc} = \sqrt{r_{ssl}^2 + r_{fsl}^2} = \sqrt{2}r_{ssl} = \sqrt{2}r_{fsl} .
        \label{eq:r_scc_aprx_dsg}
    \end{equation}
 };
\draw [connect] (step2) -- (x2);

%%%%%%%%%%%%%%%%
% Step 3
%%%%%%%%%%%%%%%%

\node (step3a) [developB,left of = x3a , xshift = -\xShiftB]
{Hence $r_{ssl}$ is then given by
    \begin{equation}
        r_{ssl} = \frac{r_{ssc}}{\sqrt{2}}.
    \end{equation} };
\draw [connect] (step3a) -- (x3a);

\node (step3b) [developB,right of = x3b , xshift = \xShiftB]
{Hence $r_{fsl}$ is then given by
    \begin{equation}
        r_{fsl} = \frac{r_{ssc}}{\sqrt{2}}.
    \end{equation} };
\draw [connect] (step3b) -- (x3b);

%%%%%%%%%%%%%%%%
% Step 4
%%%%%%%%%%%%%%%%

\node (step4a) [developB,left of = x4a , xshift = -\xShiftB]
{Refining the \emph{SSL} function as
    \begin{align}
        \begin{split}
        r_{ssl}  = & \frac{1}{f_{sw} c_{T}} \mathbf{f_{ssl}}(x_1,\cdots,x_n)\\
        x_i      = & \frac{c_i}{c_T}, ~~ c_T  =  \sum_{i=1}^n c_i.
        \end{split}
    \end{align} };
\draw [connect] (step4a) -- (x4a);

\node (step4b) [developB,right of = x4b , xshift = \xShiftB]
{The discrete implementation only uses one switch,
hence all switches have the same $r_{on}$, then for $D=50\%$ \emph{FSL} function results in
    \begin{align}
        r_{fsl}  = r_{on} \frac{7}{2} .
    \end{align} };
\draw [connect] (step4b) -- (x4b);

%%%%%%%%%%%%%%%%
% Step 5
%%%%%%%%%%%%%%%%

\node (step5a) [developB,left of = x5a , xshift = -4.75cm]
{ The capacitor breakdown is given by solving
\begin{equation}
        min( \mathbf{f_{ssl}} )s.t. (1-\sum_{i=1}^n x_i),
    \end{equation}
for $D=50\%$, and $c_T$ is then given by
\begin{equation}
    c_t= \frac{min( \mathbf{f_{ssl}})}{f_{sw} r_{ssl}}.
\end{equation}

};
\draw [connect] (step5a) -- (x5a);

\node (step5b) [developB,right of = x5b , xshift = \xShiftB]
{Hence $r_{on}$ is given by
    \begin{align}
        r_{on}  = r_{fsl} \frac{2}{7}.
    \end{align} };
\draw [connect] (step5b) -- (x5b);

\end{tikzpicture}

    \caption[SCC stage design flow]{Design flow for the SCC stage.}
    \label{fig:design_flow_drv}
    \end{figure}
\end{landscape}


\subsection{Sizing of the capacitors}
The capacitors values were determined using the optimization procedure described in Appendix~\ref{ap:opt_cap}, which resulted in a total capacitance of:
\begin{equation}
c_T = \frac{min(\mathbf{f_{ssl}})}{f_{sw}~r_{ssl}} = \frac{1.9}{2.77MHz~845m\Omega} =810nF,
\end{equation}
and a capacitor breakdown distribution of:
\begin{equation}
\frac{c_i}{c_T} = \irow{ 0.28 & 0.39 & 0.23 & 0.05 & 0.05}.
\end{equation}
Hence the optimal values for the capacitor of the converter are given in Table~\ref{tab:caps_results}, the used values are the best fit to commercial available values. Capacitor $c_5$ was doubled in value to reduce the ripple voltage present at the $v_{aux}$ output.
\begin{table}[!h]
    \renewcommand{\arraystretch}{1.2}

    \centering
    \caption{Capacitor breakdowns, optimization results and used values.}
    \label{tab:caps_results}
    \begin{threeparttable}
    \begin{tabular}{ l | c  c  c  c  c | c | r }
                    & $c_1$   & $c_2$ & $c_3$   & $c_4$  & $c_5$ & $c_T$ &  $r_{ssl}\tnote{1}$  \\
                    & \multicolumn{6}{ c | }{$nF$} &  $m\Omega$  \\

    \midrule
  Optimizer $D=75\%$  & 223  & 320 & 181  & 43   & 43  & 810 &  845  \\
  %Optimizer $D=50\%$  & 194  & 258 & 181  & 6      & 6  & 647 &  845 \\
  %Optimizer $D=10\cdots90\%$  & 247  & 305 & 230  & 46  & 46 & 875 &  735  \\

  Used        & 220  & 330 & 180  & 39  &  78 & 947 &  716\\
   \midrule
   \multirow{2}{*}{Voltages}  & \multicolumn{6}{ c | }{$V$} &  - \\
              & 9.6  & 9.6 & 9.6  & 4.8 & 4.8 & 9.6 &  -
    \end{tabular}
    \begin{tablenotes}
        \item [1] Value computed for a duty cycle $D=75\%$.
    \end{tablenotes}
    \end{threeparttable}
\end{table}


\subsection{Sizing of the transistors}
For sake of simplicity, it was only used a single type of transistors for the prototype, what  simplifies the process to determine $r_{on}$ since it is not necessary to minimize~\eqref{eq:r_fsl_dsg}. Solving~\eqref{eq:r_fsl_dsg} for $D=75\%$ and considering the same $r_{on}$ in all the switches results in
\begin{equation}
    r_{fsl}  = 3.8 r_{on}  .
\end{equation}
Hence to satisfy the target $r_{fsl,trg}=845m\Omega$, the switches must have a maximum $r_{on}$ of
\begin{equation}
    r_{on}  = \frac{r_{fsl,trg}}{3.8} = 222m\Omega,
\end{equation}
based on a balance between switching losses and conduction losses we used \emph{ZXMN2B01F} from \emph{ZETEX} featuring $r_{on}=100m\Omega$.

Table~\ref{tab:switchs_results} presents average current and blocking voltage in each device.
\begin{table}[!h]
    \renewcommand{\arraystretch}{1.2}
    \centering
    \caption{Switches blocking voltages and average currents.}
    \label{tab:switchs_results}
    \begin{threeparttable}
    \begin{tabular}{ l c | c  c  c  c  c  c c c c }
            & & $s_1$   & $s_2$ & $s_3$   & $s_4$  & $s_5$ & $s_6$ & $s_7$ & $s_8$ & $s_9$  \\
    \midrule
            $v_{ds}$ & V & 4.8 & 9.6& 9.6 & 9.6 & 4.8 & 4.8 & 4.8 & 4.8 & 4.8 \\
            $i_{on}$ & A & 0.2 & 0.2& 0.2 & 0.2 & 0.2 & 0.4 & 0.4 & 0.4 & 0.4 \\

    \end{tabular}
    \end{threeparttable}
\end{table}

\subsection{Inductor}
The inductor was designed for an small ripple of  $\Delta _i \pm10\%$. That allows the current to be dimmed up to $100mA$ without bringing the converter in discontinuous conduction mode (DCM). The value of the output inductor is determined by~\eqref{eq:hscc_l} resulting in
\begin{equation}
 l_{o,hscc}  = m_i \frac{ v_{src} D (1-D)}{\Delta i~f_{sw} }= \frac{1}{5} \frac{24V~0.75(1-0.75)}{0.2~1A~2.77MHz} = 1.62\mu H.
\label{eq:hscc_l_II}
\end{equation}
Considering the tolerances the mounted component was the \emph{CVH252009} from \emph{BOURNS} featuring $l_o=2.2\mu H$ in a 1008 SMD case.

\section{Close-loop control design}
The converter requires a close-loop control to properly operate the LED load.  Therefore a second board was build for that purpose, implementing an error amplifier, a close-loop controller, a ramp generator and a dual PWM with dead-band generator. Commercial ICs were not suitable for our application due to its high switching frequency, 2.77MHz.
\begin{figure}[!h]
    \centering
    \newcommand\sHeigh{.75cm}
\newcommand\sWidth{1cm}
\newcommand\cHeigh{.75cm}


\usetikzlibrary{shapes.geometric, arrows}
\tikzstyle{block} = [rectangle, minimum height=\sHeigh,text centered, draw=black,font=\footnotesize]
\tikzstyle{sp} = [rectangle, minimum width=\sWidth, minimum height=\sHeigh,text width=\sWidth,text centered,font=\footnotesize]
\tikzstyle{sum} = [circle,minimum width=\cHeigh,font=\footnotesize,draw=black]
\tikzstyle{arrow} = [thick,->,>=stealth]

\begin{tikzpicture} [node distance = 2.25cm]
\node (sp) [sp,text width=1.25cm] {Set point\\ ${i_{out}}^*$};
\node (error) [sum,right of =sp, xshift=-0.5cm] {};
\node (comp) [block, right of = error, xshift=-0.5cm] {$G_c(s)$};
\node (pwm) [block, right of = comp] {$G_{pwm}$ };
\node (H-SCC) [block,right of = pwm] { $G_p (s)$};
\node (sens) [block,below of = pwm,yshift=+0.75cm] {$H_s(s)$};
\node (out) [right of = H-SCC] {$i_{out}$};

\draw [arrow] (sp) -- (error)node[font=\tiny,xshift=-0.25cm]{\textbf{+}};
\draw [arrow] (error) --node[font=\footnotesize,anchor=south]{$\epsilon _r$} (comp);
\draw [arrow] (comp) --node[font=\footnotesize,anchor=south]{$D$} (pwm);
\draw [arrow] (pwm)--node[font=\footnotesize,anchor=south]{$\phi_1,\phi_2$} (H-SCC);
\draw [arrow] (H-SCC) --node(x_out)[yshift=0.12cm]{} (out);
\draw [arrow] (x_out) |- (sens);
\draw [arrow] (sens) -| (error)node[font=\footnotesize,yshift=-0.25cm]{\textbf{-}};






\end{tikzpicture}

    \caption[]{Close-loop block diagram.}
    \label{fig:close_loop_diagram}
\end{figure}

Owing to the fact that the H-SCC has an output inductor, the close-loop control is designed like in an inductive converter. First the transfer function of the power train is obtained, in this case including the SCC stage. Second with the transfer function the close-loop controller is designed to guarantee the stability and minimize the error between the set-point and the output current of the LED.

\subsection{Small signal analysis}

The small signal analysis of a H-SCC is practically the same of a buck converter. Figure~\ref{fig:small_signal_hscc} shows the equivalent circuit of a H-SCC used for the small signal analysis, the SCC stage is modeled with the voltage source controlled by the duty cycle in series with the equivalent output resistance $r_{scc}$. The output filter, composed by inductor $l_o$ and the capacitor $c_o$, is connected to the output of the SCC stage and afterwards the load $r_o$. For sake of simplicity, the equivalent series resistance of $l_o$ and $c_o$ are not included.
\begin{figure}[!h]
    \centering
    \ctikzset { bipoles/length=1cm}
\centering
\begin{circuitikz}[american,scale=0.6]
 
    \draw 
          (-1,1)   to[V,l=$v_{src} m_{x}(D)$] 
          (-1,4) to[short,,i=$i_l$](1,4)   to[R,l=$r_{scc}$] 
          (3,4) to[cute inductor,l=$l_o$,v=$v_l$] 
          (6,4) to[C,l_=$c_o$,v^=$v_c$] (6,1)
          (6,4) to[short,i=$i_o$] (8,4) to[R,l=$r_o$] 
          (8,1) -- (-1,1);
           
          
          
\end{circuitikz}
    
    \caption[]{Equivalent circuit of a \emph{hybrid}-SCC including the output filter.}
    \label{fig:small_signal_hscc}
\end{figure}

Compared to the analysis of a buck converter, the SCC stage adds the $r_{scc}$ and modifies the conversion ration of the converter. The conversion ratio provided by the SCC stage has to components, a fixed offset $m_{off}$ added to a variable fraction $m_{\Delta}$ controlled by the duty cycle $D$, thus
\begin{equation}
 m_{x}  = m_{off} + m_{\Delta}D.
\label{eq:m_ratio}
\end{equation}
For the case under study the conversion ratio at the third node is
\begin{equation}
 m_{3}  = \frac{2}{5} + \frac{D}{5}.
\label{eq:m3_ratio}
\end{equation}
thus $m_{offset} = \frac{2}{5}$ and $m_{\Delta}=\frac{1}{5}$.

Using~\eqref{eq:m_ratio} can be written the equations for the two state variables, inductor current and capacitor voltage, resulting in
\begin{align}
 l_o \frac{i_l}{dt} = &  v_{src} (m_{off} + m_{\Delta} D ) - i_l r_{scc} - v_o \label{eq:il}\\
 c_o \frac{v_c}{dt} = &  i_l - i_o.
 \label{eq:vc}
\end{align}

Applying the small signal analysis into~\eqref{eq:il} and~\eqref{eq:vc}, we can obtain the different transfer functions of the converter.
\begin{align}
 G_{id}(s) =  \frac{\widehat{i_l}}{\widehat{d}} = &
 \frac{m_{\Delta} v_{src}}{r_o} \frac{s c_o r_o + 1  }
 { s^2 l_o c_o  + s \left ( \frac{l_o}{r_o}  + c_o r_{scc}  \right )  +  \frac{r_{scc}}{r_o} + 1 } \label{eq:g_id}\\
 G_{vd}(s) =  \frac{\widehat{v_o}}{\widehat{d}} = &
 m_{\Delta} v_{src} \frac{ 1  }
 { s^2 l_o c_o  + s \left ( \frac{l_o}{r_o}  + c_o r_{scc}  \right )  +  \frac{r_{scc}}{r_o} + 1 } \label{eq:g_id}\\
 G_{od}(s) =  \frac{\widehat{i_o}}{\widehat{d}} = &
 \frac{ m_{\Delta}v_{src}}{r_o} \frac{1  }
 { s^2 l_o c_o  + s \left ( \frac{l_o}{r_o}  + c_o r_{scc}  \right )  +  \frac{r_{scc}}{r_o} + 1 } \label{eq:g_od}
\end{align}

Notice that the resulting transfer functions are practically the same of a buck converter, with the exception of two new parameters the output resistance of the SCC stage ($r_{scc}$), and the gain added by $m_\Delta$. In a Dickson and Ladder converter $m_\Delta$ is equal to the intrinsic conversion ration $m_i$ fixed by the topology.

\subsection{Close-loop controller}
\begin{figure}[!h]
    \newcommand\pHeigh{3cm}
    \newcommand\pWidth{8cm}
    \centering
    \begin{tikzpicture}
    \pgfplotsset{
        width=\pWidth,
        height=\pHeigh,
        scale only axis,
        xlabel near ticks,
        ylabel near ticks,
        scaled y ticks= true ,
        enlarge x limits={0},
        enlarge y limits={0.1},
        %xticklabels={,,},
        every tick label/.append style={font=\footnotesize},
        }

	\begin{semilogxaxis}[
		axis y line*=left,
		axis x line*=bottom,
		yticklabel style={text width=2.0em,align=right},
        ylabel = {magnitude $[dB]$},
        ylabel style = {font=\footnotesize},
        title = {$Gm =19.6dB$ (at $39.7kHz$), $Pm =95.5dB$ (at $8.8kHz$)},
        title style = {
    			font=\footnotesize,
    			at ={(0.5,0.9)}},
        xticklabels={,,},
        name=mag
		]

		\addplot[thick,black,smooth]
    		table [y=y1]{./5_hscc_led_driver/OpenLoop.dat};

        \addplot[semithick,dotted,mark=none, black]
            coordinates { (1e2,0) (1e7,0)};
        \addplot[semithick,dotted,black] coordinates {(8.81e3,-150)(8.81e3,0) };
        \addplot[thin,black] coordinates {(3.97e4,0)(3.97e4,-19.6) };
        \addplot[semithick,dotted,black] coordinates {(3.97e4,-150)(3.97e4,-19.6) };

	\end{semilogxaxis}

    \begin{semilogxaxis}[
        yshift={-(\pHeigh+0.25cm)},
		axis y line*=left,
		axis x line*=bottom,
		yticklabel style={text width=2.0em,align=right},
        ylabel = {phase $[deg]$},
        xlabel = {frequency $[Hz]$},
        ylabel style = {font=\footnotesize},
        xlabel style = {font=\footnotesize},
        ytick = {0,-90,-180,-270},
        title style = {
    			font=\footnotesize,
    			at ={(0.5,0.9)}},
		]

		\addplot[thick,black,smooth]
    		table [y=y2]{./5_hscc_led_driver/OpenLoop.dat};
        \addplot[semithick,dotted,mark=none,black]
            coordinates { (1e2,-180) (1e7,-180)};

        \addplot[thin,black] coordinates {(8.81e3,-180)(8.81e3,-84.5)};
        \addplot[semithick,dotted,black] coordinates {(8.81e3,-84.5)(8.81e3,0) };

        \addplot[semithick,dotted,black] coordinates {(3.97e4,0)(3.97e4,-180) };

	\end{semilogxaxis}
    
    %\begin{semilogxaxis}[
%        yshift={-(\pHeigh+0.5cm)},
%        at=(mag.below south west),
%		axis y line*=left,
%		axis x line*=bottom,
%        ytick = {0,-90,-180,-270},
%		yticklabel style={text width=2.0em,align=right},
%        ylabel = {Phase $[deg]$},
%        xlabel = {Frequency $[Hz]$},
%        ylabel style = {font=\footnotesize},
%        xlabel style = {font=\footnotesize},
%		title style = {
%    			font=\footnotesize },
%    			at ={(0.75,0.75)},
%		]
%
%		\addplot[thick,black,smooth]
%    		table [y=y2]{./5_hscc_led_driver/OpenLoop.dat};
%        \addplot[semithick,dotted,mark=none,black]
%            coordinates { (1e2,-180) (1e7,-180)};
%
%        \addplot[thin,black] coordinates {(8.81e3,-180)(8.81e3,-84.5)};
%        \addplot[semithick,dotted,black] coordinates {(8.81e3,-84.5)(8.81e3,0) };
%
%        \addplot[semithick,dotted,black] coordinates {(3.97e4,0)(3.97e4,-180) };
%
%
%	\end{semilogxaxis}

\end{tikzpicture}

    \caption[]{Bode plot of the open-loop ($G_{od}G_{pwm}G_s$) system without compensation.}
    \label{fig:ol_bode}
\end{figure}
Figure~\ref{fig:ol_bode} plots the uncompensated gain-loop $G_{ov}G_s G_{pwm}$ of the system. The system is stable without a compensation network, the system has a strong rejection of the high frequency noise with an attenuation the switching frequency, $124dB$ at $2.77MHz$. However the poor gain of $4.5dB$ at low frequencies is not enough to guaranty a small error between the output current and the set point and high rejection to the line variations.

\begin{figure}[!h]
    \newcommand\pHeigh{3cm}
    \newcommand\pWidth{8cm}
    \centering
    \begin{tikzpicture}
    \pgfplotsset{
        width=\pWidth,
        height=\pHeigh,
        scale only axis,
        xlabel near ticks,
        ylabel near ticks,
        scaled y ticks= true ,
        enlarge x limits={0},
        enlarge y limits={0.1},
        %xticklabels={,,},
        every tick label/.append style={font=\footnotesize},
        }

	\begin{semilogxaxis}[
		axis y line*=left,
		axis x line*=bottom,
		yticklabel style={text width=2.0em,align=right},
        ylabel = {magnitude $[dB]$},
        ylabel style = {font=\footnotesize},
        title = {$Gm =22.3dB$ (at $9.6kHz$), $Pm =75.1^\circ$ (at $1.31kHz$)},
        title style = {
    			font=\footnotesize,
    			at ={(0.5,0.9)}},
        xticklabels={,,},
		]

		\addplot[thick,black,smooth]
    		table [y=y1]{./5_hscc_led_driver/CloseLoop.dat};

        \addplot[semithick,dotted,mark=none, black]
            coordinates { (1e2,0) (1e7,0)};

        \addplot[semithick,dotted,black] coordinates {(1.31e3,0)(1.31e3,-239) };
        \addplot[semithick,dotted,black] coordinates {(9.62e3,-22.3)(9.62e3,-239) };
        \addplot[thin,black] coordinates {(9.62e3,-22.3)(9.62e3,0) };

	\end{semilogxaxis}
    \begin{semilogxaxis}[
        yshift={-(\pHeigh+0.25cm)},
        axis y line*=left,
		axis x line*=bottom,
        ytick = {-90,-180,-270,-360},
		yticklabel style={text width=2.0em,align=right},
        ylabel = {phase $[deg]$},
        xlabel = {Frequency $[Hz]$},
        ylabel style = {font=\footnotesize},
        xlabel style = {font=\footnotesize},
        %title = {$Gm =22.3dB$ (at $9.6kHz$), $Pm =75.1^\circ$ (at $1.31kHz$)},
        title style = {
    			font=\footnotesize,
    			at ={(0.5,0.9)}},
		]

		\addplot[thick,black,smooth]
    		table [y=y2]{./5_hscc_led_driver/CloseLoop.dat};
        \addplot[semithick,dotted,mark=none, black]
            coordinates { (1e2,-180) (1e7,-180)};

        \addplot[thin,black] coordinates {(1.31e3,-180)(1.31e3,-104.9)};
        \addplot[semithick,dotted,black] coordinates {(1.31e3,-104.9)(1.31e3,-90) };

        \addplot[semithick,dotted,black] coordinates {(9.62e3,-180)(9.62e3,-90) };


	\end{semilogxaxis}

\end{tikzpicture}

    \caption[]{Bode plot of the compensated open-loop ($G_{c}G_{od}G_{pwm}G_s$) system.}
    \label{fig:ol_cm_bode}
\end{figure}

The robustness of the system is improved compensating the system with an integrator ($G_c(s) = \frac{5000}{s}$), improving the gain at low frequencies and further attenuating the high frequency noise. The compensated gain-loop plotted in Figure~\ref{fig:ol_cm_bode} has a phase margin of $Pm = 75dB$ and cut-off frequency $f_{c} =1.31kHz$, resulting in a fast and smooth close-loop response of the converter.

\section{Results}
The converter was tested under two different scenarios. First the power train was tested operating the converter in open-loop, and loading it with an electronic load in resistive mode. Second the full system was tested operating the converter in close-loop, and loading it with the LED load. In the fist scenario, the efficiency and the output resistance of the converter was measured. The measured results where compared with Spcie circuit simulations.  In the second scenario, the efficiency was measured again but in this case with the electronic load. The close-loop control was also validated.

\subsection{Experimental setup configuration}
\begin{figure}[!h]
\centering
\ctikzset { bipoles/length=1cm}
\begin{circuitikz}[american,scale=0.65]
\draw [yshift=-1.5cm]
    (2.5,0) to[short]
    (-1.5,0) to[V = $v_{src}$] %(-1.5,3) -- (2.5,3)
    (-1.5,3) to[ammeter,l=$i_{in}$]  (1,3) -- (2.5,3)
    (1,3) to[voltmeter,l_=$v_{in}$] (1,0);


\draw [thick]
    (2.5,-3.5) --
    (2.5,3.5)  --
    (5.5,3.5)  --
    (5.5,-3.5) --
    (2.5,-3.5);

\draw (4,0) node[align=center]{5:1 SCC \\ U.T.} ;


\draw [yshift=0.5cm]
    (5.5,2.5) --
    (7,2.5) to[cute inductor,l=$l_o$]
    (9.5,2.5) -- (11.5,2.5) to[ammeter,l_=$i_{2}$]  (14,2.5) to[I,l = Load $v_{LED}$]
    (14,0) -- (5.5,0)
    (7,2.5) to[voltmeter,l_=$v_{x}$] (7,0)
    (9.5,2.5) to[C,l_=$c_o$] (9.5,0)
    (11.5,2.5) to[voltmeter,l_=$v_{2}$] (11.5,0);



\draw[yshift=-3.5cm]
    (5.5,2.5) --
    (7,2.5) to[ammeter,l_=$i_{1}$]
     (10.5,2.5) to[I,l = Load $v_{aux}$]
    (10.5,0.5) -- (5.5,0.5)
    (7,2.5) to[voltmeter,l_=$v_{1}$] (7,0.5);


\end{circuitikz}
\caption{Experimental arrangement for measuring the H-SCC LED Driver.}
\label{fig:exp_setup}
\end{figure}

Figure~\ref{fig:exp_setup} shows the experimental setup used to measure the power converter. Voltages were measured with four different Keithley\textsuperscript{\textregistered} \emph{Meter 2000}, and the currents with three different Keithley\textsuperscript{\textregistered} \emph{SourceMeter 2440}.


The efficiency of the LED output is given by
\begin{equation}
 \eta = \frac{P_{LED}}{P_{in}}= 100 \frac{v_2 i_2}{v_{in} i_{in}},
\end{equation}
and the efficiency of the auxiliary output is given by
\begin{equation}
 \eta = \frac{P_{aux}}{P_{in}}= 100 \frac{v_1 i_1}{v_{in} i_{in}}.
\end{equation}

The equivalent output resistance $r_{scc}$ was determined for different switching frequencies. At each frequency, the output load was swept from no load to full load, and  $r_{scc}$ was obtained as a result of a curve fitting. Two different fitting functions were used to validate the obtained results. One fitting function is based on the measured power losses $P_{loss} = P_{in} - P_{out}$, defined by the following function
\begin{equation}
 P_{loss}(i_{out}) = r_{rscc} i_{out}^2 + P_{sw}.
\end{equation}
%where the power losses are assumed to have two terms. The fix term are the switching losses $P_{sw}$, which are assumed to be independent of the output current $i_{out}$. The conduction losses are proportional to the $r_{scc}$ with respect to the square of $i_{out}$.
The other fitting function, is based on the measured output voltage of a SCC, defined by the following function
\begin{equation}
 v_{out}(i_{out}) = v_{trg} -  r_{rscc} i_{out}.
\end{equation}
%where, as in the previous case, the output voltage is described by two terms. The fix terms is the target voltage  $v_{trg}$
The $r_{scc}$ of the LED output was determined using the measurements from the voltmeter connected to the  switching node $v_x$ instead of  the voltmeter connected to the output  $v_2$, in this way the measured $r_{scc}$ does not include the series resistance of the output inductor $l_o$.

\subsection{Open-loop measurements}
The open-loop measurements were done connecting the converter to an electronic load in resistive mode and operating the converter at $2.77MHz$ switching frequency with a $50\%$ duty cycle. The measurements were done independently for each individual output, always loading only one of the outputs of the converter. Nevertheless by combining the measured results for each individual output, the total efficiency at full load, $i_{LED}=1A$ and $i_{aux}=400mA$, was predicted to be $87\%$, being this result above the specified $85\%$ minimum efficiency defined in the requirements.


\subsubsection{Measurements and results of the LED output}
The full load efficiency, $i_{LED}=1A$, is $88.8\%$, and the peak efficiency is $91.1\%$ at $i_{LED}=641mA$, as shown in the plot of Figure~\ref{fig:eff_vled}.

\begin{figure}[!h]
    \newcommand\pHeigh{4.5cm}
    \newcommand\pWidth{8cm}
    \centering
    \input{./5_hscc_led_driver/v_led_eff.tex}
    \caption[]{Efficiency versus a sweep in the output current at $2.77MHz$.}
    \label{fig:eff_vled}
\end{figure}
Figure~\ref{fig:eff_vled_param} presents the measured efficiency (\emph{top}) and power losses (\emph{bottom}) for a sweep of the output current and parameterized for different switching frequencies. It can be seen that for the high current range ($1A-550mA$), the best efficiencies are obtained operating the converter at high frequency, being indeed the curve corresponding to $f_{sw}=2.77MHz$ the one that achieves the best results. Around $i_{LED}=550mA$, the curve operating with $f_{sw}=2MHz$ achieves the best efficiency, above the $90\%$ for the range between $550-300mA$. Operating with $f_{sw}=1MHz$ achieves the best efficiency for the current range below $100mA$. Hence Figure~\ref{fig:eff_vled_param} presents the measured efficiency (\emph{top}) and power losses (\emph{bottom}) for a sweep of the output current and parameterized for different switching frequencies,  the efficiency is improved by reducing the operating switching frequency.

This tendency is caused by the fact that the two source of losses, switching and conduction losses have opposite trends with the frequency. On the one hand, the switching losses increase with the frequency, since they are given by
\begin{equation}
    P_{sw} = v^2 c_{out} f_{sw}
\end{equation}
where $v$ is the blocking voltage at the switch and $c_{out}$ is the output switch capacitance. On the other hand, the conduction losses of a SCC decrease with the switching frequency, since they are totally produced by the $r_{scc}$. As it was already described the $r_{scc}$ is inversely proportional to the switching frequency for the SSL region, where these measurements were taken.

\begin{figure}[!h]
    \newcommand\pHeigh{4.5cm}
    \newcommand\pWidth{8cm}
    \centering
    \begin{tikzpicture}
    \pgfplotsset{
        width=\pWidth,
        height=\pHeigh,
        scale only axis,
        xlabel near ticks,
        ylabel near ticks,
        scaled y ticks= true ,
        enlarge x limits={0.1},
        enlarge y limits={0.1},
        every tick label/.append style={font=\footnotesize},
        }

    \begin{axis}[
		axis y line*=left,
		axis x line*=bottom,
        xticklabels={,,},
        %ytick = {-90,-180,-270,-360},
		yticklabel style={text width=1.0em,align=right},
        ylabel = {efficiency $[\%]$},
        ylabel style = {font=\footnotesize},
        xlabel style = {font=\footnotesize},
		legend style = {
    		font   = \footnotesize,
            anchor = south,
            at = {(0.5,1)},
            draw= none,
            legend columns = 2,
            column sep = 1ex,
            legend cell align = left},
		]
        \addplot[thick,black,mark=square,mark repeat=4,smooth]
    		table [y=y1]{./5_hscc_led_driver/vled_out/vled_eff_ploss_2.77MHz.dat};
        
        \addplot[thick,black,mark=o,mark repeat=4,smooth]
    		table [y=y1]{./5_hscc_led_driver/vled_out/vled_eff_ploss_2MHz.dat};
    
        \addplot[thick,black,mark=triangle,mark repeat=4,smooth]
    		table [y=y1]{./5_hscc_led_driver/vled_out/vled_eff_ploss_1MHz.dat};
    
        \addplot[thick,black,mark=star,mark repeat=4,smooth]
    		table [y=y1]{./5_hscc_led_driver/vled_out/vled_eff_ploss_500kHz.dat};       
        
        \legend{$f_{sw}=2.77MHz$,$f_{sw}=2MHz$,$f_{sw}=1MHz$,$f_{sw}=500kHz$};
	\end{axis}

    \begin{axis}[yshift={-(\pHeigh+0.5cm)},
		axis y line*=left,
		axis x line*=bottom,
        yticklabel style={text width=1.0em,align=right},
        ylabel = {$P_{loss}~ [W]$},
        xlabel = {load current $[mA]$},
        ylabel style = {font=\footnotesize},
        xlabel style = {font=\footnotesize},
		legend style = {
    		font   = \footnotesize,
            anchor = south,
            at = {(0.5,1)},
            draw= none,
            legend columns = 2,
            column sep = 1ex,
            legend cell align = left},
		]
        \addplot[thick,black,mark=square,mark repeat=4,smooth]
    		table [y=y2]{./5_hscc_led_driver/vled_out/vled_eff_ploss_2.77MHz.dat};

        \addplot[thick,black,mark=o,mark repeat=4,smooth]
    		table [y=y2]{./5_hscc_led_driver/vled_out/vled_eff_ploss_2MHz.dat};

        \addplot[thick,black,mark=triangle,mark repeat=4,smooth]
    		table [y=y2]{./5_hscc_led_driver/vled_out/vled_eff_ploss_1MHz.dat};

        \addplot[thick,black,mark=star,mark repeat=4,smooth]
    		table [y=y2]{./5_hscc_led_driver/vled_out/vled_eff_ploss_500kHz.dat};

        %\legend{$f_{sw}=2.77MHz$,$f_{sw}=2MHz$,$f_{sw}=1MHz$,$f_{sw}=500kHz$};
	\end{axis}
\end{tikzpicture}

    \caption[]{Efficiency (\emph{top}) and losses (\emph{bottom}) versus a sweep in the output current parameterized for different switching frequencies.}
    \label{fig:eff_vled_param}
\end{figure}

\begin{figure}[!h]
    \newcommand\pHeigh{4.5cm}
    \newcommand\pWidth{8cm}
    \centering
    \input{./5_hscc_led_driver/r_scc_vled.tex}
    \caption[]{Comparison between the measured $r_{scc}$ and the predicted between the model.}
    \label{fig:rscc_vled}
\end{figure}


%%%%%%%%%%%%%%%% Vaux %%%%%%%%%%%%%

\begin{figure}[!h]
    \newcommand\pHeigh{4.5cm}
    \newcommand\pWidth{8cm}
    \centering
    \begin{tikzpicture}
    \pgfplotsset{
        width=\pWidth,
        height=\pHeigh,
        scale only axis,
        xlabel near ticks,
        ylabel near ticks,
        scaled y ticks= true ,
        enlarge x limits={0.1},
        enlarge y limits={0.1},
        %xticklabels={,,},
        every tick label/.append style={font=\footnotesize},
        }

    \begin{axis}[
		axis y line*=left,
		axis x line*=bottom,
        %ytick = {-90,-180,-270,-360},
		yticklabel style={text width=2.0em,align=right},
        ylabel = {efficiency $[\%]$},
        xlabel = {load current $[mA]$},
        ylabel style = {font=\footnotesize},
        xlabel style = {font=\footnotesize},
		title style = {
    			font=\footnotesize },
    			at ={(0.75,0.75)},
		]

		\addplot[thick,black,smooth]
    		table [y=y1]{./5_hscc_led_driver/vaux_out/vaux_eff_ploss_2.77MHz.dat};


	\end{axis}
\end{tikzpicture}

    \caption[]{Efficiency versus a sweep in the output current at $2.77MHz$.}
    \label{fig:eff_vled}
\end{figure}

\begin{figure}[!h]
    \newcommand\pHeigh{4.5cm}
    \newcommand\pWidth{8cm}
    \centering
    \begin{tikzpicture}
    \pgfplotsset{
        width=\pWidth,
        height=\pHeigh,
        scale only axis,
        xlabel near ticks,
        ylabel near ticks,
        scaled y ticks= true ,
        enlarge x limits={0.1},
        enlarge y limits={0.1},
        every tick label/.append style={font=\footnotesize},
        }

    \begin{loglogaxis}[
		axis y line*=left,
		axis x line*=bottom,
        %xticklabels={,,},
        %ytick = {-90,-180,-270,-360},
		yticklabel style={text width=1.0em,align=right},
        ylabel = { $r_{scc}~[\Omega]$},
        xlabel = { $frequency~[Hz]$},
        ylabel style = {font=\footnotesize},
        xlabel style = {font=\footnotesize},
		legend style = {
    		font   = \footnotesize,
            anchor = south,
            at = {(0.5,1)},
            draw= none,
            legend columns = -1,
            column sep = 1ex,
            legend cell align = left},
		]

        \addplot[thick,only marks,mark=x,mark repeat=1]
    		table [y=y2]{./5_hscc_led_driver/vaux_out/vaux_rscc_psw.dat};
        
        \addplot[thick,only marks,mark=o,mark repeat=1]
    		table [y=y1]{./5_hscc_led_driver/vaux_out/vaux_rscc_psw.dat};

        \addplot[thick,black,smooth]
    		table [y=y2]{./5_hscc_led_driver/vaux_out/vaux_rscc_ideal.dat};
        \legend{fit $v_{out}$, fit $p_{loss}$,model};
	\end{loglogaxis}

\end{tikzpicture}

    \caption[]{Comparison between the measured $r_{scc}$ and the predicted between the model.}
    \label{fig:rscc_vled}
\end{figure}


%\section{Power train circuits}
%\subsection{Full schematic}
%%\begin{figure}[!h]
%%\centering
%%\ctikzset{bipoles/length=1cm}
\begin{circuitikz} [american,scale=0.65]
 \draw (5,5)    to[Tnigfetd,n=m5]
       (5,7.5)  to[Tnigfetd,n=m4]
       (5,10)   to[Tnigfetd,n=m3]
       (5,12.5) to[Tnigfetd,n=m2]
       (5,15)   to[Tnigfetd,n=m1] (5,17.5)

       (2,0)    to[Tnigfetd,n=m7]
       (2,2.5)  to[Tnigfetd,n=m6] (2,5)

       (8,0)    to[Tnigfetd,n=m8,mirror]
       (8,2.5)  to[Tnigfetd,n=m9,mirror] (8,5)

       (2,5) -| (m5.S) |- (8,5)
       (2,0) -- (8,0)
       (5,0) node[sground]{};


  %Apply labels to the MOSFETS
  \draw
        (m1.B) node[anchor=south west] {$M_1$}
        (m1.B) node[anchor=north  west,font=\tiny] {ZXMN2B01F}
        (m2.B) node[anchor=west] {$M_2$}
        (m3.B) node[anchor=west] {$M_3$}
        (m4.B) node[anchor=west] {$M_4$}
        (m5.B) node[anchor=west] {$M_5$}
        (m6.B) node[anchor=west] {$M_6$}
        (m7.B) node[anchor=west] {$M_7$}
        (m8.B) node[anchor=east] {$M_8$}
        (m9.B) node[anchor=east] {$M_9$};

  %Capacitor legs
  \draw
       (2,2.5) --
       (0,2.5) -- (0,4.5) to[C,text width={2.5em},l_={$c_4$  $39nF$}]
       (0,7.5) -- (5,7.5)
       (0,7.5) to[C,text width={2.5em},l_={$c_2$ $330nF$}]
       (0,12.5) -- (5,12.5)

       (8,2.5) --
       (10,2.5) -- (10,5) to[C,text width={2.5em},align=right,l={$c_3$ $180nF$}]
       (10,10) -- (5,10)
       (10,10) to[C,text width={2.5em},align=right,l={$c_1$ $220nF$}]
       (10,15) -- (5,15)

       (5,0) to[C,text width={2.5em},l_={$c_5$ $78nF$}] (5,5);

  \draw
        (5,17.5) node[spdt,rotate=-90,anchor=out 2](SW1){}
        (SW1.in) node[rground,yscale=-1]{}
        (SW1.in) node[anchor=south,above=0.25cm] {24V}
        (SW1.out 2) node[above=3.5mm] {$sw_1~~$}

        (SW1.out 1) |- (6,17) to[short,-*] (6.5,17) node[anchor=west,font=\tiny] {\textbf{24V\_SUP}};
  \draw [dashed] (5.3,18.4) -- (6.8,18.4) node[anchor=west,font=\tiny,text width=2cm] {to relay start-up circuit} ;

  \draw (-3,12.5) to[cute inductor,l=$l_o$,n=lo]  (0,12.5)
        (-3,12.5) to[C,l=$c_{o}$] (-3,8)
        (-3,12.5) -- (-4.5,12.5) --
        (-4.5,11.5) to[short,-o]  (-5,11.5)
        (-3.75,8)  node[sground] {}
        (-4.5,12.5) to[short,-*] (-4.5,13)  node[anchor=south,font=\tiny] {\textbf{+V\_LED}}
        (lo.s) node[anchor=north] {$2.2\mu H$}

        (-5,10.5) to[short,o-]
        (-4.5,10.5) to[R,text width={2.8em},align=right,l_={$r_{s}$ $100m\Omega$}]
        (-4.5,8) -- (-3,8)
        (-5.25,11.5) to[open,v=$v_{LED}$] (-5.25,10.5);


  \draw
        (-5,4) to[short,o-] (-3.5,4) -- (-3.5,5) -- ([hs]0,6 |- -5,5) arc(180:0:\radius) -- (2,5)
        (-5,2) to[short,o-] (-4.5,2) -- (-4.5,0)
        (-5.25,4) to[open,v=$v_{aux}$] (-5.25,2)
        %(-3,5) to[C,l=$c_o2$] (-3,0) -- (-4.5,0)
        (-4.5,0) node[sground]{}
        (-3.5,4) to[short,-*] (-3.5,3.5) node[anchor=north,font=\tiny] {\textbf{+V\_AUX}};

  \draw (-4,6) node[spdt,anchor=in,rotate=180,yscale=-1](SW2){} -| (-3.5,5)
        (SW2.out 1) -| (-6,6.75) to[short,-*] (-6,7)  node[anchor=south,font=\tiny] {\textbf{5V\_SUP}}
        (SW2.out 2) node[anchor=south] {$sw_2$};

  \draw [dashed] (-5,5.7) -- (-5,6.5) -- (-4.5,6.5) node[anchor=west,font=\tiny,text width=2cm] {to relay start-up circuit};

  \draw (10,15) to[short,-*] (10,15.5) node[anchor=south,font=\tiny] {\textbf{V\_CAPS}};

\end{circuitikz}

%%\caption[Power train schematic]{5:1 H$^2$-Dickson power train schematic.}
%%\label{fig:pwr_train_sch}
%%\end{figure}
%
%%\begin{landscape}
%%\thispagestyle{empty}
%
%%\end{landscape}
%
%%\subsection{Gate driver}
%%\begin{figure}[!h]
%%\centering
%%\input{./5_hscc_led_driver/gate_driver_sch.tex}
%%\caption[Power train schematic]{5:1 H$^2$-Dickson power train schematic.}
%%\label{fig:pwr_train_sch}
%%\end{figure}
%
%\subsection{Start-up helper circuit}
%\begin{figure}[!h]
%\centering
%
\ctikzset{bipoles/length=0.75cm}
\begin{circuitikz} [american,scale=0.65]
 \draw (0,0) node[sground] {}  to[zD*,n=dz1,l={$d_1$}] (0,3)
       (-2.5,3) to[R,l={$r_1$},n=r1,-*]
       (-2.5,6.75) node[anchor=south,font=\tiny] {\textbf{24V\_SUP}}
       (dz1.s) node[anchor=north,rotate=90,font=\tiny] {BZX84C3V9}
       (0,3) --
       (-2.5,3) to[C,l={$c_{10}$},n=c1]
       (-2.5,0) -- (0,0)
       (-2.5,3) -- (-3,3) to[D*,n=d1,l={$d_2$},-*]
       (-5,3) node[anchor=east,font=\tiny] {\textbf{5V\_SUP}}
       (d1.s) node[anchor=south,font=\tiny] {BAS40}
       (r1.s) node[anchor=north,rotate=90] {\tiny{${1k\Omega}$}}
       (c1.s) node[rotate=90] {\tiny{${1\mu F}$}};

 \draw (1.65,4.5) rectangle (2.25,5.5);
 \draw[dashed] (2.25,5) -- (0.75,5) node[anchor=east,font=\tiny,text width=1.2cm,align=right] {to switches $sw_1$,$sw_2$} ;
 \draw (2.2,5.7) node[anchor=north east,rotate=90,font=\tiny] {GN200S24};
 \draw (1.75,5.75) node[] {$k_1$};


 \draw  (0,0) --
        (2,0) to[Tnigfetd,n=m10,mirror]
        (2,4) to[cute inductor] (2,6)
        (2,3.5) --
        (3,3.5) to[D*,n=d2,l_={$d_3$}]
        (3,6) --
        (2,6) --
        (2,6.5) node[rground,yscale=-1](t1){}
        (t1) node[anchor=south,above=1.2mm,font=\tiny] {\textbf{24V}}
        (m10.B) node[anchor=east,font=\tiny] {$M_{10}$};


 \draw  (6,3) node[op amp,scale=0.75,rotate=180] (oa1) {}
        (oa1) node[]{$A_1$}
        (oa1.out) |- (m10.G)
        (oa1.up) |- ++(-0.25,-.5)  -|
        ++(-0.75,-1) node[sground] {}  --
        ++(0.25,0) to[R,l=$r_2$,n=r2]
        ++(1.5,0) to[pR,n=POT]
        ++(1.5,0) -|
        ++(0.25,0.25) node[rground,yscale=-1](t2){}
        (r2.s) node[anchor=north] {\tiny{$4.78k\Omega$}}
        (POT.s) node[anchor=north] {\tiny{${10k\Omega}$}}
        (POT.s) node[above=2mm] {$r_3~~~~$}

        (POT.wiper) |- (oa1.-)
        (t2) node[anchor=south,above=1.2mm,font=\tiny] {\textbf{24V}}

        (oa1.out) |- ++(0.75,3.75) to[R,l=$r_4$,n=r4]
        ++(1,0) -|
        (oa1.+) --
        ++(3.5,0) to[R,l={$r_5$},n=r5,-*]
        ++(2,0) node[anchor=west,font=\tiny] {\textbf{V\_CAPS}}
        (r5.s) node[anchor=north]  {\tiny{${2k\Omega}$}}
        (oa1.+) --
        ++(3.5,0) to[C,text width={2em},align=right,l_={$c_{10}$ \tiny{${160nF}$}} ]
        ++(0,-2) node[sground] {}
        (r4.s) node[anchor=north] {\tiny{${48k\Omega}$}}

        (oa1.down) -- ++ (0,1.5) node[rground,yscale=-1](t3){}
        (t3) node[anchor=south,above=1.2mm,font=\tiny] {\textbf{24V}}
        --  ++(-0.5,0) to[C] ++(0,-1) node[sground]{};

\end{circuitikz}

%\caption[Start-up helper schematic]{Start-up helper circuit schematic.}
%\label{fig:pwr_train_sch}
%\end{figure}
%
%
%\subsection{Sensing and signal conditioning}
%
%\section{Close-loop controller circuits}
%\subsection{Full schematic}
%\subsection{Triangle wave generator}
%\subsection{Error amplifier}
%


\clearpage
\bibliographystyle{plainnat}
\bibliography{references}

