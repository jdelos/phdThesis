\chapter{H-SCC LED driver}
\label{ch:hscc_led_driver}


\begin{figure}[!h]
\centering

\ctikzset{bipoles/length=0.75cm}
\begin{circuitikz} [american,scale=0.65]
 
 \draw (0,0) rectangle (3,6);
 
 \draw (-1,1) to[short,o-] (0,1)
       (-1,5) to[short,o-] (0,5)
       (-1,5) to[open,v=$24V$] (-1,1)
       (-2.75,3) node[rotate=90] {e-Merge Alliance}
       (1.5,3) node[rotate=90,text width=2cm,align=center] {H-SCC LED DRIVER};
       
 \draw (3,5) -- (10,5)  to[leD*] (10,2) node[ground]{}
       (7,5) node[anchor=south,text width=4cm] {$11-13V$ $10-12W$ Dimmable LED string    };

 \draw (3,2.5) -- (4,2.5)
       (4,1.5) rectangle (8,3.5)
       (6,2.5) node[text width=2cm,align=center] {$4.5V 100mW$ Low-Voltage Circuits}
       (6,1.5) -- ++(0,-0.5)node[ground]{};
 
      
 
 
 
\end{circuitikz}

\caption[H-SCC LED driver block diagram]{H-SCC LED driver block diagram.}
\label{fig:bd_emerge_drv}
\end{figure}


An experimental converter was built with the goal to validate the performances of a H-SCC as a LED driver. The LED driver, described in the block diagram of Figure~\ref{fig:bd_emerge_drv}, was built using discrete components following the specifications of Table~\ref{tab:dsg_param_drv}. 

\begin{table}[!h]
 \caption{LED driver design specifications}\label{tab:dsg_param_drv}
 \centering 
 \renewcommand{\arraystretch}{1.2}
 \begin{tabular}{l | cl}
  Items & Value & Unit \\
  \midrule
  $v_{src}$ & 24 & V \\
  \hline
  $v_{LED}$ voltage & 11-13 & V \\
  $v_{LED}$ power & 12 & W \\
  $i_{LED}$ max & 1 & A \\
  $\Delta i_{LED}$ & $\pm$ 10 & \% \\
  \hline
  $v_{aux}$ voltage & 4.5 & V \\
  $v_{aux}$ power & 100m & W \\
  \hline
  $\eta$  & 85 & \% \\
  $f_{sw}$ & 2.77 & MHz\\
\end{tabular}
\end{table}

The driver was designed to be compliment the 24Vdc e-Merge Alliance standard used in track lighting systems, and featured two outputs. The main output supplies a \emph{LUXENON Altion} LED with a maximum current of 1A with a forward voltage around 12V, thus providing 12W at full load. The secondary output is designed for low-voltage and low-power to supply other auxiliary electronic circuits. The minimum efficiency was fixed to be higher than 85\%, and the switching frequency to be 2.77MHz, taking advantage of a 3dB higher tolerance of the conduced EMI standard X. 

\section{Design procedure}
The LED driver is composed by two main subsystems, the power train and the close-controller. Therefore the design process is accordingly  divided in three main parts: Power train design, small-signal analysis and close-loop controller design. 


\subsection{Power train}




\subsection{Small-signal analysis}

\subsection{Close-loop controller}


\section{Power train design}
\subsection{Capacitors}
\subsection{Transistors}
\subsection{Inductor}

\section{Power train circuits}
\subsection{Full schematic}
\begin{figure}[!h]
\centering
\ctikzset{bipoles/length=1cm}
\begin{circuitikz} [american,scale=0.65]
 \draw (5,5)    to[Tnigfetd,n=m5]
       (5,7.5)  to[Tnigfetd,n=m4]
       (5,10)   to[Tnigfetd,n=m3]
       (5,12.5) to[Tnigfetd,n=m2]
       (5,15)   to[Tnigfetd,n=m1] (5,17.5)

       (2,0)    to[Tnigfetd,n=m7]
       (2,2.5)  to[Tnigfetd,n=m6] (2,5)

       (8,0)    to[Tnigfetd,n=m8,mirror]
       (8,2.5)  to[Tnigfetd,n=m9,mirror] (8,5)

       (2,5) -| (m5.S) |- (8,5)
       (2,0) -- (8,0)
       (5,0) node[sground]{};


  %Apply labels to the MOSFETS
  \draw
        (m1.B) node[anchor=south west] {$M_1$}
        (m1.B) node[anchor=north  west,font=\tiny] {ZXMN2B01F}
        (m2.B) node[anchor=west] {$M_2$}
        (m3.B) node[anchor=west] {$M_3$}
        (m4.B) node[anchor=west] {$M_4$}
        (m5.B) node[anchor=west] {$M_5$}
        (m6.B) node[anchor=west] {$M_6$}
        (m7.B) node[anchor=west] {$M_7$}
        (m8.B) node[anchor=east] {$M_8$}
        (m9.B) node[anchor=east] {$M_9$};

  %Capacitor legs
  \draw
       (2,2.5) --
       (0,2.5) -- (0,4.5) to[C,text width={2.5em},l_={$c_4$  $39nF$}]
       (0,7.5) -- (5,7.5)
       (0,7.5) to[C,text width={2.5em},l_={$c_2$ $330nF$}]
       (0,12.5) -- (5,12.5)

       (8,2.5) --
       (10,2.5) -- (10,5) to[C,text width={2.5em},align=right,l={$c_3$ $180nF$}]
       (10,10) -- (5,10)
       (10,10) to[C,text width={2.5em},align=right,l={$c_1$ $220nF$}]
       (10,15) -- (5,15)

       (5,0) to[C,text width={2.5em},l_={$c_5$ $78nF$}] (5,5);

  \draw
        (5,17.5) node[spdt,rotate=-90,anchor=out 2](SW1){}
        (SW1.in) node[rground,yscale=-1]{}
        (SW1.in) node[anchor=south,above=0.25cm] {24V}
        (SW1.out 2) node[above=3.5mm] {$sw_1~~$}

        (SW1.out 1) |- (6,17) to[short,-*] (6.5,17) node[anchor=west,font=\tiny] {\textbf{24V\_SUP}};
  \draw [dashed] (5.3,18.4) -- (6.8,18.4) node[anchor=west,font=\tiny,text width=2cm] {to relay start-up circuit} ;

  \draw (-3,12.5) to[cute inductor,l=$l_o$,n=lo]  (0,12.5)
        (-3,12.5) to[C,l=$c_{o}$] (-3,8)
        (-3,12.5) -- (-4.5,12.5) --
        (-4.5,11.5) to[short,-o]  (-5,11.5)
        (-3.75,8)  node[sground] {}
        (-4.5,12.5) to[short,-*] (-4.5,13)  node[anchor=south,font=\tiny] {\textbf{+V\_LED}}
        (lo.s) node[anchor=north] {$2.2\mu H$}

        (-5,10.5) to[short,o-]
        (-4.5,10.5) to[R,text width={2.8em},align=right,l_={$r_{s}$ $100m\Omega$}]
        (-4.5,8) -- (-3,8)
        (-5.25,11.5) to[open,v=$v_{LED}$] (-5.25,10.5);


  \draw
        (-5,4) to[short,o-] (-3.5,4) -- (-3.5,5) -- ([hs]0,6 |- -5,5) arc(180:0:\radius) -- (2,5)
        (-5,2) to[short,o-] (-4.5,2) -- (-4.5,0)
        (-5.25,4) to[open,v=$v_{aux}$] (-5.25,2)
        %(-3,5) to[C,l=$c_o2$] (-3,0) -- (-4.5,0)
        (-4.5,0) node[sground]{}
        (-3.5,4) to[short,-*] (-3.5,3.5) node[anchor=north,font=\tiny] {\textbf{+V\_AUX}};

  \draw (-4,6) node[spdt,anchor=in,rotate=180,yscale=-1](SW2){} -| (-3.5,5)
        (SW2.out 1) -| (-6,6.75) to[short,-*] (-6,7)  node[anchor=south,font=\tiny] {\textbf{5V\_SUP}}
        (SW2.out 2) node[anchor=south] {$sw_2$};

  \draw [dashed] (-5,5.7) -- (-5,6.5) -- (-4.5,6.5) node[anchor=west,font=\tiny,text width=2cm] {to relay start-up circuit};

  \draw (10,15) to[short,-*] (10,15.5) node[anchor=south,font=\tiny] {\textbf{V\_CAPS}};

\end{circuitikz}

\caption[Power train schematic]{5:1 H$^2$-Dickson power train schematic.}
\label{fig:pwr_train_sch}
\end{figure}

%\begin{landscape}
%\thispagestyle{empty}

%\end{landscape}

%\subsection{Gate driver}
%\begin{figure}[!h]
%\centering
%\input{./5_hscc_led_driver/gate_driver_sch.tex}
%\caption[Power train schematic]{5:1 H$^2$-Dickson power train schematic.}
%\label{fig:pwr_train_sch}
%\end{figure}

\subsection{Start-up helper circuit}
\begin{figure}[!h]
\centering

\ctikzset{bipoles/length=0.75cm}
\begin{circuitikz} [american,scale=0.65]
 \draw (0,0) node[sground] {}  to[zD*,n=dz1,l={$d_1$}] (0,3)
       (-2.5,3) to[R,l={$r_1$},n=r1,-*]
       (-2.5,6.75) node[anchor=south,font=\tiny] {\textbf{24V\_SUP}}
       (dz1.s) node[anchor=north,rotate=90,font=\tiny] {BZX84C3V9}
       (0,3) --
       (-2.5,3) to[C,l={$c_{10}$},n=c1]
       (-2.5,0) -- (0,0)
       (-2.5,3) -- (-3,3) to[D*,n=d1,l={$d_2$},-*]
       (-5,3) node[anchor=east,font=\tiny] {\textbf{5V\_SUP}}
       (d1.s) node[anchor=south,font=\tiny] {BAS40}
       (r1.s) node[anchor=north,rotate=90] {\tiny{${1k\Omega}$}}
       (c1.s) node[rotate=90] {\tiny{${1\mu F}$}};

 \draw (1.65,4.5) rectangle (2.25,5.5);
 \draw[dashed] (2.25,5) -- (0.75,5) node[anchor=east,font=\tiny,text width=1.2cm,align=right] {to switches $sw_1$,$sw_2$} ;
 \draw (2.2,5.7) node[anchor=north east,rotate=90,font=\tiny] {GN200S24};
 \draw (1.75,5.75) node[] {$k_1$};


 \draw  (0,0) --
        (2,0) to[Tnigfetd,n=m10,mirror]
        (2,4) to[cute inductor] (2,6)
        (2,3.5) --
        (3,3.5) to[D*,n=d2,l_={$d_3$}]
        (3,6) --
        (2,6) --
        (2,6.5) node[rground,yscale=-1](t1){}
        (t1) node[anchor=south,above=1.2mm,font=\tiny] {\textbf{24V}}
        (m10.B) node[anchor=east,font=\tiny] {$M_{10}$};


 \draw  (6,3) node[op amp,scale=0.75,rotate=180] (oa1) {}
        (oa1) node[]{$A_1$}
        (oa1.out) |- (m10.G)
        (oa1.up) |- ++(-0.25,-.5)  -|
        ++(-0.75,-1) node[sground] {}  --
        ++(0.25,0) to[R,l=$r_2$,n=r2]
        ++(1.5,0) to[pR,n=POT]
        ++(1.5,0) -|
        ++(0.25,0.25) node[rground,yscale=-1](t2){}
        (r2.s) node[anchor=north] {\tiny{$4.78k\Omega$}}
        (POT.s) node[anchor=north] {\tiny{${10k\Omega}$}}
        (POT.s) node[above=2mm] {$r_3~~~~$}

        (POT.wiper) |- (oa1.-)
        (t2) node[anchor=south,above=1.2mm,font=\tiny] {\textbf{24V}}

        (oa1.out) |- ++(0.75,3.75) to[R,l=$r_4$,n=r4]
        ++(1,0) -|
        (oa1.+) --
        ++(3.5,0) to[R,l={$r_5$},n=r5,-*]
        ++(2,0) node[anchor=west,font=\tiny] {\textbf{V\_CAPS}}
        (r5.s) node[anchor=north]  {\tiny{${2k\Omega}$}}
        (oa1.+) --
        ++(3.5,0) to[C,text width={2em},align=right,l_={$c_{10}$ \tiny{${160nF}$}} ]
        ++(0,-2) node[sground] {}
        (r4.s) node[anchor=north] {\tiny{${48k\Omega}$}}

        (oa1.down) -- ++ (0,1.5) node[rground,yscale=-1](t3){}
        (t3) node[anchor=south,above=1.2mm,font=\tiny] {\textbf{24V}}
        --  ++(-0.5,0) to[C] ++(0,-1) node[sground]{};

\end{circuitikz}

\caption[Start-up helper schematic]{Start-up helper circuit schematic.}
\label{fig:pwr_train_sch}
\end{figure}


\subsection{Sensing and signal conditioning}

\section{Close-loop controller circuits}
\subsection{Full schematic}
\subsection{Triangle wave generator}
\subsection{Error amplifier}


