\newcommand\sHeigh{0.75cm}
\newcommand\sWidth{2.5cm}
\newcommand\dWidth{8cm}

\newcommand\dWidthB{6cm}

\newcommand\xShift{4cm}
\newcommand\xShiftB{3cm}


\usetikzlibrary{shapes.geometric, arrows}

\tikzstyle{startstop} = [rectangle, rounded corners, minimum width=\sWidth, minimum height=\sHeigh,text width=\sWidth,text centered, draw=black, fill=black!20,font=\footnotesize]
\tikzstyle{process} = [rectangle, minimum width=\sWidth, minimum height=\sHeigh, text width=\sWidth,text centered, draw=black, fill=gray!20,font=\footnotesize]
\tikzstyle{decision} = [diamond, minimum width=\sWidth, minimum height=\sHeigh, text width=\sWidth,text centered, draw=black, fill=gray!20,font=\footnotesize]

\tikzstyle{develop} = [rectangle, rounded corners, minimum width=\dWidth, minimum height=\sHeigh,text width=\dWidth, draw=black,thick,dashed,font=\footnotesize]

\tikzstyle{developB} = [rectangle, rounded corners, minimum width=\dWidthB, minimum height=\sHeigh,text width=\dWidthB, draw=black,thick,dashed,font=\footnotesize]

\tikzstyle{arrow} = [thick,->,>=stealth]
\tikzstyle{connect} = [thick,dashed]



\begin{tikzpicture} [node distance = 1.75cm]
\node (start) [startstop,  minimum width=4cm,text width=5cm] {
      $P_{o} = 12W,~ i_o = 1A,~ v_{src} = 24V$\\
      \vskip 0.5mm
      \hrule
      \vskip 0.5mm
      $v_{o,w.c.} = 13.2V,~ D_{w.c.}=75\%$\\
      $f_{sw}   ~=  2.77MHz, ~ \eta_{trg}=  90\% $ \\
      };

\node (rscc) [process, below of = start,yshift=-0.75cm] { $r_{scc,trg} = 1.2\Omega$};
\node (op_point) [process, below of =rscc,yshift=-.25cm] {O.P. in $r_{scc}$ curve?};
\node (rssl) [process,below of = op_point, left of = op_point, xshift = -.25cm] {$r_{ssl,trg} = 845m\Omega$};
\node (opssl) [process,below of = rssl,yshift = 0cm] { $ssl$ optimizer};
\node (sslend) [startstop,below of = opssl, right of = opssl ,yshift=-1.5cm,minimum width = 4cm, text width = 4cm, align=left] {$\mathbf{x} ~=[0.28~0.39~0.23~0.05~0.05]$
\\$c_T = 810nF$};
\node (aux1) [below of = opssl,yshift=-0.35cm] {};

\node (rfsl) [process,below of = op_point, right of = op_point, xshift = .25cm] {$r_{fsl,trg} = 845m\Omega$};
\node (opfsl) [process, below of = rfsl,yshift = -0cm] { solve $fsl$ };
\node (fslend) [startstop, below of = opfsl, yshift=-0.35cm] {$r_{on} = 222m\Omega$};



\draw [arrow] (start) --   node[yshift=-0.1cm] (x1){}  (rscc);
\draw [arrow] (rscc) --    node[yshift=+0.25cm] (x2){}  (op_point);

\draw [arrow] (op_point)-| node[yshift=-0.5cm] (x3a){} (rssl);
\draw [arrow] (rssl) --    node (x4a){} (opssl);
\draw [arrow] (opssl) |- (aux1) -|   node (x5a){} (sslend);

\draw [arrow] (op_point)-| node[yshift=-0.5cm] (x3b){} (rfsl);
\draw [arrow] (rfsl) --    node (x4b){} (opfsl);
\draw [arrow] (opfsl) --   node (x5b){} (fslend);

\node (step1) [develop,left of = x1 , xshift = -\xShift ] {With the fixed specs the target $r_{scc}$ is given by,
    \begin{equation}
        r_{scc} = \frac{P_o (1-\eta_{trg})}{i_o ^2} \label{eq:r_scc_trg}
    \end{equation} };
\draw [connect] (step1) -- (x1);

\node (step2) [develop,left of = x2 , xshift = -\xShift]
{The converter is designed to operate in the elbow of the $r_{scc}$ curve, fixing  $r_{ssl} = r_{fsl}$, thus
    \begin{equation}
        r_{scc} = \sqrt{r_{ssl}^2 + r_{fsl}^2} = \sqrt{2}r_{ssl} = \sqrt{2}r_{fsl} .
        \label{eq:r_scc_aprx_dsg}
    \end{equation}
 };
\draw [connect] (step2) -- (x2);

%%%%%%%%%%%%%%%%
% Step 3
%%%%%%%%%%%%%%%%

\node (step3a) [developB,left of = x3a , xshift = -\xShiftB]
{Hence $r_{ssl}$ is then given by
    \begin{equation}
        r_{ssl} = \frac{r_{ssc}}{\sqrt{2}}.
    \end{equation} };
\draw [connect] (step3a) -- (x3a);

\node (step3b) [developB,right of = x3b , xshift = \xShiftB]
{Hence $r_{fsl}$ is then given by
    \begin{equation}
        r_{fsl} = \frac{r_{ssc}}{\sqrt{2}}.
    \end{equation} };
\draw [connect] (step3b) -- (x3b);

%%%%%%%%%%%%%%%%
% Step 4
%%%%%%%%%%%%%%%%

\node (step4a) [developB,left of = x4a , xshift = -\xShiftB]
{Refining the \emph{SSL} function as
    \begin{align}
        \begin{split}
        r_{ssl}  = & \frac{1}{f_{sw} c_{T}} \mathbf{f_{ssl}}(x_1,\cdots,x_n)\\
        x_i      = & \frac{c_i}{c_T}, ~~ c_T  =  \sum_{i=1}^n c_i.
        \end{split}
    \end{align} };
\draw [connect] (step4a) -- (x4a);

\node (step4b) [developB,right of = x4b , xshift = \xShiftB]
{The discrete implementation only uses one switch,
hence all switches have the same $r_{on}$, then for $D=75\%$ \emph{FSL} function results in
    \begin{align}
        r_{fsl}  = 3.8 r_{on} .
    \end{align} };
\draw [connect] (step4b) -- (x4b);

%%%%%%%%%%%%%%%%
% Step 5
%%%%%%%%%%%%%%%%

\node (step5a) [developB,left of = x5a , xshift = -4.75cm]
{ The capacitor breakdown is given by solving
\begin{equation}
        min( \mathbf{f_{ssl}} )s.t. (1-\sum_{i=1}^n x_i),
    \end{equation}
for $D=75\%$, and $c_T$ is then given by
\begin{equation}
    c_t= \frac{min( \mathbf{f_{ssl}})}{f_{sw} r_{ssl,trg}}.
\end{equation}

};
\draw [connect] (step5a) -- (x5a);

\node (step5b) [developB,right of = x5b , xshift = \xShiftB]
{Hence $r_{on}$ is given by
    \begin{align}
        r_{on}  = \frac{r_{fsl,trg}}{3.8}.
    \end{align} };
\draw [connect] (step5b) -- (x5b);

\end{tikzpicture}
