\section{Single Output Converters}
Switched Capacitor Converters has been always considered two-port converters with single input and a single output as shown in the block diagram of Figure~\ref{fig:BD_two_port}. The input port $v_i$ is connected to a voltage source $v_{src}$, and the output $v_o$ port feeds the load. Where the converter provides a voltage conversion ($m$) between the two ports that  steps up, steps down and/or inverts the polarity of the input voltage. Currently, all available models were only proposed for this two port configuration. That is why, this section is starts to revisit the classical concepts of single output SCCs, helping the reader to to understand the limitations in the old models to cover the H-SCCs. Afterwards, a new modeling approach is introduced, enabling to model the H-SCC.

\begin{figure}[!h]
\centering
\ctikzset { bipoles/length=1cm}
\begin{circuitikz}[american voltages,scale=0.65]
\draw
    (1,0) to[short,o-]
    (0,0) to[V = $V_{supply}$]
    (0,3) to[short,-o]
    (1,3) ;

\draw
    (2,3) --
    (2.5,3)

    (2,0) --
    (2.5,0)

    node[ocirc]  (IC)  at (2,0) {}
    node[ocirc]  (I) at (2,3) {}
    (I) to[open,v=$v_{i}$] (IC);


\draw [thick]
    (2.5,-0.5) --
    (2.5,3.5)  --
    (5.5,3.5)  --
    (5.5,-0.5) --
    (2.5,-0.5);

\draw (4,2)node[anchor=north]{$\frac{v_o}{v_{i}}=m$} ;
\draw
    (5.5,3) -- (6,3)
    (5.5,0) -- (6,0)
    node[ocirc]  (O)  at (6,3) {}
    node[ocirc]  (OC) at (6,0) {}
    (O) to[open,v^<=$v_{o}$] (OC);

\draw
    (7,0) to[short,o-]
    (8,0) to[ R= $Load$,mirror]
    (8,3) to[short,-o]
    (7,3) ;
\end{circuitikz}
\caption[Blcok diagram of a two port SCC]{Block diagram of a two port SCC.}
\label{fig:BD_two_port}
\end{figure}

\subsection{The Output Resistance Model}
\begin{SCfigure}[][!h]
\centering
\ctikzset { bipoles/length=1cm}
\begin{circuitikz}[american voltages, scale=0.65]
\draw
    (-0.5,0) to[V = $ m \cdot v_{src}  $]
    (-0.50,3) -- (0,3) to[R,l=$r_{scc}$,-o]  (3,3)
    (3.5,3) to[short,o-,i=$i_o$]
    (4.25,3)   to[R,l=$r_o$]
    (4.25,0) to[short,-o] (3.5,0)
    (3,0) to[short,o-] (-0.5,0)
    (0,3) to[open,v^=$v_{trg}$] (0,0)
    (3.5,3) to[open,v=$v_{out}$] (3.5,0);

\end{circuitikz}
\caption[Output resistance model]{Output resistance model of a switched capacitor converter.}
\label{fig:scc_model_oi}
\end{SCfigure}
The behavior of SSCs is modeled with the well-known output resistance model~\cite{2000Oota,2012Peter} that is composed of a controlled voltage source and equivalent resistance $r_{scc}$, as shown in Figure~\ref{fig:scc_model_oi}. The output voltage provided by the converter under no-load conditions is defined as \emph{target voltage} ($v_{trg}$). The  controlled voltage source provides the target voltage, being the value of voltage supply $v_{src}$ multiplied by the conversion ratio $m$, thus
\begin{equation}
v_{trg} =  m \cdot v_{src} .
\label{eq:vtrg}
\end{equation}
When the converter is loaded, the voltage at the converter's output, $v_{out}$, drops proportionally with the load current. This effect is modeled with resistor $r_{scc}$, which accounts for the losses produced in the converter. Since the losses are proportional to the output current $i_o$, they can be modeled with a resistor. Using the presented model, the output voltage of the converter can be obtained as
\begin{equation}
v_{out} =  m \cdot v_{src} - i_o \cdot r_{scc} .
\label{eq:vout_scc}
\end{equation}
In order to solve~\eqref{eq:vout_scc}, it is necessary to obtain the two parameters of the model from the converter: the conversion ration $m$  and the equivalent output resistance $r_{scc}$. The first can be easily solved using Kirchhoff's Voltage Laws as previously explained in Section~\ref{ch:conversion_ratio}. The second is more complex and actually is the main challenge in the modeling of SCCs.

Currently, there are two different methodologies to infer the equivalent output resistance $r_{scc}$, plotted in~\ref{fig:plot_rscc}. On the one hand, S. Ben-Yaakov  ~\cite{2009Ben-Yaakov,2012Ben-Yaakov,2013Evzelman} has claimed a generalized methodology based on the analytical solution of each of the different R-C transient circuits of the converter, reducing them to a single transient solution. The methodology achieves a high accuracy, but results in a set of non-linear equations and high complexity for the analysis of advanced architectures.
On the other hand,  M. Makowski and D. Maksimovic~\cite{95Makowski} presented a methodology based on the analysis of the charge flow between capacitors in steady-state. The methodology is simple to apply and results in a set of linear expressions easy to operate for further analysis of the converters. Based on the charge flow analysis, M.Seeman~\cite{Seeman:EECS-2009-78} developed different metrics allowing to compare performance between capacitive and inductive converters.
Although both methodologies are valid for the modeling of SCCs, none of them has been used to model the effects of a loaded \emph{pwm}-node, which is fundamental to the study of H-SCC.  The charge flow analysis has a cleaner and simpler way of describing the loss mechanism. For that reason, this methodology has been chosen in this dissertation to model the \emph{hybrid} switched capacitor converter.

%\sidecaptionvpos{figure}{!b}
\begin{SCfigure}[][!h]
\centering
\begin{tikzpicture}[scale=0.85]
    \begin{loglogaxis}[
        %width=12cm,
        %height=8cm,
        xlabel near ticks,
        ylabel near ticks,
        xlabel= {$f_{sw} ~~ [Hz] $},
        ylabel= {$ [\Omega] $} ,
        axis line style={->},
        axis y line*=left,
        axis x line*=bottom,
        xtick=\empty, ytick=\empty,
        %ytick = {0,.125,.25},
        %yticklabels={0,$v_{src}\frac{1}{3}$,$v_{src}\frac{2}{3}$,$v_{src}$},
        domain=3e4:5e6,
        samples=100,
        %xticklabels={0,$D \cdot T_{sw}$,$T_{sw}$ ,$2 T_{sw}$,$3 T_{sw} $},
        legend style={at={(0.75,0.75)}, anchor= north east},
        enlarge x limits={0.0},
        enlarge y limits={0.0}
        ]

  \newcommand\C{1e-6}
  \newcommand\Resr{1}
  \newcommand\Dx{0.5}
  \newcommand\Rfsl{4*\Resr}
  \newcommand\lbssl{1e5}
  \newcommand\lbfsl{1e6}

  \addplot [thick]   { 1/(2*x*\C)*((exp(\Dx/(\Resr*x*\C))+1)/(exp(\Dx/(\Resr*x*\C))-1) + (exp((1-\Dx)/(\Resr*x*\C))+1)/(exp((1-\Dx)/(\Resr*x*\C))-1))};
  %\addplot [thick,dashed,marker=square] { sqrt((1/(2*\C*x))^2 + \Rfsl^2) };
  \addplot [thick,dashed,domain=3e4:3.5e5] { (1/(\C*x) };
  \addplot [thick,dotted,domain=2.5e4:5e6] { \Rfsl+0*x };
\legend {$r_{scc}$,$r_{ssl}$,$r_{fsl}$};
\end{loglogaxis}

\node[anchor=north] at (6.25cm,1.35cm){$FSL$};
\node[anchor=north,rotate=-60] at (1.25cm,5cm){$SSL$};

\end{tikzpicture}
\caption[Equivalent output resistance curve]{SCC Equivalent output resistance $r_{scc}$ as function of the frequency and the two asymptotic limits: \emph{Slow Switching Limit} (SSL) and \emph{Fast Switching Limit}(FSL). }
\label{fig:plot_rscc}
\end{SCfigure}

The aforementioned $r_{scc}$ accounts for the loss when the converter is loaded. All losses in the converter are, in fact, dissipated in the resistive elements of the converter: \emph{on}-resistance ($r_{on}$) of the switches and equivalent series resistance of the capacitors($r_{esr}$). The origin and magnitude of the losses depends on the operation region of the converter, which is a function of the switching frequency as shown in the plot of Figure~\ref{fig:plot_rscc}.
A SCC has two well-defined regimes of operation: the \emph{Slow Switching Limit} (SSL) and the \emph{Fast Switching Limit} (FSL). Each of the two regimes defines an asymptotic limit for the $r_{scc}$ curve. In the SSL, the converter operates at a switching frequency ($f_{sw}$) much lower than the time constant ($\tau$) of charge and discharge of the converter's capacitors, thereby allowing the full charge and discharge of the capacitors. As shown in Figure~\ref{fig:ic_ssl}, the capacitor currents present an exponential-shape waveform. In this regime of operation, the losses are determined by the charge transfer between capacitors, and dissipated in the resistive paths of the converter, mainly in the switches. That is why reducing the switch channel resistance does not decreases the losses. Instead, it will produce sharper discharge current impulses producing higher electromagnetic disturbances. In the SSL, losses are inversely proportional to the product of the switching frequency and the capacitance values, limited by the SSL asymptote as can be seen in Figure~\ref{fig:plot_rscc}.

In the FSL, the converter operates with a switching frequency ($f_{sw}$) much higher than the time constant ($\tau$) of charge and discharge of the converter's capacitors, limiting the full charge and discharge transients. As shown in Figure~\ref{fig:ic_fsl}, currents have block-shape waveforms. In this operation regime, the losses are dominantly produced by the parasitic resistive elements ($r_{on}$, $r_{esr}$), therefore changes in the capacitances or frequency do not modify the produced losses\footnote{The switching losses are not included in the modeling of $r_{scc}$. }. In the FSL, $r_{scc}$ is constant and limited by the FSL asymptote as it can be seen in Figure~\ref{fig:plot_rscc}.

\begin{figure}[!h]
\centering
\ctikzset { bipoles/length=1cm}
\begin{subfigure}[t]{.45\textwidth}
    %\centering
    \raggedright
    \begin{tikzpicture}
            \begin{axis}[
                width={\textwidth},
                height={6cm},
                axis lines=middle,
                xlabel near ticks,
                ylabel near ticks,
                xlabel= {time},
                ylabel= {capacitor current},
                every axis x label/.style={
                    at={(ticklabel* cs:1.05)},
                    anchor=west,
                },
                x axis line style={->},
                y axis  line style={<->},
                xtick=\empty, ytick=\empty,
                %ytick = {0,.125,.25},
                %yticklabels={0,$v_{src}\frac{1}{3}$,$v_{src}\frac{2}{3}$,$v_{src}$},
                domain=-0.25:2,
                samples=100,
                %xticklabels={0,$D \cdot T_{sw}$,$T_{sw}$ ,$2 T_{sw}$,$3 T_{sw} $},
                xmin=-0.1,xmax=2.2,
                ymin=-1,ymax=1.2,
                ]

          \newcommand\xtauA{1/(3*5)};
          \newcommand\ioA{1};
          \newcommand\xtauB{3/(5*5)};
          \newcommand\ioB{-0.75};

          \addplot [thick,]   coordinates { (0,0) (0.1,0) (0.1,\ioA)};
          \addplot [thick,domain=0.1:1]   { \ioA*(  exp(-(x-0.1)/(\xtauA))) };
          \addplot [thick,domain=1:2]   coordinates { (1,0) (1,\ioB)};
          \addplot [thick,domain=1:2]   { \ioB*(  exp(-(x-1)/(\xtauB))) };



        \end{axis}
    \end{tikzpicture}
    \caption{Slow Switching Limit}
    \label{fig:ic_ssl}
\end{subfigure}
\hfill
\begin{subfigure}[t]{.45\textwidth}
    %\centering
    \raggedright
    \begin{tikzpicture}
            \begin{axis}[
                width={\textwidth},
                height={6cm},
                axis lines=middle,
                xlabel near ticks,
                ylabel near ticks,
                xlabel= {$time $},
                ylabel= {capacitor current},
                every axis x label/.style={
                    at={(ticklabel* cs:1.05)},
                    anchor=west,
                },
                x axis line style={->},
                y axis  line style={<->},
                xtick=\empty, ytick=\empty,
                %ytick = {0,.125,.25},
                %yticklabels={0,$v_{src}\frac{1}{3}$,$v_{src}\frac{2}{3}$,$v_{src}$},
                domain=0:2.2,
                samples=100,
                %xticklabels={0,$D \cdot T_{sw}$,$T_{sw}$ ,$2 T_{sw}$,$3 T_{sw} $},
                xmin=-0.1,xmax=2.2,
                ymin=-1,ymax=1.2,
                ]

          \newcommand\xtauA{7)};
          \newcommand\ioA{0.65};
          \newcommand\xtauB{10};
          \newcommand\ioB{-0.35};

          \addplot [thick,]   coordinates { (0,0) (0.1,0) (0.1,\ioA)};
          \addplot [thick,domain=0.1:1]   { \ioA*(  exp(-(x-0.1)/(\xtauA))) };
          \addplot [thick,domain=1:2]   coordinates { (1, { \ioA*exp(-(1-0.1)/(\xtauA))} ) (1,\ioB)};
          \addplot [thick,domain=1:2]   { \ioB*(  exp(-(x-1)/(\xtauB))) };
          \addplot [thick]   coordinates { (2, {\ioB*exp(-(1)/(\xtauB))} ) (2,\ioA)};



        \end{axis}
    \end{tikzpicture}
    \caption{Fast Switching Limit}
    \label{fig:ic_fsl}
\end{subfigure}
\caption[Current waveforms in the converter's capacitors]{Current waveforms though the capacitors in each of the two operation regimes. }
\label{fig:capacitor_current}
\end{figure}


\subsection{Revising the charge flow analysis approach}
The charge flow analysis is based on the conservation of charge in the converter's capacitors during an entire switching period in steady state~\cite{95Makowski}. Under this conditions, the converter is studied in the two well-defined operating regimes: the Slow Switching Limit (SSL) and the Fast Switching Limit (FSL). In SSL, losses are then dominated by the charge transfer between the capacitors, therefore only the charge transfer loss mechanisms are studied.  In FSL, losses depend on the conduction through the parasitic resistive elements, therefore only the conduction losses are studied. This division in the study of the converter reduces the complexity of the problem and enables a simplified still very accurate analysis.

The charge flow analysis uses charges instead of currents. Actually being precise, the analysis is done using the so-called \emph{charge flow multipliers}, which consist of a normalization of the charges with respect to the total charge delivered at the converter's output ($q_{out}$) , hence
$$
a_x  = \frac{q_x}{q_{out}},
$$
where $a_x$ is the charge flow multiplier corresponding of the charge $q_x$ flowing through the $x$-th circuit element of the converter.

\subsection{Load Model: Voltage Sink versus Current Sink}
\ref{ch:load_model}
In order to model a SCC, the original charge flow method~\cite{95Makowski} makes three main assumptions:
\begin{enumerate}
  \item The load is modeled as an ideal voltage source since it is normally connected to the \emph{dc}-output in parallel with a large capacitor, as shown in Figure~\ref{fig:vsink_load}. This assumption, eliminates the capacitor connected in parallel with the load, neglecting the effect of this output capacitor on the equivalent output resistance.

  \item The model only considers the \emph{dc}-output as the single load point of the converter, imposing a unique output to the converter.

  \item The duty cycle is not included in the computation of the capacitor charge flow. Consequently, the modulation of the switching period is assumed to have no influence on the amount of charge flowing in the capacitors, leading in an accuracy of the SSL region for duty cycles different than the $50\%$.
\end{enumerate}

\begin{figure}[!h]
    \centering
    \ctikzset { bipoles/length=1cm}
    \begin{subfigure}[t]{.4\textwidth}
        \centering
        \begin{circuitikz}[american voltages,scale=0.65]
        \draw
                %Draw Switches
                (2,0)  to[V=$v_{src}$]
                (2,6)  --
                (5,6)   to[gswitch=$s_1$]
                (5,4.5)   to[gswitch=$s_2$]
                (5,3)   to[gswitch=$s_3$]
                (5,1.5)   to[gswitch=$s_4$]
                (5,0)  --
                (2,0)


        %Draw Capacitors
                (5,1.5) --
                (3.5,1.5) to[pC,l_=$c_{1}$]
                (3.5,4.5)--
                (5,4.5)

                (5,0) --
                (7,0) to[V,l_=$v_{out}$]
                (7,3)--
                (5,3);
        \end{circuitikz}
        \caption {Load as a voltage sink.}
        \label{fig:vsink_load}
    \end{subfigure}
    \hfill
    \begin{subfigure}[t]{.4\textwidth}
        \centering
        \begin{circuitikz}[american,scale=0.65]
        \draw
                %Draw Switches
                (2,0)  to[V=$v_{src}$]
                (2,6)  --
                (5,6)   to[gswitch=$s_1$]
                (5,4.5)   to[gswitch=$s_2$]
                (5,3)   to[gswitch=$s_3$]
                (5,1.5)   to[gswitch=$s_4$]
                (5,0)  --
                (2,0)


        %Draw Capacitors
                (5,1.5) --
                (3.5,1.5) to[pC,l_=$c_{1}$]
                (3.5,4.5)--
                (5,4.5)

                (5,0) --
                (7,0) to[pC,l^=$c_{2}$]
                (7,3)--
                (5,3)

        %%Sink load
                (7,3) --
                (8.5,3) to[I,l=$i_{out}$]
                (8.5,0) --
                (7,0);

        \end{circuitikz}
        \caption {Load as a current sink.}
        \label{fig:isink_load}
    \end{subfigure}
\caption[Two different load models]{Two load models for the charge flow analysis. }
\label{fig:loads}
\end{figure}

These assumptions reduce the accuracy and flexibility to model different concepts of the SCCs, including the H-SCCs (previously introduced in Chapter~\ref{ch:H-SCC}). In order to overcome these limitations, the presented methodology makes two different assumptions:
\begin{enumerate}
  \item The load is assumed to be a constant current sink with a value equal to the average load current, as shown in Figure~\ref{fig:isink_load}. Using this approach the charge delivered to the load can be evaluated for each switching phase $j$ as
      \begin{equation}
        q_{out}^j = D^j \frac{i_{out}}{f_{sw}} = D^j i_{out}{T_{sw}}  = D^j q_{out},
      \label{eq:q_out}
      \end{equation}
  where $i_{out}$ is the average output current and $D^j$ is the duty cycle corresponding to the $j$-th phase.


  \item Any of the converter nodes can be loaded. Since the load is modeled as a current sink, it can now be connected to any of the converter nodes without biasing it.

  \item When the load is connected to a \emph{dc}-node the associated \emph{dc}-capacitor of the node is no longer neglected, thus the effects of the output capacitor are included in the equivalent output resistance.

\end{enumerate}

\subsection{Re-formulating the charge flow analysis}
The equivalent output impedance encompasses the basic root losses produced in the converter due to capacitor charge transfer and charge conduction. As aforementioned, the original charge flow analysis~\cite{95Makowski} assumes an infinitely large output capacitance in parallel with the load. This assumption leads to inaccuracies in the prediction of the equivalent output resistance when the output capacitor is comparable in value to the flying capacitors~\cite{2013Breussegem:c_out}. Actually, the root cause for this inaccuracy lies in the wrong quantification of the charges that produces losses in the converter.\\

Looking in detail to the charge flow in a SCC, we can identify two different charge flows during each circuit mode:
\begin{description}

  \item[Redistributed charge] flows between capacitors in order to equalize their voltage differences, by evaluating them the capacitor transfer losses can be obtained.

      This charge flow is associated with a charge or discharge of the capacitors, happening right after the switching event and lasting for a short period of time\footnote{The duration of the charge depends on the time constant of the associated R-C circuit.}.

  \item[Pumped charge] flows from the capacitors to the load, where it is consumed by the load, hence producing useful work.  This charge delivery is associated with a discharge of the capacitors, lasting for the entire phase time.

\end{description}
Besides these two charge flows, there we quantify another  charge, necessary for the analysis :
\begin{description}
  \item[Net charge]  is the difference in charge between the beginning and the end of a switching event. For clarity, in a two phase converter, the \textbf{net charges} are quantified as
      \begin{equation*}
        \begin{cases}
               \text{Net charge, phase 1} \rightarrow q^1 = q(DT_{sw}) - q(0)\\
               \text{Net charge, phase 2} \rightarrow q^2 = q(T_{sw}) - q(DT_{sw}),
            \end{cases}
      \label{eq:net_charges}
      \end{equation*}
      where $q(t)$ is the charge in a capacitor at instant $t$, and $q^j$ the \textbf{net charge} of the phase $j$.


      Based on that principle all \emph{net} charges in the capacitors can be obtained applying Kirchhoff's Currents Law (KCL), but using charges instead of currents. Therefore, the circuit can be solved for \emph{net} charge flow, applying the \emph{capacitor charge balance} as
      \begin{equation}
       \forall~c_{i} : \sum_{j=1}^{phases}q_{i}^j = 0,
      \label{eq:charge_balance}
      \end{equation}

     The resulting charges are then gathered in the charge flow vector $\mathbf{a}$ as
       \begin{equation}
        \mathbf{a}^j =  \left[ a_{in}^j~a_1^j~a_2^j \cdots a_n^j \right] = \frac{\left[ q_{in}^j~q_1^j~q_2^j \cdots q_n^j \right]}{q_{out}},
      \label{eq:a_vector}
      \end{equation}
    where the superindex denotes the $j$-th phase, $q_{in}$ is the charge supplied by the voltage source and $q_i$ is the \emph{net} charge flowing in the $i$-th capacitor $c_i$. Notice that the vector is composed by charge flow multipliers, being the charges  normalized with respect to total output charge $q_{out}$.
\end{description}

\begin{figure}[!h]
\centering
\ctikzset { bipoles/length=1cm}
\begin{subfigure}[t]{.4\textwidth}
    %\centering
    \raggedright
    \begin{circuitikz} [american,scale=0.65]
    \draw
        (0,0) to[V=$v_{src}$] (0,3)
        (3,3) to[pC,l=$c_1$] (0,3)
        (3,0) to[pC,l=$c_2$] (3,3) -- (6,3)
        (6,0) to[pC,l=$c_3$] (6,3) --
        (8,3) to[I,l=$i_o$] (8,0) -- (0,0);
    \begin{scope}[>=latex,thick,text=black]
        \draw [->,rounded corners=7pt,dashed]
            (0.4,2.4) -- (2.4,2.4) -- (2.4,0.4);
        \draw [->,rounded corners=7pt,dashed]
            (3.6,0.5) |- (5.4,2.7) -- (5.4,0.4);
        \draw [->,rounded corners=7pt]
             (6.3,2) |- (8,3.3);
        \draw [>=latex,text=black,dashed]
          (0,4)  -- (0.9,4) node[anchor=west]{Redistributed charge};
        \draw [>=latex,text=black]
          (0,4.5)  -- (0.9,4.5) node[anchor=west]{Pumped charge};
    \end{scope}
    \end{circuitikz}
    \caption{}
\end{subfigure}
\hfill
\hfill
\begin{subfigure}[t]{.4\textwidth}
    %\centering
    \raggedleft
    \begin{circuitikz} [american,scale=0.65]
    \draw
        (1,0) to[V=$v_{src}$] (1,3)
        (6,3) to[pC,l=$c_2$] (3,3)
        (3,0) to[pC,l=$c_1$] (3,3)
        (6,0) to[pC,l=$c_3$] (6,3) --
        (8,3) to[I,l=$i_o$] (8,0) -- (1,0);
    \begin{scope}[>=latex,thick,text=black]
        \draw [->,rounded corners=7pt,dashed]
            (3.6,0.5) |- (5.4,2.4) -- (5.4,0.4);
        \draw [->,rounded corners=7pt]
            (6.3,2) |- (8,3.3);% -- (8.6,0.4);

    \end{scope}
    \end{circuitikz}
    \caption{}
\end{subfigure}
\caption[Charge flow in a 3:1 Dickson with an infinite output capacitor.]{Charge flows in a Dickson 3:1 converter when loaded at a \emph{dc}-node with a infinitely large output capacitor $c_3$ during the two switching phases. }
\label{fig:charge_flow_I}
\end{figure}
The loss mechanisms of SCCs can be better understood based on the \emph{redistributed} and \emph{pumped} charge flows. For instance Figure~\ref{fig:charge_flow_I} shows the charge flows for a 3:1 Dickson converter with a infinitely large output capacitor $c_3$. In such a converter, the charge flow through capacitors $c_1$ and $c_2$ is always either redistributed between them or towards the big capacitor $c_3$, and only capacitor $c_3$ supplies charge to the load. Therefore since the flowing charge in $c_1$ and $c_2$ is always transferred between capacitors, it produces losses and it never supplies directly the load. However, for a finite value of the output capacitor, or for converters loaded from an internal node, there is always the probability that all capacitors contribute to pumping charge to the load~\cite{2013Breussegem:c_out}; phenomenon that was not considered in the initial charge flow analysis.
\begin{figure}[!h]
\centering
\ctikzset { bipoles/length=1cm}
\begin{subfigure}[t]{.4\textwidth}
    %\centering
    \raggedright
    \begin{circuitikz} [american,scale=0.65]
    \draw
        (0,0) to[V=$v_{src}$] (0,3)
        (3,3) to[pC,l=$c_1$] (0,3)
        (3,0) to[pC,l=$c_2$] (3,3) -- (6,3)
        (6,0) to[pC,l=$c_3$] (6,3) --
        (8,3) to[I,l=$i_o$] (8,0) -- (0,0);
    \begin{scope}[>=latex,thick,text=black]
        \draw [->,rounded corners=7pt,dashed]
            (0.4,2.4) -- (2.4,2.4) -- (2.4,0.4);
        \draw [->,rounded corners=7pt,dashed]
            (3.6,0.5) |- (5.4,2.7) -- (5.4,0.4);

        \draw [->,rounded corners=7pt]
            (2,3.3) -- (7,3.3)
            (2.7,2) |- (7.5,3.3)
            (6.3,2) |- (8,3.3);
        \draw [>=latex,text=black,dashed]
          (0,4)  -- (0.9,4) node[anchor=west]{Redistributed charge};
        \draw [>=latex,text=black]
          (0,4.5)  -- (0.9,4.5) node[anchor=west]{Pumped charge};
    \end{scope}


    \end{circuitikz}
    \caption{}
\end{subfigure}
\hfill
\hfill
\begin{subfigure}[t]{.4\textwidth}
    %\centering
    \raggedleft
    \begin{circuitikz} [american,scale=0.65]
    \draw
        (1,0) to[V=$v_{src}$] (1,3)
        (6,3) --  (3,3)
        (3,0) to[pC,l=$c_1$] (3,3)
        (6,0)-- (6,0.25) to[pC,l=$c_3$] (6,1.5) to[pC,l=$c_2$] (6,2.75) |-
        (8,3) to[I,l=$i_o$] (8,0) -- (1,0);
    \begin{scope}[>=latex,thick,text=black]
        \draw [->,rounded corners=7pt,dashed]
            (3.6,0.5) |- (5.4,2.7) -- (5.4,0.4);
        \draw [->,rounded corners=7pt]
             (2.7,2) |- (7.5,3.3)
             (6.3,2.5) |- (8,3.3);% -- (8.6,0.4);
    \end{scope}
    \end{circuitikz}
    \caption{}
\end{subfigure}
\caption[Charge flow in a 3:1 H-Dickson.]{Charge flows in a Dickson 3:1 converter when loaded at one of the \emph{pwm}-nodes during the two switching phases. }
\label{fig:charge_flow_II}
\end{figure}
In another scenario, the one of Figure~\ref{fig:charge_flow_II},  a 3:1 H$^2$-Dickson has its load connected to the second ~\emph{pwm}-node. In such a converter, there is a redistributed charge flow between the different capacitors as in the previous case, but at the same time, all capacitors pump charge to the load as well. Therefore all capacitors  contribute in delivering charge to the load, which actually reduces the equivalent output impedance of the converter.


The original charge flow analysis only uses the \emph{net} charge flow in order to quantify the produced losses in the SSL region, which in fact results in an over estimation of the charge flow responsible for the losses (the \emph{redistributed} charge flow). The methodology proposed in this dissertation identifies these different charge flows, and achieves a closer estimation of the losses in the converter by independently quantifying each of them.
\begin{figure}[!h]
\centering
% This file was created by matlab2tikz.
%
%The latest updates can be retrieved from
%  http://www.mathworks.com/matlabcentral/fileexchange/22022-matlab2tikz-matlab2tikz
%where you can also make suggestions and rate matlab2tikz.
%

\begin{tikzpicture}

\begin{axis}[%
width=8cm,
height=4cm,
at={(1.532in,0.729in)},
scale only axis,
xmin=0,
xmax=100,
ymin=6.64,
ymax=6.673,
axis background/.style={fill=white},
axis x line*=bottom,
axis y line*=none
]
\addplot [thick,solid,forget plot]
  table[row sep=crcr]{%
1	6.65026844706453\\
2	6.6541906445053\\
3	6.65706994984393\\
4	6.65925863469407\\
5	6.66097190927411\\
6	6.66234478933293\\
7	6.66346632700282\\
8	6.66439739781252\\
9	6.66517963724446\\
10	6.66584465605996\\
11	6.66641623072511\\
12	6.66691268208139\\
13	6.66734838569269\\
14	6.6677347707153\\
15	6.66808100843884\\
16	6.66839450570606\\
17	6.66868127063699\\
18	6.66894619123669\\
19	6.66919325215929\\
20	6.66942570599671\\
21	6.66964621014256\\
22	6.66985693700001\\
23	6.67005966319872\\
24	6.67025584208054\\
25	6.67044666273543\\
26	6.67063309816195\\
27	6.67081594459879\\
28	6.67099585366935\\
29	6.67117335866543\\
30	6.67134889604556\\
31	6.67152282302303\\
32	6.67169543195692\\
33	6.67186696212833\\
34	6.67203760937764\\
35	6.67220753399144\\
36	6.66994898046201\\
37	6.66832576586377\\
38	6.66712775868683\\
39	6.66621535139474\\
40	6.66549387715858\\
41	6.66490036899855\\
42	6.66439286198158\\
43	6.6639431930172\\
44	6.66353165635979\\
45	6.66314608322637\\
46	6.6627778080521\\
47	6.66242109657901\\
48	6.66207215837762\\
49	6.6617284123912\\
50	6.66138813136178\\
51	6.66105016231045\\
52	6.66071373588587\\
53	6.66037833874885\\
54	6.66004362838354\\
55	6.659709376253\\
56	6.65937542987049\\
57	6.65904168749229\\
58	6.6587080812319\\
59	6.65837456579342\\
60	6.65804111199682\\
61	6.65770769693504\\
62	6.65737430934867\\
63	6.65704093984341\\
64	6.65670758237776\\
65	6.65637423294288\\
66	6.65604088886614\\
67	6.65570754836448\\
68	6.65537421024822\\
69	6.65504087372358\\
70	6.65470753826091\\
71	6.65437420350682\\
72	6.65404086922551\\
73	6.65370753525966\\
74	6.6533742015043\\
75	6.65304086788937\\
76	6.65270753436815\\
77	6.65237420090945\\
78	6.65204086749247\\
79	6.6517075341038\\
80	6.65137420073293\\
81	6.65104086737468\\
82	6.65070753402473\\
83	6.65037420068031\\
84	6.65004086733958\\
85	6.64970753400131\\
86	6.64937420066469\\
87	6.64904086732916\\
88	6.64870753399436\\
89	6.64837420066005\\
90	6.64804086732606\\
91	6.64770753399229\\
92	6.64737420065867\\
93	6.64704086732514\\
94	6.64670753399168\\
95	6.64637420065826\\
96	6.64604086732487\\
97	6.6457075339915\\
98	6.64537420065814\\
99	6.64504086732479\\
100	6.64470753399144\\
};
\end{axis}
\end{tikzpicture}%
\caption[Voltage ripple in the SCC's capacitors]{Two possible voltage waveforms that show the capacitors in a SCC. Ripples are associated with the charge flow mechanisms: top) unipolar capacitor discharge (DC capacitor); bottom) bipolar capacitor discharge (flying capacitor).}
\label{fig:cap_riples}
\end{figure}
The nature and effects of the three different charge flow can be better analysed and understood  by looking at the voltage waveforms in the converter capacitors during an entire switching cycle. From Figure~\ref{fig:cap_riples}, we can associate the voltage ripples to the previously defined charge flows:
\begin{description}
  \item[Net voltage ripple $\Delta vn$] is the voltage variation measured at the beginning and at the end of each of the switching events (\emph{on}$\rightarrow$\emph{off}, \emph{off}$\rightarrow$\emph{on} ). As a matter of fact, this \emph{net} ripple is associated with the \emph{net} charge flow, therefore using (\ref{eq:a_vector}) the \emph{net} voltage ripple can be formulated as
      % computed from the \emph{charge balance} in a capacitor in steady-state condition as
%      \begin{equation}
%        \Delta {vn}^j_i  = \frac{q_i ^j }{c_i}.
%        \label{eq:net_voltage}
%      \end{equation}
       \begin{equation}
        \Delta {vn}^j_i  = \frac{q_i ^j }{c_i} = \frac{a_i ^j }{c_i} {q_{out}}.
        \label{eq:net_voltage_cf}
      \end{equation}

      Notice that the capacitor charge balance principle is reflected in the \emph{net} voltage ripple of Figure~\ref{fig:cap_riples}. The sum of all \emph{net} ripples in each capacitor during a switching cycle  must be zero. Which explains why $\Delta vn^1 = \Delta vn^2$ in the two-phase converter used in the example of Figure~\ref{fig:cap_riples}.

  \item[Pumped voltage ripple $\Delta vp$] is the voltage variation associated with the discharge of the capacitor by a constant current. Thanks to modeling the load as current sink, the \emph{pumped} ripple can be associated to a linear voltage discharge, thus the \emph{pumped} ripple can be obtained for each switching phase as
      \begin{equation}
        \Delta {vp}^j_i  = D^j \frac{i_i^j}{c_i }T_{sw},
      \label{eq:pumped_voltage}
      \end{equation}
      where $i_i^j$ is the current flowing through the $i$-th capacitor $c_i$. Actually, the current flowing in each individual capacitor $c_i$ during each $j$-th phase is a function of the output current, therefore it can be expresses as a function of $i_{out}$ as
      \begin{equation}
        i_i^j = b_i^j i_{out} ,
      \label{eq:b_cnst}
      \end{equation}
      where $b_i^j$ is a constant obtained from determining the currents in each circuit mode of the converter. Replacing~\eqref{eq:b_cnst} and~\eqref{eq:q_out} into~\eqref{eq:pumped_voltage}, the \emph{pumped} voltage ripple can be expressed in the charge flow notation as
      \begin{equation}
        \Delta {vp}^j_i  = D^j \frac{b_i^j}{c_i } {i_{out}} {T_{sw}} = D^j \frac{b_i^j}{c_i } {q_{out}}.
      \label{eq:pumped_voltage_cf}
      \end{equation}
      Like in the previous case, the $b_i^j$ elements are gathered in the \emph{pumped} charge flow vector $\mathbf{b}$ as
      \begin{equation}
        \mathbf{b}^j =  \left[ b_1^j~b_2^j \cdots b_n^j \right] = \frac{\left[ ~i_1^j~i_2^j \cdots i_n^j \right]}{i_{out}},
      \label{eq:b_vector}
      \end{equation}
      where the $j$ denotes the circuit phase, $i_i$ is the \emph{pumped} current flowing in the $i$-th capacitor $c_i$. The vector is normalized with respect to the output current $i_{out}$.


  \item[Redistributed ripple $\Delta vr$ ]is the voltage variation associated to a transient exponential charge or discharge. It is produced by the charge redistribution between capacitors and happens just after each switching event. The \emph{redistribution} ripple can be quantified by the addition of the two previous defined ripple types as
      \begin{equation}
        \Delta {vn}^j_i  = \Delta {vp}^j_i + \Delta {vr}^j_i .
      \label{eq:rdst_ripple_I}
      \end{equation}
      Substituting~\eqref{eq:net_voltage_cf} and~\eqref{eq:pumped_voltage_cf} into~\eqref{eq:rdst_ripple_I}, the \emph{redistributed} ripple is formulated in terms of the charge flow analysis, as
      \begin{equation}
        \Delta {vr}^j_i  = \frac{q_{out}}{c_i} \left[ a^j_i - D^j b^j_i \right] = \frac{q_{out}}{c_i} g^j_i,
      \label{eq:rdst_ripple_II}
      \end{equation}
      where $g^j_i$ is the \emph{redistributed} charge flow of the $j$-th phase and the $i$-th capacitor. The \emph{redistributed charge flow vector} $\mathbf{g}$ is actually defined as
      \begin{equation}
        \mathbf{g}^j   = \mathbf{ac}^j - D^j \mathbf{b}^j,
      \label{eq:rdst_chrg_flow}
      \end{equation}

      where $\mathbf{ac}$ is the \emph{capacitor charge flow vector}, a sub-vector of $\mathbf{a}$ that only contains the charge flow multiplier associated to the capacitors.
\end{description}
In conclusion, in order to study a SCC is necessary to obtain the three charge flow vectors from a converter, which is presented in the following section.

\subsection[Solving the charge flow vectors] { Solving the charge flow vectors}
\begin{SCfigure}[][!h]
\ctikzset { bipoles/length=1cm}
\centering
    \begin{circuitikz}[american,scale=0.6]

    \draw
            (0,0) to[V=$v_{src}$] (0,7.5) -- (5,7.5)
            (5,7.5)  to[gswitch=$s_1$] %S1
            (5,6)   to[gswitch=$s_2$] %S2
            (5,4.5)   to[gswitch=$s_3$] %S3
            (5,3) --
            %left branch
            (3,3)   to[gswitch=$s_7$]
            (3,1.5)   to[gswitch=$s_6$]
            (3,0);

    \draw   %right branch
            (5,3) --
            (7,3)   to[gswitch,l_=$s_4$]
            (7,1.5)   to[gswitch,l_=$s_5$]
            (7,0) -- (0,0);


    \draw %Capacitor C1
           (3,1.5) -- (2,1.5)
            to[pC,l_=$c_1$] (2,6) --
           (5,6);

    \draw %Capacitor C2
           (7,1.5) --
           (8.25,1.5)  to[pC,l_=$c_2$](8.25,4.5) --
           (5,4.5)
           (8.25,4.5) -- (10,4.5) to[I=$i_{out}$] (10,0) -- (5,0);


    \draw %Capacitor C3
           (5,0) to[pC,l_=$c_3$] (5,3);

     \end{circuitikz}
 \caption{ 3:1 H$^2$-Dickson with the load connected to the second \emph{pwm}-node.}
 \label{fig:3_1_hscc_solv}
\end{SCfigure}
The charge flow vectors are solved for the converter of Figure~\ref{fig:3_1_hscc_solv}, a  3:1 H$^2$-Dickson loaded at second node, in two steps. First are solved the \emph{net} charge flow vectors. Second are solved the \emph{pumped} charge flow vectors. As aforementioned, the \emph{net} charge flow vectors are determined by solving the converter applying the capacitor charge balance condition~\eqref{eq:charge_balance}.  Therefore considering the two circuit modes of the converter, shown in Figure~\ref{fig:hscc_phases_charges}, the converter can be solved by creating a single system of linear equations. The node equations for the first phase (Figure~\ref{fig:hscc_full_p1_slv}) are:
\begin{align}
\label{eqn:ph1_kil}
\begin{split}
  q_{in}^1 - q_1^1   &=0, \\
  q_1^1 - q_2^1 - q_3^1 - q_{out}^1 &=0.
\end{split}
\end{align}
The node equations for second circuit mode (Figure~\ref{fig:hscc_full_p2_slv}) are:
\begin{align}
\label{eqn:ph2_kil}
\begin{split}
  q_{in}^2 & = 0,\\
  q_2^2 - q_3^2    &=0, \\
  q_1^2 + q_2^2 + q_{out}^2 &=0.
\end{split}
\end{align}
Applying~\eqref{eq:q_out} into $q_{out}^1$ and $q_{out}^2$, the phase output charges are expressed as function of the total output charge $q_{out}$, as
\begin{align}
\label{eqn:qout_CL}
\begin{split}
  q_{out}^1 & = D~q_{out} ,\\
  q_{out}^2 & = (1-D)~q_{out},
\end{split}
\end{align}
where $D$ corresponds to the duty cycle of odd switches. The charge flow in the capacitors are constrained to the null charge balance condition of~\eqref{eq:charge_balance}, hence
\begin{align}
\label{eqn:q_i_NCB}
\forall~c_{i} : \sum_{j=1}^{phases}q_{i}^j & \rightarrow
    \begin{cases}
        q_1 \leftarrow q_1^1 =  - q_1^2 & \text{for } c_1;\\
        \\
        q_2 \leftarrow q_2^1 =  - q_2^2 & \text{for } c_2;\\
        \\
        q_3 \leftarrow q_3^1 =  - q_3^2 & \text{for } c_3.
    \end{cases}
\end{align}
Substituting~\eqref{eqn:qout_CL} and~\eqref{eqn:q_i_NCB} into~\eqref{eqn:ph1_kil} and~\eqref{eqn:ph2_kil}, we can formulate a system of linear equations as
\begin{equation}
  \syssubstitute{.,{a_1}{q_{in}^1}{a_2}{q_{in}^2}{b_1}{q_1}{b_2}{q_2}{b_3}{q_3}}
  \systeme{
    a_1  - b_1  = 0,
    a_2 = 0,
    b_1 - b_2  - b_3  =    D q_{out},
    b_1 + b_2         = (1-D) q_{out},
    b_2 - b_3  = 0},
\end{equation}
solving the system yields
\begin{align}
\label{eqn:qi_rslt}
    \begin{split}
        q_{in}^1  = q_{1} & = \frac{2 -D}{3} q_{out} ,\\
        q_{2}     = q_3   & = \frac{1 - 2D}{3} q_{out}.
    \end{split}
\end{align}
Substituting~\eqref{eqn:qi_rslt} into~\eqref{eq:a_vector}, the solution is presented in charge flow vector form, resulting in
\begin{align}
\mathbf{a}^1 & = \frac{1}{3}\irow{2 - D & 2 - D & 1 - 2D & 1- 2D}, \\
\mathbf{a}^2 & = \frac{1}{3}\irow{0 & D - 2  & 2D -1  & 2D -1 }.
\label{eq:a_31dikson}
\end{align}


\begin{figure}[!h]
\centering
\ctikzset { bipoles/length=1cm}
%\ctikzset { scale=0.5}
\begin{subfigure}[t]{\textwidth}
    \centering
    %\ctikzset { bipoles/length=1cm}
        \begin{circuitikz}[american,scale=0.6]
       \draw %Input Supply
                (0,0) to[V=$v_{src}$,i=$q_{in}^1$]  (0,3)
                (4,3) to[pC,l_=$c_1$,i<_=$q_1^1$,v^>=$v_1$]     (0,3)
                (4,0) to[pC=$c_2$,i<=$q_2^1$,v>=$v_2$]     (4,3) -- (7,3)
                (3,0) -- (7,0) to[pC=$c_3$,i<=$q_3^1$,v>=$v_3$]  (7,3) to[short,i>=$q_{out}^1$]
                (9,3) to[I=$i_{out}$] (9,0) -- (0,0);




         \end{circuitikz}
     \subcaption{First mode, odd switches are closed and even switches are open.}
     \label{fig:hscc_full_p1_slv}
     \end{subfigure}

\begin{subfigure}[t]{\textwidth}
      \centering
      \begin{circuitikz}[american,scale=0.6]
        \draw (0,4.5) node[anchor=north]{ };
        \draw   %Input Supply
                (-1,0)  to[V=$v_{src}$,i=$q_{in}^2$]
                %Draw Switches
                (-1,4);

        \draw   (5,2) to[pC=$c_2$,i<=$q_2^2$,v>=$v_2$]
                (5,4) to[short,i>=$q_{out}^2$]
                (7,4) to[I=$i_{out}$] (7,0) -- (-1,0);


        \draw %Capacitor C1
               (2,0)to[pC=$c_1$,i<=$q_1^2$,v>=$v_1$](2,4) --(5,4);

        \draw %Capacitor C3
               (5,0) to[pC=$c_3$,i<=$q_3^2$,v>=$v_3$] (5,2);




         \end{circuitikz}
     \subcaption{Second mode, even switches are closed and odd switches are open.}
     \label{fig:hscc_full_p2_slv}
     \end{subfigure}
\caption{The two switching modes of 3:1 H-Dickson of Figure~\ref{fig:3_1_hscc}}
\label{fig:hscc_phases_charges}
\end{figure}



The \emph{pumped} charge flow multipliers are obtained by individually solving the currents in each circuit mode. For sake of brevity, only the circuit associated to the first mode of the converter will be solved in detail. The sing conventions for voltages and currents are defined in Figure~\ref{fig:hscc_full_p1_slv}, but instead of using charges $q_x$ the circuit will be solved for currents $i_x$. We can formulate two node equations,
\begin{align}
i_{in} - i_1 & = 0, \label{eqn:b_param_kcl_1}\\
i_1  - i_2 - i_3 - i_{out} &=0, \label{eqn:b_param_kcl_2}
\end{align}
and two more mesh equations
\begin{align}
\label{eqn:b_param_kvl}
\begin{split}
  v_{src} - v_1 - v_2  & = 0,\\
  v_2 - v_3&=0.
\end{split}
\end{align}
Owing to the fact that the relation current-voltage in a capacitor is $c \frac{dv}{dt} = i$, and using the mesh equations~\eqref{eqn:b_param_kvl}, we can define the relations between currents as follows
\begin{align}
\label{eqn:b_i_rel}
\begin{split}
  i_2 & = i_1 \frac{c_2}{c_1},\\
  i_3 & = i_2 \frac{c_3}{c_2} = i_1 \frac{c_3}{c_1}.
\end{split}
\end{align}
Substituting~\eqref{eqn:b_i_rel} into~\eqref{eqn:b_param_kcl_2} and isolating $i_1$, we obtain the \emph{pumped} charge flow multiplier for $c_1$ phase $1$:
\begin{equation}
  i_1  = i_o \frac{c_1}{c_1+c_2+c_3} = i_o b_1^1.
\label{eqn:b_c1_p1}
\end{equation}
The rest of the \emph{pumped} charge multipliers can be found solving for the remaining  currents, and for the other circuit modes. Arranging them in the corresponding vector form, will result i:
\begin{equation}
\begin{array}{rclcrcl}
\mathbf{b}^1 &=& \frac{1}{\beta_1}\irow{ c_1 & -c_2 & -c_3 } & & \beta_1 &=& c_1+c_2+c_3,\\
\\
\mathbf{b}^2 &=& \frac{-1}{\beta_2}\irow{ c_1c_2 + c_1c_3 & c_2c_3 &  c_2c_3  } & &  \beta_2  &=& c_1c_2+c_1c_3+c_2c_3.
\end{array}
\end{equation}

%\begin{align}
%\mathbf{b}^1 & = \frac{1}{\beta_1}\irow{ c_1 & -c_2 & -c_3 },\\
% \beta_1 & = c_1+c_2+c_3 ,\\
%\mathbf{b}^2 & = \frac{-1}{\beta_2}\irow{ c_1c_2 + c_1c_3 & c_2c_3 &  c_2c_3  },\\
% \beta_2 & =c_1c_2+c_1c_3+c_2c_3.
%\label{eq:b_31dikson}
%\end{align}

\subsection[SSL Equivalent Resistance ]{Slow Switching Limit Equivalent Resistance}
The SSL equivalent output resistance $r_{ssl}$ accounts for the losses produced by the capacitor charge transfer, therefore $r_{scc}$ can be obtained by evaluating the losses in the capacitors.  The energy lost in a charge or discharge of capacitor $c$ is given by
\begin{equation}
E_{loss}=\frac{1}{2} c ~{\Delta{v}_c}^2 .
\label{eq:e_lost}
\end{equation}
where $\Delta v_c$ is the voltage variation in the process. Previously, we defined that the \emph{redistributed} ripple is associated with the capacitor charge transfer. Therefore, substituting~\eqref{eq:rdst_ripple_II} into~\eqref{eq:e_lost}, we obtain the losses due to capacitor charge transfer
\begin{equation}
E_i^j=\frac{1}{2}{({\Delta{vr}}_i^j)}^2 c_i = \frac{1}{2}\frac{{q_{out}}^2}{{c_i}^2}{\left[a_{i\
}^j-{D^j} {b_i^j}\right]}^2c_i=\frac{1}{2}\frac{{q_{out}}^2}{c_i}{\left[a_{i\
}^j-{D^j} {b_i^j}\right]}^2 .
\label{eq:e_lost_ssl}
\end{equation}
The total power loss in the circuit is the sum of the losses in all of the capacitors during each phase multiplied by the switching frequency $f_{sw}$. This yield{\small s}
\begin{equation}
P_{ssl}= f_{sw} \sum_{i=1}^{caps.}\sum_{j=1}^{phases} E_i^j =\frac{f_{sw}{q_{out}}^2}{2}\sum_{i=1}^{caps.}\sum_{j=1}^{phases}\frac{1}{c_i}{\left[a_{i\
}^j-{D^j}{b_i^j}\right]}^2.
\label{eq:pwr_ssl}
\end{equation}
The losses can be expressed as the output SSL resistance, dividing~\eqref{eq:pwr_ssl} with the
square of the output current as
\begin{equation}
r_{ssl}=\frac{P_{ssl}}{{i_o}^2}=\frac{P_{ssl}}{{(f_{sw} {q_{out}})}^2}=\frac{1}{2 f_{sw}}\sum_{i=1}^{caps.}\sum_{j=1}^{phases}\frac{1}{c_i}{\left[a_{i\
}^j-{D^j} {b_i^j}\right]}^2.
\label{eq:r_ssl}
\end{equation}


\subsection[FSL Equivalent Resistance]{Fast Switching Limit Equivalent Resistance}
The fast switching limit (FSL) equivalent output resistance $r_{fsl}$ accounts for losses produced in the resistive circuit elements, being the \emph{on}-resistance of the switches and the Equivalent Series Resistance (ESR) of the capacitors $r_{esr,c}$.

The power dissipated by a resistor $r_i$ from a square-wave pulsating current is given by
\begin{equation}
P_{r_i} = r_i~D^j~i_i^2,
\label{eq:pwr_r}
\end{equation}
where $D^j$ is the duty cycle. The value of $i_i$ (peak current) though the resistor can be also defined by its flowing charge $q_i$ as
\begin{equation}
i_i = \frac{q_i}{D^j~T_{sw}} = \frac{q_i}{D^j} f_{sw}.
\label{eq:i_q}
\end{equation}
As outlined in~\cite{Seeman:EECS-2009-78}, the charge flowing through the parasitic resistive elements can be derived from the charge flow vectors ($\mathbf{a}$), providing the \emph{switch}\footnote{These charge flow vectors also account for other resistive elements, not only the switches, such as the capacitors equivalent series resistance. Nevertheless they are called after the switches since they are the dominant resistive elements in the design of a converter.} charge flow vectors $\mathbf{ar}$. Using the \emph{switch} charge flow multiplier, \eqref{eq_i_q} can be redefined as function of the output charge (or the output current) as
\begin{equation}
i_i = \frac{ar_i^j}{D^j} q_{out}~f_{sw} = \frac{ar_i^j}{D^j} i_{out}.
\label{eq:i_ar}
\end{equation}
Substituting~\eqref{eq:i_ar} into~\eqref{eq:pwr_r} yields
\begin{equation}
P_{r_i} = \frac{r_i}{D^j}{ar_i^j}^2 i_{out}^2 ,
\label{eq:pwr_r_ar}
\end{equation}
the total loss accounting all resistive elements and phases is then
\begin{equation}
P_{fsl} = \sum_{i=1}^{elm.} \sum_{j=1}^{phs.}  \frac{r_i}{D^j}{ar_i^j}^2 i_{out}^2,
\label{eq:pwr_fsl}
\end{equation}
dividing by $i_{out}^2$ yields the FSL equivalent output resistance:
\begin{equation}
r_{fsl}=\sum_{i=1}^{elm.}\sum_{j=1}^{phases}\frac{r_i}{D^j}{ar_i^j}^2
\label{eq:r_fsl}
\end{equation}
where $r_i$ is the resistance value of the $i$-th resistive element.


\subsection{Equivalent Switched Capacitor Converter Resistance}
\label{ch:rscc_apprx}
With the goal of obtaining a simple design equation, a first analytical approximation of $r_{scc}$ in~\cite{1998Arntzen,1999Maksimovic} was given as

\begin{equation}
r_{scc} \approx \sqrt{{r_{ssl}}^2+{r_{fsl}}^2},
\label{eq:r_scc}
\end{equation}
being used in all the presented results of this dissertation.
\begin{SCfigure}[][h]
\ctikzset { bipoles/length=1cm}
    \begin{circuitikz} [american,scale=0.65]
    \draw
        (0,0) to[V=$v_{src}$]
        (0,3) to[gswitch,l=$s_1$]
        (2,3) to[R,l=$r_1$]
        (3.5,3) -- (4,3) to[C,l=$c$] (4,0)
        (4,3) -- (4.5,3) to[R,l=$r_2$] (6,3)
        (6,3) to[gswitch,l=$s_2$] (8,3)
        (0,0) -- (8,0) to[V_=$v_{out}$] (8,3) ;
    \end{circuitikz}
    \caption[1:1 SCC]{1:1 SCC used as a reference circuit for the \emph{Makowski} approximation.}
    \label{fig:single_capacitor}
\end{SCfigure}
Due to the \emph{arbitrary} of the first approximation, \citeauthor{2012Makowski} proposed, in a recent publication~\cite{2012Makowski}, a new approximation using a more rigourous approach given by
\begin{equation}
r_{scc,Mak} \approx \sqrt[\leftroot{-3}\uproot{3} \mu]{{r_{ssl}}^{\mu}+{r_{fsl}}^{\mu}},
\label{eq:r_scc_II}
\end{equation}
with $\mu = 2.54$.

\begin{SCfigure}[][!h]
\newcommand\pHeigh{3cm}
\newcommand\pWidth{5cm}
\centering
% This file was created by matlab2tikz.
%
%The latest updates can be retrieved from
%  http://www.mathworks.com/matlabcentral/fileexchange/22022-matlab2tikz-matlab2tikz
%where you can also make suggestions and rate matlab2tikz.
%

\begin{tikzpicture}
\pgfplotsset{
    width=\pWidth,
    height=\pHeigh,
    scale only axis,
    xlabel near ticks,
    ylabel near ticks,
    enlarge x limits={0.05},
    enlarge y limits={0.05},
    every tick label/.append style={font=\footnotesize},
    %yticklabel style={text width=2em,align=right},
}

\begin{loglogaxis}[
        ylabel= {$  [\Omega] $},
        xlabel= {$ f_{sw}~[Hz]$},
        axis y line*=left,
        axis x line*=bottom,
        xtick=\empty, ytick=\empty,
        xtick={250e3},
        xticklabels={$f_{\angle}$},
        legend style={
                draw=none,
                font=\footnotesize,
                column sep=1ex,
                legend cell align = left}
        ]

     \addplot [thick,smooth,black,domain=100e3:1e6]
        {1/(1e-6*x)*( exp(0.5/(x*1e-6))+1 )/(exp(0.5/(x*1e-6))-1)};
     \addlegendentry{$r_{scc}~(CF)$};

     \addplot [semithick,dashed,smooth,gray,domain=100e3:275e3] {1/(1e-6*x)};
     \addlegendentry{$r_{ssl}$};

     \addplot [semithick,dotted,smooth,gray,domain=200e3:1e6] {4)};
     \addlegendentry{$r_{fsl}$};

     \draw[->,thin] ({axis cs:250e3,0}|-{axis cs:0,4}) -- ({axis cs:250e3,0}|-{ axis cs:0,5.2521});
     %\draw[thin,dashed] ({axis cs:250e3,0}|-{axis cs:0,4}) -- ({axis cs:250e3,0}|-{ axis cs:0,0});
     \addplot [thin,mark=o,only marks,mark size=1] coordinates  { (250e3,4) (250e3,5.2521)};

     \node[font=\footnotesize, anchor=north east ] at (axis cs:250e3,4) {$r_x$};
     \node[font=\footnotesize, anchor=south west] at (axis cs:250e3,5.2521) {$r_{elbow}$};
     \node[font=\footnotesize, anchor=north, rotate=90] at (axis cs:250e3,4.5) {$p$};

\end{loglogaxis}

\end{tikzpicture}

\caption[\emph{Makowski} approximation of the $r_{scc}$]{Graphic demonstration of the \emph{Minkowski distance} $p$ between the two asymptotic limits ($r_{ssl}$ and $r_{fsl}$), and the closed form (CF) of $r_{scc}$. }
\label{fig:rscc_elbow}
\end{SCfigure}
As shown in Figure~\ref{fig:rscc_elbow}, the \emph{Makowski} formulation is based on solving the \emph{Minkowski distance} form
\begin{equation}
r_{elbow} = \left( {r_x}^{\mu}+{r_x}^{\mu} \right) ^\frac{1}{\mu} = 2^\frac{1}{\mu} r_x = p~r_x
\label{eq:r_scc_II}
\end{equation}
at the corner frequency $f_{\angle}$ where $r_x = r_{ssl} = r_{fsl}$, for a single capacitor under periodic and symmetric ($D=50\%$) voltage square excitation in steady-sate (see schematic in Figure~\ref{fig:single_capacitor}). The $r_{scc}$ closed form (CF) of the circuit used in to the approximation is
\begin{align}
r_{scc} & =  \frac{1}{2~c~f_{sw}} \left[ \frac{\me^{\frac{D}{\tau_1~f_{sw}}}+1}{\me^{\frac{D}{\tau_1~f_{sw}}}-1} +
\frac{\me^{\frac{1-D}{\tau_2~f_{sw}}}+1}{\me^{\frac{1-D}{\tau_2~f_{sw}}}-1} \right],\\
\tau_1 &= r_1c,\\
\tau_2 &= r_2c.
\label{eq:r_scc_CF}
\end{align}
A correction of the \citeauthor{2012Makowski} is proposed to cover the variations in the duty cycle by solving $\mu$ is as a function of $D$, as
\begin{align}
p & = \frac{1}{2} \left[ \frac{\me^{\frac{1}{D}+1}}{\me^{\frac{1}{D}-1}} + \frac{\me^{\frac{1}{1-D}+1}}{\me^{\frac{1}{1-D}-1}} \right], \label{eq:p}\\
\mu &= \frac{1}{\log_2 p}.
\label{eq:u_factor}
\end{align}
\begin{figure}[!h]
\newcommand\pHeigh{3.25cm}
\newcommand\pWidth{4.5cm}
\centering
    \begin{subfigure}{\textwidth}
       \parbox[b]{.45\linewidth}{
            \raggedright
            \newcommand\dutyCycle{10}
            \newcommand\uDx{1.74}
            % This file was created by matlab2tikz.
%
%The latest updates can be retrieved from
%  http://www.mathworks.com/matlabcentral/fileexchange/22022-matlab2tikz-matlab2tikz
%where you can also make suggestions and rate matlab2tikz.
%

\begin{tikzpicture}
\pgfplotsset{
    width=\pWidth,
    height=\pHeigh,
    scale only axis,
    ylabel near ticks,
    enlarge y limits={0.2},
    xlabel near ticks,
    ylabel near ticks,
    enlarge x limits={0.15},
    every tick label/.append style={font=\footnotesize},
    yticklabel style={text width=2em,align=right},
}

\begin{semilogxaxis}[
        %xlabel= {$f_{sw}[Hz] $},
        xticklabels={,,},
        ylabel= {$ \epsilon_r ~ [\%] $} ,
        axis y line*=left,
        axis x line*=bottom,
        %yticklabel style={xshift=0.5ex},
        enlarge y limits={0.1},
        title={$D=\dutyCycle\%~(\mu=\uDx)$ },
        title style = {
                at ={(0.5,1.1)},
                font=\footnotesize },
        legend style={
                legend columns = -1,
                at={(0.5,1)},
               anchor=south,
                draw=none,
                font=\tiny,
                column sep=1ex,
                legend cell align = left},
        ]


    \addplot [thin,smooth,black,mark=o,mark repeat=2]
      table [y=y1]{./3_modeling/err_rx_aprox_ORG_hom.dat};\label{pl_MDL}
    \addlegendentry{Org};
    \addplot [thin,smooth,mark=+,mark repeat=2]
      table [y=y1]{./3_modeling/err_rx_aprox_MAK_hom.dat};
    \addlegendentry{Mak};
    \addplot [thin,smooth,mark=x,mark repeat=2]
      table [y=y1]{./3_modeling/err_rx_aprox_RMAK_hom.dat};
    \addlegendentry{*Mak};


\end{semilogxaxis}

\end{tikzpicture}

        }
       \parbox[b]{.45\linewidth}{
            \raggedleft
            \newcommand\dutyCycle{23}
            \newcommand\uDx{2.12}
            % This file was created by matlab2tikz.
%
%The latest updates can be retrieved from
%  http://www.mathworks.com/matlabcentral/fileexchange/22022-matlab2tikz-matlab2tikz
%where you can also make suggestions and rate matlab2tikz.
%

\begin{tikzpicture}
\pgfplotsset{
    width=\pWidth,
    height=\pHeigh,
    scale only axis,
    ylabel near ticks,
    enlarge y limits={0.2},
    xlabel near ticks,
    ylabel near ticks,
    enlarge x limits={0.15},
    every tick label/.append style={font=\footnotesize},
    yticklabel style={text width=2em,align=right},
}

\begin{semilogxaxis}[
        %xlabel= {$f_{sw}[Hz] $},
        xticklabels={,,},
        %ylabel= {$ r_{scc} ~ [\Omega] $} ,
        axis y line*=left,
        axis x line*=bottom,
        %yticklabel style={xshift=0.5ex},
        enlarge y limits={0.1},
        title={$D=\dutyCycle\%~(\mu=\uDx)$ },
        title style = {
                at ={(0.5,1.1)},
                font=\footnotesize },
        legend style={
                legend columns = -1,
                at={(0.5,0.97)},
                anchor=south,
                draw=none,
                font=\tiny,
                column sep=1ex},
        ]


    \addplot [thin,smooth,black,mark=o,mark repeat=2]
      table [y=y2]{./3_modeling/err_rx_aprox_ORG_hom.dat};\label{pl_MDL}
    %%\addlegendentry{Model};
    \addplot [thin,smooth,mark=+,mark repeat=2]
      table [y=y2]{./3_modeling/err_rx_aprox_MAK_hom.dat};
    %\addlegendentry{Seeman};
    \addplot [thin,smooth,mark=x,mark repeat=2]
      table [y=y2]{./3_modeling/err_rx_aprox_RMAK_hom.dat};
    %\addlegendentry(Mak $)};


\end{semilogxaxis}

\end{tikzpicture}

        }
    \end{subfigure}

    \begin{subfigure}{\textwidth}
       \parbox[b]{.45\linewidth}{
            \raggedright
            \newcommand\dutyCycle{36}
            \newcommand\uDx{2.43}
            % This file was created by matlab2tikz.
%
%The latest updates can be retrieved from
%  http://www.mathworks.com/matlabcentral/fileexchange/22022-matlab2tikz-matlab2tikz
%where you can also make suggestions and rate matlab2tikz.
%

\begin{tikzpicture}
\pgfplotsset{
    width=\pWidth,
    height=\pHeigh,
    scale only axis,
    ylabel near ticks,
    enlarge y limits={0.2},
    xlabel near ticks,
    ylabel near ticks,
    enlarge x limits={0.15},
    every tick label/.append style={font=\footnotesize},
    yticklabel style={text width=2em,align=right},
}

\begin{semilogxaxis}[
        xlabel= {$f_{sw}[Hz] $},
        ylabel= {$ \epsilon_r ~ [\%] $} ,
        axis y line*=left,
        axis x line*=bottom,
        %yticklabel style={xshift=0.5ex},
        enlarge y limits={0.1},
        title={$D=\dutyCycle\%~(\mu=\uDx)$ },
        title style = {
                at ={(0.5,1.1)},
                font=\footnotesize },
        legend style={
                legend columns = -1,
                at={(0.5,0.97)},
                anchor=south,
                draw=none,
                font=\tiny,
                column sep=1ex},
        ]


    \addplot [thin,smooth,black,mark=o,mark repeat=2]
      table [y=y3]{./3_modeling/err_rx_aprox_ORG_hom.dat};\label{pl_MDL}
    %%\addlegendentry{Model};
    \addplot [thin,smooth,mark=+,mark repeat=2]
      table [y=y3]{./3_modeling/err_rx_aprox_MAK_hom.dat};
    %\addlegendentry{Seeman};
    \addplot [thin,smooth,mark=x,mark repeat=2]
      table [y=y3]{./3_modeling/err_rx_aprox_RMAK_hom.dat};
    %\addlegendentry{Mak. rect. $u=\uDx$};


\end{semilogxaxis}

\end{tikzpicture}

        }
       \parbox[b]{.45\linewidth}{
            \raggedleft
            \newcommand\dutyCycle{50}
            \newcommand\uDx{2.54}
            % This file was created by matlab2tikz.
%
%The latest updates can be retrieved from
%  http://www.mathworks.com/matlabcentral/fileexchange/22022-matlab2tikz-matlab2tikz
%where you can also make suggestions and rate matlab2tikz.
%

\begin{tikzpicture}
\pgfplotsset{
    width=\pWidth,
    height=\pHeigh,
    scale only axis,
    enlarge y limits={0.2},
    xlabel near ticks,
    ylabel near ticks,
    enlarge x limits={0.15},
    every tick label/.append style={font=\footnotesize},
    yticklabel style={text width=2em,align=right},
}

\begin{semilogxaxis}[
        xlabel= {$f_{sw}[Hz] $},
        %ylabel= {$ r_{scc} ~ [\Omega] $} ,
        axis y line*=left,
        axis x line*=bottom,
        enlarge y limits={0.1},
        title={$D=\dutyCycle\%~(\mu=\uDx)$ },
        title style = {
                at ={(0.5,1.1)},
                font=\footnotesize },
        legend style={
                legend columns = -1,
                at={(0.5,0.97)},
                anchor=south,
                draw=none,
                font=\tiny,
                column sep=1ex},
        ]


    \addplot [thin,smooth,black,mark=o,mark repeat=2]
      table [y=y4]{./3_modeling/err_rx_aprox_ORG_hom.dat};\label{pl_MDL}
    %%\addlegendentry{Model};
    \addplot [thin,smooth,mark=+,mark repeat=2]
      table [y=y4]{./3_modeling/err_rx_aprox_MAK_hom.dat};
    %\addlegendentry{Seeman};
    \addplot [thin,smooth,mark=x,mark repeat=2]
      table [y=y4]{./3_modeling/err_rx_aprox_RMAK_hom.dat};
    %\addlegendentry{Mak. rect. $u=\uDx$};


\end{semilogxaxis}

\end{tikzpicture}

        }
    \end{subfigure}


\caption[Different approximations methods for $r_{scc}$ scenario I]{Relative error of a single capacitor switching with homogenous $\tau$ constants  between the closed form of $r_{scc}$ and the different approximations: \emph{Org} - Original, \emph{Mak} - Makowski and \emph{*Mak} - rectified Mackowski.  Solved for the circuit in Figure~\ref{fig:single_capacitor} with $c=1\mu F$ and $r_1=r_2=1\Omega$.}
\label{fig:rscc_aprox_homo}
\end{figure}
\begin{figure}[!h]
\newcommand\pHeigh{3.25cm}
\newcommand\pWidth{4.5cm}
\centering
    \begin{subfigure}{\textwidth}
       \parbox[b]{.45\linewidth}{
            \raggedright
            \newcommand\dutyCycle{10}
            \newcommand\uDx{1.74}
            % This file was created by matlab2tikz.
%
%The latest updates can be retrieved from
%  http://www.mathworks.com/matlabcentral/fileexchange/22022-matlab2tikz-matlab2tikz
%where you can also make suggestions and rate matlab2tikz.
%

\begin{tikzpicture}
\pgfplotsset{
    width=\pWidth,
    height=\pHeigh,
    scale only axis,
    ylabel near ticks,
    enlarge y limits={0.15},
    xlabel near ticks,
    ylabel near ticks,
    enlarge x limits={0.15},
    every tick label/.append style={font=\footnotesize},
    yticklabel style={text width=2em,align=right},
}

\begin{semilogxaxis}[
        xticklabels={,,},
        ylabel= {$ \epsilon_r ~ [\%] $} ,
        axis y line*=left,
        axis x line*=bottom,
        title={$D=\dutyCycle\%~(\mu=\uDx)$ },
        title style = {
                at ={(0.5,1.1)},
                font=\footnotesize },
        legend style={
                legend columns = -1,
                at={(0.5,1)},
               anchor=south,
                draw=none,
                font=\tiny,
                column sep=1ex,
                legend cell align = left},
        ]


    \addplot [thin,smooth,black,mark=o,mark repeat=2]
      table [y=y1]{./3_modeling/err_rx_aprox_ORG_10het.dat};\label{pl_MDL}
    \addlegendentry{Org};
    \addplot [thin,smooth,mark=+,mark repeat=2]
      table [y=y1]{./3_modeling/err_rx_aprox_MAK_10het.dat};
    \addlegendentry{Mak};
    \addplot [thin,smooth,mark=x,mark repeat=2]
      table [y=y1]{./3_modeling/err_rx_aprox_RMAK_10het.dat};
    \addlegendentry{*Mak};


\end{semilogxaxis}

\end{tikzpicture}

        }
       \parbox[b]{.45\linewidth}{
            \raggedleft
            \newcommand\dutyCycle{23}
            \newcommand\uDx{2.12}
            % This file was created by matlab2tikz.
%
%The latest updates can be retrieved from
%  http://www.mathworks.com/matlabcentral/fileexchange/22022-matlab2tikz-matlab2tikz
%where you can also make suggestions and rate matlab2tikz.
%

\begin{tikzpicture}
\pgfplotsset{
    width=\pWidth,
    height=\pHeigh,
    scale only axis,
    ylabel near ticks,
    enlarge y limits={0.2},
    xlabel near ticks,
    ylabel near ticks,
    enlarge x limits={0.15},
    every tick label/.append style={font=\footnotesize},
    yticklabel style={text width=2em,align=right},
}

\begin{semilogxaxis}[
        xticklabels={,,},
        axis y line*=left,
        axis x line*=bottom,
        enlarge y limits={0.1},
        title={$D=\dutyCycle\%~(\mu=\uDx)$ },
        title style = {
                at ={(0.5,1.1)},
                font=\footnotesize },
        legend style={
                legend columns = -1,
                at={(0.5,0.97)},
                anchor=south,
                draw=none,
                font=\tiny,
                column sep=1ex},
        ]


    \addplot [thin,smooth,black,mark=o,mark repeat=2]
      table [y=y2]{./3_modeling/err_rx_aprox_ORG_10het.dat};\label{pl_MDL}
    %%\addlegendentry{Model};
    \addplot [thin,smooth,mark=+,mark repeat=2]
      table [y=y2]{./3_modeling/err_rx_aprox_MAK_10het.dat};
    %\addlegendentry{Seeman};
    \addplot [thin,smooth,mark=x,mark repeat=2]
      table [y=y2]{./3_modeling/err_rx_aprox_RMAK_10het.dat};
    %\addlegendentry(Mak $)};


\end{semilogxaxis}

\end{tikzpicture}

        }
    \end{subfigure}

    \begin{subfigure}{\textwidth}
       \parbox[b]{.45\linewidth}{
            \raggedright
            \newcommand\dutyCycle{36}
            \newcommand\uDx{2.43}
            % This file was created by matlab2tikz.
%
%The latest updates can be retrieved from
%  http://www.mathworks.com/matlabcentral/fileexchange/22022-matlab2tikz-matlab2tikz
%where you can also make suggestions and rate matlab2tikz.
%

\begin{tikzpicture}
\pgfplotsset{
    width=\pWidth,
    height=\pHeigh,
    scale only axis,
    ylabel near ticks,
    enlarge y limits={0.2},
    xlabel near ticks,
    ylabel near ticks,
    enlarge x limits={0.15},
    every tick label/.append style={font=\footnotesize},
    yticklabel style={text width=2em,align=right},
}

\begin{semilogxaxis}[
        xlabel= {$f_{sw}[Hz] $},
        ylabel= {$ \epsilon_r ~ [\%] $} ,
        axis y line*=left,
        axis x line*=bottom,
        %yticklabel style={xshift=0.5ex},
        enlarge y limits={0.1},
        title={$D=\dutyCycle\%~(\mu=\uDx)$ },
        title style = {
                at ={(0.5,1.1)},
                font=\footnotesize },
        legend style={
                legend columns = -1,
                at={(0.5,0.97)},
                anchor=south,
                draw=none,
                font=\tiny,
                column sep=1ex},
        ]


    \addplot [thin,smooth,black,mark=o,mark repeat=2]
      table [y=y3]{./3_modeling/err_rx_aprox_ORG_10het.dat};\label{pl_MDL}
    %%\addlegendentry{Model};
    \addplot [thin,smooth,mark=+,mark repeat=2]
      table [y=y3]{./3_modeling/err_rx_aprox_MAK_10het.dat};
    %\addlegendentry{Seeman};
    \addplot [thin,smooth,mark=x,mark repeat=2]
      table [y=y3]{./3_modeling/err_rx_aprox_RMAK_10het.dat};
    %\addlegendentry{Mak. rect. $u=\uDx$};


\end{semilogxaxis}

\end{tikzpicture}

        }
       \parbox[b]{.45\linewidth}{
            \raggedleft
            \newcommand\dutyCycle{50}
            \newcommand\uDx{2.54}
            % This file was created by matlab2tikz.
%
%The latest updates can be retrieved from
%  http://www.mathworks.com/matlabcentral/fileexchange/22022-matlab2tikz-matlab2tikz
%where you can also make suggestions and rate matlab2tikz.
%

\begin{tikzpicture}
\pgfplotsset{
    width=\pWidth,
    height=\pHeigh,
    scale only axis,
    ylabel near ticks,
    enlarge y limits={0.2},
    xlabel near ticks,
    ylabel near ticks,
    enlarge x limits={0.15},
    every tick label/.append style={font=\footnotesize},
    yticklabel style={text width=2em,align=right},
}

\begin{semilogxaxis}[
        xlabel= {$f_{sw}[Hz] $},
        axis y line*=left,
        axis x line*=bottom,
        enlarge y limits={0.1},
        title={$D=\dutyCycle\%~(\mu=\uDx)$ },
        title style = {
                at ={(0.5,1.1)},
                font=\footnotesize },
        legend style={
                legend columns = -1,
                at={(0.5,0.97)},
                anchor=south,
                draw=none,
                font=\tiny,
                column sep=1ex},
        ]


    \addplot [thin,smooth,black,mark=o,mark repeat=2]
      table [y=y4]{./3_modeling/err_rx_aprox_ORG_10het.dat};\label{pl_MDL}
    %%\addlegendentry{Model};
    \addplot [thin,smooth,mark=+,mark repeat=2]
      table [y=y4]{./3_modeling/err_rx_aprox_MAK_10het.dat};
    %\addlegendentry{Seeman};
    \addplot [thin,smooth,mark=x,mark repeat=2]
      table [y=y4]{./3_modeling/err_rx_aprox_RMAK_10het.dat};
    %\addlegendentry{Mak. rect. $u=\uDx$};


\end{semilogxaxis}

\end{tikzpicture}

        }
    \end{subfigure}



\caption[Different approximations methods for $r_{scc}$ scenario II]{Relative error of a single capacitor switching with heterogenous $\tau$ constants ($10 \tau_1 = \tau_2$)  between the closed form of $r_{scc}$ and the different approximations: \emph{Org} - Original, \emph{Mak} - Makowski and \emph{*Mak} - rectified Mackowski.  Solved for the circuit in Figure~\ref{fig:single_capacitor} with $c=1\mu F$ and $r_1=r_2=10\Omega$.}
\label{fig:rscc_aprox_hete}
\end{figure}

An initial assessment of the different approximations is given for the circuit of Figure~\ref{fig:single_capacitor} used as a reference in this new formulation. The results are presented for two different scenarios:
\begin{itemize}
  \item Converter with homogenous time constants, thus $\tau_1 = \tau_2$, reproducing the scenario assumed for the new formulation.

  \item Converter with heterogenous time constants, thus $10 \tau_1 = \tau_2$, reproducing a case with a less idealized converter.
\end{itemize}
Giving the relative error between the closed form solution~\eqref{eq:r_scc_CF} and the three approximations: Original (\emph{Org.}), Makowski (\emph{Mak.}), and rectified Makowski (\emph{*Mak}). In the first case, Figure~\ref{fig:rscc_aprox_homo}, with homogenous time constants. The \emph{rectified Makowski} formulation presents the best results for all four tested duty cycles, obviously matching the \emph{Makowski } approximation for $D=50\%$. The \emph{Original} approximation is the second best approximation for the two small values of $D$, since $\mu$ is closer to 2.

This improved accuracy that presents the \emph{rectified Makowski} approximation, changes as the $\tau$ constants of the converter diverge from each other, as happens in the second scenario of Figure~\ref{fig:rscc_aprox_hete}. In this case, the \emph{Original} approximation keeps $\epsilon_r$ below $\pm5\%$, but for $D=10\%$ it rises about $-9\%$. \emph{Makowski} approximation is the best in the lowest $D=10\%$, but it becomes the worst for the other $D$ values, rising above $5\%$. \emph{Rectified Makowski} is the best for $D=23\%$, but it rises about $10\%$ for other values of $D$. Looking at this second scenario, the \emph{Original} formulation would be the preferred one since it keeps the error within the lowest boundaries for all simulated $D$ values. The results of Figures~\ref{fig:rscc_aprox_homo} and~\ref{fig:rscc_aprox_hete} are only given for a range of $D$ between $0\%$ and $50\%$, since $p(D)$, eq.~\eqref{eq:p}, is symmetric about $D=50\%$.

Considering the results  none of them shows a clear advantage with respect to the others. Actually, the \emph{Makowski} approximations obtains the $\mu$ values form a the correlation between \emph{Minkowski distance} for a specific  converter. Therefore, as the converter under study diverges from the reference circuit, the accuracy of the new approximations decreases, becoming even worst that the original formulation. That is why using the \emph{Makowski} formulation to obtain $\mu$ values for complex SCCs and H-SCCs, can be as arbitrary as it was to use the initial proposed value of $\mu=2$.

\afterpage{\clearpage}

\subsection{Conversion ratio}

The conversion ratio of the converter can be computed with the source \emph{net} charge multiplier, first element in $\mathbf{a}^j$, as
\begin{equation}
m=\frac{{v_{trg}}}{v_{src}}=\sum_{j=1}^{phases}a_{in}^j.
\label{eq:m_ratio}
\end{equation}

For instance, we can obtain the conversion ratio of the converter 3:1 H-Dickson of Figure~\ref{fig:3_1_hscc_solv} used in the previous example, applying~\eqref{eq:m_ratio} in the already solved $\mathbf{a}$ vectors of~\eqref{eq:a_31dikson}, resulting in
\begin{equation}
m_2= \sum_{j=1}^{2}a_{in}^j= \frac{2-D}{3} + 0 = \frac{2-D}{3},
\label{eq:m_31dickson}
\end{equation}
where the subscript in $m$ refers to the second node of the converter. Notice that the result coincides with the conversion ratio obtained in the previous chapter~\eqref{eq:int_vx_t}, where the same converter was solved using a different approach.

\afterpage{\clearpage}
