
Switch capacitor converters are circuits composed by a large number of switches and capacitors, and require accurate models to properly design them. SCCs have the peculiarity to be lossy by nature due to the non adiabatic energy transfer between capacitors, phenomenon not present in the inductor based converters. Generally, the modeling of SCCs focuses just on the description of the loss mechanisms associated to conduction and capacitor charge transfer, neglecting other sources of loss such as driving and switching loss. These loss mechanisms are proportional to the output current, thus they are normally modeled with resistor in the well known output impedance model.

This chapter presents an enhanced methodology that allows to model a SCC when any of the nodes, \emph{dc} or \emph{pwm}, are loaded. The proposed methodology include also the effects of PWM control, thus accurately predicting the equivalent internal impedance and the converter conversion ratio.

The chapter is divided in two sections, the first section is devoted to the study and model a SCC when loaded from a \emph{pwm}-node, reviewing and extending the charge flow analysis~\cite{95Makowski,Seeman:EECS-2009-78} to include two new aspects that arise with the proposed hybrid converter. First, any of the nodes of the converter are considered as possible outputs that can be loaded. Second, the analysis includes the effects of duty cycle modulation (PWM) in any of the operation regimes of the converter, which indeed affects the converter's conversion ratio and the produced losses. The previous models are discussed, and the limiting factors are identified. Subsequently the charge flow analysis is reformulated using a new approach that leads a better accuracy for the analysis SCC and allows to model the new proposed H-SCC. The new model is validated against circuit simulations and experimental data.

The second section is devoted to the study of multiple loaded H-SCC. Based on the well-known output impedance model, a new circuit representation for converters with multiple current-loaded outputs is presented. The related characterization methodology is developed to determine the parameters of said new model based on the current-loaded analysis presented in the first section. The resulting model is validated against simulation and experimental data.


\section{Single Output Converters}
Switched Capacitor Converters has been always treated as a two-port converter with single input and a single output as shown in Fig.\ref{fig:two_port}. The input port is connected to a voltage source and the output port feeds the load. The SCC provides between input, $v_i$, and output, $v_o$, a voltage conversion, $m$,  that  steps up, steps down or/and inverts the polarity of the input voltage. The current circuit theory  related to SCCs is valid only for the two-port configuration, therefore this section is dedicated to revisit the classical concepts of single output SCC and to enhance them to also cover the H-SCC.

\begin{figure}[!h]
\centering
\ctikzset { bipoles/length=1cm}
\begin{circuitikz}[american voltages,scale=0.65]
\draw
    (1,0) to[short,o-]
    (0,0) to[V = $V_{supply}$]
    (0,3) to[short,-o]
    (1,3) ;

\draw
    (2,3) --
    (2.5,3)

    (2,0) --
    (2.5,0)

    node[ocirc]  (IC)  at (2,0) {}
    node[ocirc]  (I) at (2,3) {}
    (I) to[open,v=$v_{i}$] (IC);


\draw [thick]
    (2.5,-0.5) --
    (2.5,3.5)  --
    (5.5,3.5)  --
    (5.5,-0.5) --
    (2.5,-0.5);

\draw (4,2)node[anchor=north]{$\frac{v_o}{v_{i}}=m$} ;
\draw
    (5.5,3) -- (6,3)
    (5.5,0) -- (6,0)
    node[ocirc]  (O)  at (6,3) {}
    node[ocirc]  (OC) at (6,0) {}
    (O) to[open,v^<=$v_{o}$] (OC);

\draw
    (7,0) to[short,o-]
    (8,0) to[ R= $Load$,mirror]
    (8,3) to[short,-o]
    (7,3) ;
\end{circuitikz}
\caption{General two port configuration of a Switched Capacitor Converter. }
\label{fig:two_port}
\end{figure}

\subsection{The Output Impedance Model}
\begin{SCfigure}%[!h]
\centering
\ctikzset { bipoles/length=1cm}
\begin{circuitikz}[american voltages, scale=0.65]
\draw
    (-0.5,0) to[V = $ m \cdot v_{src}  $]
    (-0.50,3) -- (0,3) to[R,l=$r_{scc}$,-o]  (3,3)
    (3.5,3) to[short,o-,i=$i_o$]
    (4.25,3)   to[R,l=$r_o$]
    (4.25,0) to[short,-o] (3.5,0)
    (3,0) to[short,o-] (-0.5,0)
    (0,3) to[open,v^=$v_{trg}$] (0,0)
    (3.5,3) to[open,v=$v_{out}$] (3.5,0);

\end{circuitikz}
\caption{Output impedance model of a switched capacitor converter.}
\label{fig:scc_model_oi}
\end{SCfigure}
The behavior of SSCs is modeled with the well-known output impedance model~\cite{2000Oota,2012Peter} that is composed of a controlled voltage source and equivalent resistance $r_{scc}$, as shown in Figure~\ref{fig:scc_model_oi}.The output voltage provided by the converter under no-load conditions is defined as \emph{target voltage} ($v_{trg}$). The  controlled voltage source provides the target voltage, being the value of voltage supply $v_{src}$ multiplied by the conversion ratio $m$, thus
\begin{equation}
v_{trg} =  m \cdot v_{src} .
\label{eq:vtrg}
\end{equation}

When the converter is loaded, the voltage at the converter's output, $v_{outs}$, drops proportionally with the load current. This is modeled with resistor $r_{scc}$, which accounts for the losses produced in the converter. Since the losses are proportional to the output current $i_o$, they can be modeled with a resistor. Using the presented model, the output voltage of the converter can be obtained as
\begin{equation}
v_{out} =  m \cdot v_{src} - i_o \cdot r_{scc} .
\label{eq:vout_scc}
\end{equation}

Therefore to solve \ref{eq:vout_scc} is necessary to obtain from the converter the two parameters of the model, the conversion ration $m$  and the equivalent output resistance $r_{scc}$. The first, can be easily solved using Kirchhoff's Voltage Laws as previously explained in Section~\ref{ch:conversion_ratio}. The second, is more complex and actually is the main problem in the modeling of SCCs.

Up to day, there are two different methodologies to infer the equivalent output resistance $r_{scc}$, plotted in~\ref{fig:plot_rscc}. On the one hand, S. Ben-Yaakov  ~\cite{2009Ben-Yaakov,2012Ben-Yaakov,2013Evzelman} has claimed a generalized methodology based on the analytical solution of each of the different R-C transient circuits of the converter, reducing all of them to a single transient solution. The methodology achieves a high accuracy, but yields to a set of none linear equations and high complexity for the analysis of advanced architectures.

On the other hand,  M. Makowski and D. Maksimovic~\cite{95Makowski} presented a methodology based on the analysis of the charge flow between capacitors in steady-state. The methodology is simple to apply and yields with a set of linear expressions, being them easy to operate for further analysis of the converters. Based on the charge flow analysis, M.Seeman~\cite{Seeman:EECS-2009-78} developed different metrics allowing to compare performances between capacitive and inductive converters.

Although both methodologies are valid in the modeling of SCCs, none of them has been used to model the effects of a loaded \emph{pwm}-node, which is fundamental to study the H-SCC.  Nevertheless the charge flow analysis has a more clean and simplified way of describing the loss mechanism, based on the hypothesis that a SCC in steady-state has to have null charge balance in all the capacitors. For that reason, this methodology has been chosen in this dissertation in order to model the \emph{hybrid} switched capacitor converter.

\sidecaptionvpos{figure}{!b}
\begin{SCfigure} %[!h]
\centering
\begin{tikzpicture}[scale=0.85]
    \begin{loglogaxis}[
        %width=12cm,
        %height=8cm,
        xlabel near ticks,
        ylabel near ticks,
        xlabel= {$f_{sw} ~~ [Hz] $},
        ylabel= {$ [\Omega] $} ,
        axis line style={->},
        axis y line*=left,
        axis x line*=bottom,
        xtick=\empty, ytick=\empty,
        %ytick = {0,.125,.25},
        %yticklabels={0,$v_{src}\frac{1}{3}$,$v_{src}\frac{2}{3}$,$v_{src}$},
        domain=3e4:5e6,
        samples=100,
        %xticklabels={0,$D \cdot T_{sw}$,$T_{sw}$ ,$2 T_{sw}$,$3 T_{sw} $},
        legend style={at={(0.75,0.75)}, anchor= north east},
        enlarge x limits={0.0},
        enlarge y limits={0.0}
        ]

  \newcommand\C{1e-6}
  \newcommand\Resr{1}
  \newcommand\Dx{0.5}
  \newcommand\Rfsl{4*\Resr}
  \newcommand\lbssl{1e5}
  \newcommand\lbfsl{1e6}

  \addplot [thick]   { 1/(2*x*\C)*((exp(\Dx/(\Resr*x*\C))+1)/(exp(\Dx/(\Resr*x*\C))-1) + (exp((1-\Dx)/(\Resr*x*\C))+1)/(exp((1-\Dx)/(\Resr*x*\C))-1))};
  %\addplot [thick,dashed,marker=square] { sqrt((1/(2*\C*x))^2 + \Rfsl^2) };
  \addplot [thick,dashed,domain=3e4:3.5e5] { (1/(\C*x) };
  \addplot [thick,dotted,domain=2.5e4:5e6] { \Rfsl+0*x };
\legend {$r_{scc}$,$r_{ssl}$,$r_{fsl}$};
\end{loglogaxis}

\node[anchor=north] at (6.25cm,1.35cm){$FSL$};
\node[anchor=north,rotate=-60] at (1.25cm,5cm){$SSL$};

\end{tikzpicture}
\caption{SCC Equivalent output resistance $r_{scc}$ as function of the frequency and the two asymptotic limits: \emph{Slow Switching Limit} (SSL) and \emph{Fast Switching Limit}(FSL). }
\label{fig:plot_rscc}
\end{SCfigure}


As aforementioned $r_{scc}$ accounts for the loss when the converter is loaded. All losses in the converter are, in fact, dissipated in the resistive elements of the converter: \emph{on}-resistance $r_{on}$ of the switches and equivalent series resistance $r_{esr}$ of the capacitors, thus all loesses represented by $r_{scc}$ are conduction losses. Nevertheless, the origin and magnitude of the losses depends on the operation region of the converter, which is function of the switching frequency as shown in the plot of Figure~\ref{fig:plot_rscc}.

As SCC has two well-defined regimes of operation: the \emph{Slow Switching Limit} (SSL) and the \emph{Fast Switching Limit} (FSL). Each of the two regimes defines an asymptotic limit for the $r_{scc}$ curve. In SSL, the converter operates at a switching frequency $f_{sw}$ much lower than the time constant $\tau$ of charge and discharge of the converter's capacitors, thereby allowing the full charge and discharge of the capacitors. As shown in Figure~\ref{fig:ic_ssl} the capacitor currents present an exponential-shape waveform. In this regime of operation, the losses are determined by the charge transfer between capacitors, and dissipated in the resistive paths of the converter, mainly in the switches. That is why, reducing the switch channel resistance does not decreases the losses, instead, it will produce sharper discharge currents producing higher electromagnetic disturbances. In SSL, losses are inversely proportional of product between the switching frequency and capacitances, limited by the SSL asymptote as it can be seen in Figure~\ref{fig:plot_rscc}.

In FSL, the converter operates with a switching frequency $f_{sw}$ much higher than the time constant $\tau$ of charge and discharge of the converter's capacitors, limiting the full charge and discharge transients. As shown in Figure~\ref{fig:ic_fsl} currents have block-shape waveforms. In such operation regime, the losses are totally produced by the parasitic resistive elements ($r_{on}$, $r_{esr}$), therefore changes in the capacitances or frequency do not modify the produced losses\footnote{The switching losses are not included in the modeling of $r_{scc}$. }. In FSL, $r_{scc}$ is constant and limited by the FSL asymptote as it can be seen in Figure~\ref{fig:plot_rscc}.

\begin{figure}[!h]
\centering
\ctikzset { bipoles/length=1cm}
\begin{subfigure}[t]{.45\textwidth}
    %\centering
    \raggedright
    \begin{tikzpicture}
            \begin{axis}[
                width={\textwidth},
                height={6cm},
                axis lines=middle,
                xlabel near ticks,
                ylabel near ticks,
                xlabel= {time},
                ylabel= {capacitor current},
                every axis x label/.style={
                    at={(ticklabel* cs:1.05)},
                    anchor=west,
                },
                x axis line style={->},
                y axis  line style={<->},
                xtick=\empty, ytick=\empty,
                %ytick = {0,.125,.25},
                %yticklabels={0,$v_{src}\frac{1}{3}$,$v_{src}\frac{2}{3}$,$v_{src}$},
                domain=-0.25:2,
                samples=100,
                %xticklabels={0,$D \cdot T_{sw}$,$T_{sw}$ ,$2 T_{sw}$,$3 T_{sw} $},
                xmin=-0.1,xmax=2.2,
                ymin=-1,ymax=1.2,
                ]

          \newcommand\xtauA{1/(3*5)};
          \newcommand\ioA{1};
          \newcommand\xtauB{3/(5*5)};
          \newcommand\ioB{-0.75};

          \addplot [thick,]   coordinates { (0,0) (0.1,0) (0.1,\ioA)};
          \addplot [thick,domain=0.1:1]   { \ioA*(  exp(-(x-0.1)/(\xtauA))) };
          \addplot [thick,domain=1:2]   coordinates { (1,0) (1,\ioB)};
          \addplot [thick,domain=1:2]   { \ioB*(  exp(-(x-1)/(\xtauB))) };



        \end{axis}
    \end{tikzpicture}
    \caption{Slow Switching Limit}
    \label{fig:ic_ssl}
\end{subfigure}
\hfill
\begin{subfigure}[t]{.45\textwidth}
    %\centering
    \raggedright
    \begin{tikzpicture}
            \begin{axis}[
                width={\textwidth},
                height={6cm},
                axis lines=middle,
                xlabel near ticks,
                ylabel near ticks,
                xlabel= {$time $},
                ylabel= {capacitor current},
                every axis x label/.style={
                    at={(ticklabel* cs:1.05)},
                    anchor=west,
                },
                x axis line style={->},
                y axis  line style={<->},
                xtick=\empty, ytick=\empty,
                %ytick = {0,.125,.25},
                %yticklabels={0,$v_{src}\frac{1}{3}$,$v_{src}\frac{2}{3}$,$v_{src}$},
                domain=0:2.2,
                samples=100,
                %xticklabels={0,$D \cdot T_{sw}$,$T_{sw}$ ,$2 T_{sw}$,$3 T_{sw} $},
                xmin=-0.1,xmax=2.2,
                ymin=-1,ymax=1.2,
                ]

          \newcommand\xtauA{7)};
          \newcommand\ioA{0.65};
          \newcommand\xtauB{10};
          \newcommand\ioB{-0.35};

          \addplot [thick,]   coordinates { (0,0) (0.1,0) (0.1,\ioA)};
          \addplot [thick,domain=0.1:1]   { \ioA*(  exp(-(x-0.1)/(\xtauA))) };
          \addplot [thick,domain=1:2]   coordinates { (1, { \ioA*exp(-(1-0.1)/(\xtauA))} ) (1,\ioB)};
          \addplot [thick,domain=1:2]   { \ioB*(  exp(-(x-1)/(\xtauB))) };
          \addplot [thick]   coordinates { (2, {\ioB*exp(-(1)/(\xtauB))} ) (2,\ioA)};



        \end{axis}
    \end{tikzpicture}
    \caption{Fast Switching Limit}
    \label{fig:ic_fsl}
\end{subfigure}
\caption{Current waveforms though the capacitors in each of the two regimes of operation. }
\label{fig:capacitor_current}
\end{figure}


\subsection{Revising the charge flow analysis}

The charge flow analysis is based on the charge conservation in the converter's capacitors during an entire switching period in steady state~\cite{95Makowski}. The converter is studied in the two well-defined operating regimes: the Slow Switching Limit (SSL) and the Fast Switching Limit (FSL). In SSL, losses are then dominated by the charge transfer between capacitors, therefore only the charge transfer loss mechanisms are studied.  In FSL, losses depend on the conduction through the parasitic resistive elements, therefore only the conduction losses are studied. This division in the study of the converter reduces the complexity of the problem, and enables a simplified but accurate analysis.

In the charge flow analysis, the flowing charges are used instead of the currents. Moreover, the charges are grouped in charge flow vectors, being each vector is normalized with respect to the total charge flow delivered at the output of the converter.


\subsection{Load Model: Voltage Sink versus Current Sink}

In order to model a SCC, the original charge flow method~\cite{95Makowski} makes three main assumptions:
\begin{enumerate}
  \item The load is modeled as an ideal voltage source since it is normally connected to the \emph{dc}-output in parallel with a large capacitor, as shown in Figure~\ref{fig:vsink_load}. Such assumption, eliminates the capacitor connected in parallel with the load, neglecting the effects of the output capacitor to the equivalent resistance of the model.

  \item The model only considers the \emph{dc}-output as the single load point of the converter, imposing a unique output to the converter.

  \item The phase time ratio is not included in the computation of the capacitor charge flow. Consequently, the modulation of the switching period is assumed to have no influence on the amount of charge flowing in the capacitors.
\end{enumerate}

\begin{figure}[!h]
    \centering
    \ctikzset { bipoles/length=1cm}
    \begin{subfigure}[t]{.4\textwidth}
        \centering
        \begin{circuitikz}[american voltages,scale=0.65]
        \draw
                %Draw Switches
                (1,0)  to[V=$v_{src}$]
                (1,8)  --
                (5,8)   to[switch=$s_1$]
                (5,6)   to[switch=$s_2$]
                (5,4)   to[switch=$s_3$]
                (5,2)   to[switch=$s_4$]
                (5,0)  --
                (1,0)


        %Draw Capacitors
                (5,2) --
                (3,2) to[pC,l_=$c_{1}$]
                (3,6)--
                (5,6)

                (5,0) --
                (7,0) to[V,l_=$v_{out}$]
                (7,4)--
                (5,4);
        \end{circuitikz}
        \caption {Load modeled as a voltage sink.}
        \label{fig:vsink_load}
    \end{subfigure}
    \hfill
    \begin{subfigure}[t]{.4\textwidth}
        \centering
        \begin{circuitikz}[american,scale=0.65]
        \draw
                %Draw Switches
                (1,0)  to[V=$v_{src}$]
                (1,8)  --
                (5,8)   to[switch=$s_1$]
                (5,6)   to[switch=$s_2$]
                (5,4)   to[switch=$s_3$]
                (5,2)   to[switch=$s_4$]
                (5,0)  --
                (1,0)


        %Draw Capacitors
                (5,2) --
                (3,2) to[pC,l_=$c_{1}$]
                (3,6)--
                (5,6)

                (5,0) --
                (7,0) to[pC,l=$c_{2}$]
                (7,4)--
                (5,4)

        %%Sink load
                (7,4) --
                (9,4) to[I,l=$i_{out}$]
                (9,0) --
                (7,0);

        \end{circuitikz}
        \caption {Load modeled as a current sink.}
        \label{fig:isink_load}
    \end{subfigure}
\caption {Different load models for the charge flow analysis.  }
\label{fig:loads}
\end{figure}

Such assumptions reduce the usability of the model to the specific application of dc-to-dc conversion, and, at the same time, limit the flexibility to model different concepts of the SCC, such as the H-SCC previously introduced in Chapter~\ref{ch:H-SCC}. In order to overcome these limitations, the presented methodology makes two different assumptions:
\begin{enumerate}
  \item The load is assumed to be a constant current source with a value equal to the average load current, as shown in Figure~\ref{fig:isink_load}. In fact, using such approach the charge delivered to the load can be evaluated for each switching phases $j$ as
      \begin{equation}
        q_{out}^j = D^j \frac{i_{out}}{f_{sw}} = D^j i_{out}{T_{sw}}  = D^j q_{out},
      \label{eq:q_out}
      \end{equation}
  where $i_{out}$ is the average output current and $D^j$ is the duty cycle corresponding to the $j$-th phase.


  \item Any of the converter nodes can be loaded. Since the load is modeled as a current sink, it can be now connected to any of the converter's nodes without biasing it.

  \item When the load is connected to a \emph{dc}-node the associated \emph{dc}-capacitor of the node is not longer cancelled by the load model, thus the effects of the output capacitor are included in the model.

\end{enumerate}


\subsection{Re-formulating the charge flow analysis}

The equivalent impedance encompasses the root losses produced in a SCC due to capacitor charge transfer and charge conduction. As aforementioned, the classical charge flow methodology assumes an infinitely large output capacitance connected to a \emph{dc}-node, producing inaccuracy in the prediction of the equivalent output impedance when the output capacitor is comparable in value to the flying capacitors~\cite{2013Breussegem:c_out}. Actually, the root cause for this inaccuracy relies in the wrong quantification of the charge vectors that produces the converter losses.\\

\begin{figure}[!h]
\centering
\ctikzset { bipoles/length=1cm}
\begin{subfigure}[t]{.4\textwidth}
    %\centering
    \raggedright
    \begin{circuitikz} [american,scale=0.65]
    \draw
        (0,0) to[V=$v_{src}$] (0,3)
        (3,3) to[pC,l=$c_1$] (0,3)
        (3,0) to[pC,l=$c_2$] (3,3) -- (6,3)
        (6,0) to[pC,l=$c_3$] (6,3) --
        (8,3) to[I,l=$i_o$] (8,0) -- (0,0);
    \begin{scope}[>=latex,thick,text=black]
        \draw [->,rounded corners=7pt,dashed]
            (0.4,2.4) -- (2.4,2.4) -- (2.4,0.4);
        \draw [->,rounded corners=7pt,dashed]
            (3.6,0.5) |- (5.4,2.7) -- (5.4,0.4);
        \draw [->,rounded corners=7pt]
             (6.3,2) |- (8,3.3);
        \draw [>=latex,text=black,dashed]
          (0,4)  -- (0.9,4) node[anchor=west]{Redistributed charge};
        \draw [>=latex,text=black]
          (0,4.5)  -- (0.9,4.5) node[anchor=west]{Pumped charge};
    \end{scope}
    \end{circuitikz}
    \caption{}
\end{subfigure}
\hfill
\hfill
\begin{subfigure}[t]{.4\textwidth}
    %\centering
    \raggedleft
    \begin{circuitikz} [american,scale=0.65]
    \draw
        (1,0) to[V=$v_{src}$] (1,3)
        (6,3) to[pC,l=$c_2$] (3,3)
        (3,0) to[pC,l=$c_1$] (3,3)
        (6,0) to[pC,l=$c_3$] (6,3) --
        (8,3) to[I,l=$i_o$] (8,0) -- (1,0);
    \begin{scope}[>=latex,thick,text=black]
        \draw [->,rounded corners=7pt,dashed]
            (3.6,0.5) |- (5.4,2.4) -- (5.4,0.4);
        \draw [->,rounded corners=7pt]
            (6.3,2) |- (8,3.3);% -- (8.6,0.4);

    \end{scope}
    \end{circuitikz}
    \caption{}
\end{subfigure}
\caption{Charge flows in a Dickson 3:1 converter when loaded at a \emph{dc}-node with a large capacitor during the two switching phases. }
\label{fig:charge_flow_I}
\end{figure}

Looking, in detail, the charge flow in a SCC, we can identify two different charge flows during each circuit mode:
\begin{description}

  \item[Redistributed charge] flows between capacitors in order to equalize their voltage differences, being them the source of losses, thus by evaluating them the capacitor charge losses can be obtained.

      This charge flow is associated with a charge or discharge of the capacitors, happening right after the switching event and lasting for a short period of time\footnote{The duration of the charge depends on the time constant of the associated R-C circuit.}.

  \item[Pumped charge] flows from the capacitors to the load, this charge is consumed by the load, hence producing useful work.  This charge delivery is associated with a discharge of the capacitors, lasting for the entire phase time.

\end{description}

In fact, we can also define another theoretical charge flow, which is used to solve the flowing charges in the converter:
\begin{description}

  \item[Net charge] flow is quantified based on the principle of \emph{capacitor charge balance} for a converter in steady state. Based on that principle all \emph{net} charges in the capacitors can be obtained applying KCL, but using charges instead of currents. Therefore, the circuit can be solved for the \emph{net} charges flow applying
      the \emph{capacitor charge balance} as
      \begin{equation}
       \forall~c_{i} : \sum_{j=1}^{phases}q_{i}^j = 0,
      \label{eq:charge_balance}
      \end{equation}

     The resulting charges are then gathered in the charge flow vector $\mathbf{a}$ as
       \begin{equation}
        \mathbf{a}^j =  \left[ a_{in}^j~a_1^j~a_2^j \cdots a_n^j \right] = \frac{\left[ q_{in}^j~q_1^j~q_2^j \cdots q_n^j \right]}{q_{out}},
      \label{eq:a_vector}
      \end{equation}
    where the superindex denotes the $j$-th phase, $q_{in}$ is the charge supplied by the voltage source and $q_i$ is the \emph{net} charge flowing in the $i$-th capacitor $c_i$. Notice that the vector is normalized with respect to the output charge $q_{out}$.



\end{description}


The loss mechanisms of the converters can be better understood based on these two different charge flows. For instance Figure~\ref{fig:charge_flow_I} shows the charge flows for a 3:1 Dickson converter with a infinitely large output capacitor $c_3$. In such converter, the charge flow through capacitors $c_1$ and $c_2$ is always redistributed towards the big capacitor $c_3$ and only the capacitor $c_3$ will supply charge to the load. Hence the transported charge in $c_1$ and $c_2$ is producing losses and never supplying the load. However, for a finite value of the output capacitor, or for converters loaded from an internal node, all of the capacitors contribute to pumping charge to the load~\cite{2013Breussegem:c_out}.

\begin{figure}[!h]
\centering
\ctikzset { bipoles/length=1cm}
\begin{subfigure}[t]{.4\textwidth}
    %\centering
    \raggedright
    \begin{circuitikz} [american,scale=0.65]
    \draw
        (0,0) to[V=$v_{src}$] (0,3)
        (3,3) to[pC,l=$c_1$] (0,3)
        (3,0) to[pC,l=$c_2$] (3,3) -- (6,3)
        (6,0) to[pC,l=$c_3$] (6,3) --
        (8,3) to[I,l=$i_o$] (8,0) -- (0,0);
    \begin{scope}[>=latex,thick,text=black]
        \draw [->,rounded corners=7pt,dashed]
            (0.4,2.4) -- (2.4,2.4) -- (2.4,0.4);
        \draw [->,rounded corners=7pt,dashed]
            (3.6,0.5) |- (5.4,2.7) -- (5.4,0.4);

        \draw [->,rounded corners=7pt]
            (2,3.3) -- (7,3.3)
            (2.7,2) |- (7.5,3.3)
            (6.3,2) |- (8,3.3);
        \draw [>=latex,text=black,dashed]
          (0,4)  -- (0.9,4) node[anchor=west]{Redistributed charge};
        \draw [>=latex,text=black]
          (0,4.5)  -- (0.9,4.5) node[anchor=west]{Pumped charge};
    \end{scope}


    \end{circuitikz}
    \caption{}
\end{subfigure}
\hfill
\hfill
\begin{subfigure}[t]{.4\textwidth}
    %\centering
    \raggedleft
    \begin{circuitikz} [american,scale=0.65]
    \draw
        (1,0) to[V=$v_{src}$] (1,3)
        (6,3) --  (3,3)
        (3,0) to[pC,l=$c_1$] (3,3)
        (6,0)-- (6,0.25) to[pC,l=$c_3$] (6,1.5) to[pC,l=$c_2$] (6,2.75) |-
        (8,3) to[I,l=$i_o$] (8,0) -- (1,0);
    \begin{scope}[>=latex,thick,text=black]
        \draw [->,rounded corners=7pt,dashed]
            (3.6,0.5) |- (5.4,2.7) -- (5.4,0.4);
        \draw [->,rounded corners=7pt]
             (2.7,2) |- (7.5,3.3)
             (6.3,2.5) |- (8,3.3);% -- (8.6,0.4);
    \end{scope}
    \end{circuitikz}
    \caption{}
\end{subfigure}
\caption{Charge flows in a Dickson 3:1 converter when loaded at one of the \emph{pwm}-nodes during the two switching phases. }
\label{fig:charge_flow_II}
\end{figure}

In another scenario, the one of Figure~\ref{fig:charge_flow_II},  a 3:1 H-Dickson with the load connected to second ~\emph{pwm}-node. In such converter, there is a redistributed charge flow between the different capacitors as in the previous case, but at the same time, all capacitors pump charge to the load as well. Therefore all the capacitors  contribute in delivering charge to the load.

As a matter of fact, the original analysis omitted to quantify the \emph{pumped} charge contribution of the flying capacitors; thereby overestimating the \emph{redistributed} charge, which leaded larger equivalent output resistance for the SSL. Therefore, in order to estimate the right output impedance, the \emph{redistributed} charge flow has to be properly quantified.\\

\begin{figure}[!h]
\centering
% This file was created by matlab2tikz.
%
%The latest updates can be retrieved from
%  http://www.mathworks.com/matlabcentral/fileexchange/22022-matlab2tikz-matlab2tikz
%where you can also make suggestions and rate matlab2tikz.
%

\begin{tikzpicture}

\begin{axis}[%
width=8cm,
height=4cm,
at={(1.532in,0.729in)},
scale only axis,
xmin=0,
xmax=100,
ymin=6.64,
ymax=6.673,
axis background/.style={fill=white},
axis x line*=bottom,
axis y line*=none
]
\addplot [thick,solid,forget plot]
  table[row sep=crcr]{%
1	6.65026844706453\\
2	6.6541906445053\\
3	6.65706994984393\\
4	6.65925863469407\\
5	6.66097190927411\\
6	6.66234478933293\\
7	6.66346632700282\\
8	6.66439739781252\\
9	6.66517963724446\\
10	6.66584465605996\\
11	6.66641623072511\\
12	6.66691268208139\\
13	6.66734838569269\\
14	6.6677347707153\\
15	6.66808100843884\\
16	6.66839450570606\\
17	6.66868127063699\\
18	6.66894619123669\\
19	6.66919325215929\\
20	6.66942570599671\\
21	6.66964621014256\\
22	6.66985693700001\\
23	6.67005966319872\\
24	6.67025584208054\\
25	6.67044666273543\\
26	6.67063309816195\\
27	6.67081594459879\\
28	6.67099585366935\\
29	6.67117335866543\\
30	6.67134889604556\\
31	6.67152282302303\\
32	6.67169543195692\\
33	6.67186696212833\\
34	6.67203760937764\\
35	6.67220753399144\\
36	6.66994898046201\\
37	6.66832576586377\\
38	6.66712775868683\\
39	6.66621535139474\\
40	6.66549387715858\\
41	6.66490036899855\\
42	6.66439286198158\\
43	6.6639431930172\\
44	6.66353165635979\\
45	6.66314608322637\\
46	6.6627778080521\\
47	6.66242109657901\\
48	6.66207215837762\\
49	6.6617284123912\\
50	6.66138813136178\\
51	6.66105016231045\\
52	6.66071373588587\\
53	6.66037833874885\\
54	6.66004362838354\\
55	6.659709376253\\
56	6.65937542987049\\
57	6.65904168749229\\
58	6.6587080812319\\
59	6.65837456579342\\
60	6.65804111199682\\
61	6.65770769693504\\
62	6.65737430934867\\
63	6.65704093984341\\
64	6.65670758237776\\
65	6.65637423294288\\
66	6.65604088886614\\
67	6.65570754836448\\
68	6.65537421024822\\
69	6.65504087372358\\
70	6.65470753826091\\
71	6.65437420350682\\
72	6.65404086922551\\
73	6.65370753525966\\
74	6.6533742015043\\
75	6.65304086788937\\
76	6.65270753436815\\
77	6.65237420090945\\
78	6.65204086749247\\
79	6.6517075341038\\
80	6.65137420073293\\
81	6.65104086737468\\
82	6.65070753402473\\
83	6.65037420068031\\
84	6.65004086733958\\
85	6.64970753400131\\
86	6.64937420066469\\
87	6.64904086732916\\
88	6.64870753399436\\
89	6.64837420066005\\
90	6.64804086732606\\
91	6.64770753399229\\
92	6.64737420065867\\
93	6.64704086732514\\
94	6.64670753399168\\
95	6.64637420065826\\
96	6.64604086732487\\
97	6.6457075339915\\
98	6.64537420065814\\
99	6.64504086732479\\
100	6.64470753399144\\
};
\end{axis}
\end{tikzpicture}%
\caption{Two possible voltage waveforms that show the capacitors in a SCC. Ripples are associated with the charge flow mechanisms: top) unipolar capacitor discharge (DC capacitor); bottom) bipolar capacitor discharge (flying capacitor).}
\label{fig:cap_riples}
\end{figure}

Looking to the voltage ripple in the capacitors during an entire switching cycle, in Figure~\ref{fig:cap_riples}, we can identify three different voltage ripples associated to the previous described charge flows:
\begin{description}
  \item[Net voltage ripple $\Delta vn$] is the voltage variation measured at the beginning and at the end of the switch events. As a matter of fact, this \emph{net} ripple can be computed from the null \emph{charge balance} in a capacitor in steady-state condition as
      \begin{equation}
        \Delta {vn}^j_i  = \frac{q_i ^j }{c_i}.
        \label{eq:net_voltage}
      \end{equation}
      Using (\ref{eq:a_vector}) the \emph{net} ripple can be formulated using the charge flow notation
      \begin{equation}
        \Delta {vn}^j_i  = \frac{a_i ^j }{c_i} {q_{out}}.
        \label{eq:net_voltage_cf}
      \end{equation}

      Notice that \emph{capacitor charge balance} principle is reflected in the \emph{net }voltage ripples of Figure~\ref{fig:cap_riples}. Thus the sum of all \emph{net} ripples of each capacitor during a switching cycle  must be zero; that is why in the two phase converter used in the example $\Delta vn^1 = \Delta vn^2$.

  \item[Pumped voltage ripple $\Delta vp$] is the voltage variation associated with the discharge of the capacitor by a constant current. Thanks to modeling the load as current sink, it can be identified by the linear voltage discharge, thus the \emph{pumped} ripple can be obtained for each switching phase as
      \begin{equation}
        \Delta {vp}^j_i  = D^j \frac{i_i^j}{c_i }T_{sw},
      \label{eq:pumped_voltage}
      \end{equation}
      where $i_i^j$ is the current flowing through the $i$-th capacitor $c_i$. Actually, the current flowing in each individual capacitor $c_i$ during each $j$-th phase can be expressed as function of the output current by solving the network of capacitors associated to the circuit of each mode, thus
      \begin{equation}
        i_i^j = b_i^j i_{out} ,
      \label{eq:b_cnst}
      \end{equation}
      where $ b_i^j $ is a constant coming from solving the capacitor network.  Replacing (\ref{eq:b_cnst}) and (\ref{eq:q_out}) into (\ref{eq:pumped_voltage}), the \emph{pumped} voltage ripple can be expressed in the charge flow notation as
      \begin{equation}
        \Delta {vp}^j_i  = D^j \frac{b_i^j}{c_i } {i_{out}} {T_{sw}} = D^j \frac{b_i^j}{c_i } {q_{out}}.
      \label{eq:pumped_voltage_cf}
      \end{equation}

      Like in the previous case with the \emph{net} charge flow, the $b_i^j$ elements are gathered in the \emph{pumped} charge flow vector $\mathbf{b}$ as

      \begin{equation}
        \mathbf{b}^j =  \left[ b_1^j~b_2^j \cdots b_n^j \right] = \frac{\left[ ~i_1^j~i_2^j \cdots i_n^j \right]}{i_{out}},
      \label{eq:b_vector}
      \end{equation}

      where the superindex denotes the $j$-th phase, $i_i$ is the \emph{pumped} current flowing in the $i$-th capacitor $c_i$. The vector is normalized with respect to the output current $i_{out}$. Notice that $b$ vector is dual for currents or charges.


  \item[Redistributed ripple $\Delta vr$ ]is the voltage variation associated to an exponential charge or discharge transient. Produced by the charge redistribution between capacitors and happening just right after the phase transition event. The \emph{redistribution} ripple can be quantified by the addition of the two previous ripples as
      \begin{equation}
        \Delta {vr}^j_i  = \Delta {vn}^j_i + \Delta {vp}^j_i .
      \label{eq:rdst_ripple_I}
      \end{equation}
      Substituting (~\ref{eq:net_voltage_cf}) and (\ref{eq:pumped_voltage_cf}) into (\ref{eq:rdst_ripple_I}) the \emph{redistributed} ripple is formulated in terms of the charge flow analysis, as
      \begin{equation}
        \Delta {vr}^j_i  = \frac{q_{out}}{c_i} \left[ a^j_i - D^j b^j_i \right] = \frac{q_{out}}{c_i} g^j_i,
      \label{eq:rdst_ripple_II}
      \end{equation}

      where $g^j_i$ is the \emph{redistributed} charge flow of the $j$-th phase and the $i$-th capacitor. The \emph{redistributed charge flow vector} $\mathbf{g}$ is actually defined as
      \begin{equation}
        \mathbf{g^j}   = \mathbf{a^j_c} - D^j \mathbf{b^j},
      \label{eq:rdst_chrg_flow}
      \end{equation}

      where $\mathbf{a_c}$ is the \emph{capacitor charge flow vector}, a sub-vector of $\mathbf{a}$ that only contains the charge flows corresponding to the capacitors.


\end{description}


\subsubsection[SSL]{Slow Switching Limit Equivalent Output Resistance}

The SSL equivalent output resistance $r_{ssl}$ accounts for the losses produced by the capacitor charge transfer, therefore $r_{scc}$ can be obtained evaluating the losses in the capacitors.  The energy lost in a charge or discharge of capacitor $c$ is given by
\begin{equation}
E_{loss}=\frac{1}{2}{{\Delta{v}}_c}^2 c.
\label{eq:e_lost}
\end{equation}
where $\Delta v_c$ is the voltage variation in the process. Previously, we defined that the \emph{redistributed} ripple is associated to the capacitor charge transfer, thus by substituting (\ref{eq:rdst_ripple_II}) into (\ref{eq:e_lost}) we obtain the losses due to capacitor charge transfer
\begin{equation}
E_i^j=\frac{1}{2}{({\Delta{vr}}_i^j)}^2 c_i = \frac{1}{2}\frac{{q_{out}}^2}{{c_i}^2}{\left[a_{i\
}^j-{D^j} {b_i^j}\right]}^2c_i=\frac{1}{2}\frac{{q_{out}}^2}{c_i}{\left[a_{i\
}^j-{D^j} {b_i^j}\right]}^2 .
\label{eq:e_lost_ssl}
\end{equation}
The total power loss in the circuit is the sum of the losses in all of the
capacitors during each phase multiplied by the switching frequency$f_{sw}$.
This yield{\small s}
\begin{equation}
P_{ssl}= f_{sw} \sum_{i=1}^{caps.}\sum_{j=1}^{phases} E_i^j =\frac{f_{sw}{q_{out}}^2}{2}\sum_{i=1}^{caps.}\sum_{j=1}^{phases}\frac{1}{c_i}{\left[a_{i\
}^j-{D^j}{b_i^j}\right]}^2.
\label{eq:pwr_ssl}
\end{equation}

The losses can be expressed as the output SSL resistance by dividing~\eqref{eq:pwr_ssl} by the
square of the output current as
\begin{equation}
r_{ssl}=\frac{P_{ssl}}{{i_o}^2}=\frac{P_{ssl}}{{(f_{sw} {q_{out}})}^2}=\frac{1}{2 f_{sw}}\sum_{i=1}^{caps.}\sum_{j=1}^{phases}\frac{1}{c_i}{\left[a_{i\
}^j-{D^j} {b_i^j}\right]}^2.
\label{eq:r_ssl}
\end{equation}


\subsubsection[FSL]{Fast Switching Limit Equivalent Output Resistance}
The fast switching limit (FSL) equivalent output resistance $r_{fsl}$ accounts for losses produced in the resistive circuit elements, being these the \emph{on}-resistance of the switches and the Equivalent Series Resistance (ESR) of the capacitors $r_{esr,c}$.

The power dissipated by resistor $r_i$  from a square-wave pulsating current is given by
\begin{equation}
P_{r_i} = r_i~D^j~i_i^2,
\label{eq:pwr_r}
\end{equation}
where $D^j$ is the duty cycle. The value of $i_i$ (peak current) though the resistor can be also defined by its flowing charge $q_i$ as
\begin{equation}
i_i = \frac{q_i}{D^j~T_{sw}} = \frac{q_i}{D^j} f_{sw}.
\label{eq:i_q}
\end{equation}
As outlined in~\cite{Seeman:EECS-2009-78}, the charge flowing through the parasitic resistive elements can be derived from the charge flow vectors ($\mathbf{a}$), providing the \emph{switch}\footnote{These charge flow vectors also account for other resistive elements, not only the switches, such as the capacitors' equivalent series resistance.} charge flow vectors $\mathbf{ar}$. Using the \emph{switch} charge flow multiplier, \eqref{eq_i_q} can be redefined as function of the output charge (or the output current) as
\begin{equation}
i_i = \frac{ar_i^j}{D^j} q_{out}~f_{sw} = \frac{ar_i^j}{D^j} i_{out}.
\label{eq:i_ar}
\end{equation}
Substituting~\eqref{eq:i_ar} into~\eqref{eq:pwr_r} yields
\begin{equation}
P_{r_i} = \frac{r_i}{D^j}{ar_i^j}^2 i_{out}^2 ,
\label{eq:pwr_r_ar}
\end{equation}
the total loss accounting all resistive elements and phases is then
\begin{equation}
P_{fsl} = \sum_{i=1}^{elm.} \sum_{j=1}^{phs.}  \frac{r_i}{D^j}{ar_i^j}^2 i_{out}^2,
\label{eq:pwr_fsl}
\end{equation}
dividing by $i_{out}^2$ yields the FSL equivalent output resistance:
\begin{equation}
r_{fsl}=\sum_{i=1}^{elm.}\sum_{j=1}^{phases}\frac{r_i}{D^j}{ar_i^j}^2
\label{eq:r_fsl}
\end{equation}
where $r_i$ is the resistance value of the $i$-th resistive element.


\subsubsection{Equivalent Switched Capacitor Converter Resistance}
With the goal of obtaining a simple design equation, an analytical approximation of $r_{scc}$ suggested in~\cite{1998Arntzen,1999Maksimovic}, is given by
\begin{equation}
r_{scc} \approx \sqrt{{r_{ssl}}^2+{r_{fsl}}^2},
\label{eq:r_scc}
\end{equation}
being used in all the presented results of this dissertation.

A recent publication~\cite{2012Makowski} claimed a \emph{better} approximation as
\begin{equation}
r_{scc,bis} \approx \sqrt[\leftroot{-3}\uproot{3} u]{{r_{ssl}}^{u}+{r_{fsl}}^{u}},
\label{eq:r_scc_II}
\end{equation}
where $u = 2.54$. This value comes from solving the equation of a single switched capacitor operating with a 0.5 duty cycle. This formulation has a better accuracy for converters with equal capacitor values, however it becomes worst than the original approximation for duty cycles different of $0.5$ or for converters with different time constants between phases. A slightly better accuracy is obtained with $u$ as function of the duty cycle $D$, as
\begin{align}
p & = \frac{1}{2} \left[ \frac{e^{\frac{1}{D}+1}}{e^{\frac{1}{D}-1}} + \frac{e^{\frac{1}{1-D}+1}}{e^{\frac{1}{1-D}-1}} \right],
\\
\\
u &= \frac{1}{\log_2 p}.
\label{eq:u_factor}
\end{align}

This approach shows better results in converters with similar time constants between phases. In circuits with different time constants, any of the approximations shows to be better than the others for the whole range of $D$.

\subsubsection{Conversion ratio}

The conversion ratio of the converter can be obtained with the source \emph{net} charge element from vector $\mathbf{a}$ as
\begin{equation}
m=\frac{{v_{trg}}}{v_{src}}=\sum_{j=1}^{phases}a_{in}^j.
\label{eq:r_ssl}
\end{equation}
where $a_{in}$  corresponds to the input voltage source term of the charge vector multiplier $\mathbf{a}$.

\subsection{Experimental Model Validation}

\begin{figure}[t]
\ctikzset { bipoles/length=1cm}
\centering
\begin{subfigure}[t]{0.45\textwidth}
    \centering
    \begin{circuitikz}[american voltages,scale=0.6]
    \draw
            %Input Supply
            %(0,0)  to[V=$v_{src}$]
            %Draw Switches
            %(0,10)  --
            (5,10.3) node[anchor=south] {$v_{src}$}
            (5,10) node[rground, yscale=-1] {}
            to[switch=$s_1$] %S1
            (5,8)   to[switch=$s_2$] %S2
            (5,6)   to[switch=$s_3$] %S3
            (5,4) --
            %left branch
            (3,4)   to[switch=$s_7$]
            (3,2)   to[switch=$s_6$]
            (3,0);

    \draw   %right branch
            (5,4) --
            (7,4)   to[switch,l_=$s_4$]
            (7,2)   to[switch,l_=$s_5$]
            (7,0) -- (3,0);


    \draw %Capacitor C1
           (3,2) -- (2,2) -- (2,4)
            to[pC,l_=$c_1$] (2,8) --
           (5,8);

    \draw %Capacitor C2
           (7,2) --
           (8.25,2) -- (8.25,3.5)  to[pC,l^=$c_2$] (8.25,6) --
           (5,6);

    \draw %Capacitor C3
           (5,0) node[sground] {} to[pC,l_=$c_3$] (5,4);

     %\draw (7,4) to[short,-o] (10,4) node[anchor=west] {};

     %\draw (9,6) to[open,v^=$v_{1}$] (9,0);
     \draw (8.25,6) to[short,-o] (9,6) node[anchor=west] {$v_{out}$} ;

     \end{circuitikz}
\caption{}
\label{fig:3_1_hscc_exp_a}
\end{subfigure}
\hfill
\begin{subfigure}[t]{0.45\textwidth}
    \centering
    \begin{circuitikz}[american voltages,scale=0.6]
    \draw
            %Input Supply
            %(0,0)  to[V=$v_{src}$]
            %Draw Switches
            %(0,10)  --
            (5,10.3) node[anchor=south] {$v_{src}$}
            (5,10) node[rground, yscale=-1] {}
            to[switch=$s_1$] %S1
            (5,8)   to[switch=$s_2$] %S2
            (5,6)   to[switch=$s_3$] %S3
            (5,4) --
            %left branch
            (3,4)   to[switch=$s_7$]
            (3,2)   to[switch=$s_6$]
            (3,0);

    \draw   %right branch
            (5,4) --
            (7,4)   to[switch,l_=$s_4$]
            (7,2)   to[switch,l_=$s_5$]
            (7,0) -- (3,0);


    \draw %Capacitor C1
           (3,2) -- (2,2) -- (2,4)
            to[pC,l_=$c_1$] (2,8) --
           (5,8);

    \draw %Capacitor C2
           (7,2) --
           (8.25,2) -- (8.25,3.5)  to[pC,l^=$c_2$] (8.25,6) --
           (5,6);

    \draw %Capacitor C3
           (5,0) node[sground] {} to[pC,l_=$c_3$] (5,4);

     %\draw (7,4) to[short,-o] (10,4) node[anchor=west] {};

     %\draw (9,6) to[open,v^=$v_{1}$] (9,0);
     \draw (5,4)  --([hs]8.25,4 |- 5,4) arc(180:0:\radius) to[short,-o] (9.5,4) node[anchor=west] {$v_{out}$} ;

     \end{circuitikz}
\caption{}
\label{fig:3_1_hscc_exp_b}
\end{subfigure}
\caption{Experiment 3:1 Dickson used for the experimental results, \emph{left}- output taken from a \emph{pwm}-node; \emph{right}- output taken from a \emph{dc}-node.}
\label{fig:3_1_hscc_exp}
\end{figure}

The model was validated using a 3:1 Dickson converter for the two different scenarios presented in Figure~\ref{fig:3_1_hscc_exp}. In the first scenario, the load is connected to the second \emph{pwm}-node, Figure~\ref{fig:3_1_hscc_exp_a}. In the other scenario, the converter is loaded at the \emph{dc}-node, Figure~\ref{fig:3_1_hscc_exp_b}. In both cases the output impedance values are compared with results obtained from transient PLECS\footnote{Behavioral circuit simulator} simulations. Furthermore results from the second scenario are compared with results from previous modeling works.  The charge flow vectors $\mathbf{a}, \mathbf{b} $ and $\mathbf{ar}$ are presented in the Appendix~\ref{apx:31_dick_charge_flows}.

\begin{figure}[!h]
\centering
    \begin{subfigure}{\textwidth}
        \parbox[c]{.03\linewidth}{\subcaption{}}
        \hspace{.02\linewidth}
        \parbox[c]{.95\linewidth}{
        \centering
        % This file was created by matlab2tikz.
%
%The latest updates can be retrieved from
%  http://www.mathworks.com/matlabcentral/fileexchange/22022-matlab2tikz-matlab2tikz
%where you can also make suggestions and rate matlab2tikz.
%

\begin{tikzpicture}
\pgfplotsset{
    width=9cm,
    height=2.5cm,
    scale only axis,
    xlabel near ticks,
    ylabel near ticks,
    enlarge y limits={0.2},
    legend style={
                legend columns = 3,
                at={(0.5,1.075)},
                anchor=south,
                draw=none,
                font=\small,
                column sep=2ex,
                legend cell align=left},
}

\begin{axis}[%
    axis x line*=bottom,
    axis y line*=left,
    %xlabel= {duty cycle},
    ylabel= {$r_{scc}~[\Omega]$},
    yticklabel style={text width=2em,align=right},
    ]
    
    \addplot [semithick,mark=square,only marks,white]
      table[y=y1] {./3_modeling/rx_sw_dx_O1.dat};\label{pl_PLECS_hidden}

    \addplot [semithick,mark=square,only marks]
      table[y=y1] {./3_modeling/rx_sw_dx_O1.dat};\label{pl_PLECS}


    \addplot [semithick,smooth,mark=o,mark repeat=10]
      table [y=y1] {./3_modeling/rm1_sw_dx_O1.dat};\label{pl_MDL_JD}

    \addplot [semithick,smooth,mark=+,mark repeat=10]%black!66]
      table [y=y1] {./3_modeling/rm2_sw_dx_O1.dat};\label{pl_Makw}

    \addplot [semithick,smooth,mark=x,mark repeat=10]%black!33]
      table [y=y1] {./3_modeling/rm3_sw_dx_O1.dat};\label{pl_Makw rec}



\end{axis}

\begin{axis}[%
    axis y line*=right,
    axis x line=none,
    ylabel = {$\epsilon_r~[\%]$},
    yticklabel pos=right,
    yticklabel style={text width=2em,align=left},
    ]
    \addlegendimage{/pgfplots/refstyle=pl_PLECS}\addlegendentry{PLECS}
    \addlegendimage{/pgfplots/refstyle=pl_MDL_JD}\addlegendentry{This Work}

    \addplot [semithick,mark=o,only marks,black!60]
      table [y=y1] {./3_modeling/err1_sw_dx_O1.dat};
    \addlegendentry{ $\epsilon_r$ (Rel. Error)}


    \addlegendimage{/pgfplots/refstyle=pl_PLECS_hidden}\addlegendentry{\color{white}PLECS}
    \addlegendimage{/pgfplots/refstyle=pl_Makw}\addlegendentry{Makowski}
    \addplot [semithick,mark=+,only marks,black!60]
      table [y=y1] {./3_modeling/err2_sw_dx_O1.dat};
    \addlegendentry{$\epsilon_r$ \color{white}(Rel. Error) }

    \addlegendimage{/pgfplots/refstyle=pl_PLECS_hidden}\addlegendentry{\color{white}PLECS}
    \addlegendimage{/pgfplots/refstyle=pl_Makw rec}\addlegendentry{Mak. rect.}
    \addplot [semithick,mark=x,only marks,black!60]
      table [y=y1] {./3_modeling/err3_sw_dx_O1.dat};
    \addlegendentry{$\epsilon_r$ \color{white}(Rel. Error)}

\end{axis}
\end{tikzpicture}




}
        \label{fig:exp_rscc_pwm_node_100kHz}
    \end{subfigure}

    \begin{subfigure}{\textwidth}
        \parbox[c]{.03\linewidth}{\subcaption{}}
        \hspace{.02\linewidth}
        \parbox[c]{.95\linewidth}{
        \centering
        % This file was created by matlab2tikz.
%
%The latest updates can be retrieved from
%  http://www.mathworks.com/matlabcentral/fileexchange/22022-matlab2tikz-matlab2tikz
%where you can also make suggestions and rate matlab2tikz.
%

\begin{tikzpicture}
\pgfplotsset{
    width=9cm,
    height=2.5cm,
    scale only axis,
    ylabel near ticks,
    enlarge y limits={0.2},
    xlabel near ticks,
    ylabel near ticks,
}
\begin{axis}[%
axis x line*=bottom,
axis y line*=left,
xlabel= {duty cycle},
ylabel= {$r_{scc}~[\Omega]$},
yticklabel style={text width=2em,align=right}
]

    \addplot [semithick,mark=square,only marks]
      table[y=y4] {./3_modeling/rx_sw_dx_O1.dat};\label{pl_PLECS}
    \addplot [semithick,smooth,black]
      table [y=y2] {./3_modeling/rm1_sw_dx_O1.dat};\label{pl_MDL}
    \addplot [semithick,smooth,black!66]
      table [y=y2] {./3_modeling/rm2_sw_dx_O1.dat};
    \addplot [semithick,smooth,black!33]
      table [y=y2] {./3_modeling/rm3_sw_dx_O1.dat};

\end{axis}

\begin{axis}[%
    axis y line*=right,
    axis x line=none,
    ylabel = {$\epsilon_r~[\%]$},
    yticklabel pos=right,
    yticklabel style={text width=2em,align=left},
    ]

    \addplot [semithick,mark=star,only marks,black]
    table [y=y4] {./3_modeling/err1_sw_dx_O1.dat};
          
    \addplot [semithick,mark=star,only marks,black!66]
    table [y=y4] {./3_modeling/err2_sw_dx_O1.dat};

    \addplot [semithick,mark=star,only marks,black!33]
    table [y=y4] {./3_modeling/err3_sw_dx_O1.dat};

\end{axis}
\end{tikzpicture}
}
        \label{fig:exp_rscc_pwm_node_1MHz}
    \end{subfigure}

    \begin{subfigure}{\textwidth}
        \parbox[c]{.03\linewidth}{\subcaption{}}
        \hspace{.02\linewidth}
        \parbox[c]{.95\linewidth}{
        \centering
        % This file was created by matlab2tikz.
%
%The latest updates can be retrieved from
%  http://www.mathworks.com/matlabcentral/fileexchange/22022-matlab2tikz-matlab2tikz
%where you can also make suggestions and rate matlab2tikz.
%

\begin{tikzpicture}
\pgfplotsset{
    width=9cm,
    height=2.5cm,
    scale only axis,
    ylabel near ticks,
    enlarge y limits={0.2},
    xlabel near ticks,
    ylabel near ticks,
}
\begin{axis}[%
axis x line*=bottom,
axis y line*=left,
%xlabel= {duty cycle},
ylabel= {$r_{scc}~[\Omega]$},
yticklabel style={text width=2em,align=right},
]

    \addplot [semithick,mark=square,only marks]
      table[y=y7] {./3_modeling/rx_sw_dx_O1.dat};\label{pl_PLECS}
    \addplot [semithick,smooth,mark=o,mark repeat=10]
      table [y=y3] {./3_modeling/rm1_sw_dx_O1.dat};\label{pl_MDL_JD}
    \addplot [semithick,smooth,mark=+,mark repeat=10]%black!66]
      table [y=y3] {./3_modeling/rm2_sw_dx_O1.dat};\label{pl_Makw}
    \addplot [semithick,smooth,mark=x,mark repeat=10]%black!33]
      table [y=y3] {./3_modeling/rm3_sw_dx_O1.dat};\label{pl_Makw rec}

\end{axis}

\begin{axis}[%
axis y line*=right,
axis x line=none,
ylabel = {$\epsilon_r~[\%]$},
yticklabel pos=right,
yticklabel style={text width=2em,align=left},
]

    \addplot [semithick,mark=o,only marks,black!60]
    table [y=y7] {./3_modeling/err1_sw_dx_O1.dat};

    \addplot [semithick,mark=+,only marks,black!60]
    table [y=y7] {./3_modeling/err2_sw_dx_O1.dat};

    \addplot [semithick,mark=x,only marks,black!60]
    table [y=y7] {./3_modeling/err3_sw_dx_O1.dat};

\end{axis}
\end{tikzpicture}
}
        \label{fig:exp_rscc_pwm_node_10MHz}
    \end{subfigure}

    \begin{subfigure}{\textwidth}
        \parbox[c]{.03\linewidth}{\subcaption{}}
        \hspace{.02\linewidth}
        \parbox[c]{.95\linewidth}{
        \centering
        % This file was created by matlab2tikz.
%
%The latest updates can be retrieved from
%  http://www.mathworks.com/matlabcentral/fileexchange/22022-matlab2tikz-matlab2tikz
%where you can also make suggestions and rate matlab2tikz.
%

\begin{tikzpicture}
\pgfplotsset{
    width=9cm,
    height=2.5cm,
    scale only axis,
    ylabel near ticks,
    enlarge y limits={0.2},
    xlabel near ticks,
    ylabel near ticks,
}
\begin{axis}[%
axis x line*=bottom,
axis y line*=left,
xlabel= {duty cycle},
ylabel= {$r_{scc}~[\Omega]$},
yticklabel style={text width=2em,align=right},
]

\addplot [semithick,mark=square,only marks,forget plot]
  table {./3_modeling/sim_rx_100MHz_O1.dat};

\addplot [semithick,solid,forget plot]
  table {./3_modeling/mdl_rx_100MHz_O1.dat};

\end{axis}

\begin{axis}[%
axis y line*=right,
axis x line=none,
ylabel = {$\epsilon_r~[\%]$},
yticklabel pos=right,
yticklabel style={text width=2em,align=left},
]

\addplot [semithick,mark=star,only marks,forget plot]
  table {./3_modeling/error_rx_100MHz_O1.dat};


\end{axis}
\end{tikzpicture}
}
        \label{fig:exp_rscc_pwm_node_100MHz}
    \end{subfigure}

\caption{Equivalent Output Resistance ($r_{scc}$) from the \emph{pwm}-node of the converter of Figure~\ref{fig:3_1_hscc_exp_a}. Experimental results (\emph{square marks}) compared with the model (\emph{solid line}) at different switching frequencies ($f_{sw}$): \emph{top to bottom}- $100kHz$, $1MHz$, $10MHz$ and $100MHz$.Plots are obtained for the different analytical $r_{scc}$ approximations (see~\ref{ch:an_apprx}): \emph{Black} - Original $u=2$ ,\emph{grey} - Makowski  $u=2.54$, \emph{light grey} - rectified Makowski $u=f(D)$. }
\label{fig:exp_rscc_pwm_node}
\end{figure}


\begin{figure}[!h]
\centering
    \begin{subfigure}{\textwidth}
        \parbox[c]{.03\linewidth}{\subcaption{}}
        \hspace{.02\linewidth}
        \parbox[c]{.95\linewidth}{
        \centering
        % This file was created by matlab2tikz.
%
%The latest updates can be retrieved from
%  http://www.mathworks.com/matlabcentral/fileexchange/22022-matlab2tikz-matlab2tikz
%where you can also make suggestions and rate matlab2tikz.
%

\begin{tikzpicture}
\pgfplotsset{
    width=9cm,
    height=2.5cm,
    scale only axis,
    xlabel near ticks,
    ylabel near ticks,
    enlarge y limits={0.2},
    legend style={
                legend columns = 3,
                at={(0.5,1.075)},
                anchor=south,
                draw=none,
                font=\small,
                column sep=2ex,
                legend cell align=left},
}

\begin{axis}[%
    axis x line*=bottom,
    axis y line*=left,
    %xlabel= {duty cycle},
    ylabel= {$r_{scc}~[\Omega]$},
    yticklabel style={text width=2em,align=right},
    ]

    \addplot [semithick,mark=square,only marks,white]
      table[y=y1] {./3_modeling/rx_sw_dx_O2.dat};\label{pl_PLECS_hidden}

    \addplot [semithick,mark=square,only marks]
      table[y=y1] {./3_modeling/rx_sw_dx_O2.dat};\label{pl_PLECS}


    \addplot [semithick,smooth,mark=o,mark repeat=10]
      table [y=y1] {./3_modeling/rm1_sw_dx_O2.dat};\label{pl_MDL_JD}

    \addplot [semithick,smooth,mark=+,mark repeat=10]%black!66]
      table [y=y1] {./3_modeling/rm2_sw_dx_O2.dat};\label{pl_Makw}

    \addplot [semithick,smooth,mark=x,mark repeat=10]%black!33]
      table [y=y1] {./3_modeling/rm3_sw_dx_O2.dat};\label{pl_Makw rec}

    \addplot [semithick,smooth,dashed,black!75]
      table [y=y1] {./3_modeling/rm1_ms_sw_dx_O2.dat};\label{pl:95Makw}

    \addplot [semithick,smooth,dotted,black!75]
      table [y=y1] {./3_modeling/rx_ST_sw_dx_1Co.dat};\label{pl:Stey}



\end{axis}

\begin{axis}[%
    axis y line*=right,
    axis x line=none,
    ylabel = {$\epsilon_r~[\%]$},
    yticklabel pos=right,
    yticklabel style={text width=2em,align=left},
    ]
    \addlegendimage{/pgfplots/refstyle=pl_PLECS}\addlegendentry{PLECS}
    \addlegendimage{/pgfplots/refstyle=pl_MDL_JD}\addlegendentry{This Work}

    \addplot [semithick,mark=o,only marks,black!60]
      table [y=y1] {./3_modeling/err1_sw_dx_O2.dat};
    \addlegendentry{ $\epsilon_r$ (Rel. Error)}


    \addlegendimage{/pgfplots/refstyle=pl_PLECS_hidden}\addlegendentry{\color{white}PLECS}
    \addlegendimage{/pgfplots/refstyle=pl_Makw}\addlegendentry{Makowski}
    \addplot [semithick,mark=+,only marks,black!60]
      table [y=y1] {./3_modeling/err2_sw_dx_O2.dat};
    \addlegendentry{$\epsilon_r$ \color{white}(Rel. Error) }

    \addlegendimage{/pgfplots/refstyle=pl_PLECS_hidden}\addlegendentry{\color{white}PLECS}
    \addlegendimage{/pgfplots/refstyle=pl_Makw rec}\addlegendentry{*Mak. }
    \addplot [semithick,mark=x,only marks,black!60]
      table [y=y1] {./3_modeling/err3_sw_dx_O2.dat};
    \addlegendentry{$\epsilon_r$ \color{white}(Rel. Error)}

    \addlegendimage{/pgfplots/refstyle=pl_PLECS_hidden}\addlegendentry{\color{white}PLECS}
    \addlegendimage{/pgfplots/refstyle=pl:95Makw}\addlegendentry{95Makowski.}
    \addlegendimage{/pgfplots/refstyle=pl:Stey}\addlegendentry{Steyaert}


\end{axis}
\end{tikzpicture}




}
        \label{fig:exp_rscc_dc_node_100kHz}
    \end{subfigure}

    \begin{subfigure}{\textwidth}
        \parbox[c]{.03\linewidth}{\subcaption{}}
        \hspace{.02\linewidth}
        \parbox[c]{.95\linewidth}{
        \centering
        % This file was created by matlab2tikz.
%
%The latest updates can be retrieved from
%  http://www.mathworks.com/matlabcentral/fileexchange/22022-matlab2tikz-matlab2tikz
%where you can also make suggestions and rate matlab2tikz.
%

\begin{tikzpicture}
\pgfplotsset{
    width=9cm,
    height=2.5cm,
    scale only axis,
    ylabel near ticks,
    enlarge y limits={0.2},
    xlabel near ticks,
    ylabel near ticks,
}
\begin{axis}[%
axis x line*=bottom,
axis y line*=left,
xlabel= {duty cycle},
ylabel= {$r_{scc}~[\Omega]$},
yticklabel style={text width=2em,align=right}
]

    \addplot [semithick,mark=square,only marks]
      table[y=y4] {./3_modeling/rx_sw_dx_O2.dat};\label{pl_PLECS}
    \addplot [semithick,smooth,black]
      table [y=y2] {./3_modeling/rm1_sw_dx_O2.dat};\label{pl_MDL}
    \addplot [semithick,smooth,black!66]
      table [y=y2] {./3_modeling/rm2_sw_dx_O2.dat};
    \addplot [semithick,smooth,black!33]
      table [y=y2] {./3_modeling/rm3_sw_dx_O2.dat};
    
    \addplot [semithick,smooth,black,dashed]
      table [y=y2] {./3_modeling/rm1_ms_sw_dx_O2.dat};\label{pl_MDL}

\end{axis}

\begin{axis}[%
    axis y line*=right,
    axis x line=none,
    ylabel = {$\epsilon_r~[\%]$},
    yticklabel pos=right,
    yticklabel style={text width=2em,align=left},
    ]

    \addplot [semithick,mark=star,only marks,black]
    table [y=y4] {./3_modeling/err1_sw_dx_O2.dat};

    \addplot [semithick,mark=star,only marks,black!66]
    table [y=y4] {./3_modeling/err2_sw_dx_O2.dat};

    \addplot [semithick,mark=star,only marks,black!33]
    table [y=y4] {./3_modeling/err3_sw_dx_O2.dat};

\end{axis}
\end{tikzpicture}
}
        \label{fig:exp_rscc_dc_node_1MHz}
    \end{subfigure}

    \begin{subfigure}{\textwidth}
        \parbox[c]{.03\linewidth}{\subcaption{}}
        \hspace{.02\linewidth}
        \parbox[c]{.95\linewidth}{
        \centering
        % This file was created by matlab2tikz.
%
%The latest updates can be retrieved from
%  http://www.mathworks.com/matlabcentral/fileexchange/22022-matlab2tikz-matlab2tikz
%where you can also make suggestions and rate matlab2tikz.
%

\begin{tikzpicture}
\pgfplotsset{
    width=9cm,
    height=2.5cm,
    scale only axis,
    ylabel near ticks,
    enlarge y limits={0.2},
    xlabel near ticks,
    ylabel near ticks,
}
\begin{axis}[%
axis x line*=bottom,
axis y line*=left,
xlabel= {duty cycle},
ylabel= {$r_{scc}~[\Omega]$},
yticklabel style={text width=2em,align=right},
]

    \addplot [semithick,mark=square,only marks]
      table[y=y7] {./3_modeling/rx_sw_dx_O2.dat};\label{pl_PLECS}
    \addplot [semithick,smooth,black]
      table [y=y3] {./3_modeling/rm1_sw_dx_O2.dat};\label{pl_MDL}
    \addplot [semithick,smooth,black!66]
      table [y=y3] {./3_modeling/rm2_sw_dx_O2.dat};
    \addplot [semithick,smooth,black!33]
      table [y=y3] {./3_modeling/rm3_sw_dx_O2.dat};
    
    \addplot [semithick,smooth,black,dashed]
      table [y=y3] {./3_modeling/rm1_ms_sw_dx_O2.dat};\label{pl_MDL}

\end{axis}

\begin{axis}[%
axis y line*=right,
axis x line=none,
ylabel = {$\epsilon_r~[\%]$},
yticklabel pos=right,
yticklabel style={text width=2em,align=left},
]

    \addplot [semithick,mark=star,only marks,black]
    table [y=y7] {./3_modeling/err1_sw_dx_O2.dat};

    \addplot [semithick,mark=star,only marks,black!66]
    table [y=y7] {./3_modeling/err2_sw_dx_O2.dat};

    \addplot [semithick,mark=star,only marks,black!33]
    table [y=y7] {./3_modeling/err3_sw_dx_O2.dat};

\end{axis}
\end{tikzpicture}
}
        \label{fig:exp_rscc_dc_node_10MHz}
    \end{subfigure}

    \begin{subfigure}{\textwidth}
        \parbox[c]{.03\linewidth}{\subcaption{}}
        \hspace{.02\linewidth}
        \parbox[c]{.95\linewidth}{
        \centering
        % This file was created by matlab2tikz.
%
%The latest updates can be retrieved from
%  http://www.mathworks.com/matlabcentral/fileexchange/22022-matlab2tikz-matlab2tikz
%where you can also make suggestions and rate matlab2tikz.
%

\begin{tikzpicture}
\pgfplotsset{
    width=9cm,
    height=2.5cm,
    scale only axis,
    ylabel near ticks,
    enlarge y limits={0.2},
    xlabel near ticks,
    ylabel near ticks,
}
\begin{axis}[%
axis x line*=bottom,
axis y line*=left,
xlabel= {duty cycle},
ylabel= {$r_{scc}~[\Omega]$},
yticklabel style={text width=2em,align=right},
]

\addplot [semithick,mark=square,only marks,forget plot]
  table {./3_modeling/sim_rx_100MHz_O2.dat};

\addplot [semithick,solid,forget plot]
  table {./3_modeling/mdl_rx_100MHz_O2.dat};

%\addplot [semithick,dashed,forget plot]
%  table {./3_modeling/mdl_see_rx_100MHz_O1.dat};

\end{axis}

\begin{axis}[%
axis y line*=right,
axis x line=none,
ylabel = {$\epsilon_r~[\%]$},
yticklabel pos=right,
yticklabel style={text width=2em,align=left},
]

\addplot [semithick,mark=star,only marks,forget plot]
  table {./3_modeling/error_rx_100MHz_O2.dat};


\end{axis}
\end{tikzpicture}
}
        \label{fig:exp_rscc_dc_node_100MHz}
    \end{subfigure}

\caption{Equivalent Output Resistance ($r_{scc}$) from the \emph{dc}-node of the converter of Figure~\ref{fig:3_1_hscc_exp_b} plotted as function of the duty cycle. Experimental results (\emph{square marks}) compared with the model (\emph{solid line}) at different switching frequencies ($f_{sw}$): \emph{top to bottom}- $100kHz$, $1MHz$, $10MHz$ and $100MHz$.Plots are obtained for the different analytical $r_{scc}$ approximations (see~\ref{ch:an_apprx}): \emph{Black} - Original $u=2$ ,\emph{grey} - Makowski  $u=2.54$, \emph{light grey} - rectified Makowski $u=f(D)$. }
\label{fig:exp_rscc_dc_node}
\end{figure}

\begin{figure}[!h]
\centering
    \begin{subfigure}{0.45\textwidth}
        % This file was created by matlab2tikz.
%
%The latest updates can be retrieved from
%  http://www.mathworks.com/matlabcentral/fileexchange/22022-matlab2tikz-matlab2tikz
%where you can also make suggestions and rate matlab2tikz.
%

\begin{tikzpicture}
\pgfplotsset{
    width=4.5cm,
    height=3.25cm,
    scale only axis,
    ylabel near ticks,
    enlarge y limits={0.2},
    xlabel near ticks,
    ylabel near ticks,
    enlarge x limits={0.15},
}
\begin{loglogaxis}[
        %xlabel= {$f_{sw}[Hz] $},
        xticklabels={,,},
        ylabel= {$ r_{scc} ~ [\Omega] $} ,
        axis y line*=left,
        axis x line*=bottom,
        %xtick=\empty, ytick=\empty,
        %ytick = {0,.125,.25},
        %yticklabels={0,$v_{src}\frac{1}{3}$,$v_{src}\frac{2}{3}$,$v_{src}$},
        %xticklabels={0,$D \cdot T_{sw}$,$T_{sw}$ ,$2 T_{sw}$,$3 T_{sw} $},
        enlarge y limits={0.2},
        title={$D=10\%$},
        title style = {
            at= {(0.5,1.25)}},
        ]

\addplot [semithick,mark=square,only marks,white]
  table [y=y1]{./3_modeling/rx_sw_fsw_O1.dat};\label{pl_PLECS_hd}

\addplot [semithick,mark=square,only marks,black]
  table [y=y1]{./3_modeling/rx_sw_fsw_O1.dat};\label{pl_PLECS}

\addplot [semithick,smooth,black,mark=o]
  table [y=y1]{./3_modeling/rm1_sw_fsw_O1.dat};\label{pl_MDL}

\addplot [semithick,smooth,black,mark=+]
  table [y=y1]{./3_modeling/rm2_sw_fsw_O1.dat};\label{pl_Makow}

\addplot [semithick,smooth,black,mark=x]
  table [y=y1]{./3_modeling/rm3_sw_fsw_O1.dat};\label{pl_Makow_II}

\end{loglogaxis}

\begin{semilogxaxis}[%
    axis y line*=right,
    axis x line=none,
    %ylabel = {$\epsilon_r~[\%]$},
    yticklabel pos=right,
    yticklabel style={text width=2em,align=left},
    enlarge y limits={0.15},
    legend style={
                    legend columns = 3,
                    at={(0.5,0.95)},
                    anchor=south,
                    draw=none,
                    font=\tiny,
                    column sep=0.5ex,
                    legend cell align=left},
    ]

\addlegendimage{/pgfplots/refstyle=pl_PLECS}\addlegendentry{PLECS}
\addlegendimage{/pgfplots/refstyle=pl_MDL}\addlegendentry{This work}
\addplot [semithick,mark=o,only marks,black!60]
  table [y=y1]{./3_modeling/err1_sw_fsw_O1.dat};
  \addlegendentry{ $\epsilon_r$}
  
\addlegendimage{/pgfplots/refstyle=pl_PLECS_hd}\addlegendentry{\color{white}PLECS}
\addlegendimage{/pgfplots/refstyle=pl_Makow}\addlegendentry{Mak.}  
\addplot [semithick,mark=+,only marks,black!60]
  table [y=y1]{./3_modeling/err2_sw_fsw_O1.dat};
   \addlegendentry{ $\epsilon_r$}
   
   
\addlegendimage{/pgfplots/refstyle=pl_PLECS_hd}\addlegendentry{\color{white}PLECS}
\addlegendimage{/pgfplots/refstyle=pl_Makow_II}\addlegendentry{Mak. Rect.}  
\addplot [semithick,mark=x,only marks,black!60]
  table [y=y1]{./3_modeling/err3_sw_fsw_O1.dat};
\addlegendentry{ $\epsilon_r$}


\end{semilogxaxis}

\end{tikzpicture}

    \end{subfigure}
    \hfill
    \begin{subfigure}{0.45\textwidth}
        % This file was created by matlab2tikz.
%
%The latest updates can be retrieved from
%  http://www.mathworks.com/matlabcentral/fileexchange/22022-matlab2tikz-matlab2tikz
%where you can also make suggestions and rate matlab2tikz.
%

\begin{tikzpicture}
\pgfplotsset{
    width=4.5cm,
    height=3.25cm,
    scale only axis,
    ylabel near ticks,
    enlarge y limits={0.2},
    xlabel near ticks,
    ylabel near ticks,
    enlarge x limits={0.15},
}
\begin{loglogaxis}[
        xticklabels={,,},
        axis y line*=left,
        axis x line*=bottom,
        enlarge y limits={0.1},
        title={$D=23\%$}
        ]

\addplot [semithick,mark=square,only marks,black]
  table [y=y2]{./3_modeling/rx_sw_fsw_O1.dat};
\addplot [semithick,smooth,black,mark=o]
  table [y=y2]{./3_modeling/rm1_sw_fsw_O1.dat};
\addplot [semithick,smooth,black,mark=+]
  table [y=y2]{./3_modeling/rm2_sw_fsw_O1.dat};
\addplot [semithick,smooth,mark=x]
  table [y=y2]{./3_modeling/rm3_sw_fsw_O1.dat};


\end{loglogaxis}

\begin{semilogxaxis}[%
    axis y line*=right,
    axis x line=none,
    ylabel = {$\epsilon_r~[\%]$},
    yticklabel pos=right,
    yticklabel style={text width=2em,align=left},
    enlarge y limits={0.15},
    title={\color{white} $D=10\%$},
    title style = {
          at= {(0.5,1.25)}},
    ]
\addplot [semithick,mark=o,only marks,black!60]
  table [y=y2]{./3_modeling/err1_sw_fsw_O1.dat};

\addplot [semithick,mark=+,only marks,black!60]
  table [y=y2]{./3_modeling/err2_sw_fsw_O1.dat};

\addplot [semithick,mark=x,only marks,black!60]
  table [y=y2]{./3_modeling/err3_sw_fsw_O1.dat};

\end{semilogxaxis}

\end{tikzpicture}

    \end{subfigure}

    \begin{subfigure}{0.45\textwidth}
        % This file was created by matlab2tikz.
%
%The latest updates can be retrieved from
%  http://www.mathworks.com/matlabcentral/fileexchange/22022-matlab2tikz-matlab2tikz
%where you can also make suggestions and rate matlab2tikz.
%

\begin{tikzpicture}
\pgfplotsset{
    width=4.5cm,
    height=3.25cm,
    scale only axis,
    ylabel near ticks,
    enlarge y limits={0.2},
    xlabel near ticks,
    ylabel near ticks,
    enlarge x limits={0.15},
}
\begin{loglogaxis}[
        %xlabel= {$f_{sw}[Hz] $},
        xticklabels={,,},
        ylabel= {$ r_{scc} ~ [\Omega] $} ,
        axis y line*=left,
        axis x line*=bottom,
        %xtick=\empty, ytick=\empty,
        %ytick = {0,.125,.25},
        %yticklabels={0,$v_{src}\frac{1}{3}$,$v_{src}\frac{2}{3}$,$v_{src}$},
        %xticklabels={0,$D \cdot T_{sw}$,$T_{sw}$ ,$2 T_{sw}$,$3 T_{sw} $},
        enlarge y limits={0.1},
        title={$D=50\%$}
        ]

\addplot [semithick,mark=square,only marks,black]
  table [y=y4]{./3_modeling/rx_sw_fsw_O1.dat};
\addplot [semithick,smooth,black]
  table [y=y4]{./3_modeling/rm1_sw_fsw_O1.dat};
\addplot [semithick,smooth,black!66]
  table [y=y4]{./3_modeling/rm2_sw_fsw_O1.dat};
\addplot [semithick,smooth,black!33]
  table [y=y4]{./3_modeling/rm3_sw_fsw_O1.dat};


\end{loglogaxis}

\begin{semilogxaxis}[%
axis y line*=right,
axis x line=none,
%ylabel = {$\epsilon_r~[\%]$},
yticklabel pos=right,
yticklabel style={text width=2em,align=left},
enlarge y limits={0.15}
]

\addplot [semithick,mark=star,only marks,black]
  table [y=y4]{./3_modeling/err1_sw_fsw_O1.dat};
  
\addplot [semithick,mark=star,only marks,black!66]
  table [y=y1]{./3_modeling/err2_sw_fsw_O1.dat};

\addplot [semithick,mark=star,only marks,black!33]
  table [y=y1]{./3_modeling/err3_sw_fsw_O1.dat};

\end{semilogxaxis}

\end{tikzpicture}

        %}
    \end{subfigure}
    \hfill
    \begin{subfigure}{0.45\textwidth}
        % This file was created by matlab2tikz.
%
%The latest updates can be retrieved from
%  http://www.mathworks.com/matlabcentral/fileexchange/22022-matlab2tikz-matlab2tikz
%where you can also make suggestions and rate matlab2tikz.
%

\begin{tikzpicture}
\pgfplotsset{
    width=4.5cm,
    height=3.25cm,
    scale only axis,
    ylabel near ticks,
    enlarge y limits={0.2},
    xlabel near ticks,
    ylabel near ticks,
    enlarge x limits={0.15},
}
\begin{loglogaxis}[
        %xlabel= {$f_{sw}[Hz] $},
        xticklabels={,,},
        %ylabel= {$ [\Omega] $} ,
        axis y line*=left,
        axis x line*=bottom,
        %xtick=\empty, ytick=\empty,
        %ytick = {0,.125,.25},
        %yticklabels={0,$v_{src}\frac{1}{3}$,$v_{src}\frac{2}{3}$,$v_{src}$},
        %xticklabels={0,$D \cdot T_{sw}$,$T_{sw}$ ,$2 T_{sw}$,$3 T_{sw} $},
        enlarge y limits={0.1},
        title={$D=77\%$}
        ]

\addplot [semithick,mark=square,only marks,black]
  table [y=y6]{./3_modeling/rx_sw_fsw_O1.dat};
\addplot [semithick,smooth,black,,mark=o]
  table [y=y6]{./3_modeling/rm1_sw_fsw_O1.dat};
\addplot [semithick,smooth,black,mark=+]
  table [y=y6]{./3_modeling/rm2_sw_fsw_O1.dat};
\addplot [semithick,smooth,black,mark=x]
  table [y=y6]{./3_modeling/rm3_sw_fsw_O1.dat};

\end{loglogaxis}

\begin{semilogxaxis}[%
axis y line*=right,
axis x line=none,
ylabel = {$\epsilon_r~[\%]$},
yticklabel pos=right,
yticklabel style={text width=2em,align=left},
enlarge y limits={0.15}
]
\addplot [semithick,mark=o,only marks,black!60]
  table [y=y6]{./3_modeling/err1_sw_fsw_O1.dat};
\addplot [semithick,mark=+,only marks,black!60]
  table [y=y6]{./3_modeling/err2_sw_fsw_O1.dat};
\addplot [semithick,mark=x,only marks,black!60]
  table [y=y6]{./3_modeling/err3_sw_fsw_O1.dat};

\end{semilogxaxis}

\end{tikzpicture}

    \end{subfigure}

        \begin{subfigure}{0.45\textwidth}
        % This file was created by matlab2tikz.
%
%The latest updates can be retrieved from
%  http://www.mathworks.com/matlabcentral/fileexchange/22022-matlab2tikz-matlab2tikz
%where you can also make suggestions and rate matlab2tikz.
%

\begin{tikzpicture}
\pgfplotsset{
    width=4.5cm,
    height=3.25cm,
    scale only axis,
    ylabel near ticks,
    enlarge y limits={0.2},
    xlabel near ticks,
    ylabel near ticks,
    enlarge x limits={0.15},
}
\begin{loglogaxis}[
        xlabel= {$f_{sw}[Hz] $},
        ylabel= {$ r_{scc} ~ [\Omega] $} ,
        axis y line*=left,
        axis x line*=bottom,
        %xtick=\empty, ytick=\empty,
        %ytick = {0,.125,.25},
        %yticklabels={0,$v_{src}\frac{1}{3}$,$v_{src}\frac{2}{3}$,$v_{src}$},
        %xticklabels={0,$D \cdot T_{sw}$,$T_{sw}$ ,$2 T_{sw}$,$3 T_{sw} $},
        enlarge y limits={0.1},
        title={$D=90\%$}
        ]

\addplot [semithick,mark=square,only marks,black]
  table [y=y7]{./3_modeling/rx_sw_fsw_O1.dat};
\addplot [semithick,smooth,black]
  table [y=y7]{./3_modeling/rm1_sw_fsw_O1.dat};
\addplot [semithick,smooth,black!66]
  table [y=y7]{./3_modeling/rm2_sw_fsw_O1.dat};
\addplot [semithick,smooth,black!33]
  table [y=y7]{./3_modeling/rm3_sw_fsw_O1.dat};
  
\end{loglogaxis}

\begin{semilogxaxis}[%
axis y line*=right,
axis x line=none,
%ylabel = {$\epsilon_r~[\%]$},
yticklabel pos=right,
yticklabel style={text width=2em,align=left},
enlarge y limits={0.15}
]
\addplot [semithick,mark=star,only marks,black]
  table [y=y7]{./3_modeling/err1_sw_fsw_O1.dat};
  
\addplot [semithick,mark=star,only marks,black!66]
  table [y=y1]{./3_modeling/err2_sw_fsw_O1.dat};

\addplot [semithick,mark=star,only marks,black!33]
  table [y=y1]{./3_modeling/err3_sw_fsw_O1.dat};
\end{semilogxaxis}

\end{tikzpicture}

    \end{subfigure}
    \hfill
    \begin{subfigure}{0.45\textwidth}
        % This file was created by matlab2tikz.
%
%The latest updates can be retrieved from
%  http://www.mathworks.com/matlabcentral/fileexchange/22022-matlab2tikz-matlab2tikz
%where you can also make suggestions and rate matlab2tikz.
%

\begin{tikzpicture}
\pgfplotsset{
    width=4.5cm,
    height=3.25cm,
    scale only axis,
    ylabel near ticks,
    enlarge y limits={0.2},
    xlabel near ticks,
    ylabel near ticks,
    enlarge x limits={0.15},
    legend style={
                legend columns = 1,
                anchor=north east,
                %at={(1,1)},
                draw=none,
                font=\tiny},
}
\begin{loglogaxis}[
       xlabel= {$f_{sw}[Hz] $},
        %ylabel= {$ [\Omega] $} ,
        axis y line*=left,
        axis x line*=bottom,
        %xtick=\empty, ytick=\empty,
        %ytick = {0,.125,.25},
        %yticklabels={0,$v_{src}\frac{1}{3}$,$v_{src}\frac{2}{3}$,$v_{src}$},
        %xticklabels={0,$D \cdot T_{sw}$,$T_{sw}$ ,$2 T_{sw}$,$3 T_{sw} $},
        enlarge y limits={0.1},
        ]

%\addplot [semithick,mark=square,only marks,black]
%  table [y=y7]{./3_modeling/r_scc_O1.dat};
\addplot [semithick,smooth,mark=square,black]
  table [y=y1]{./3_modeling/rm1_sw_fsw_O1.dat};\addlegendentry{ $D = 10\%$}

%\addplot [semithick,mark=o,only marks,black!75]
% table [y=y3]{./3_modeling/r_scc_O1.dat};
\addplot [semithick,smooth,mark=o]
  table [y=y2]{./3_modeling/rm1_sw_fsw_O1.dat};\addlegendentry{ $D = 23\%$}

%\addplot [semithick,mark=square,only marks]
%  table [y=y4]{./3_modeling/r_scc_O1.dat};
\addplot [semithick,smooth,mark=x]
  table [y=y4]{./3_modeling/rm1_sw_fsw_O1.dat};\addlegendentry{ $D = 50\%$}

%\addplot [semithick,mark=star,only marks]
%  table [y=y5]{./3_modeling/r_scc_O1.dat};
\addplot [semithick,smooth,mark=+]
  table [y=y6]{./3_modeling/rm1_sw_fsw_O1.dat};\addlegendentry{ $D = 77\%$}
%
%\addplot [semithick,mark=diamond,only marks]
%  table[y=y7] {./3_modeling/r_scc_O1.dat};
\addplot [semithick,smooth,mark=diamond]
  table [y=y7]{./3_modeling/rm1_sw_fsw_O1.dat};\addlegendentry{ $D = 90\%$}

\end{loglogaxis}

\end{tikzpicture}

    \end{subfigure}

\caption{Equivalent Output Resistance ($r_{scc}$) from the \emph{pwm}-node of the converter of Figure~\ref{fig:3_1_hscc_exp_a} as function of the switching frequency ($f_{sw}$). \emph{Plots 1-5 top-to-bottom}-  Experimental points ($\Box$ ) compared with the model (\emph{solid line}) and the absolute relative error ($star$)  at different duty cycles ($D$): - $10\%$, $23\%$, $50\%$, $63\%$ and $90\%$. \emph{Bottom-right}- Parametric plot with all the curves. Plots are obtained for the different analytical $r_{scc}$ approximations (see~\ref{ch:an_apprx}): \emph{Black} - Original $u=2$ ,\emph{grey} - Makowski  $u=2.54$, \emph{light grey} - rectified Makowski $u=f(D)$. }\label{fig:exp_rscc_pwm_node_fsw}
\end{figure}


\begin{landscape}
\thispagestyle{empty}
\begin{figure}[!h]
\centering
    \begin{subfigure}{0.45\textwidth}
        % This file was created by matlab2tikz.
%
%The latest updates can be retrieved from
%  http://www.mathworks.com/matlabcentral/fileexchange/22022-matlab2tikz-matlab2tikz
%where you can also make suggestions and rate matlab2tikz.
%

\begin{tikzpicture}
\pgfplotsset{
    width=4.5cm,
    height=3.25cm,
    scale only axis,
    ylabel near ticks,
    enlarge y limits={0.2},
    xlabel near ticks,
    ylabel near ticks,
    enlarge x limits={0.15},
}
\begin{loglogaxis}[
        %xlabel= {$f_{sw}[Hz] $},
        xticklabels={,,},
        ylabel= {$ r_{scc} ~ [\Omega] $} ,
        axis y line*=left,
        axis x line*=bottom,
        %xtick=\empty, ytick=\empty,
        %ytick = {0,.125,.25},
        %yticklabels={0,$v_{src}\frac{1}{3}$,$v_{src}\frac{2}{3}$,$v_{src}$},
        %xticklabels={0,$D \cdot T_{sw}$,$T_{sw}$ ,$2 T_{sw}$,$3 T_{sw} $},
        enlarge y limits={0.2},
        title={$D=10\%$},
        title style = {
            at= {(0.5,1.25)}},
        ]

\addplot [semithick,mark=square,only marks,white]
  table [y=y1]{./3_modeling/rx_sw_fsw_O1.dat};\label{pl_PLECS_hd}

\addplot [semithick,mark=square,only marks,black]
  table [y=y1]{./3_modeling/rx_sw_fsw_O1.dat};\label{pl_PLECS}

\addplot [semithick,smooth,black,mark=o]
  table [y=y1]{./3_modeling/rm1_sw_fsw_O1.dat};\label{pl_MDL}

\addplot [semithick,smooth,black,mark=+]
  table [y=y1]{./3_modeling/rm2_sw_fsw_O1.dat};\label{pl_Makow}

\addplot [semithick,smooth,black,mark=x]
  table [y=y1]{./3_modeling/rm3_sw_fsw_O1.dat};\label{pl_Makow_II}

\end{loglogaxis}

\begin{semilogxaxis}[%
    axis y line*=right,
    axis x line=none,
    %ylabel = {$\epsilon_r~[\%]$},
    yticklabel pos=right,
    yticklabel style={text width=2em,align=left},
    enlarge y limits={0.15},
    legend style={
                    legend columns = 3,
                    at={(0.5,0.95)},
                    anchor=south,
                    draw=none,
                    font=\tiny,
                    column sep=0.5ex,
                    legend cell align=left},
    ]

\addlegendimage{/pgfplots/refstyle=pl_PLECS}\addlegendentry{PLECS}
\addlegendimage{/pgfplots/refstyle=pl_MDL}\addlegendentry{This work}
\addplot [semithick,mark=o,only marks,black!60]
  table [y=y1]{./3_modeling/err1_sw_fsw_O1.dat};
  \addlegendentry{ $\epsilon_r$}
  
\addlegendimage{/pgfplots/refstyle=pl_PLECS_hd}\addlegendentry{\color{white}PLECS}
\addlegendimage{/pgfplots/refstyle=pl_Makow}\addlegendentry{Mak.}  
\addplot [semithick,mark=+,only marks,black!60]
  table [y=y1]{./3_modeling/err2_sw_fsw_O1.dat};
   \addlegendentry{ $\epsilon_r$}
   
   
\addlegendimage{/pgfplots/refstyle=pl_PLECS_hd}\addlegendentry{\color{white}PLECS}
\addlegendimage{/pgfplots/refstyle=pl_Makow_II}\addlegendentry{Mak. Rect.}  
\addplot [semithick,mark=x,only marks,black!60]
  table [y=y1]{./3_modeling/err3_sw_fsw_O1.dat};
\addlegendentry{ $\epsilon_r$}


\end{semilogxaxis}

\end{tikzpicture}

    \end{subfigure}
    \hfill
    \begin{subfigure}{0.45\textwidth}
        % This file was created by matlab2tikz.
%
%The latest updates can be retrieved from
%  http://www.mathworks.com/matlabcentral/fileexchange/22022-matlab2tikz-matlab2tikz
%where you can also make suggestions and rate matlab2tikz.
%

\begin{tikzpicture}
\pgfplotsset{
    width=4.5cm,
    height=3.25cm,
    scale only axis,
    ylabel near ticks,
    enlarge y limits={0.2},
    xlabel near ticks,
    ylabel near ticks,
    enlarge x limits={0.15},
}
\begin{loglogaxis}[
        xticklabels={,,},
        axis y line*=left,
        axis x line*=bottom,
        enlarge y limits={0.1},
        title={$D=23\%$}
        ]

\addplot [semithick,mark=square,only marks,black]
  table [y=y2]{./3_modeling/rx_sw_fsw_O1.dat};
\addplot [semithick,smooth,black,mark=o]
  table [y=y2]{./3_modeling/rm1_sw_fsw_O1.dat};
\addplot [semithick,smooth,black,mark=+]
  table [y=y2]{./3_modeling/rm2_sw_fsw_O1.dat};
\addplot [semithick,smooth,mark=x]
  table [y=y2]{./3_modeling/rm3_sw_fsw_O1.dat};


\end{loglogaxis}

\begin{semilogxaxis}[%
    axis y line*=right,
    axis x line=none,
    ylabel = {$\epsilon_r~[\%]$},
    yticklabel pos=right,
    yticklabel style={text width=2em,align=left},
    enlarge y limits={0.15},
    title={\color{white} $D=10\%$},
    title style = {
          at= {(0.5,1.25)}},
    ]
\addplot [semithick,mark=o,only marks,black!60]
  table [y=y2]{./3_modeling/err1_sw_fsw_O1.dat};

\addplot [semithick,mark=+,only marks,black!60]
  table [y=y2]{./3_modeling/err2_sw_fsw_O1.dat};

\addplot [semithick,mark=x,only marks,black!60]
  table [y=y2]{./3_modeling/err3_sw_fsw_O1.dat};

\end{semilogxaxis}

\end{tikzpicture}

    \end{subfigure}
    \hfill
    \begin{subfigure}{0.45\textwidth}
        % This file was created by matlab2tikz.
%
%The latest updates can be retrieved from
%  http://www.mathworks.com/matlabcentral/fileexchange/22022-matlab2tikz-matlab2tikz
%where you can also make suggestions and rate matlab2tikz.
%

\begin{tikzpicture}
\pgfplotsset{
    width=4.5cm,
    height=3.25cm,
    scale only axis,
    ylabel near ticks,
    enlarge y limits={0.2},
    xlabel near ticks,
    ylabel near ticks,
    enlarge x limits={0.15},
}
\begin{loglogaxis}[
        %xlabel= {$f_{sw}[Hz] $},
        xticklabels={,,},
        ylabel= {$ r_{scc} ~ [\Omega] $} ,
        axis y line*=left,
        axis x line*=bottom,
        %xtick=\empty, ytick=\empty,
        %ytick = {0,.125,.25},
        %yticklabels={0,$v_{src}\frac{1}{3}$,$v_{src}\frac{2}{3}$,$v_{src}$},
        %xticklabels={0,$D \cdot T_{sw}$,$T_{sw}$ ,$2 T_{sw}$,$3 T_{sw} $},
        enlarge y limits={0.1},
        title={$D=50\%$}
        ]

\addplot [semithick,mark=square,only marks,black]
  table [y=y4]{./3_modeling/rx_sw_fsw_O1.dat};
\addplot [semithick,smooth,black]
  table [y=y4]{./3_modeling/rm1_sw_fsw_O1.dat};
\addplot [semithick,smooth,black!66]
  table [y=y4]{./3_modeling/rm2_sw_fsw_O1.dat};
\addplot [semithick,smooth,black!33]
  table [y=y4]{./3_modeling/rm3_sw_fsw_O1.dat};


\end{loglogaxis}

\begin{semilogxaxis}[%
axis y line*=right,
axis x line=none,
%ylabel = {$\epsilon_r~[\%]$},
yticklabel pos=right,
yticklabel style={text width=2em,align=left},
enlarge y limits={0.15}
]

\addplot [semithick,mark=star,only marks,black]
  table [y=y4]{./3_modeling/err1_sw_fsw_O1.dat};
  
\addplot [semithick,mark=star,only marks,black!66]
  table [y=y1]{./3_modeling/err2_sw_fsw_O1.dat};

\addplot [semithick,mark=star,only marks,black!33]
  table [y=y1]{./3_modeling/err3_sw_fsw_O1.dat};

\end{semilogxaxis}

\end{tikzpicture}

        %}
    \end{subfigure}
    
    \begin{subfigure}{0.45\textwidth}
        % This file was created by matlab2tikz.
%
%The latest updates can be retrieved from
%  http://www.mathworks.com/matlabcentral/fileexchange/22022-matlab2tikz-matlab2tikz
%where you can also make suggestions and rate matlab2tikz.
%

\begin{tikzpicture}
\pgfplotsset{
    width=4.5cm,
    height=3.25cm,
    scale only axis,
    ylabel near ticks,
    enlarge y limits={0.2},
    xlabel near ticks,
    ylabel near ticks,
    enlarge x limits={0.15},
}
\begin{loglogaxis}[
        %xlabel= {$f_{sw}[Hz] $},
        xticklabels={,,},
        %ylabel= {$ [\Omega] $} ,
        axis y line*=left,
        axis x line*=bottom,
        %xtick=\empty, ytick=\empty,
        %ytick = {0,.125,.25},
        %yticklabels={0,$v_{src}\frac{1}{3}$,$v_{src}\frac{2}{3}$,$v_{src}$},
        %xticklabels={0,$D \cdot T_{sw}$,$T_{sw}$ ,$2 T_{sw}$,$3 T_{sw} $},
        enlarge y limits={0.1},
        title={$D=77\%$}
        ]

\addplot [semithick,mark=square,only marks,black]
  table [y=y6]{./3_modeling/rx_sw_fsw_O1.dat};
\addplot [semithick,smooth,black,,mark=o]
  table [y=y6]{./3_modeling/rm1_sw_fsw_O1.dat};
\addplot [semithick,smooth,black,mark=+]
  table [y=y6]{./3_modeling/rm2_sw_fsw_O1.dat};
\addplot [semithick,smooth,black,mark=x]
  table [y=y6]{./3_modeling/rm3_sw_fsw_O1.dat};

\end{loglogaxis}

\begin{semilogxaxis}[%
axis y line*=right,
axis x line=none,
ylabel = {$\epsilon_r~[\%]$},
yticklabel pos=right,
yticklabel style={text width=2em,align=left},
enlarge y limits={0.15}
]
\addplot [semithick,mark=o,only marks,black!60]
  table [y=y6]{./3_modeling/err1_sw_fsw_O1.dat};
\addplot [semithick,mark=+,only marks,black!60]
  table [y=y6]{./3_modeling/err2_sw_fsw_O1.dat};
\addplot [semithick,mark=x,only marks,black!60]
  table [y=y6]{./3_modeling/err3_sw_fsw_O1.dat};

\end{semilogxaxis}

\end{tikzpicture}

    \end{subfigure}
    \hfill
    \begin{subfigure}{0.45\textwidth}
       % This file was created by matlab2tikz.
%
%The latest updates can be retrieved from
%  http://www.mathworks.com/matlabcentral/fileexchange/22022-matlab2tikz-matlab2tikz
%where you can also make suggestions and rate matlab2tikz.
%

\begin{tikzpicture}
\pgfplotsset{
    width=4.5cm,
    height=3.25cm,
    scale only axis,
    ylabel near ticks,
    enlarge y limits={0.2},
    xlabel near ticks,
    ylabel near ticks,
    enlarge x limits={0.15},
}
\begin{loglogaxis}[
        xlabel= {$f_{sw}[Hz] $},
        ylabel= {$ r_{scc} ~ [\Omega] $} ,
        axis y line*=left,
        axis x line*=bottom,
        %xtick=\empty, ytick=\empty,
        %ytick = {0,.125,.25},
        %yticklabels={0,$v_{src}\frac{1}{3}$,$v_{src}\frac{2}{3}$,$v_{src}$},
        %xticklabels={0,$D \cdot T_{sw}$,$T_{sw}$ ,$2 T_{sw}$,$3 T_{sw} $},
        enlarge y limits={0.1},
        title={$D=90\%$}
        ]

\addplot [semithick,mark=square,only marks,black]
  table [y=y7]{./3_modeling/rx_sw_fsw_O1.dat};
\addplot [semithick,smooth,black]
  table [y=y7]{./3_modeling/rm1_sw_fsw_O1.dat};
\addplot [semithick,smooth,black!66]
  table [y=y7]{./3_modeling/rm2_sw_fsw_O1.dat};
\addplot [semithick,smooth,black!33]
  table [y=y7]{./3_modeling/rm3_sw_fsw_O1.dat};
  
\end{loglogaxis}

\begin{semilogxaxis}[%
axis y line*=right,
axis x line=none,
%ylabel = {$\epsilon_r~[\%]$},
yticklabel pos=right,
yticklabel style={text width=2em,align=left},
enlarge y limits={0.15}
]
\addplot [semithick,mark=star,only marks,black]
  table [y=y7]{./3_modeling/err1_sw_fsw_O1.dat};
  
\addplot [semithick,mark=star,only marks,black!66]
  table [y=y1]{./3_modeling/err2_sw_fsw_O1.dat};

\addplot [semithick,mark=star,only marks,black!33]
  table [y=y1]{./3_modeling/err3_sw_fsw_O1.dat};
\end{semilogxaxis}

\end{tikzpicture}

    \end{subfigure}
    \hfill
    \begin{subfigure}{0.45\textwidth}
        % This file was created by matlab2tikz.
%
%The latest updates can be retrieved from
%  http://www.mathworks.com/matlabcentral/fileexchange/22022-matlab2tikz-matlab2tikz
%where you can also make suggestions and rate matlab2tikz.
%

\begin{tikzpicture}
\pgfplotsset{
    width=4.5cm,
    height=3.25cm,
    scale only axis,
    ylabel near ticks,
    enlarge y limits={0.2},
    xlabel near ticks,
    ylabel near ticks,
    enlarge x limits={0.15},
    legend style={
                legend columns = 1,
                anchor=north east,
                %at={(1,1)},
                draw=none,
                font=\tiny},
}
\begin{loglogaxis}[
       xlabel= {$f_{sw}[Hz] $},
        %ylabel= {$ [\Omega] $} ,
        axis y line*=left,
        axis x line*=bottom,
        %xtick=\empty, ytick=\empty,
        %ytick = {0,.125,.25},
        %yticklabels={0,$v_{src}\frac{1}{3}$,$v_{src}\frac{2}{3}$,$v_{src}$},
        %xticklabels={0,$D \cdot T_{sw}$,$T_{sw}$ ,$2 T_{sw}$,$3 T_{sw} $},
        enlarge y limits={0.1},
        ]

%\addplot [semithick,mark=square,only marks,black]
%  table [y=y7]{./3_modeling/r_scc_O1.dat};
\addplot [semithick,smooth,mark=square,black]
  table [y=y1]{./3_modeling/rm1_sw_fsw_O1.dat};\addlegendentry{ $D = 10\%$}

%\addplot [semithick,mark=o,only marks,black!75]
% table [y=y3]{./3_modeling/r_scc_O1.dat};
\addplot [semithick,smooth,mark=o]
  table [y=y2]{./3_modeling/rm1_sw_fsw_O1.dat};\addlegendentry{ $D = 23\%$}

%\addplot [semithick,mark=square,only marks]
%  table [y=y4]{./3_modeling/r_scc_O1.dat};
\addplot [semithick,smooth,mark=x]
  table [y=y4]{./3_modeling/rm1_sw_fsw_O1.dat};\addlegendentry{ $D = 50\%$}

%\addplot [semithick,mark=star,only marks]
%  table [y=y5]{./3_modeling/r_scc_O1.dat};
\addplot [semithick,smooth,mark=+]
  table [y=y6]{./3_modeling/rm1_sw_fsw_O1.dat};\addlegendentry{ $D = 77\%$}
%
%\addplot [semithick,mark=diamond,only marks]
%  table[y=y7] {./3_modeling/r_scc_O1.dat};
\addplot [semithick,smooth,mark=diamond]
  table [y=y7]{./3_modeling/rm1_sw_fsw_O1.dat};\addlegendentry{ $D = 90\%$}

\end{loglogaxis}

\end{tikzpicture}

    \end{subfigure}
    
\caption{Equivalent Output Resistance ($r_{scc}$) from the \emph{pwm}-node of the converter of Figure~\ref{fig:3_1_hscc_exp_a} as function of the switching frequency ($f_{sw}$). \emph{plots 1-5 top-to-bottom}-  Experimental points ($\Box$ ) compared with the model (\emph{solid line}) and the absolute relative error ($star$)  at different duty cycles ($D$): - $10\%$, $23\%$, $50\%$, $63\%$ and $90\%$. \emph{bottom-right}- Parametric plot with all the curves.   }
\label{fig:exp_rscc_pwm_node}
\end{figure}
\end{landscape}


