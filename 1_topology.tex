\chapter{Miniaturization of LED drivers}
%
%The challenges in powering LED loads are so relevant that have an impact in functionalities and design of the future \emph{Solid-State-Lighting} (SSL) products. So much, that the user adoption of such a beneficial technology by is far slower than comparable disruptive technologies~\cite{11Voger}. In a part, that could be attributed due to the difficulties in achieving the high miniaturization and performance necessary in the LED drivers, at low cost, in order to outcompete the cheaper old technologies.

The LED driver is the circuit that transforms raw energy from a power source and properly delivers it to the LED load. From the power management standpoint power a LED load is a trivial task, however the different requirements of SSL products make the design of them a complex task. Initially the main driving forces in the driver designs were: manufacturing cost, power quality, light quality. Reducing manufacturing costs can enable to decrease the lamp prices to the entry point for the consumers, while fulfilling the power quality fixed by the legislation and keeping the adequate light quality to do not perturb the human eye~\cite{10Wilkins}.

More recently two other factors are becoming more relevant in the driver: miniaturization and \emph{smartability}\footnote{Provide the abilities/ functionalities to a device in order to be smart}. Currently the volume of the drivers is limited by the old lamp shapes in order to provide retrofit solutions. Although, in many cases the shape and look of the lamps has been modified in order to relax the requirements in terms of miniaturization for the driver. Therefore further the reduction of the driver volume will probably enable higher freedom in the lamp design. On the other hand, the future connected lamps~\cite{14Harbers} (or smart lamps) will require control and connectivity, what challenges the driver to provide multiple color channels and power management for the added control circuitry such as sensors, actuators, communication interfaces and $\mu$Controllers.

Looking at the current designs, we can say that the initial driving forces in driver design, cost, power and light quality, found effective solutions based on discrete components. However discrete drivers cannot easily satisfy the design in terms of miniaturization and \emph{smartability}. Literally, the implications of these two concepts have opposing consequences, providing smart functions to the driver requires to add more components, therefore more volume and costs; while miniaturization requires to make the driver smaller. Actually, such dilemma has its analogy in the mobile phone industry. After the first generations of mobile phones, the requirements in terms of functionalities for the current smart-phones could had only been satisfied by a \emph{System-On-Chip} (SoC), where the maximum number of functionalities were integrated in a single IC package. Based on smart phone analogy, this dissertation brings the research for the future highly miniaturized LED drivers in the context of \emph{Power Systems on-Chip/in-Package}(PSoC/PSiP), where miniaturization and integration of functionalities can be easily achieved. Settling the goal in providing new driver circuits that that are more suitable for integration, than the currently available. And with the vision that integrable drivers will help in further development of more functionalities within the same package.

This chapter starts with an overview of the LED characteristics as a load to give an understanding of the necessary requirements of a LED driver. Subsequently, the current three driver technologies are studied: Linear, Switched inductor converters and Switched capacitor converters. An state-of-the-art for each technology will be provided in order to construct a rational of the technology toward miniaturization, and presenting the facts why switched capacitor converters has been the selected technology in this dissertation.

%Therefore  will be thoroughly studied, since constitute the central conversion technology selected for this dissertation.


\section{The LED as a load}
\label{sc:LED_load}
A LED is as its acronym stands for a \emph{Light Emitting Diode}. Therefore a LED is a non-linear load with the well-known \emph{voltage-current} ($v-i$) curve of a diode shown in Figure~\ref{fig:led_I-V}. For voltages below the \emph{forward voltage} ($v_{f}$), practically no current flows through it and the LED behaves as an open circuit. For voltages above $v_{f}$ the curve becomes very steep and the current increases dramatically with respect to the voltage, thus the LED behaves similar to a short circuit. The LED has to be supplied at an specific point $P$ in order to provide a desired light output as shown in Figure~\ref{fig:led_I-V}, depending on the bias current light colour and intensity will vary. Due to the steepness in the $v-i$ curve, the practical way to bias a LED is supplying it by a \emph{dc}-current. Owning to the fact that the bast majority of energy sources supply voltage in order two properly supply an LED  it is necessary to select a circuits that converts voltage to current.

\begin{figure}[!h]
\centering
\begin{circuitikz}[american voltages]

    \draw (0,2.5) to[leD*,v=$v$,i=$i$,*-*] (3
    ,2.5) ;

\begin{scope}[xshift=4cm, domain=0:6]
    \draw [->] (0,0) -- (5.5,0) node[anchor=west]{$v$};
    \draw [->] (0,0) -- (0,5.5) node[anchor=east]{$i$};

    %Mark Vth
    \draw (2,2pt) -- (2,-5pt) node[anchor=north] {$v_{f}$};

    %Mark Vthmin and Vt_max
    \def\dvf{0.25}
    \def\dvftop{0.45}

    \draw (1.5,2pt) -- (1.5,-5pt) node[anchor=north] {};
    \draw (2.5,2pt) -- (2.5,-5pt) node[anchor=north] {};
    \draw[dotted] (1.5,0) -- (1.5,\dvftop);
    \draw[dotted] (2.5,0) -- (2.5,\dvftop);
    %Mark Delta Vf
    \draw[pil,>-< ,dotted] (1.25,\dvf) -- (2.75,\dvf);
    \draw (1,\dvf) -- (1.5,\dvf) node[anchor=south east] {$\Delta v_f$};


    %Draw ideal plot
    \draw[thick] (0,0) -- (2,0) -- (4.5,5);

    %Draw lower limit
    \draw[dashed] (0,0) -- (1.5,0) -- (4,5);
    %Draw higher limit
    \draw[dashed] (0,0) -- (2.5,0) -- (5,5);

    %Draw bias point projection
    \draw[dotted] (4,4) -- (4,-0);
    \draw (4,2pt) -- (4,-5pt) node[anchor=north]  {$v_{bias}$};

    \draw[dotted] (4,4) -- (-0,4);
    \draw (2pt,4) -- (-5pt,4) node[anchor=east]  {$i_{bias}$};

    %Draw bias point
    \filldraw (4,4) circle(2pt) node[anchor=south west] {$P$};

    %Draw bias point variations
    \draw (4.5,4)  node[anchor=south west] {};
    \draw (3.5,4)  node[anchor=south west] {};

    %\draw[loosely dotted] (4.5,4) -- (4.5,-0);
    %\draw[loosely dotted] (3.5,4) -- (3.5,-0);
    \draw (4,2pt) -- (4,-5pt) node[anchor=north]  {$v_{bias}$};
    %Mark Vthmin and Vt_max
    \def\bpmi{4.5}
    \def\bpMa{3.5}
    %\draw (1.5,2pt) -- (1.5,-5pt) node[anchor=north] {};
    %\draw (2.5,2pt) -- (2.5,-5pt) node[anchor=north] {};
    %\draw[dotted] (1.5,0) -- (1.5,\dvftop);
    %\draw[dotted] (2.5,0) -- (2.5,\dvftop);
    %Mark Delta Vf
    %\draw[pil,>-< ,dotted] (3.25,\dvf) -- (4.75,\dvf);
    %\draw (4.5,\dvf) -- (5,\dvf) node[anchor=south west] {$\Delta V_{bias}$};

    %Mark Vthmin and Vt_max
    \def\dvf{4}
    %\draw (1.5,2pt) -- (1.5,-5pt) node[anchor=north] {};
    %\draw (2.5,2pt) -- (2.5,-5pt) node[anchor=north] {};
    %\draw[dotted] (1.5,0) -- (1.5,\dvftop);
    %\draw[dotted] (2.5,0) -- (2.5,\dvftop);
    %Mark Delta Vf
    \draw[pil,>-< ,dotted] (3.25,\dvf) -- (4.75,\dvf);
    \draw (4.5,\dvf) -- (5,\dvf) node[anchor=south west] {$\Delta v_{bias}$};


\end{scope}
\end{circuitikz}
\caption{Idealized LED voltage-current characteristic, with the \emph{forward voltage} $v_f$  identified and a projection of the \emph{bias point} P }
\label{fig:led_I-V}
\end{figure}

The characteristics of an LED subjected to variations due to manufacturing tolerances and second order effects at thermal deviations and ageing. Therefore the $i-v$ characteristics is not static and requires the driver to adapt to the load in order to keep the desired light output. The variations in $v-i$ curve can be associated to three main effects. First, $v_f$ has a negative dependence with the temperature, drooping its values as the \emph{pn}-junction temperature increases. Second, the LED has an aging factor which derates the light output over time, and which has to be adjusted by changing the bias point. And third, during production LEDs will vary in colour, flux, and forward voltage; even for products from the same batch. The manufacturers have reduced the tolerances between devices by binning \footnote{Quality control performed at LED production line, where each LED is individual tested and sorted in groups (bins) that have the same electrical and lighting characteristics.}, nevertheless  after binning, the parts are still subjected to some tolerances. For example Table~\ref{tab:vf_values} shows the tolerances in forward voltages for 4 different commercial devices, observing a deviation around $\pm10\%$.

Figure ~\ref{fig:led_I-V} graphically presents how the tolerances in $v_f$ produce a displacement in the $v-i$ characteristic,  which require to modify the $v_{bias}$ within a certain range $\Delta v_{bias}$ in order to keep $i_{bias}$ constant. Despite the variations associated to the load, the driver has at the same time to cope with perturbations and tolerances subjected to the energy supply. The driver has to provide line regulation for static deviations and immunity to high frequency perturbations, without affecting the load. Currently there are three different driver families used to implement LED drivers that are presented in the s subsequent sections.

\begin{table}[!h]
\centering
\caption{Electrical characteristics of different commercial LEDs}
\label{tab:vf_values}
\renewcommand{\arraystretch}{1.5}% Wider
\begin{tabular}{l | c | c |c | c | c | c | c  }
 \multirow{2}{*}{ Model} & \multirow{2}{*}{ Mnf.} & \multicolumn{3}{c|}{ $v_f~[V]$ }  & $i_f$ & Lum. & \multirow{2}{*}{ CCT }  \\ \cline{3-7}
  & & $min.$ & $typ.$ & $max.$ & $[mA]$ & $[lm]$ &  \\
  \midrule
  L130 & Lumileds & 5.8 & 6.1 & 6.6 & 120 & 92 & 4000K\\
  CLA1A & XB-D    &  -  & 2.9 & 3.5 & 350  &  122 & 5000K\\
  NF2L757GR & Nichia & - & 6.32 & 7.1 & 150 & 124 & 3000K\\
  ASMT – M & Avago  & 2.8 & 3.2  & 3.5 & 120 & 350 &  4000K

\end{tabular}
\end{table}

\section{Linear Regulators}
Linear drivers place a shunt element between the source and the load(\emph{i.e} the LED). The shunt element limits the LED current providing the necessary voltage droop between the source and the load. The excess of voltage between the source and the load is dissipated in the series element, literally burned in form of heat; therefore these drivers become very inefficient if the LED voltage is not close to the source. Other limitation is that linear drivers only provide step-down conversion, thus they cannot work when the voltage at the load is higher than the input supply.

\begin{figure}[!h]
\centering
\ctikzset { bipoles/length=1cm}
\begin{subfigure}[t]{.45\textwidth}
    \centering
    \begin{circuitikz} [american voltages,scale=0.65]
    \draw
        (0,0) to[V = $v_{src}$]
        (0,3) to[generic=$r_{series}$,i=$i_o$]
        (5,3) to[leD*,v=$v_{o}$]
        (5,0) -- (0,0);
    \end{circuitikz}
    \caption{}
    \label{fig:linear_ckt}
\end{subfigure}
\hfill
\begin{subfigure}[t]{.45\textwidth}
    \begin{circuitikz} [scale=0.65]
    \begin{scope}%[xshift = 8cm, yshift=0cm]
        \draw[->] (0,0) -- (4,0) node[anchor=south] {$  m $};
        \draw[->] (0,0) -- (0,3.2) node[anchor=east] {$\eta $};

        %Ticks X
        \draw (3,-5pt) -- (3,2pt)  node[anchor=south west] {$1$};
        \draw (1.5,-5pt) -- (1.5,2pt)   node[anchor=south west] {$0.7$};

        %Ticks Y
        \draw (2pt,2.5) -- (-5pt,2.5) node[anchor=east] {$100\%$};
        \draw (2pt,1.5) -- (-5pt,1.5) node[anchor=east] {$70\%$};

        %Markers
        \draw[dotted] (3,2.5) -- (3,0);
        \draw[dotted] (3,2.5) -- (0,2.5);
        \draw[dotted] (1.5,1.5) -- (1.5,0);
        \draw[dotted] (1.5,1.5) -- (0,1.5);


        \draw[thick] (3,2.5) -- (0,0.5);
    \end{scope}
    \end{circuitikz}
    \caption{}
\label{fig:linear_chr}
\end{subfigure}
\caption{Linear driver, \emph{left}- schematic; \emph{righ}- conversion ratio vs. efficiency characteristics}
\label{fig:linear_drv}
\end{figure}

The circuit of the Figure~\ref{fig:linear_ckt} shows the schematic of a linear driver. The shunt element can be implemented with just a resistor of with an active device. The first will impose a current depending on the input source and the load conditions; the second will provide regulation of the bias point for variations in the source and in the load. Linear drivers are very simple to implement with few components as shown in the schematic of Figure~\ref{fig:ldo_circuit}. They have a very low costs and take almost no area, being indeed the perfect solution for integration.

\begin{SCfigure}[][!h]
    \centering
    \ctikzset { bipoles/length=1cm}
    \ctikzset{tripoles/mos style/arrows}
    \begin{circuitikz} [american voltages,scale=0.65]
    \draw
        (0,0) to[V = $v_{src}$] (0,5)
        (5,5) to[short,i<=$i_o$] (4,5) to[Tnigfete,n=m1] (0,5)
        (5,5) to[leD*]
        (5,2.5) to[R,l=$r_{sense}$]
        (5,0) -- (0,0);
    \draw (3,2.5) node[op amp,xscale=0.5,yscale=-0.5,rotate=-90,](opamp){}
          (opamp.out) -|  (m1.G)
          (opamp.-) |- (4,1.5) |- (5,2.5)
          (opamp.+) |- (2,1.25) node[anchor=east] {$v_{ref}$};
    \draw (7,2.5) node[]{$i_o = \frac{v_{ref}}{r_{sence}}$};

    \end{circuitikz}
    \caption{Low-dropout (LDO) LED driver regulator. The load current $i_o$ is fixed by the voltage $v_{ref}$, despite perturbations in $v_{src}$ and the diode voltage.}
    \label{fig:ldo_circuit}
\end{SCfigure}

The plotted graph in Figure~\ref{fig:linear_chr} presents the variation of the driver efficiency with respect to the conversion ration $m$.
   \begin{equation}
        m = \frac{v_o}{v_{src}}.
   \end{equation}
The efficiency of the driver is the ratio between the input and output power, thus
   \begin{equation}
        \eta = \frac{P_o}{P_i} = \frac{v_o i_o}{v_{src} i_o} = \frac{v_o}{v_{src}} = m,
        \label{eq:linear_reg}
   \end{equation}
which is indeed equal to the conversion ratio. Therefore in a linear regulator the efficiency is subjected to the conversion ratio of the converter, the higher the difference between input and output the lower the efficiency. For instance assuming a minimum efficiency of $80\%$, the maximum accepted conversion ratio is 0.8.


\section{Inductor Based Converters}

\emph{Inductor Based Converters} (IBCs) are switched mode power supplies (SMPS) \footnote{Electronic power supply that provides efficient electric power conversion by commuting between different circuit configurations (modes).}  that employ magnetic passive elements, i.e. inductors and transformers, to store energy and provide efficient electrical power conversion. Therefore the magnetic component is the main passive element in the converter, allowing to process electrical energy by storing in in form of a magnetic field.

IBCs provide step-up and step-down conversion for large dynamic ranges while keeping the efficiency very high. On top of their power conversion capabilities, they can also provide galvanic isolation, which in some mains supplied applications is compulsory in order to guarantee the safety of the users against electrical hazards. These characteristics place these drivers as the preferred solution for the LED industry now a days. Figure~\ref{fig:induct_ckt} shows a \emph{Buck} converter, being it one of the most popular implementations for LED drivers in \emph{dc-dc} applications.  Figure~\ref{fig:induc_chr} presents the regulation characteristic of a generic  inductor based converter. As shown, the theoretical efficiency of these converters is 100\% for all the conversion ratio range. In practice, the parasitics in the components make the efficiency to drop with fluctuations with respect to the point of operation of the converter.

\begin{figure}[!h]
\centering
\ctikzset { bipoles/length=1cm}
\begin{subfigure}[t]{.45\textwidth}
    %\centering
    \raggedright
    \begin{circuitikz} [american voltages,scale=0.65]
    \draw
        (6.5,0) to[short]
        (0,0) to[V = $v_{src}$]
        (0,3) to[gswitch]
        (3,3) to[inductor=${L}$,i=$i_o$]
        (6.5,3);

    \draw (2.5,3) to[gswitch] (2.5,0);

    \draw (6.5,3) to[leD*,v=$v_{o}$] (6.5,0);

    \end{circuitikz}
    \caption{}
    \label{fig:induct_ckt}
\end{subfigure}
\hfill
\begin{subfigure}[t]{.45\textwidth}
    %\centering
    \begin{circuitikz} [scale=0.65]
    \begin{scope}%[xshift = 8cm, yshift=0cm]
        \draw[->] (0,0) -- (4,0) node[anchor=south] {$  m $};
        \draw[->] (0,0) -- (0,3.2) node[anchor=east] {$\eta $};


        %Ticks Y
        \draw (2pt,2.5) -- (-5pt,2.5) node[anchor=east] {$100\%$};
        \draw (2pt,1.5) -- (-5pt,1.5) node[anchor=east] {$90\%$};

        %Markers
        \draw[dotted] (3,2.5) -- (0,2.5);
        \draw[dotted] (1.5,1.5) -- (0,1.5);


        \draw[thick] (0.5,2.5) -- (3,2.5) node[anchor=south] {$Theoretical$};
        \draw[thick,dashed] (0.5,1.20) parabola[bend at end] (3,1.7) node[anchor=north] {$Real$};
    \end{scope}
    \end{circuitikz}
    \caption{}
\label{fig:induc_chr}
\end{subfigure}
\caption{Inductor based converter, \emph{left} - buck converter schematic; \emph{right} - conversion ration \emph{vs.} efficiency curve comparing the \emph{theoretical} and a \emph{practical} limit. }
\label{fig:inductive_smps}
\end{figure}

The main disadvantage is the use of magnetic components due to their volume and integrability . In practice, inductors dominate the entire volume of the LED drivers as shown in Figure~\ref{fig:smps_driver}. At the same time the three-dimensional nature of these components limit their integrability, specially in standard processes, current research in the field is providing with solutions for power inductors, however they are far to be mature enough for commercial uses.

\begin{SCfigure}[][!h]
\centering
\begin{tikzpicture}
\node[anchor=south west,inner sep=0] (image) at (0,0) {\includegraphics[height=5cm,angle=90]{./0_intro/img/LED_driver.png}};
\begin{scope}[x={(image.south east)},y={(image.north west)}]
%\draw [<-,thick] (0.75,0.5) -- (0.855,0.7)  node [anchor=south west] {Power Magnet};
\draw[black,ultra thick,rounded corners] (0.70,0.3805) rectangle (0.855,0.7);
\draw[black,ultra thick,rounded corners] (0.11,0.1) rectangle (0.28,0.50);
\draw[black,ultra thick,rounded corners] (0.28,0.1) rectangle (0.63,0.62);
\end{scope}
\end{tikzpicture}
\caption{Magnetic components marked with a black square in a mains connected LED driver. These components dominate the volume of the converter.}
\label{fig:smps_driver}
\end{SCfigure}

From the standpoint of view of the switches technology, inductive converters bring yet another disadvantage with respect of the integration of the switches.  Generally, the switches in an inductive converter have to fully block the highest operational voltage of the converter, from the input or the output voltage. Depending on the application, the range is from tens to a few hundreds of volts. Using high voltage devices has three main drawbacks: First, the losses in the devices scale quadratically with the voltage  stress. Second, bad switching performances, because high voltage devices are less efficient and slower switching. Third, the standard VLSI technologies do not offer these \emph{high voltage} (HV) devices and the VLSI technologies that offer them are less performance and more expensive than the dedicated discrete technologies.

\subsection{Energy transfer in inductor based converters}
%\begin{wrapfigure}{o}{0.7\textwidth}
Inductor based converter are ideally lossless, since the transfer of energy between a voltage source  and an inductor is an adiabatic process, which can be demonstrated using the circuit of Figure~\ref{fig:ind_chrg}.

\begin{figure}[!h]
    \centering
    \begin{subfigure}[b]{.33\textwidth}
    \raggedright
    %\ctikzset { bipoles/length=1cm}
    \begin{circuitikz} [american,scale=0.65]
    \draw
        (0,0) to[battery1 = $v_{src}$]
        (0,3) to[cspst=$s_1$] (2,3) to[short,i=$i$]
        (3,3) to[inductor=${l}$,v=$v_l$]
        (3,0) -- (0,0);
    \draw[white]  (2.5,-2pt) node[anchor=north,font=\footnotesize] {$t+t_{on}$};
    \end{circuitikz}
    \label{fig:induct_charge}
    \end{subfigure}
    \begin{subfigure}[b]{.33\textwidth}
    \raggedright
    \begin{circuitikz} [scale=0.65]
    \begin{scope}%[xshift = 8cm, yshift=0cm]
        \draw[->] (0,0) -- (0,1.25) node[anchor=east] {$ s_1 $};
        \draw[->] (0,0) -- (3.5,0) node[anchor=south] {$  t $};

        \draw[->] (0,1.75) -- (0,3) node[anchor=east] {$ i $};
        \draw[->] (0,1.75) -- (3.5,1.75) node[anchor=south] {$  t $};

        %Ticks Y
        \draw (0.5,0pt) -- (0.5,-2pt) node[anchor=north,font=\footnotesize] {$t_o$};
        \draw (2.5,0pt) -- (2.5,-2pt) node[anchor=north,font=\footnotesize] {$t_o+t_{x}$};

        %Markers
        \draw[dotted] (.5,0) -- (0.5,1.75);
        \draw[dotted] (2.5,0) -- (2.5,2);

        \draw[semithick] (0.5,0) |- (3.25,0.65) ;
        %\draw (1.5,.32) node[font=\footnotesize] {$t_{on}$};
        \fill[gray!50] (0.5,1.75) -- (2.5,2.75)  -- (2.5,1.75);
        \draw[semithick] (0.5,1.75) -- (2.5,2.75) -- (3,3);

        \draw[->] (2,2) to[bend left=45] (1,2.5) node[font=\footnotesize,anchor=south]{$Q_{src}$};

        %\draw[thick,dashed] (0.5,1.20) parabola[bend at end] (3,1.7) node[anchor=north] {$Real$};
    \end{scope}
    \end{circuitikz}

    \end{subfigure}
    \caption{Energy transfer in an inductor.}
    \label{fig:ind_chrg}
\end{figure}
On the one hand, the energy stored in an inductor is given by
\begin{equation}
E_l = \frac{1}{2} l i^2.
\label{eq:e_induct}
\end{equation}
The current flowing in the inductor after switch $s_1$ is closed is given by
\begin{equation}
i(t)= \frac{1}{l} \int v_l dt = \frac{v_{src}}{l}t.
\label{eq:i_inductor}
\end{equation}
Substituting~\eqref{eq:i_inductor} into~\eqref{eq:e_induct}, we can obtain the energy stored in the inductor after closing $s_1$ during the time $t_x$, which results in
\begin{equation}
E_{l,t_x}= \frac{v_{src}^2{t_x}^2}{2l} .
\label{eq:e_l_tx}
\end{equation}
On the other hand, the energy delivered by the energy source $v_{src}$ is given
\begin{equation}
E_{src} = v_{src} q_{src}.
\label{eq:e_src}
\end{equation}
The charge $q_{src}$ delivered by the energy source after closing $s_1$ during a time $t_x$ can be obtained by integrating the inductor current~\eqref{eq:i_inductor},  between $t_o$ and $t_o+t_x$ as
\begin{equation}
q_{src} = \int_t^{t_o+t_x} i(t) dt = \frac{v_{src}}{2l}{t_x}^2 .
\label{eq:q_src}
\end{equation}
|Therefore substituting~\eqref{eq:q_src} into~\eqref{eq:e_src} gives the energy deliver by the source, which gives
\begin{equation}
E_{src,t_x} = \frac{v_{src}^2{t_x}^2}{2l}.
\label{eq:e_src_tx}
\end{equation}
The energy lost while transferring energy between the source and the inductor, is the difference between the energy deliver from the source~\eqref{eq:e_src_tx} and the energy stored in the inductor~\eqref{eq:e_l_tx}, which results in
\begin{equation}
E_{loss} = E_{src,t_x} - E_{l,t_x} = \frac{v_{src}^2{t_x}^2}{2l} - \frac{v_{src}^2{t_x}^2}{2l} =0.
\label{eq:e_loss_l}
\end{equation}
In conclusion, transferring energy between a voltage source and inductor is lossless, and that is why generally inductor based converters achieve very high conversion efficiencies. Nevertheless the parasitics in the components make these converters to do not achieve 100\% efficiencies.

\section{Capacitor Based Converters }
Switched Capacitor Converters (SCCs) are SMPS composed only of switches and capacitors. SCC were initially used for voltage multiplication~\cite{30Cockcroft,44Waidelich,76Dickson} and more recently in applications that need voltage regulation as well~\cite{Ng:EECS-2011-94}. Compared to inductor based converters, the absence of magnetic elements places them in a good position for high density power systems and integrated solutions, such as Power-System-in-Package (PSiP) or Power-System-on-Chip (PSoC).

SCCs have a fixed ratio of conversion between the input and the output determined by the topology. The output voltage of the converter under no load conditions is defined as the \emph{target voltage} ( $v_t$). The converter performs at high efficiency when the load is supplied close to the target voltage. Similar to the linear drivers, the efficiency of the converter drops as the difference between the load and target voltage increases. Also, the converter can not supply the load with an output voltage value above to the target voltage. Figure~\ref{fig:SCC_ckt} shows a step-down converter with a conversion ratio of one half. A common practice to extend the regulation margins of these converters is to have topologies with multiple conversion rations~\cite{2013Ma,2013Breussegem:m_trg}. From Figure~\ref{fig:SCC_chr} it can be seen that the efficiency increases as the ration $m$ gets close to the first fixed conversion ration of the converter $m_1$; right after $m_1$ the efficiency drops again dramatically and it again linearly increases as it approaches the second fixed conversion ratio of the converter $m_2$. Beyond $m_2$ the converter does not work.
\begin{figure}[!h]
%\centering
\ctikzset { bipoles/length=1cm}
\begin{subfigure}[t]{.45\textwidth}
    %\centering
    \raggedright
    %\raggedleft
    \begin{circuitikz} [american voltages,scale=0.65]
    \draw
        (6.5,0) to[short]
        (0,0) to[V = $v_{src}$]
        (0,3) to[gswitch]
        (2,3) to[capacitor=${c_1}$]
        (3,3) to[gswitch]
        (5,3) to[short]
        (6.5,3);

    %Parallel switch to ground
    \draw (3.25,3) to[gswitch] (3.25,0);

    %Switch branch to load
    \draw (1.75,3) --
          (1.75,4.5) to[gswitch]
          (5,4.5) --
          (5,3);

    %Load and capacitor C2
    \draw (5,0) to[capacitor=$c_2$,-*] (5,3);
    \draw (6.5,3) to[leD*,v_=$v_{o}$] (6.5,0);

    \end{circuitikz}
    \caption{}
    \label{fig:SCC_ckt}
\end{subfigure}
\hfill
\begin{subfigure}[t]{.45\textwidth}
    %\centering
    \raggedleft
    %\raggedright
    \begin{circuitikz} [scale=0.65]
    \begin{scope}[xshift = 10cm, yshift=0cm]
            \draw[->] (0,0) -- (4,0) node[anchor=south west] {$  m $};
            \draw[->] (0,0) -- (0,3.2) node[anchor=east] {$\eta $};

            %Ticks X
            \draw  (1.75,5pt) -- (1.75,0pt) node[anchor=south west ] {$m_1$};
            \draw  (3,5pt) -- (3,0pt)   node[anchor=south west ] {$m_2$};
            %\draw (1.5,2pt) -- (1.5,-5pt) node[anchor=north] {$0.7$};

            %Ticks Y
            \draw (2pt,2.5) -- (-5pt,2.5) node[anchor=east] {$100\%$};
            \draw (2pt,1.5) -- (-5pt,1.5) node[anchor=east] {$90\%$};

            %Markers
            \draw[dotted] (1.75,2.4) -- (1.75,0);
            \draw[dotted] (3,2.3) -- (3,0);
            \draw[dotted] (3,2.5) -- (0,2.5);
            %\draw[dotted] (1.5,1.5) -- (1.5,0);
            %\draw[dotted] (1.5,1.5) -- (0,1.5);


            \draw[thick] (0.5,1.4) -- (1.75,2.4) -- (1.75,1.6) -- (3,2.3)  node[anchor=south] {};
            \draw (10,0)[anchor=north] {};
        \end{scope}
    \end{circuitikz}
    \caption{}
\label{fig:SCC_chr}
\end{subfigure}
\caption{Switched capacitor converter, \emph{left} - 2:1 converter schematic; \emph{right} - conversion ration \emph{vs.} efficiency curve for of a generic multiple  conversion ration stage }
\label{fig:SCC_smps}
\end{figure}

The main limitation of SCCs is that they cannot directly  provide the voltage-to-current regulation, necessary feature of the LED driver. Nevertheless by indirect means to output current can be controlled adding a linear regulator in series with the converter output, compromising the converter efficiency.  Such approach is very popular driver for backlighting LEDs in battery supplied devices, where high  integrability was more relevant than the power efficiency. The backlighint LED driver solution uses a multi-target SCC converter the battery voltage to a voltage above the LED strings, and by means of  linear drivers the output current is adjusted to properly bias the LEDs. Adopting that architecture for general lighting could be a solution, however when scaling the voltages and currents to the requirements of general lighting applications the resulting driver would be infeasible and inefficient.

From the standpoint view of integration the main advantage of these converters is the use no inductors. Integrated capacitors have a better energy density than integrated inductors. The mechanical structure of the capacitors, a stack of metal-isolator-metal, is much easier to replicate on a small scale. With respect to the switches technology, SCCs have the advantage of dividing the operational voltages of  the converter among the different components, thus reducing the voltage stress in the switches and capacitors. Actually reducing the voltages in the converter has different advantages. First,  capacitors have higher energy density. Second, lower voltage switches have better switching performances and lower associated losses. Finally, lower voltage devices require less silicon area and have a larger offer in the standard very large integration scale (VLSI) processes.



\subsection{Energy transfer in capacitor based converters}
%\begin{wrapfigure}{o}{0.7\textwidth}
SCCs are by nature lossy, since the transfer of energy between a voltage and a capacitor is a non-adiabatic process, which can be demonstrated using the circuit of Figure~\ref{fig:cap_chrg}.
\begin{figure}[!h]
    \centering
    \begin{subfigure}[b]{.33\textwidth}
    \raggedright
    %\ctikzset { bipoles/length=1cm}
    \begin{circuitikz} [american,scale=0.65]
    \draw
        (0,0) to[battery1 = $v_{src}$]
        (0,3) to[cspst=$s_1$] (2,3) to[short,i=$i$]
        (3,3) to[capacitor=${c}$,v=$v$]
        (3,0) -- (0,0);
    %\draw[white]  (2.5,-2pt) node[anchor=north,font=\footnotesize] {$t_o$};
    \end{circuitikz}
    \label{fig:induct_charge}
    \end{subfigure}
    \begin{subfigure}[b]{.33\textwidth}
    \raggedright
    \begin{circuitikz} [scale=0.65]
    \begin{scope}%[xshift = 8cm, yshift=0cm]
        \draw[->] (0,0) -- (0,1.25) node[anchor=east] {$ s_1 $};
        \draw[->] (0,0) -- (3.5,0) node[anchor=south] {$  t $};

        \draw[->] (0,1.75) -- (0,3) node[anchor=east] {$ v $};
        \draw[->] (0,1.75) -- (3.5,1.75) node[anchor=south] {$  t $};

        %Ticks Y
        \draw (0.5,0pt) -- (0.5,-2pt) node[anchor=north,font=\footnotesize] {$t_o$};
        %\draw (2.5,0pt) -- (2.5,-2pt) node[anchor=north,font=\footnotesize] {$t+t_{x}$};
        \draw (3pt,2.5) -- (-2pt,2.5) node[anchor=east,font=\footnotesize]{$v_{src}$};

        %Markers
        \draw[dotted] (.5,0) -- (0.5,1.75);
        \draw[dotted] (0,2.5) -- (2.5,2.5);

        \draw[semithick] (0.5,0) |- (3.25,0.65) ;
        %\draw (1.5,.32) node[font=\footnotesize] {$t_{on}$};
        %\fill[gray!50] (0.5,1.75) -- (2.5,2.75)  -- (2.5,1.75);
        \draw[semithick] (0.5,1.75) |- (3.25,2.5) ;

        %\draw[->] (2,2) to[bend left=45] (1,2.5) node[font=\footnotesize,anchor=south]{$Q_{src}$};

        %\draw[thick,dashed] (0.5,1.20) parabola[bend at end] (3,1.7) node[anchor=north] {$Real$};
    \end{scope}
    \end{circuitikz}

    \end{subfigure}
    \caption{Energy transfer in a capacitor.}
    \label{fig:cap_chrg}
\end{figure}

On the one hand, the energy stored in a capacitor is given by
\begin{equation}
E_c = \frac{1}{2} c v^2.
\label{eq:e_cap}
\end{equation}
After closing the switch at $t_o$, the capacitor is charged at $v_{src}$.  The charge delivered by the source $q_{src}$ to the capacitor is given by
\begin{equation}
q_{src}= c~v_{src},
\label{eq:q_src_cap}
\end{equation}
hence the energy stored in the charged capacitor results in
\begin{equation}
E_{c^+} = \frac{1}{2} \frac{q_{src}}{v_{src}}  {v_{src}}^2 = \frac{1}{2}q_{src}~v_{src}.
\label{eq:e_cap+}
\end{equation}
On the other hand, the energy delivered by the energy source is just
\begin{equation}
E_{src} = v_{src} q_{src}.
\label{eq:e_src_c}
\end{equation}
The energy lost while transferring energy between the source and the capacitor, is the difference between the energy deliver from the source~\eqref{eq:e_src_c} and the energy stored in the capacitor~\eqref{eq:e_cap+}, which results in
\begin{equation}
E_{loss} = E_{src} - E_{c^+} = v_{src}~q_{src}  - \frac{1}{2}q_{src}~v_{src}  = \frac{1}{2}q_{src}~v_{src}.
\label{eq:e_loss_l}
\end{equation}
In conclusion, transferring energy between a voltage source and a capacitor is a lossy process. Actually, half of the used energy is lost, and the other half is stored in the capacitor. The energy lost is dissipated in the resistive elements of the current path such as the \emph{on}-channel resistance and track resistances.









\section{Overview in integrated power supplies}
With regard to the integration of power supplies, we can indemnify two clear approaches, Power System on Chip (PSoC), and  Power System in Package (PSiP). PSoCs integrate all converter functions, power train and control, in a single die, using the available in-die reactive components. Currently, the standard VLSI processes offer capacitors and inductors with low energy densities, generally provided for radio frequency (RF) proposes. PSiPs integrate the converter in a single integrated circuit package, allowing to use multiple dies with different technologies and dedicated miniaturized discrete components. Currently there are few commercially available PSiP LED drivers offered by \emph{Linear Technology}.

There is yet another approach in integration of power supplies, where the IC integrates the power train and the control, using off-package  passive components. In reality, this is a widely adopted and dominant solution among the different IC manufacturers that provide LED dedicated drivers for the three main applications: General lighting, screen back-lighting and automotive. In order to fulfil these applications manufacturers offer two possible IC solutions, stand-alone controllers, and integrated power train and controller. These application specific drivers facilitates the development by reducing component count and design time, however the flexibility in the design of the new topologies that help integrability and improve the power density is very limited since the driver topologies are already fixed by the available ICs. 


In general lighting and automotive applications the dedicated commercial ICs implement only inductor based converters such as back, boost, flyback, etc., providing just a solution for the power conversion point of view without integrating other functionalities. In the case of back-lighting, the commercial ICs integrate a power train and control unit and a control interface. 





\section{Summary}
The three main technologies

Based on the aforementioned arguments both SMPS technologies, inductor based and switched capacitor converters,   are




the most suitable  for high efficient LED drivers and miniaturization. On the one hand, inductive converters are the preferred choice due to their high efficiency and regulation capabilities, although they are less suitable to integrate due to the use of

   adopting a SCC based driver as solution for LED lighting applications seems to be, a priori,  not an evident choice. Due to the limitations of SCC  in voltage-to-current conversion would directly disqualify them. However their advantageous characteristics form the standpoint of view of integration, made these circuits very attractive. Actually, if the initial limitations in voltage-to-current conversion could be overcome, such architecture would be an interesting candidate to explore as a solution for a \emph{Power System on-Chip/in-Package} LED driver.  Exploring the possibilities that switched capacitor converters can offer in terms of integrated and miniaturized LED drivers with efficient voltage-to-current conversion was the rational of this dissertation.

The use of a bear SCC can never satisfy the requirements of LED drivers due to the following facts:
\begin{itemize}
  \item Only provide voltage-to-voltage conversion
  \item Fixed conversion ratios
  \item Regulation is provided by series shunting
\end{itemize}
These limitations combined with the abrupt characteristics I-V of the LEDs makes barely impossible to provide high efficient solutions with the single use of SCC. The converters would require to have a large number of conversion ratios with a very large granularity to avoid uncontrolled currents flowing through the LEDs.

The research presented in this work aims to explore the possibilities of the SCC for LED drivers and the conducting path is based in the combination of the with inductors. The overall solution improves the power density and reduced form factor of the present solutions.


This thesis is divided in the four main sections that where necessary to build a switched capacitor LED driver. The first section introduces the new LED driver architecture used during the entire thesis, the \emph{Hybrid-Switched Capacitor Converter}, H-SCC from now on. The second part of this book, the core of the PhD. work, presents the methodology to model H-SCC. The methodology extends the previous works in the topic providing an enhanced modeling for the design of SCCs and H-SCCs. The third section is devoted to the practical use of the new methodology, thus for the design phase of a converter. The modeling is used  to help in the development facilitating the sizing and optimization of the design variables. The last section presents a discrete implementation of 12W H-SCC LED driver and the design procedure. Although is not a regular practice, experimental work is not only presented in the in the last section. The experimental work has been also  used to validate the presented modeling and methodology. The final section is the conclusion of the entire work and the future opportunities that the presented work can offer.

\bibliographystyle{plainnat}
\bibliography{references} 