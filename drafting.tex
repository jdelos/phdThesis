%%%%%%%%%%%%%%%%%%%%%%%%%%%%%%%%%%%%%%%%%%%%%%%%%%%%%%%
%               CHAPTER 3: H-SCC                      %
%%%%%%%%%%%%%%%%%%%%%%%%%%%%%%%%%%%%%%%%%%%%%%%%%%%%%%%

Actually using the \emph{pwm}-nodes of an SCC reduces the amplitude of the voltage ripple at the filter inductor, what at the same time requires a smaller inductor value for a converter operating under the same frequency and current ripple conditions. The reduction ratio of the inductor can be obtained from eqs. (\ref{eq:buck_l}) and (\ref{eq:hscc_l}) as
\begin{equation}
 \frac{l_{o,hscc}}{l_{o,buck}} =  \frac{{ m_i \cdot v_{src} \cdot DD' \cdot T}/{\Delta i} }{{  v_{src} \cdot DD' \cdot T}/{\Delta i}} = m_i.
\label{eq:l_m}
\end{equation}
In fact, the ratio coincides with the intrinsic conversion ratio ($m_i$) of the converter, which for a step-down is always smaller than 1. For the 3:1 Dickson H-SCC of figure~\ref{fig:3_1_hscc} the inductor value is 3 times smaller than in a buck converter. As a matter of fact, using an H-SCC introduces an new design parameter to reduce the value of the power inductor, providing a new architecture that improves the integration or power converters in this case by the reduction of the magnetic components.


As a general practice in the design of a LED driver, the voltage of
the LED string is often optimized by selecting low, mid or high
power LEDs and wiring them in series or parallel. It is favorable
to select a high voltage in the LED strings, but small enough to be
driven by a buck converter. Increasing the voltage in the LED
string decreases currents through the converter reducing the
conduction losses.


%\subsection{High Current Paths} %\label{sc:high_current_path} %
%The current % %\begin{figure}[t] %\ctikzset { bipoles/length=1cm}
%\centering % \begin{circuitikz}[american voltages,scale=0.6] % %
\draw % %Input Supply % (0,0) to[V=$v_{src}$] % %Draw Switches %
(0,10) -- % (5,10) to[switch=$s_1$,-*] %S1 % (5,8) to[switch=$s_2$]
%S2 % (5,6) to[switch=$s_3$] %S3 % (5,4) -- % %left branch % (3,4)
to[switch=$s_5$] % (3,2) to[switch=$s_4$] % (3,0); % % \draw %right
branch % (5,4) -- % (7,4) to[switch,l_=$s_6$] % (7,2)
to[switch,l_=$s_7$] % (7,0) -- (0,0); % % % % \draw %Capacitor C1 %
(3,2) -- (2,2) % to[pC,l_=$c_1$] (2,8) -- % (5,8); % % \draw
%Capacitor C2 % (7,2) -- % (8.25,2) to[pC,l_=$c_2$](8.25,6) -- %
(5,6); % % \draw %Inductor % (5,8) -- (8,8) node[anchor=south]
{$v_x$} to[L=$l_o$] (12,8) % (12,6) node[sground] {}
to[pC,l=$c_{o}$] (12,8) % (12,8) -- (14,8) to[leD*] (14,6) % (14,5)
to[leD*] (14,3) node[sground]{} % (14,9) to[open,v^=$v_o$] (14,2);
% % \draw[dotted] (14,6) -- (14,5); % % % \draw %Capacitor C3 %
(5,0) to[pC,l_=$c_3$,-*] (5,4) node[anchor=south east] {}; % % %
Draw currents flow arrows % % Ph1 % \draw[thick,->] (0.25,6) --
(0.25,8.75) to[bend left=45] (1.25,9.75) -- (3.75,9.75) % to[bend
left=45] (4.75,8.75) to[bend right = 45] (5.75,7.75) -- (7.5,7.75);
% % \draw[thick] (2.25,5.5) -- (2.25,6.75) to[bend left=45]
(3.25,7.75) -- (7,7.75) ; % % %Ph 2 % \draw[thick, dashed,->]
(1.75,5.5) -- (1.75,7.75) to[bend left=45] (2.25,8.25) --
(7.5,8.25) ; % \draw[thick, dashed] (4.75,6.25) -- (4.75,7.25)
to[bend left=45] (5.75,8.25) ; % % \end{circuitikz} % \caption{
H-SCC with a 3:1 Dickson topology with the inductor connected to
the second \emph{pwm}-node.} % \label{fig:3_1_hscc} %\end{figure}\

%\begin{SCfigure} %\centering %\begin{circuitikz}[american
voltages,xscale=0.55,yscale=0.65] %\begin{scope} % \draw [->] (0,0)
-- (10,0) node[anchor=west]{$t$}; % \draw [->] (0,0) -- (0,6.5)
node[anchor=east]{$v_x$}; % % %Vertical ticks % \draw (2pt,6) --
(-5pt,6) node[anchor=east] {$v_{src} $}; % \draw (2pt,4) --
(-5pt,4) node[anchor=east] {$v_{src} \frac{2}{3}$}; % \draw (2pt,2)
-- (-5pt,2) node[anchor=east] {$v_{src} \frac{1}{3}$}; % %
%Horizontal ticks % \draw (1.25,2pt) -- (1.25,-5pt)
node[anchor=north] {$DT$}; % \draw (3,2pt) -- (3,-5pt)
node[anchor=north] {$T$}; % \draw (6,2pt) -- (6,-5pt)
node[anchor=north] {$2T$}; % \draw (9,2pt) -- (9,-5pt)
node[anchor=north] {$3T$}; % \draw (0,0) node[anchor=north east]
{$0$}; % % \draw[semithick, dashed] (0,2.1) -- (1.15,2.1) --
(1.35,3.9) -- (2.9,3.9) -- % (3.1,2.1) -- (4.15,2.1) -- (4.35,3.9)
-- (5.95,3.9) -- % (6.1,2.1) -- (7.15,2.1) -- (7.35,3.9) -- (9,3.9)
; % % \draw[semithick,dotted] (0,5.9) -- (1.15,5.9) -- (1.35,4.1)
-- (2.9,4.1) -- % (3.1,5.9) -- (4.15,5.9) -- (4.35,4.1) --
(5.9,4.1) -- % (6.1,5.9) -- (7.15,5.9) -- (7.35,4.1) -- (9,4.1) ; %
% \draw[semithick ] (0,1.95) -- (1.15,1.95) -- (1.35,0.05) --
(2.9,0.05) -- % (3.1,1.95) -- (4.15,1.95) -- (4.35,0.05) --
(5.9,0.05) -- % (6.1,1.95) -- (7.15,1.95) -- (7.35,0.05) --
(9,0.05) ; % % \draw[semithick, dashdotted ] (0,0.05) --
(1.15,0.05) -- (1.35,1.95) -- (2.9,1.95) -- % (3.1,0.05) --
(4.15,0.05) -- (4.35,1.95) -- (5.9,1.95) -- % (6.1,0.05) --
(7.15,0.05) -- (7.35,1.95) -- (9,1.95) ; % %\end{scope}
%\end{circuitikz} %\caption{Transient voltages of all the different
\emph{pwm}-nodes of the converter of figure~\ref{fig:3_1_hscc}.}
%\label{fig:vx_t} %\end{SCfigure}



%%Single output block diagram

\begin{figure}[!h]
\centering
\ctikzset { bipoles/length=1cm}
\begin{circuitikz}[scale=0.65]
\draw
    (1,0) to[short,o-]
    (0,0) to[V = $V_{supply}$]
    (0,3) to[short,-o]
    (1,3) ;

\draw
    (2,3) --
    (2.5,3)

    (2,0) --
    (2.5,0)

    node[ocirc]  (IC)  at (2,0) {}
    node[ocirc]  (I) at (2,3) {}
    (I) to[open,v=$v_{i}$] (IC);


\draw [thick]
    (2.5,-0.5) --
    (2.5,3.5)  --
    (5.5,3.5)  --
    (5.5,-0.5) --
    (2.5,-0.5);

\draw (4,2)node[anchor=north]{$\frac{v_o}{v_{i}}=m$} ;
\draw
    (5.5,3) -- (6,3)
    (5.5,0) -- (6,0)
    node[ocirc]  (O)  at (6,3) {}
    node[ocirc]  (OC) at (6,0) {}
    (O) to[open,v^<=$v_{o}$] (OC);

\draw
    (7,0) to[short,o-]
    (8,0) to[ R= $Load$,mirror]
    (8,3) to[short,-o]
    (7,3) ;
\end{circuitikz}
\label{fig:two_port}
\caption{General two port configuration of a Switched Capacitor Converter. }
\end{figure}


%%%%%%%%%%%%%%%%%%%%%%%%%%%%%%%%%%%%%%%%%%%%%%%%%%%%%%%
%               CHAPTER 4: MODELING                   %
%%%%%%%%%%%%%%%%%%%%%%%%%%%%%%%%%%%%%%%%%%%%%%%%%%%%%%%

%Define the two vertical regions SSL & FSL
  \vasymptote[draw=none]{1e5}{ssl};
  \vasymptote[draw=none]{1e6}{fsl};

  %Define vertical axis at zero
  \path[name path=yaxis_0] (axis cs:0,0) -- (axis cs:0,1e10);
  \path[name path=yaxis_fsl] (axis cs:1e6,0) -- (axis cs:1e6,1e10);
  \path[name path=yaxis_Inf] (axis cs:5e6,0) -- (axis cs:5e6,1e10);

  \addplot [
        thick,
        draw=none,
        fill=lightgray,
        fill opacity=0.3
    ]
    fill between[
        of = yaxis_0 and ssl,
    ];

    \addplot [
        thick,
        draw=none,
        fill=lightgray,
        fill opacity=0.3
    ]
    fill between[
        of=yaxis_fsl and yaxis_Inf,
    ];

    %\node[anchor=north,rotate=45] at (\lbssl,1/(\lbssl*\C)){$SSL$};
    %
    \node[anchor=north] at (1e6,4){$FSL$};

The \emph{solid} lines marks the pumped charge paths to the load and the \emph{dashed} lines the possible paths of the redistributed charge.

 in order to overcome the restrictions of the original charge flow methodology, 
 
As described in the previous section, the proposed methodology models the load as a current sink instead of a voltage sink. This slight modification has a relevant change in the modeling of the converter, adding three important aspects to the modeling:
\begin{enumerate}
  \item Connect the load to any of the nodes of the converter, since it does not fix any voltage at the connected node.

  \item Include the \emph{dc}-capacitors in the analysis, since the voltage across the sink is given by the capacitor and not longer by the previous used voltage sink.

  \item Estimate the delivered charge to the load for each of the circuit modes of the converter, since the current sink has a value of the averaged output current.
\end{enumerate}

