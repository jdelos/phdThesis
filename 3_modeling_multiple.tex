\clearpage
\section{Multiple Output Converter}
Another advantage that SCC offers is to provide multiple outputs using a single SCC stage. In this multi-port configuration, the energy supply is connected to input port, and the converter provides multiple output ports with different conversion ratios. A clear application was presented by \citeauthor{2012Kumar} in~\cite{2012Kumar} with the Triple Output Fixed Ratio Converter (TOFRC); where a 2:1 Ladder converter combined with two inductors provides three fixed output voltages using a single SCC stage.
\begin{figure}[!h]
\centering
\ctikzset { bipoles/length=1cm}
\begin{circuitikz}[american,scale=0.65]
\draw
    (1,0) to[short,o-]
    (0,0) to[V = $V_{src}$]
    (0,3) to[short,-o]
    (1,3) ;

\draw
    (2,3) --
    (2.5,3)

    (2,0) --
    (2.5,0)

    node[ocirc]  (IC)  at (2,0) {}
    node[ocirc]  (I) at (2,3) {}
    (I) to[open,v=$v_{i}$] (IC);


\draw [thick]
    (2.5,-0.5) --
    (2.5,3.5)  --
    (5.5,3.5)  --
    (5.5,-0.5) --
    (2.5,-0.5);

\draw (4,2.5)node[]{$\frac{v_{1}}{v_{i}}=m_1$};
\draw (4,0.5)node[]{$\frac{v_{n}}{v_{i}}=m_n$};

\draw
    (5.5,1.25) to[short,-o](6,1.25)
    (5.5,-0.25)  to[short,-o] (6,-0.25)
    (6.25,1) to[open,v^<=$v_{n}$] (6.25,0);
\draw
    (7,-0.25) to[short,o-]
    (8,-0.25) to[/tikz/circuitikz/bipoles/length=0.5cm,R= Load $n$,mirror]
    (8,1.25) to[short,-o] (7,1.25) ;
    
\draw
    (5.5,3.25) to[short,-o](6,3.25)
    (5.5,1.75)  to[short,-o] (6,1.75)
    (6.25,3) to[open,v^<=$v_{1}$] (6.25,2);

\draw
    (7,1.75) to[short,o-]
    (8,1.75) to[/tikz/circuitikz/bipoles/length=0.5cm,R= Load 1,mirror]
    (8,3.25) to[short,-o] (7,3.25) ;
\end{circuitikz}
\label{fig:two_port}
\caption[Block diagram of a multi-output SCC]{Block diagram of a general multiple output port configuration of a Switched Capacitor Converter. }
\end{figure}

\subsection{The Output Trans-Resistance Model}
When considering a converter with multiple outputs, the load effects have to be taken into account for all the outputs. Actually, when the converter is loaded, it produces a voltage drop throughout outputs of the converter. Therefore, the output current of one output node has an influence to the other outputs. In order to model these effects a new model based on trans-resistance parameters is proposed.
\begin{figure}[!h]
\centering
\ctikzset { bipoles/length=1cm}
\begin{circuitikz}[american voltages, scale=0.65]
\draw
    (0,0) to[V = $ m_1  v_{src}  $] (0,3)
    (3,3) to[american controlled voltage source,l_=$i_1 z_{11} + i_2 z_{12} + \cdots + i_n z_{1n} $] (0,3)
    (3,3) to[short,i>_=$i_1$,-o] (4,3)
    (4,0) to[short,o-] (0,0)
    (4,3) to[open,v^=$v_1$] (4,0);

\draw
    (8,0) to[V = $ m_n  v_{src}  $] (8,3)
    (11,3) to[american controlled voltage source,l_=$i_1 z_{n1} + i_2 z_{n2} + \cdots + i_n z_{nn} $] (8,3)
    (11,3) to[short,i>_=$i_n$,-o] (12,3)
    (12,0) to[short,o-] (8,0)
    (12,3) to[open,v^=$v_n$] (12,0);

\end{circuitikz}
\caption{Output trans-resistance model of a switched capacitor converter.}
\label{fig:scc_model_tr}
\end{figure}

The proposed model is shown in Figure\ref{fig:scc_model_tr}; as it can be seen, each output is represented using two controlled voltage sources connected in anti-series. One source provides the \emph{target voltage}  associated with the output, taking the value from the input voltage, $v_{src}$, multiplied by the respective conversion ratio associated to that output, $m_x$.

The other source, produces a voltage droop associated with the losses in the converter. The current delivered by each loaded node adds a specific contribution to the converter losses. Therefore, this voltage source takes the value given by the linear combination of all the converter output currents weighted by their associated trans-resistance factor $z$.

The trans-resistance factor $z_{xy}$ produces a voltage drop at the output $x$ proportional to the charge (\emph{i.e}. current) delivered by the output $y$.  It can be seen that the trans-resistance factor $z_{xx}$ corresponds to the voltage drop of the same output where the current is delivered, thus this parameter is  the output impedance for that node. Since all the trans-resistance factors relate current to voltage, they are in \emph{Ohms}.

With the proposed model, the converter behavior can be described as
\begin{equation}
 \mathbf{v_o} = -\mathbf{Z} \cdot \mathbf{i_o} + \mathbf{m} \cdot v_{src},
 \label{eq:admit_sol}
\end{equation}
where $\mathbf{Z}$ is the \emph{trans-resistance matrix}. %$\mathbf{Z}$ is symmetric for two phase converters.


\subsection{Power losses and trans-resistance parameters}
\begin{figure}[!h]
\centering
\ctikzset { bipoles/length=1cm}
\begin{circuitikz}[american voltages, scale=0.65]
\draw
    (0,0) to[V = $ m_1  v_{src}  $] (0,3)
    (3,3) to[american controlled voltage source,l_=$i_1 z_{11} + i_2 z_{12} $] (0,3)
    (3,3) to[short,i>_=$i_1$,-o] (4,3)
    (4,0) to[short,o-] (0,0)
    (4,3) to[open,v^=$v_1$] (4,0);

\draw
    (8,0) to[V = $ m_2  v_{src}  $] (8,3)
    (11,3) to[american controlled voltage source,l_=$i_1 z_{21} + i_2 z_{22} $] (8,3)
    (11,3) to[short,i>_=$i_2$,-o] (12,3)
    (12,0) to[short,o-] (8,0)
    (12,3) to[open,v^=$v_2$] (12,0);

\end{circuitikz}
\caption{Two output converter.}
\label{fig:model_duality}
\end{figure}
Using the trans-resistance matrix $\mathbf{Z}$ the losses of the converter can be computed. For a two output converter, modeled as shown in Figure~\ref{fig:model_duality}, the losses associated to each output would be
\begin{equation}
 P_{o1} = i_1^2 ~ z_{11} + i_1 ~ i_2 ~ z_{12}
 \label{eq:ploss_1}
\end{equation}

\begin{equation}
 P_{o2} = i_1 ~ i_2 z_{21} + i_2^2 ~ z_{22},
 \label{eq:ploss_2}
\end{equation}
and the total converter losses are
\begin{equation}
 P_{total} = i_1^2 ~ z_{11} + i_2^2 ~ z_{22} +  i_1 ~ i_2 ~ z_{12} ~ z_{21}  .
 \label{eq:ploss_3}
\end{equation}
%\subsubsection{Slow Switching Limit}
Using the the charge flow analysis  described in the previous section, the total losses of a two output converter can be computed as well. In order to make the analysis less cumbersome, the phases are eluded and losses are computed in a single capacitor for the SSL region. The results can be extended for any converter with any number of phases and capacitors.


In the case of a multiple-output converter, each of the individual outputs produces a \emph{redistributed} charge flow through the capacitors that can be individually quantified, being $g_{i,1}$  the \emph{redistributed} cahrge flow multiplier associated to the first output, $g_{i,2}$ associated to the second output. The total \emph{redistributed} charge is the sum of each individual contributions as
\begin{equation}
 g_i =  (g_{i,1} ~ q_{o,1} +  g_{i,2} ~ q_{o,2}).
 \label{eq:g_i_total}
\end{equation}
Substituting~\eqref{eq:g_i_total} in~\eqref{eq:pwr_ssl} the losses produced in capacitor $c_i$ of the two output converter are
\begin{equation}
 P_{c_{i}} = f_{sw} \frac{1}{2 ~ c_i} (g_{i,1} ~ q_{o,1} +  g_{i,2} ~ q_{o,2})^2.
 \label{eq:ploss_c_2}
\end{equation}
expanding terms and substituting $q_{o,1}=i_1/f_{sw}$ and $q_{o,2}=i_2/f_{sw}$ into~\eqref{eq:ploss_c_2}  yelds
\begin{equation}
 P_{c_{i}} =  \frac{1}{2 ~ f_{sw} ~ c_i} (i_1^2 ~g_{i,1}^2  +  i_2^2 ~ g_{i,2}^2 + 2 ~ i_{1} ~ i_{2} ~ g_{i,1}~g_{i,2} ).
 \label{eq:ploss_c_3}
\end{equation}
It can be seen that the trans-resistance parameters of~\eqref{eq:ploss_3} can be directly matched with the \emph{redistributed charge flow multipliers} in ~\eqref{eq:ploss_c_3} as
\begin{center}
    \renewcommand{\arraystretch}{2}
    \begin{tabular} {l c c c }
	$z_{11}$ & = & $\raisebox{0.8ex}{$g_{i,1}^2$}\big/ \raisebox{-0.6ex}{$2 f_{sw} c_i$}$ & $[\Omega] $\\
	$z_{22}$ & = & $\raisebox{0.8ex}{$g_{i,2}^2$}\big/ \raisebox{-0.6ex}{$2 f_{sw} c_i$} $& $[\Omega]$\\
	$z_{12} + z_{21} $ & = & $\raisebox{0.8ex}{$g_{i,1}g_{i,2}$}\big/ \raisebox{-0.6ex}{$ f_{sw} c_i$} $& $ [\Omega]$
    \end{tabular}
\end{center}
Therefore the general expressions of the SSL trans-resistance parameters are given as a function of the \emph{redistributed charge multipliers} as
\begin{equation}
  z_{ssl,xx} =  \frac{1}{2 f_{sw}} \sum_{i=1}^{caps.} \sum_{j=1}^{phas.}
  \frac{ \left ( g_{i,x}^j \right )^2 } {c_i}.
 \label{eq:z_ssl_xx}
\end{equation}

\begin{equation}
  z_{ssl,xy} + z_{ssl,yx} =  \frac{1}{f_{sw}} \sum_{i=1}^{caps.} \sum_{j=1}^{phas.}
  \frac{g_{i,x}^j g_{i,y}^j}{c_i}.
 \label{eq:z_ssl_xy}
\end{equation}
%\subsubsection{Fast Switching Limit}
The same analysis can be done for the FSL, but in this case the losses are compute for a single resistor.
As in the SSL case of a multiple-output converter, each of the individual outputs produces a charge flow through the switches that can be individually quantified, being $ar_{i,1}$ associated to the first output, $ar_{i,2}$ associated to the second output, etc. The total \emph{switch} charge multiplier is the sum of each individual \emph{switch} multiplier as
\begin{equation}
 ar_i =  (ar_{i,1} ~ q_{o,1} +  ar_{i,2} ~ q_{o,2}).
 \label{eq:ar_i_total}
\end{equation}
Substituting~\eqref{eq:ar_i_total} in~\eqref{eq:pwr_ssl}, the power dissipated in $r_i$ of the two output converter results in
\begin{equation}
 P_{r_{i}} =  \frac{r_i}{D} (i_1^2 ~ar_{i,1}^2  +  i_2^2 ~ ar_{i,2}^2 + 2 ~ i_{1} ~ i_{2} ~ ar_{i,1}~ar_{i,2}),
 \label{eq:ploss_r_1}
\end{equation}
leading to a similar polynomial solution of the previous case. Hence the general expressions for the FSL trans-resistance parameters are
\begin{equation}
  z_{fsl,xx} =   \sum_{i=1}^{swts.} \sum_{j=1}^{phas.}
  \frac{r_{i}}{D^j} \left ( ar_{i,x}^j \right )^2,
 \label{eq:z_fsl_xx}
\end{equation}
\begin{equation}
  z_{fsl,xy} + z_{fsl,yx} =   \sum_{i=1}^{swts.} \sum_{j=1}^{phas.}
  \frac{r_{i}}{D^j} ar_{i,x}^j ar_{i,y}^j,
 \label{eq:z_fsl_xy}
\end{equation}

Notice that~\eqref{eq:z_ssl_xy} and ~\eqref{eq:z_fsl_xy} do not provide the individual expressions for the cross trans-resistance parameters $z_{xy}$ and $z_{yx}$. Actually, the individual quantification of these parameters is related to the sequence order of the different circuit modes for the converter, but this relation has not yet been founded\footnote{Converters with more than 2 phases are beyond the scope of the H-SCC, and so, this dissertation.}. Fortunately,  two-phase converters do not have cardinality  in the sequence of the switching modes, resulting in symmetry of these parameters , and making $\mathbf{Z}$ matrix to be symmetric. Consequently, the generic expressions of the trans-resistance parameters for two phase converters are reduced to two:
\begin{equation}
  z_{ssl,xy}  =  \frac{1}{2~f_{sw}} \sum_{i=1}^{caps.} \sum_{j=1}^{phas.}
  \frac{g_{i,x}^j g_{i,y}^j}{c_i}.
 \label{eq:z_ssl_xy_2ph}
\end{equation}
\begin{equation}
  z_{fsl,xy} =   \sum_{i=1}^{swts.} \sum_{j=1}^{phas.}
  \frac{r_{i}}{D^j} ar_{i,x}^j ar_{i,y}^j,
 \label{eq:z_fsl_xy_2ph}
\end{equation}

%As aforementioned,  for converters with more than two phases, the relation between sequentiality of the circuit modes and the cross trans-conductances has not yet been found, since converters with more that 2-phases are beyond the scope of this work and the H-SCC.

\subsection{Trans-resistance Parameters Methodology}
Based on the \emph{charge flow analysis} for current-loaded SCCs, each converter output has three associated sets of charge flow vectors per switching phase. Thus, for a given converter, the different vector types can be collected in a matrix, where each column corresponds to a converter output and each row corresponds to a circuit component.

Therefore the \emph{charge flow multipliers} are collected in a matrix as
\begin{equation}
 \mathbf{A}^j =
   \bordermatrix { ~ & out_1 & out_2 & ~ & out_n \cr
     v_{src} & a_{1,1}^j  & a_{1,2}^j & \cdots & a_{1,n}^j \cr
     c_1    & a_{2,1}^j  & a_{2,2}^j & \cdots & a_{2,n}^j \cr
      ~     & \vdots     & \vdots & \ddots & \vdots \cr
     c_p    & a_{p,1}^j  & a_{p,2}^j & \cdots & a_{p,n}^j \cr},
 \label{eq:A_matrix}
\end{equation}
where the elements of the first row  $a_{1,x}^j$ corresponds to the \emph{charge flow multiplier}  delivered by the input voltage source associated to the charge flow through the $x$\emph{-th} output. The remaining elements after the first row are associated with the charge flow in the capacitors. Therefore $a_{1,1}$ is the net charge flow in capacitor $c_1$ due to the charge delivered at the $1st$ output node of a converter with $p$ capacitors and $n$ outputs.

Likewise, the \emph{charge pumped multipliers} are collected in the following matrix
\begin{equation}
 \mathbf{B}^j =
   \bordermatrix { ~ & out_1 & out_2 & ~ & out_n \cr
     c_1  & b_{1,1}^j  & b_{1,2}^j & \cdots & b_{1,n}^j \cr
     c_2  & b_{2,1}^j  & b_{2,2}^j & \cdots & b_{2,n}^j \cr
      ~   & \vdots     & \vdots & \ddots & \vdots \cr
     c_p  & b_{p,1}^j  & b_{p,2}^j & \cdots & b_{p,n}^j \cr}
     \label{eq:A_matrix},
\end{equation}
where all the elements are associated with the converter capacitors.

On the other hand, the \emph{switch charge flow multipliers} lead to the following matrix
\begin{equation}
 \mathbf{Ar}^j =
   \bordermatrix { ~ & out_1 & out_2 & ~ & out_n \cr
     sw_1  & ar_{1,1}^j  & ar_{1,2}^j & \cdots & ar_{1,n}^j \cr
     sw_2  & ar_{2,1}^j  & ar_{2,2}^j & \cdots & ar_{2,n}^j \cr
      ~    & \vdots     & \vdots & \ddots & \vdots \cr
     sw_p  & ar_{p,1}^j  & ar_{p,2}^j & \cdots & ar_{p,n}^j \cr}.
 \label{eq:A_matrix}
\end{equation}
where all the elements are associated with the converter switches. This matrix can be extended with the Equivalent Series Resistance (ESR) of the capacitors, but for the sake of clarity they are not included in the present calculations yet.

 The converter is described with two trans-resistance matrix: one for the SSL, $\mathbf{Z_{ssl}}$, and another for the FSL, $\mathbf{Z_{fsl}}$.

\subsection{Slow Switching Limit Trans-resistance Matrix}

The \emph{redistributed} charge flow multipliers matrix can be obtained from the
matrices $\mathbf{A}$ and $\mathbf{B}$  as
\begin{equation}
 \mathbf{G}^j = \mathbf{A}_{(2:end,1:end)}^j - D^j \mathbf{B}^j,
 \label{eq:R_matrix}.
\end{equation}
The \emph{redistributed charge} corresponds to the charge that flows between capacitors; therefore it is the root cause of
losses associated with the SSL operation regime~\cite{Seeman:EECS-2009-78}.

The SSL trans-resistance factors can be individually obtained from the redistributed charge multipliers as described in \eqref{eq:z_ssl_xy_2ph}. In order to obtain directly the trans-resistance matrix, the operation in \eqref{eq:z_ssl_xy_2ph} is performed in  two steps. First, the outer product of  each row of $\mathbf{G}^j$ is taken with itself as
\begin{equation}
 \mathbf{K}_i^j =[\mathbf{G}_{(i,1:end)}^j ]^T \mathbf{G}_{(i,1:end)}^j ,
 \label{eq:K_matrix}
\end{equation}
where the matrix $\mathbf{K_i}$ contains all the possible products of the  $i^{th}$ row. Since each row in $\mathbf{G}$ is associated with a capacitor, there is a matrix $\mathbf{K_i}$ for each capacitor $C_i$.
Second, with the set of $\mathbf{K}$ matrices the trans-resistance matrix is obtained as
\begin{equation}
 \mathbf{Z_{ssl}} = \frac{1}{2 F_{sw}} \sum_{j=1}^{phas.} \sum_{i=1}^{caps.} \frac{1}{C_i} \mathbf{K}_i^j.
 \label{eq:G_ssl}
\end{equation}

\subsection{Fast Switching Limit trans-resistance Matrix}
For the FSL, the trans-resistance matrix is obtained using the switch charge multipliers
contained in matrix $\mathbf{Ar}$. The operation to obtain the trans-resistance matrix as described
in \eqref{eq:z_ssl_xy_2ph} is performed in two steps. First, a set of matrices are obtained by taking the outer product
of each row of $\mathbf{Ar}$ with itself as
\begin{equation}
 \mathbf{Kr}_i^j = \mathbf{Ar}_{(i,1:end)}^j [\mathbf{Ar}_{(i,1:end)}^j]^T,
 \label{eq:Kr_matrix}
\end{equation}
yielding a matrix for each row in $\mathbf{Ar}$ associated with a switch \emph{on}-resistance ($r_{i}$). Second, with the set of matrices $\mathbf{Kr}$ the FSL trans-resistance matrix is obtained as
\begin{equation}
 \mathbf{Z_{fsl}} =  \sum_{i=1}^{swts.} \sum_{j=1}^{phas.} \frac{_{i}}{D^j}
\mathbf{Kr}_i^j,
 \label{eq:G_fsl}
\end{equation}

\subsection{Converter trans-resistance Matrix}
The total trans-resistance values are approximated using~\eqref{eq:r_scc} as
\begin{equation}
 \mathbf{Z}_{(x,y)} \approx \sqrt{\mathbf{Z_{ssl,(x,y)}}^2 + \mathbf{Z_{fsl,(x,y)}}^2}.
 \label{eq:G_total}
\end{equation}

\subsection{Conversion Ratio Vector }

The conversion ratio vector is obtained as
\begin{equation}
 \mathbf{m} = \sum_{j=1}^{phas.}[\mathbf{A}^j_{(1,1:end)}]^T.
 \label{eq:m_equation}
\end{equation}

