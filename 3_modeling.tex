\chapter[Modeling of H-SCC]{Modeling of Hybrid Switched Capacitor Converters}
\label{ch:modeling}
Switch capacitor converters are circuits composed of a large number of switches and capacitors, and require accurate models to properly design them. SCCs have the peculiarity to be lossy by nature due to the non adiabatic energy transfer between capacitors, a phenomenon not present in the inductor based converters. Generally, the modeling of SCCs focuses just on the description of the loss mechanisms associated to conduction and capacitor charge transfer, neglecting other sources of losses such as driving and switching losses. The modeled mechanisms of losses are proportional to the output current, being  normally represented with a resistor in the well-known output resistance model.

This chapter presents an enhancement of the charge flow analysis that extends its use to also cover the H-SCC.   The chapter is divided in two sections, the first section is devoted to the study and model of a H-SCC, where the original charge flow analysis~\cite{95Makowski,Seeman:EECS-2009-78} is reviewed and extended. Firstly, discussing and identifying the limiting factors of the previous published models. Subsequently, the charge flow analysis is reformulated with a new approach that enables the analysis of the H-SCC. The second section is devoted to the study of multiple outputs H-SCCs, introducing a new circuit model, and its related methodology to obtain the circuit model parameters. The chapter closes summarizing the contributions of the new modeling approach.

\section{Single Output Converters}
Switched Capacitor Converters has been always treated as a two-port converter with single input and a single output as shown in Fig.\ref{fig:two_port}. The input port is connected to a voltage source and the output port feeds the load. The SCC provides between input, $v_i$, and output, $v_o$, a voltage conversion, $m$,  that  steps up, steps down or/and inverts the polarity of the input voltage. The current circuit theory  related to SCCs is valid only for the two-port configuration, therefore this section is dedicated to revisit the classical concepts of single output SCC and to enhance them to also cover the H-SCC.

\begin{figure}[!h]
\centering
\ctikzset { bipoles/length=1cm}
\begin{circuitikz}[american voltages,scale=0.65]
\draw
    (1,0) to[short,o-]
    (0,0) to[V = $V_{supply}$]
    (0,3) to[short,-o]
    (1,3) ;

\draw
    (2,3) --
    (2.5,3)

    (2,0) --
    (2.5,0)

    node[ocirc]  (IC)  at (2,0) {}
    node[ocirc]  (I) at (2,3) {}
    (I) to[open,v=$v_{i}$] (IC);


\draw [thick]
    (2.5,-0.5) --
    (2.5,3.5)  --
    (5.5,3.5)  --
    (5.5,-0.5) --
    (2.5,-0.5);

\draw (4,2)node[anchor=north]{$\frac{v_o}{v_{i}}=m$} ;
\draw
    (5.5,3) -- (6,3)
    (5.5,0) -- (6,0)
    node[ocirc]  (O)  at (6,3) {}
    node[ocirc]  (OC) at (6,0) {}
    (O) to[open,v^<=$v_{o}$] (OC);

\draw
    (7,0) to[short,o-]
    (8,0) to[ R= $Load$,mirror]
    (8,3) to[short,-o]
    (7,3) ;
\end{circuitikz}
\caption[Two port converter]{General two port configuration of a Switched Capacitor Converter. }
\label{fig:two_port}
\end{figure}

\subsection{The Output Impedance Model}
\begin{SCfigure}%[!h]
\centering
\ctikzset { bipoles/length=1cm}
\begin{circuitikz}[american voltages, scale=0.65]
\draw
    (-0.5,0) to[V = $ m \cdot v_{src}  $]
    (-0.50,3) -- (0,3) to[R,l=$r_{scc}$,-o]  (3,3)
    (3.5,3) to[short,o-,i=$i_o$]
    (4.25,3)   to[R,l=$r_o$]
    (4.25,0) to[short,-o] (3.5,0)
    (3,0) to[short,o-] (-0.5,0)
    (0,3) to[open,v^=$v_{trg}$] (0,0)
    (3.5,3) to[open,v=$v_{out}$] (3.5,0);

\end{circuitikz}
\caption[Output resistance model]{Output resistance model of a switched capacitor converter.}
\label{fig:scc_model_oi}
\end{SCfigure}
The behavior of SSCs is modeled with the well-known output impedance model~\cite{2000Oota,2012Peter} that is composed of a controlled voltage source and equivalent resistance $r_{scc}$, as shown in Figure~\ref{fig:scc_model_oi}. The output voltage provided by the converter under no-load conditions is defined as \emph{target voltage} ($v_{trg}$). The  controlled voltage source provides the target voltage, being the value of voltage supply $v_{src}$ multiplied by the conversion ratio $m$, thus
\begin{equation}
v_{trg} =  m \cdot v_{src} .
\label{eq:vtrg}
\end{equation}

When the converter is loaded, the voltage at the converter's output, $v_{outs}$, drops proportionally with the load current. This is modeled with resistor $r_{scc}$, which accounts for the losses produced in the converter. Since the losses are proportional to the output current $i_o$, they can be modeled with a resistor. Using the presented model, the output voltage of the converter can be obtained as
\begin{equation}
v_{out} =  m \cdot v_{src} - i_o \cdot r_{scc} .
\label{eq:vout_scc}
\end{equation}

Therefore to solve \eqref{eq:vout_scc} is necessary to obtain the two parameters of the model from the converter: the conversion ration $m$  and the equivalent output resistance $r_{scc}$. The first, can be easily solved using Kirchhoff's Voltage Laws as previously explained in Section~\ref{ch:conversion_ratio}. The second, is more complex and actually is the main challenge in the modeling of SCCs.

Up to day, there are two different methodologies to infer the equivalent output resistance $r_{scc}$, plotted in~\ref{fig:plot_rscc}. On the one hand, S. Ben-Yaakov  ~\cite{2009Ben-Yaakov,2012Ben-Yaakov,2013Evzelman} has claimed a generalized methodology based on the analytical solution of each of the different R-C transient circuits of the converter, reducing all of them to a single transient solution. The methodology achieves a high accuracy, but yields to a set of none linear equations and high complexity for the analysis of advanced architectures.

On the other hand,  M. Makowski and D. Maksimovic~\cite{95Makowski} presented a methodology based on the analysis of the charge flow between capacitors in steady-state. The methodology is simple to apply and yields with a set of linear expressions, being them easy to operate for further analysis of the converters. Based on the charge flow analysis, M.Seeman~\cite{Seeman:EECS-2009-78} developed different metrics allowing to compare performances between capacitive and inductive converters.

Although both methodologies are valid in the modeling of SCCs, none of them has been used to model the effects of a loaded \emph{pwm}-node, which is fundamental to study the H-SCC.  Nevertheless the charge flow analysis has a more clean and simplified way of describing the loss mechanism. For that reason, this methodology has been chosen in this dissertation in order to model the \emph{hybrid} switched capacitor converter.

\sidecaptionvpos{figure}{!b}
\begin{SCfigure} %[!h]
\centering
\begin{tikzpicture}[scale=0.85]
    \begin{loglogaxis}[
        %width=12cm,
        %height=8cm,
        xlabel near ticks,
        ylabel near ticks,
        xlabel= {$f_{sw} ~~ [Hz] $},
        ylabel= {$ [\Omega] $} ,
        axis line style={->},
        axis y line*=left,
        axis x line*=bottom,
        xtick=\empty, ytick=\empty,
        %ytick = {0,.125,.25},
        %yticklabels={0,$v_{src}\frac{1}{3}$,$v_{src}\frac{2}{3}$,$v_{src}$},
        domain=3e4:5e6,
        samples=100,
        %xticklabels={0,$D \cdot T_{sw}$,$T_{sw}$ ,$2 T_{sw}$,$3 T_{sw} $},
        legend style={at={(0.75,0.75)}, anchor= north east},
        enlarge x limits={0.0},
        enlarge y limits={0.0}
        ]

  \newcommand\C{1e-6}
  \newcommand\Resr{1}
  \newcommand\Dx{0.5}
  \newcommand\Rfsl{4*\Resr}
  \newcommand\lbssl{1e5}
  \newcommand\lbfsl{1e6}

  \addplot [thick]   { 1/(2*x*\C)*((exp(\Dx/(\Resr*x*\C))+1)/(exp(\Dx/(\Resr*x*\C))-1) + (exp((1-\Dx)/(\Resr*x*\C))+1)/(exp((1-\Dx)/(\Resr*x*\C))-1))};
  %\addplot [thick,dashed,marker=square] { sqrt((1/(2*\C*x))^2 + \Rfsl^2) };
  \addplot [thick,dashed,domain=3e4:3.5e5] { (1/(\C*x) };
  \addplot [thick,dotted,domain=2.5e4:5e6] { \Rfsl+0*x };
\legend {$r_{scc}$,$r_{ssl}$,$r_{fsl}$};
\end{loglogaxis}

\node[anchor=north] at (6.25cm,1.35cm){$FSL$};
\node[anchor=north,rotate=-60] at (1.25cm,5cm){$SSL$};

\end{tikzpicture}
\caption[Equivalent output resistance curve]{SCC Equivalent output resistance $r_{scc}$ as function of the frequency and the two asymptotic limits: \emph{Slow Switching Limit} (SSL) and \emph{Fast Switching Limit}(FSL). }
\label{fig:plot_rscc}
\end{SCfigure}


As aforementioned $r_{scc}$ accounts for the loss when the converter is loaded. All losses in the converter are, in fact, dissipated in the resistive elements of the converter: \emph{on}-resistance $r_{on}$ of the switches and equivalent series resistance $r_{esr}$ of the capacitors. Nevertheless, the origin and magnitude of the losses depends on the operation region of the converter, which is function of the switching frequency as shown in the plot of Figure~\ref{fig:plot_rscc}.

A SCC has two well-defined regimes of operation: the \emph{Slow Switching Limit} (SSL) and the \emph{Fast Switching Limit} (FSL). Each of the two regimes defines an asymptotic limit for the $r_{scc}$ curve. In SSL, the converter operates at a switching frequency $f_{sw}$ much lower than the time constant $\tau$ of charge and discharge of the converter's capacitors, thereby allowing the full charge and discharge of the capacitors. As shown in Figure~\ref{fig:ic_ssl} the capacitor currents present an exponential-shape waveform. In this regime of operation, the losses are determined by the charge transfer between capacitors, and dissipated in the resistive paths of the converter, mainly in the switches. That is why, reducing the switch channel resistance does not decreases the losses, instead, it will produce sharper discharge currents producing higher electromagnetic disturbances. In SSL, losses are inversely proportional of product between the switching frequency and capacitances, limited by the SSL asymptote as it can be seen in Figure~\ref{fig:plot_rscc}.

In FSL, the converter operates with a switching frequency $f_{sw}$ much higher than the time constant $\tau$ of charge and discharge of the converter's capacitors, limiting the full charge and discharge transients. As shown in Figure~\ref{fig:ic_fsl} currents have block-shape waveforms. In such operation regime, the losses are totally produced by the parasitic resistive elements ($r_{on}$, $r_{esr}$), therefore changes in the capacitances or frequency do not modify the produced losses\footnote{The switching losses are not included in the modeling of $r_{scc}$. }. In FSL, $r_{scc}$ is constant and limited by the FSL asymptote as it can be seen in Figure~\ref{fig:plot_rscc}.

\begin{figure}[!h]
\centering
\ctikzset { bipoles/length=1cm}
\begin{subfigure}[t]{.45\textwidth}
    %\centering
    \raggedright
    \begin{tikzpicture}
            \begin{axis}[
                width={\textwidth},
                height={6cm},
                axis lines=middle,
                xlabel near ticks,
                ylabel near ticks,
                xlabel= {time},
                ylabel= {capacitor current},
                every axis x label/.style={
                    at={(ticklabel* cs:1.05)},
                    anchor=west,
                },
                x axis line style={->},
                y axis  line style={<->},
                xtick=\empty, ytick=\empty,
                %ytick = {0,.125,.25},
                %yticklabels={0,$v_{src}\frac{1}{3}$,$v_{src}\frac{2}{3}$,$v_{src}$},
                domain=-0.25:2,
                samples=100,
                %xticklabels={0,$D \cdot T_{sw}$,$T_{sw}$ ,$2 T_{sw}$,$3 T_{sw} $},
                xmin=-0.1,xmax=2.2,
                ymin=-1,ymax=1.2,
                ]

          \newcommand\xtauA{1/(3*5)};
          \newcommand\ioA{1};
          \newcommand\xtauB{3/(5*5)};
          \newcommand\ioB{-0.75};

          \addplot [thick,]   coordinates { (0,0) (0.1,0) (0.1,\ioA)};
          \addplot [thick,domain=0.1:1]   { \ioA*(  exp(-(x-0.1)/(\xtauA))) };
          \addplot [thick,domain=1:2]   coordinates { (1,0) (1,\ioB)};
          \addplot [thick,domain=1:2]   { \ioB*(  exp(-(x-1)/(\xtauB))) };



        \end{axis}
    \end{tikzpicture}
    \caption{Slow Switching Limit}
    \label{fig:ic_ssl}
\end{subfigure}
\hfill
\begin{subfigure}[t]{.45\textwidth}
    %\centering
    \raggedright
    \begin{tikzpicture}
            \begin{axis}[
                width={\textwidth},
                height={6cm},
                axis lines=middle,
                xlabel near ticks,
                ylabel near ticks,
                xlabel= {$time $},
                ylabel= {capacitor current},
                every axis x label/.style={
                    at={(ticklabel* cs:1.05)},
                    anchor=west,
                },
                x axis line style={->},
                y axis  line style={<->},
                xtick=\empty, ytick=\empty,
                %ytick = {0,.125,.25},
                %yticklabels={0,$v_{src}\frac{1}{3}$,$v_{src}\frac{2}{3}$,$v_{src}$},
                domain=0:2.2,
                samples=100,
                %xticklabels={0,$D \cdot T_{sw}$,$T_{sw}$ ,$2 T_{sw}$,$3 T_{sw} $},
                xmin=-0.1,xmax=2.2,
                ymin=-1,ymax=1.2,
                ]

          \newcommand\xtauA{7)};
          \newcommand\ioA{0.65};
          \newcommand\xtauB{10};
          \newcommand\ioB{-0.35};

          \addplot [thick,]   coordinates { (0,0) (0.1,0) (0.1,\ioA)};
          \addplot [thick,domain=0.1:1]   { \ioA*(  exp(-(x-0.1)/(\xtauA))) };
          \addplot [thick,domain=1:2]   coordinates { (1, { \ioA*exp(-(1-0.1)/(\xtauA))} ) (1,\ioB)};
          \addplot [thick,domain=1:2]   { \ioB*(  exp(-(x-1)/(\xtauB))) };
          \addplot [thick]   coordinates { (2, {\ioB*exp(-(1)/(\xtauB))} ) (2,\ioA)};



        \end{axis}
    \end{tikzpicture}
    \caption{Fast Switching Limit}
    \label{fig:ic_fsl}
\end{subfigure}
\caption[Current waveforms in the converter's capacitors]{Current waveforms though the capacitors in each of the two regimes of operation. }
\label{fig:capacitor_current}
\end{figure}


\subsection{Revising the charge flow analysis}

The charge flow analysis is based on the charge conservation in the converter's capacitors during an entire switching period in steady state~\cite{95Makowski}. The converter is studied in the two well-defined operating regimes: the Slow Switching Limit (SSL) and the Fast Switching Limit (FSL). In SSL, losses are then dominated by the charge transfer between capacitors, therefore only the charge transfer loss mechanisms are studied.  In FSL, losses depend on the conduction through the parasitic resistive elements, therefore only the conduction losses are studied. This division in the study of the converter reduces the complexity of the problem, and enables a simplified but accurate analysis.

In the charge flow analysis, the flowing charges are used instead of the currents. Moreover, the charges are normalized with respect to the total output charge of the converter as
$$
a_i  = \frac{q_i}{q_out}
$$
creating the so-called charge flow multiplier $a_i$ for the charge flowing through the $i$-th component of the converter.


\subsection{Load Model: Voltage Sink versus Current Sink}

In order to model a SCC, the original charge flow method~\cite{95Makowski} makes three main assumptions:
\begin{enumerate}
  \item The load is modeled as an ideal voltage source since it is normally connected to the \emph{dc}-output in parallel with a large capacitor, as shown in Figure~\ref{fig:vsink_load}. Such assumption, eliminates the capacitor connected in parallel with the load, neglecting the effects of the output capacitor into the equivalent output resistance.

  \item The model only considers the \emph{dc}-output as the single load point of the converter, imposing a unique output to the converter.

  \item The phase time ratio is not included in the computation of the capacitor charge flow. Consequently, the modulation of the switching period is assumed to have no influence on the amount of charge flowing in the capacitors.
\end{enumerate}

\begin{figure}[!h]
    \centering
    \ctikzset { bipoles/length=1cm}
    \begin{subfigure}[t]{.4\textwidth}
        \centering
        \begin{circuitikz}[american voltages,scale=0.65]
        \draw
                %Draw Switches
                (1,0)  to[V=$v_{src}$]
                (1,8)  --
                (5,8)   to[switch=$s_1$]
                (5,6)   to[switch=$s_2$]
                (5,4)   to[switch=$s_3$]
                (5,2)   to[switch=$s_4$]
                (5,0)  --
                (1,0)


        %Draw Capacitors
                (5,2) --
                (3,2) to[pC,l_=$c_{1}$]
                (3,6)--
                (5,6)

                (5,0) --
                (7,0) to[V,l_=$v_{out}$]
                (7,4)--
                (5,4);
        \end{circuitikz}
        \caption {Load modeled as a voltage sink.}
        \label{fig:vsink_load}
    \end{subfigure}
    \hfill
    \begin{subfigure}[t]{.4\textwidth}
        \centering
        \begin{circuitikz}[american,scale=0.65]
        \draw
                %Draw Switches
                (1,0)  to[V=$v_{src}$]
                (1,8)  --
                (5,8)   to[switch=$s_1$]
                (5,6)   to[switch=$s_2$]
                (5,4)   to[switch=$s_3$]
                (5,2)   to[switch=$s_4$]
                (5,0)  --
                (1,0)


        %Draw Capacitors
                (5,2) --
                (3,2) to[pC,l_=$c_{1}$]
                (3,6)--
                (5,6)

                (5,0) --
                (7,0) to[pC,l=$c_{2}$]
                (7,4)--
                (5,4)

        %%Sink load
                (7,4) --
                (9,4) to[I,l=$i_{out}$]
                (9,0) --
                (7,0);

        \end{circuitikz}
        \caption {Load modeled as a current sink.}
        \label{fig:isink_load}
    \end{subfigure}
\caption[Two different load models]{Different load models for the charge flow analysis.  }
\label{fig:loads}
\end{figure}

Such assumptions reduce the usability of the model to the specific application of dc-to-dc conversion, and, at the same time, limit the flexibility to model different concepts of the SCC, such as the H-SCC previously introduced in Chapter~\ref{ch:H-SCC}. In order to overcome these limitations, the presented methodology makes two different assumptions:
\begin{enumerate}
  \item The load is assumed to be a constant current source with a value equal to the average load current, as shown in Figure~\ref{fig:isink_load}. In fact, using such approach the charge delivered to the load can be evaluated for each switching phases $j$ as
      \begin{equation}
        q_{out}^j = D^j \frac{i_{out}}{f_{sw}} = D^j i_{out}{T_{sw}}  = D^j q_{out},
      \label{eq:q_out}
      \end{equation}
  where $i_{out}$ is the average output current and $D^j$ is the duty cycle corresponding to the $j$-th phase.


  \item Any of the converter nodes can be loaded. Since the load is modeled as a current sink, it can be now connected to any of the converter's nodes without biasing it.

  \item When the load is connected to a \emph{dc}-node the associated \emph{dc}-capacitor of the node is not longer neglected, thus the effects of the output capacitor are included in the equivalent output resistance.

\end{enumerate}


\subsection{Re-formulating the charge flow analysis}

The equivalent impedance encompasses the root losses produced in the converter due to capacitor charge transfer and charge conduction. As aforementioned, the original charge flow analysis~\cite{95Makowski} assumes an infinitely large output capacitance in parallel with the load. This assumption leads to inaccuracies in the prediction of the equivalent output resistance when the output capacitor is comparable in value to the flying capacitors~\cite{2013Breussegem:c_out}. Actually, the root cause for this inaccuracy relies in the wrong quantification of the charges that produces losses in the converter.\\

\begin{figure}[!h]
\centering
\ctikzset { bipoles/length=1cm}
\begin{subfigure}[t]{.4\textwidth}
    %\centering
    \raggedright
    \begin{circuitikz} [american,scale=0.65]
    \draw
        (0,0) to[V=$v_{src}$] (0,3)
        (3,3) to[pC,l=$c_1$] (0,3)
        (3,0) to[pC,l=$c_2$] (3,3) -- (6,3)
        (6,0) to[pC,l=$c_3$] (6,3) --
        (8,3) to[I,l=$i_o$] (8,0) -- (0,0);
    \begin{scope}[>=latex,thick,text=black]
        \draw [->,rounded corners=7pt,dashed]
            (0.4,2.4) -- (2.4,2.4) -- (2.4,0.4);
        \draw [->,rounded corners=7pt,dashed]
            (3.6,0.5) |- (5.4,2.7) -- (5.4,0.4);
        \draw [->,rounded corners=7pt]
             (6.3,2) |- (8,3.3);
        \draw [>=latex,text=black,dashed]
          (0,4)  -- (0.9,4) node[anchor=west]{Redistributed charge};
        \draw [>=latex,text=black]
          (0,4.5)  -- (0.9,4.5) node[anchor=west]{Pumped charge};
    \end{scope}
    \end{circuitikz}
    \caption{}
\end{subfigure}
\hfill
\hfill
\begin{subfigure}[t]{.4\textwidth}
    %\centering
    \raggedleft
    \begin{circuitikz} [american,scale=0.65]
    \draw
        (1,0) to[V=$v_{src}$] (1,3)
        (6,3) to[pC,l=$c_2$] (3,3)
        (3,0) to[pC,l=$c_1$] (3,3)
        (6,0) to[pC,l=$c_3$] (6,3) --
        (8,3) to[I,l=$i_o$] (8,0) -- (1,0);
    \begin{scope}[>=latex,thick,text=black]
        \draw [->,rounded corners=7pt,dashed]
            (3.6,0.5) |- (5.4,2.4) -- (5.4,0.4);
        \draw [->,rounded corners=7pt]
            (6.3,2) |- (8,3.3);% -- (8.6,0.4);

    \end{scope}
    \end{circuitikz}
    \caption{}
\end{subfigure}
\caption[Charge flow in a 3:1 Dickson with an infinite output capacitor.]{Charge flows in a Dickson 3:1 converter when loaded at a \emph{dc}-node with a infinitely large output capacitor $c_3$ during the two switching phases. }
\label{fig:charge_flow_I}
\end{figure}

Looking, in detail, the charge flow in a SCC, we can identify two different \emph{real} charge flows during each circuit mode:
\begin{description}

  \item[Redistributed charge] flows between capacitors in order to equalize their voltage differences, being them the source of losses. Therefore evaluating them the capacitor charge losses can be obtained.

      This charge flow is associated with a charge or discharge of the capacitors, happening right after the switching event and lasting for a short period of time\footnote{The duration of the charge depends on the time constant of the associated R-C circuit.}.

  \item[Pumped charge] flows from the capacitors to the load, this charge is consumed by the load, hence producing useful work.  This charge delivery is associated with a discharge of the capacitors, lasting for the entire phase time.

\end{description}

Besides these two charge flows, there is a third \emph{theoretical} charge flow that is necessary to analyse and solve the converter:
\begin{description}

  \item[Net charge] flow is quantified based on the principle of \emph{capacitor charge balance} for a converter in steady state. Based on that principle all \emph{net} charges in the capacitors can be obtained applying KCL, but using charges instead of currents. Therefore, the circuit can be solved for the \emph{net} charges flow applying the \emph{capacitor charge balance} as
      \begin{equation}
       \forall~c_{i} : \sum_{j=1}^{phases}q_{i}^j = 0,
      \label{eq:charge_balance}
      \end{equation}

     The resulting charges are then gathered in the charge flow vector $\mathbf{a}$ as
       \begin{equation}
        \mathbf{a}^j =  \left[ a_{in}^j~a_1^j~a_2^j \cdots a_n^j \right] = \frac{\left[ q_{in}^j~q_1^j~q_2^j \cdots q_n^j \right]}{q_{out}},
      \label{eq:a_vector}
      \end{equation}
    where the superindex denotes the $j$-th phase, $q_{in}$ is the charge supplied by the voltage source and $q_i$ is the \emph{net} charge flowing in the $i$-th capacitor $c_i$. Notice that the vector is normalized with respect to the output charge $q_{out}$.
\end{description}


The loss mechanisms of SCCs can be better understood based on the \emph{redistributed} and \emph{pumped} charge flows. For instance Figure~\ref{fig:charge_flow_I} shows the charge flows for a 3:1 Dickson converter with a infinitely large output capacitor $c_3$. In such converter, the charge flow through capacitors $c_1$ and $c_2$ is always either redistributed between them or towards the big capacitor $c_3$, and only capacitor $c_3$ supplies charge to the load. Therefore since the flowing charge in $c_1$ and $c_2$ is always transferred between capacitors, it produces losses and it never supplies directly the load. However, for a finite value of the output capacitor, or for converters loaded from an internal node, there is always the probability that all capacitors contribute to pumping charge to the load~\cite{2013Breussegem:c_out}; phenomenon that was not considered in the initial charge flow analysis.
\begin{figure}[!h]
\centering
\ctikzset { bipoles/length=1cm}
\begin{subfigure}[t]{.4\textwidth}
    %\centering
    \raggedright
    \begin{circuitikz} [american,scale=0.65]
    \draw
        (0,0) to[V=$v_{src}$] (0,3)
        (3,3) to[pC,l=$c_1$] (0,3)
        (3,0) to[pC,l=$c_2$] (3,3) -- (6,3)
        (6,0) to[pC,l=$c_3$] (6,3) --
        (8,3) to[I,l=$i_o$] (8,0) -- (0,0);
    \begin{scope}[>=latex,thick,text=black]
        \draw [->,rounded corners=7pt,dashed]
            (0.4,2.4) -- (2.4,2.4) -- (2.4,0.4);
        \draw [->,rounded corners=7pt,dashed]
            (3.6,0.5) |- (5.4,2.7) -- (5.4,0.4);

        \draw [->,rounded corners=7pt]
            (2,3.3) -- (7,3.3)
            (2.7,2) |- (7.5,3.3)
            (6.3,2) |- (8,3.3);
        \draw [>=latex,text=black,dashed]
          (0,4)  -- (0.9,4) node[anchor=west]{Redistributed charge};
        \draw [>=latex,text=black]
          (0,4.5)  -- (0.9,4.5) node[anchor=west]{Pumped charge};
    \end{scope}


    \end{circuitikz}
    \caption{}
\end{subfigure}
\hfill
\hfill
\begin{subfigure}[t]{.4\textwidth}
    %\centering
    \raggedleft
    \begin{circuitikz} [american,scale=0.65]
    \draw
        (1,0) to[V=$v_{src}$] (1,3)
        (6,3) --  (3,3)
        (3,0) to[pC,l=$c_1$] (3,3)
        (6,0)-- (6,0.25) to[pC,l=$c_3$] (6,1.5) to[pC,l=$c_2$] (6,2.75) |-
        (8,3) to[I,l=$i_o$] (8,0) -- (1,0);
    \begin{scope}[>=latex,thick,text=black]
        \draw [->,rounded corners=7pt,dashed]
            (3.6,0.5) |- (5.4,2.7) -- (5.4,0.4);
        \draw [->,rounded corners=7pt]
             (2.7,2) |- (7.5,3.3)
             (6.3,2.5) |- (8,3.3);% -- (8.6,0.4);
    \end{scope}
    \end{circuitikz}
    \caption{}
\end{subfigure}
\caption[Charge flow in a 3:1 H-Dickson.]{Charge flows in a Dickson 3:1 converter when loaded at one of the \emph{pwm}-nodes during the two switching phases. }
\label{fig:charge_flow_II}
\end{figure}

In another scenario, the one of Figure~\ref{fig:charge_flow_II},  a 3:1 H-Dickson with the load connected to second ~\emph{pwm}-node. In such converter, there is a redistributed charge flow between the different capacitors as in the previous case, but at the same time, all capacitors pump charge to the load as well. Therefore all capacitors  contribute in delivering charge to the load, which actually reduces the equivalent output impedance of the converter.

The original charge flow analysis only used the \emph{net} charge flow in order to quantify the losses produced in the SSL region , which in fact leaded in an over estimation of the charge flow responsible of the losses, the \emph{redistributed} charge flow. The proposed methodology in this dissertation identifies these different flows of charges, and by quantifying each of them independently achieves a closer estimation of the losses in a SCC.

\begin{figure}[!h]
\centering
% This file was created by matlab2tikz.
%
%The latest updates can be retrieved from
%  http://www.mathworks.com/matlabcentral/fileexchange/22022-matlab2tikz-matlab2tikz
%where you can also make suggestions and rate matlab2tikz.
%

\begin{tikzpicture}

\begin{axis}[%
width=8cm,
height=4cm,
at={(1.532in,0.729in)},
scale only axis,
xmin=0,
xmax=100,
ymin=6.64,
ymax=6.673,
axis background/.style={fill=white},
axis x line*=bottom,
axis y line*=none
]
\addplot [thick,solid,forget plot]
  table[row sep=crcr]{%
1	6.65026844706453\\
2	6.6541906445053\\
3	6.65706994984393\\
4	6.65925863469407\\
5	6.66097190927411\\
6	6.66234478933293\\
7	6.66346632700282\\
8	6.66439739781252\\
9	6.66517963724446\\
10	6.66584465605996\\
11	6.66641623072511\\
12	6.66691268208139\\
13	6.66734838569269\\
14	6.6677347707153\\
15	6.66808100843884\\
16	6.66839450570606\\
17	6.66868127063699\\
18	6.66894619123669\\
19	6.66919325215929\\
20	6.66942570599671\\
21	6.66964621014256\\
22	6.66985693700001\\
23	6.67005966319872\\
24	6.67025584208054\\
25	6.67044666273543\\
26	6.67063309816195\\
27	6.67081594459879\\
28	6.67099585366935\\
29	6.67117335866543\\
30	6.67134889604556\\
31	6.67152282302303\\
32	6.67169543195692\\
33	6.67186696212833\\
34	6.67203760937764\\
35	6.67220753399144\\
36	6.66994898046201\\
37	6.66832576586377\\
38	6.66712775868683\\
39	6.66621535139474\\
40	6.66549387715858\\
41	6.66490036899855\\
42	6.66439286198158\\
43	6.6639431930172\\
44	6.66353165635979\\
45	6.66314608322637\\
46	6.6627778080521\\
47	6.66242109657901\\
48	6.66207215837762\\
49	6.6617284123912\\
50	6.66138813136178\\
51	6.66105016231045\\
52	6.66071373588587\\
53	6.66037833874885\\
54	6.66004362838354\\
55	6.659709376253\\
56	6.65937542987049\\
57	6.65904168749229\\
58	6.6587080812319\\
59	6.65837456579342\\
60	6.65804111199682\\
61	6.65770769693504\\
62	6.65737430934867\\
63	6.65704093984341\\
64	6.65670758237776\\
65	6.65637423294288\\
66	6.65604088886614\\
67	6.65570754836448\\
68	6.65537421024822\\
69	6.65504087372358\\
70	6.65470753826091\\
71	6.65437420350682\\
72	6.65404086922551\\
73	6.65370753525966\\
74	6.6533742015043\\
75	6.65304086788937\\
76	6.65270753436815\\
77	6.65237420090945\\
78	6.65204086749247\\
79	6.6517075341038\\
80	6.65137420073293\\
81	6.65104086737468\\
82	6.65070753402473\\
83	6.65037420068031\\
84	6.65004086733958\\
85	6.64970753400131\\
86	6.64937420066469\\
87	6.64904086732916\\
88	6.64870753399436\\
89	6.64837420066005\\
90	6.64804086732606\\
91	6.64770753399229\\
92	6.64737420065867\\
93	6.64704086732514\\
94	6.64670753399168\\
95	6.64637420065826\\
96	6.64604086732487\\
97	6.6457075339915\\
98	6.64537420065814\\
99	6.64504086732479\\
100	6.64470753399144\\
};
\end{axis}
\end{tikzpicture}%
\caption[Voltage ripple in the SCC's capacitors]{Two possible voltage waveforms that show the capacitors in a SCC. Ripples are associated with the charge flow mechanisms: top) unipolar capacitor discharge (DC capacitor); bottom) bipolar capacitor discharge (flying capacitor).}
\label{fig:cap_riples}
\end{figure}

The nature and effects of the three different charge flow can be better understood by looking at the voltage waveforms in the converter's capacitors during an entire switching cycle. From Figure~\ref{fig:cap_riples}, we can associate the voltage ripples to the previously defined charge flows:
\begin{description}
  \item[Net voltage ripple $\Delta vn$] is the voltage variation measured at the beginning and at the end of the switch events. As a matter of fact, this \emph{net} ripple can be computed from the null \emph{charge balance} in a capacitor in steady-state condition as
      \begin{equation}
        \Delta {vn}^j_i  = \frac{q_i ^j }{c_i}.
        \label{eq:net_voltage}
      \end{equation}
      Using (\ref{eq:a_vector}) the \emph{net} ripple can be formulated using the charge flow notation
      \begin{equation}
        \Delta {vn}^j_i  = \frac{a_i ^j }{c_i} {q_{out}}.
        \label{eq:net_voltage_cf}
      \end{equation}

      Notice that \emph{capacitor charge balance} principle is reflected in the \emph{net }voltage ripples of Figure~\ref{fig:cap_riples}. Thus the sum of all \emph{net} ripples of each capacitor during a switching cycle  must be zero; that is why $\Delta vn^1 = \Delta vn^2$ in the two phase converter used in the example of Figure~\ref{fig:cap_riples}.

  \item[Pumped voltage ripple $\Delta vp$] is the voltage variation associated with the discharge of the capacitor by a constant current. Thanks to modeling the load as current sink, it can be identified by the linear voltage discharge, thus the \emph{pumped} ripple can be obtained for each switching phase as
      \begin{equation}
        \Delta {vp}^j_i  = D^j \frac{i_i^j}{c_i }T_{sw},
      \label{eq:pumped_voltage}
      \end{equation}
      where $i_i^j$ is the current flowing through the $i$-th capacitor $c_i$. Actually, the current flowing in each individual capacitor $c_i$ during each $j$-th phase can be expressed as function of the output current by solving the network of capacitors associated to the circuit of each mode, thus
      \begin{equation}
        i_i^j = b_i^j i_{out} ,
      \label{eq:b_cnst}
      \end{equation}
      where $ b_i^j $ is a constant coming from solving the capacitor network.  Replacing (\ref{eq:b_cnst}) and (\ref{eq:q_out}) into (\ref{eq:pumped_voltage}), the \emph{pumped} voltage ripple can be expressed in the charge flow notation as
      \begin{equation}
        \Delta {vp}^j_i  = D^j \frac{b_i^j}{c_i } {i_{out}} {T_{sw}} = D^j \frac{b_i^j}{c_i } {q_{out}}.
      \label{eq:pumped_voltage_cf}
      \end{equation}

      Like in the previous case with the \emph{net} charge flow, the $b_i^j$ elements are gathered in the \emph{pumped} charge flow vector $\mathbf{b}$ as

      \begin{equation}
        \mathbf{b}^j =  \left[ b_1^j~b_2^j \cdots b_n^j \right] = \frac{\left[ ~i_1^j~i_2^j \cdots i_n^j \right]}{i_{out}},
      \label{eq:b_vector}
      \end{equation}

      where the superindex denotes the $j$-th phase, $i_i$ is the \emph{pumped} current flowing in the $i$-th capacitor $c_i$. The vector is normalized with respect to the output current $i_{out}$. Notice that $b$ vector is dual for currents or charges.


  \item[Redistributed ripple $\Delta vr$ ]is the voltage variation associated to an exponential charge or discharge transient. Produced by the charge redistribution between capacitors and happening just right after the phase transition event. The \emph{redistribution} ripple can be quantified by the addition of the two previous ripples as
      \begin{equation}
        \Delta {vr}^j_i  = \Delta {vn}^j_i + \Delta {vp}^j_i .
      \label{eq:rdst_ripple_I}
      \end{equation}
      Substituting (~\ref{eq:net_voltage_cf}) and (\ref{eq:pumped_voltage_cf}) into (\ref{eq:rdst_ripple_I}) the \emph{redistributed} ripple is formulated in terms of the charge flow analysis, as
      \begin{equation}
        \Delta {vr}^j_i  = \frac{q_{out}}{c_i} \left[ a^j_i - D^j b^j_i \right] = \frac{q_{out}}{c_i} g^j_i,
      \label{eq:rdst_ripple_II}
      \end{equation}

      where $g^j_i$ is the \emph{redistributed} charge flow of the $j$-th phase and the $i$-th capacitor. The \emph{redistributed charge flow vector} $\mathbf{g}$ is actually defined as
      \begin{equation}
        \mathbf{g^j}   = \mathbf{a^j_c} - D^j \mathbf{b^j},
      \label{eq:rdst_chrg_flow}
      \end{equation}

      where $\mathbf{a_c}$ is the \emph{capacitor charge flow vector}, a sub-vector of $\mathbf{a}$ that only contains the charge flows corresponding to the capacitors.
\end{description}

In conclusion, in order to study a SCC is necessary to obtain the three charge flow vectors from a converter, which is illustrated in the following section. 

\subsection[Solving the charge flow vectors] { Solving the charge flow vectors}
The charge flow vectors are solved for the converter Figure~\ref{fig:3_1_hscc_solv}, a  3:1 H-Dickson loaded at second node.  
\begin{SCfigure}[][h]
\ctikzset { bipoles/length=1cm}
\centering
    \begin{circuitikz}[american,scale=0.6]

    \draw
            (0,0) to[V=$v_{src}$] (0,7.5) -- (5,7.5)
            (5,7.5)  to[switch=$s_1$] %S1
            (5,6)   to[switch=$s_2$] %S2
            (5,4.5)   to[switch=$s_3$] %S3
            (5,3) --
            %left branch
            (3,3)   to[switch=$s_7$]
            (3,1.5)   to[switch=$s_6$]
            (3,0);

    \draw   %right branch
            (5,3) --
            (7,3)   to[switch,l_=$s_4$]
            (7,1.5)   to[switch,l_=$s_5$]
            (7,0) -- (0,0);


    \draw %Capacitor C1
           (3,1.5) -- (2,1.5)
            to[pC,l_=$c_1$] (2,6) --
           (5,6);

    \draw %Capacitor C2
           (7,1.5) --
           (8.25,1.5)  to[pC,l_=$c_2$](8.25,4.5) --
           (5,4.5)
           (8.25,4.5) -- (10,4.5) to[I=$i_{out}$] (10,0) -- (5,0);
           

    \draw %Capacitor C3
           (5,0) to[pC,l_=$c_3$] (5,3);

     \end{circuitikz}
 \caption{ H-SCC with a 3:1 H-Dickson with the load connected to the second \emph{pwm}-node.}
 \label{fig:3_1_hscc_solv}
\end{SCfigure}

The \emph{net} charge flow vectors are composed by the solution of the charges flowing in the capacitors and input supply assuming the null charge balance in the capacitors. Therefore considering the two circuit modes of the converter, shown in  Figure~\ref{fig:hscc_phases_charges}, the converter can be solved creating a single system of linear equations. 

The node equations for the first phase (Figure~\ref{fig:hscc_full_p1_slv}) are:
\begin{align}
\label{eqn:ph1_kil}
\begin{split}
  q_{in}^1 - q_1^2   &=0, \\
  q_1^2 - q_2^1 - q_3^1 - q_{out}^1 &=0. 
\end{split}
\end{align}



\begin{figure}[!h]
\centering
\ctikzset { bipoles/length=1cm}
%\ctikzset { scale=0.5}
\begin{subfigure}[t]{\textwidth}
    \centering
    %\ctikzset { bipoles/length=1cm}
        \begin{circuitikz}[american,scale=0.6]
       \draw %Input Supply
                (0,0) to[V=$v_{src}$,i=$q_{in}^1$]  (0,3)
                (4,3) to[pC,l_=$c_1$,i<_=$q_1^1$,v^>=$v_1$]     (0,3)
                (4,0) to[pC=$c_2$,i<=$q_2^1$,v>=$v_2$]     (4,3) -- (7,3)
                (3,0) -- (7,0) to[pC=$c_3$,i<=$q_3^1$,v>=$v_3$]  (7,3) to[short,i>=$q_{out}^1$]
                (9,3) to[I=$i_{out}$] (9,0) -- (0,0);




         \end{circuitikz}
     \subcaption{First mode, odd switches are closed and even switches are open.}
     \label{fig:hscc_full_p1_slv}
     \end{subfigure}

\begin{subfigure}[t]{\textwidth}
      \centering
      \begin{circuitikz}[american,scale=0.6]
        \draw (0,4.5) node[anchor=north]{ };
        \draw   %Input Supply
                (-1,0)  to[V=$v_{src}$,i=$q_{in}^2$]
                %Draw Switches
                (-1,4);

        \draw   (5,2) to[pC=$c_2$,i<=$q_2^2$,v>=$v_2$]
                (5,4) to[short,i>=$q_{out}^2$]
                (7,4) to[I=$i_{out}$] (7,0) -- (-1,0);


        \draw %Capacitor C1
               (2,0)to[pC=$c_1$,i<=$q_1^2$,v>=$v_1$](2,4) --(5,4);

        \draw %Capacitor C3
               (5,0) to[pC=$c_3$,i<=$q_3^2$,v>=$v_3$] (5,2);




         \end{circuitikz}
     \subcaption{Second mode, even switches are closed and odd switches are open.}
     \label{fig:hscc_full_p2_slv}
     \end{subfigure}
\caption{The two switching modes of 3:1 H-Dickson of Figure~\ref{fig:3_1_hscc}}
\label{fig:hscc_phases_charges}
\end{figure}


The node equations for second circuit mode (Figure~\ref{fig:hscc_full_p2_slv}) are:
\begin{align}
\label{eqn:ph2_kil}
\begin{split}
  q_{in}^2 & = 0,\\
  q_2^2 - q_3^2    &=0, \\
  q_1^2 - q_2^2 - q_{out}^2 &=0.
\end{split}
\end{align}

Applying~\eqref{eq:q_out} into $q_{out}^1$ and $q_{out}^2$, the phase output charges are expressed as function of the total output charge $q_{out}$, as
\begin{align}
\label{eqn:qout_CL}
\begin{split}
  q_{out}^1 & = D~q_{out} ,\\
  q_{out}^2 & = (1-D)~q_{out},
\end{split}
\end{align}
where $D$ corresponds to the duty cycle of odd switches. The charge flow in the capacitors are constrained to the null charge balance condition of~\eqref{eq:charge_balance}, hence
\begin{align}
\label{eqn:q_i_NCB}
\forall~c_{i} : \sum_{j=1}^{phases}q_{i}^j & \rightarrow
    \begin{cases}
        q_1 \leftarrow q_1^1 =  - q_1^2 & \text{for } c_1;\\
        \\
        q_2 \leftarrow q_2^1 =  - q_2^2 & \text{for } c_2;\\
        \\
        q_3 \leftarrow q_3^1 =  - q_3^2 & \text{for } c_3.    
    \end{cases}
\end{align}

Substituting~\eqref{eqn:qout_CL} and~\eqref{eqn:q_i_NCB} into~\eqref{eqn:ph1_kil} and~\eqref{eqn:ph1_kil}, we can formulate a system of linear equations as
\begin{equation}
  \syssubstitute{.,{a_1}{q_{in}^1}{a_2}{q_{in}^2}{b_1}{q_1}{b_2}{q_2}{b_3}{q_3}}
  \systeme{
    a_1  - b_1  = 0,
    a_2 = 0,
    b_1 - b_2  - b_3  = D q_{out},
    b_1 + b_2  =  - (1-D) q_{out},
    b_2 - b_3  = 0},
\end{equation} 
solving the system yields
\begin{align}
\label{eqn:qi_rslt}
    \begin{split}
        q_{in}^1  = q_{1} & = \frac{2 -D}{3} q_{out} ,\\
        q_{2}     = q_3   & = \frac{1 - 2D}{3} q_{out}.
    \end{split}
\end{align}

Substituting~\eqref{eqn:qi_rslt} into~\eqref{eq:a_vector}, the \emph{net} charge flow vectors are solved 
\begin{align}
\mathbf{a}^1 & = \frac{1}{3}\irow{2 - D & 2 - D & 1 - 2D & 1- 2D}, \\
\mathbf{a}^2 & = \frac{1}{3}\irow{0 & D - 2  & 2D -1  & 2D -1 }.
\label{eq:a_31dikson}
\end{align}

The \emph{pumped} charge flow multipliers are obtained by solving the currents of each circuit mode isolated to the others modes. For sake of brevity, only the circuit associated to the first mode of the converter will be solved in detail. The sing conventions for voltages and currents are defined in Figure~\ref{fig:hscc_full_p1_slv}, although instead of using charges $q_x$ the circuit will be solved for currents $i_x$. We can formulate two node equations,
\begin{align}
\label{eqn:b_param_kcl}
\begin{split}
  i_{in} - i_1 & = 0,\\
  i_1  - i_2 - i_3 - i_{out} &=0,
\end{split}
\end{align}
and two more net equations
\begin{align}
\label{eqn:b_param_kvl}
\begin{split}
  v_{src} - v_1 - v_2  & = 0,\\
  v_2 - v_3&=0.
\end{split}
\end{align} 
Owing to the fact that the relation current-voltage in a capacitor is $c \frac{dv}{dt} = i$, and using the net equations~\eqref{eqn:b_param_kvl} we can define the relations between currents as follows
\begin{align}
\label{eqn:b_i_rel}
\begin{split}
  i_2 & = i_1 \frac{c_2}{c_1},\\
  i_3 & = i_2 \frac{c_3}{c_2} = i_1 \frac{c_3}{c_1}.  
\end{split}
\end{align}
Substituting~\eqref{eqn:b_i_rel} into~\eqref{eqn:b_param_kcl} and isolating, we obtain the \emph{pumped} charge flow multiplier for $c_1$ phase $1$:
\begin{equation}
  i_1  = i_o \frac{c_1}{c_1+c_2+c_3} = i_o b_1^1.
\label{eqn:b_c1_p1}
\end{equation}
The rest of the \emph{pumped} charge multipliers can be found in the same way. Arranging them in the corresponding vector form, will result in:
\begin{align}
\mathbf{b}^1 & = \frac{1}{c_1+c_2+c_3}\irow{ c_1 & -c_2 & -c_3 }, \\
\mathbf{b}^2 & = \frac{-1}{c_1~c_2+c_1~c_3+c_2~c_3}\irow{ c_1~c_2 + c_1~c_3 & c_2~c_3 &  c_2~c_3  }.
\label{eq:b_31dikson}
\end{align}







\subsection[SSL Equivalent Resistance ]{Slow Switching Limit Equivalent Resistance}

The SSL equivalent output resistance $r_{ssl}$ accounts for the losses produced by the capacitor charge transfer, therefore $r_{scc}$ can be obtained evaluating the losses in the capacitors.  The energy lost in a charge or discharge of capacitor $c$ is given by
\begin{equation}
E_{loss}=\frac{1}{2}{{\Delta{v}}_c}^2 c.
\label{eq:e_lost}
\end{equation}
where $\Delta v_c$ is the voltage variation in the process. Previously, we defined that the \emph{redistributed} ripple is associated to the capacitor charge transfer, thus by substituting (\ref{eq:rdst_ripple_II}) into (\ref{eq:e_lost}) we obtain the losses due to capacitor charge transfer
\begin{equation}
E_i^j=\frac{1}{2}{({\Delta{vr}}_i^j)}^2 c_i = \frac{1}{2}\frac{{q_{out}}^2}{{c_i}^2}{\left[a_{i\
}^j-{D^j} {b_i^j}\right]}^2c_i=\frac{1}{2}\frac{{q_{out}}^2}{c_i}{\left[a_{i\
}^j-{D^j} {b_i^j}\right]}^2 .
\label{eq:e_lost_ssl}
\end{equation}
The total power loss in the circuit is the sum of the losses in all of the
capacitors during each phase multiplied by the switching frequency$f_{sw}$.
This yield{\small s}
\begin{equation}
P_{ssl}= f_{sw} \sum_{i=1}^{caps.}\sum_{j=1}^{phases} E_i^j =\frac{f_{sw}{q_{out}}^2}{2}\sum_{i=1}^{caps.}\sum_{j=1}^{phases}\frac{1}{c_i}{\left[a_{i\
}^j-{D^j}{b_i^j}\right]}^2.
\label{eq:pwr_ssl}
\end{equation}

The losses can be expressed as the output SSL resistance by dividing~\eqref{eq:pwr_ssl} by the
square of the output current as
\begin{equation}
r_{ssl}=\frac{P_{ssl}}{{i_o}^2}=\frac{P_{ssl}}{{(f_{sw} {q_{out}})}^2}=\frac{1}{2 f_{sw}}\sum_{i=1}^{caps.}\sum_{j=1}^{phases}\frac{1}{c_i}{\left[a_{i\
}^j-{D^j} {b_i^j}\right]}^2.
\label{eq:r_ssl}
\end{equation}


\subsection[FSL Equivalent Resistance]{Fast Switching Limit Equivalent Resistance}
The fast switching limit (FSL) equivalent output resistance $r_{fsl}$ accounts for losses produced in the resistive circuit elements, being these the \emph{on}-resistance of the switches and the Equivalent Series Resistance (ESR) of the capacitors $r_{esr,c}$.

The power dissipated by resistor $r_i$  from a square-wave pulsating current is given by
\begin{equation}
P_{r_i} = r_i~D^j~i_i^2,
\label{eq:pwr_r}
\end{equation}
where $D^j$ is the duty cycle. The value of $i_i$ (peak current) though the resistor can be also defined by its flowing charge $q_i$ as
\begin{equation}
i_i = \frac{q_i}{D^j~T_{sw}} = \frac{q_i}{D^j} f_{sw}.
\label{eq:i_q}
\end{equation}
As outlined in~\cite{Seeman:EECS-2009-78}, the charge flowing through the parasitic resistive elements can be derived from the charge flow vectors ($\mathbf{a}$), providing the \emph{switch}\footnote{These charge flow vectors also account for other resistive elements, not only the switches, such as the capacitors' equivalent series resistance.} charge flow vectors $\mathbf{ar}$. Using the \emph{switch} charge flow multiplier, \eqref{eq_i_q} can be redefined as function of the output charge (or the output current) as
\begin{equation}
i_i = \frac{ar_i^j}{D^j} q_{out}~f_{sw} = \frac{ar_i^j}{D^j} i_{out}.
\label{eq:i_ar}
\end{equation}
Substituting~\eqref{eq:i_ar} into~\eqref{eq:pwr_r} yields
\begin{equation}
P_{r_i} = \frac{r_i}{D^j}{ar_i^j}^2 i_{out}^2 ,
\label{eq:pwr_r_ar}
\end{equation}
the total loss accounting all resistive elements and phases is then
\begin{equation}
P_{fsl} = \sum_{i=1}^{elm.} \sum_{j=1}^{phs.}  \frac{r_i}{D^j}{ar_i^j}^2 i_{out}^2,
\label{eq:pwr_fsl}
\end{equation}
dividing by $i_{out}^2$ yields the FSL equivalent output resistance:
\begin{equation}
r_{fsl}=\sum_{i=1}^{elm.}\sum_{j=1}^{phases}\frac{r_i}{D^j}{ar_i^j}^2
\label{eq:r_fsl}
\end{equation}
where $r_i$ is the resistance value of the $i$-th resistive element.


\subsection{Equivalent Switched Capacitor Converter Resistance}
\label{ch:rscc_apprx}
With the goal of obtaining a simple design equation, a first analytical approximation of $r_{scc}$ in~\cite{1998Arntzen,1999Maksimovic} was given as

\begin{equation}
r_{scc} \approx \sqrt{{r_{ssl}}^2+{r_{fsl}}^2},
\label{eq:r_scc}
\end{equation}
being used in all the presented results of this dissertation.

\begin{SCfigure}[][h]
\ctikzset { bipoles/length=1cm}
    \begin{circuitikz} [american,scale=0.65]
    \draw
        (0,0) to[V=$v_{src}$]
        (0,3) to[switch,l=$s_1$]
        (2,3) to[R,l=$r_1$]
        (3.5,3) -- (4,3) to[C,l=$c$] (4,0)
        (4,3) -- (4.5,3) to[R,l=$r_2$] (6,3)
        (6,3) to[switch,l=$s_1$] (8,3)
        (0,0) -- (8,0) to[V_=$v_{out}$] (8,3) ;
    \end{circuitikz}
    \caption{Single capacitor converter.}
    \label{fig:single_capacitor}
\end{SCfigure}

\citeauthor{2012Makowski}, in a recent publication~\cite{2012Makowski}, claimed a \emph{better} approximation as
\begin{equation}
r_{scc,Mak} \approx \sqrt[\leftroot{-3}\uproot{3} \mu]{{r_{ssl}}^{\mu}+{r_{fsl}}^{\mu}},
\label{eq:r_scc_II}
\end{equation}
where $\mu = 2.54$, being the value obtained using the \emph{Minkowski distance} form
\begin{equation}
r_{elbow} = \left( {r_x}^{\mu}+{r_x}^{\mu} \right) ^\frac{1}{\mu} = 2^\frac{1}{\mu} r_x = p~r_x
\label{eq:r_scc_II}
\end{equation}
at the corner frequency, where $r_x = r_{sssl} = r_{fsl}$,  of a single lossy capacitor under periodic voltage square excitation in steady-sate (see schematic  in Figure~\ref{fig:single_capacitor}), which the closed expression of the equivalent output resistance is
\begin{align}
r_{scc} & =  \frac{1}{2~c~f_{sw}} \left[ \frac{\me^{\frac{D}{\tau_1~f_{sw}}}+1}{\me^{\frac{D}{\tau_1~f_{sw}}}-1} +
\frac{\me^{\frac{1-D}{\tau_2~f_{sw}}}+1}{\me^{\frac{1-D}{\tau_2~f_{sw}}}-1} \right],\\
\\
\tau_1 = r_1~c,\\
\\
\tau_2 = r_2~c.
\label{eq:r_scc_CF}
\end{align}
This formulation has its best accuracy when the converter operates close to $50\%$ duty cycle for a converter with an homogenous time constant ($\tau$) across all capacitors (see Figure~\ref{ig:rscc_aprox_homo}). The accuracy of this approximation is extended to the full range if $\mu$ is solved as function of $D$, given by
\begin{align}
p & = \frac{1}{2} \left[ \frac{\me^{\frac{1}{D}+1}}{\me^{\frac{1}{D}-1}} + \frac{\me^{\frac{1}{1-D}+1}}{\me^{\frac{1}{1-D}-1}} \right],
\\
\\
\mu &= \frac{1}{\log_2 p}.
\label{eq:u_factor}
\end{align}



\begin{figure}[!h]
\newcommand\pHeigh{3.25cm}
\newcommand\pWidth{4.5cm}
\centering
    \begin{subfigure}{\textwidth}
       \parbox[b]{.45\linewidth}{
            \raggedright
            \newcommand\dutyCycle{10}
            \newcommand\uDx{1.74}
            % This file was created by matlab2tikz.
%
%The latest updates can be retrieved from
%  http://www.mathworks.com/matlabcentral/fileexchange/22022-matlab2tikz-matlab2tikz
%where you can also make suggestions and rate matlab2tikz.
%

\begin{tikzpicture}
\pgfplotsset{
    width=\pWidth,
    height=\pHeigh,
    scale only axis,
    ylabel near ticks,
    enlarge y limits={0.2},
    xlabel near ticks,
    ylabel near ticks,
    enlarge x limits={0.15},
    every tick label/.append style={font=\footnotesize},
    yticklabel style={text width=2em,align=right},
}

\begin{semilogxaxis}[
        %xlabel= {$f_{sw}[Hz] $},
        xticklabels={,,},
        ylabel= {$ \epsilon_r ~ [\%] $} ,
        axis y line*=left,
        axis x line*=bottom,
        %yticklabel style={xshift=0.5ex},
        enlarge y limits={0.1},
        title={$D=\dutyCycle\%~(\mu=\uDx)$ },
        title style = {
                at ={(0.5,1.1)},
                font=\footnotesize },
        legend style={
                legend columns = -1,
                at={(0.5,1)},
               anchor=south,
                draw=none,
                font=\tiny,
                column sep=1ex,
                legend cell align = left},
        ]


    \addplot [thin,smooth,black,mark=o,mark repeat=2]
      table [y=y1]{./3_modeling/err_rx_aprox_ORG_hom.dat};\label{pl_MDL}
    \addlegendentry{Org};
    \addplot [thin,smooth,mark=+,mark repeat=2]
      table [y=y1]{./3_modeling/err_rx_aprox_MAK_hom.dat};
    \addlegendentry{Mak};
    \addplot [thin,smooth,mark=x,mark repeat=2]
      table [y=y1]{./3_modeling/err_rx_aprox_RMAK_hom.dat};
    \addlegendentry{*Mak};


\end{semilogxaxis}

\end{tikzpicture}

        }
       \parbox[b]{.45\linewidth}{
            \raggedleft
            \newcommand\dutyCycle{23}
            \newcommand\uDx{2.12}
            \input{./3_modeling/rx_aprox_12.tex}
        }
    \end{subfigure}

    \begin{subfigure}{\textwidth}
       \parbox[b]{.45\linewidth}{
            \raggedright
            \newcommand\dutyCycle{36}
            \newcommand\uDx{2.43}
            \input{./3_modeling/rx_aprox_21.tex}
        }
       \parbox[b]{.45\linewidth}{
            \raggedleft
            \newcommand\dutyCycle{50}
            \newcommand\uDx{2.54}
            \input{./3_modeling/rx_aprox_22.tex}
        }
    \end{subfigure}


\caption{Relative error of a single capacitor switch capacitor with homogenous $\tau$ constants  between the closed form of $r_{scc}$ and the different approximations: \emph{Org} - Original, \emph{Mak} - Makowski and \emph{*Mak} - rectified Mackowski.  Solved for the circuit in Figure~\ref{fig:single_capacitor} with $c=1\mu F$ and $r_1=r_2=1\Omega$.}
\label{fig:rscc_aprox_homo}
\end{figure}


\begin{figure}[!h]
\newcommand\pHeigh{3.25cm}
\newcommand\pWidth{4.5cm}
\centering
    \begin{subfigure}{\textwidth}
       \parbox[b]{.45\linewidth}{
            \raggedright
            \newcommand\dutyCycle{10}
            \newcommand\uDx{1.74}
            % This file was created by matlab2tikz.
%
%The latest updates can be retrieved from
%  http://www.mathworks.com/matlabcentral/fileexchange/22022-matlab2tikz-matlab2tikz
%where you can also make suggestions and rate matlab2tikz.
%

\begin{tikzpicture}
\pgfplotsset{
    width=\pWidth,
    height=\pHeigh,
    scale only axis,
    ylabel near ticks,
    enlarge y limits={0.15},
    xlabel near ticks,
    ylabel near ticks,
    enlarge x limits={0.15},
    every tick label/.append style={font=\footnotesize},
    yticklabel style={text width=2em,align=right},
}

\begin{semilogxaxis}[
        xticklabels={,,},
        ylabel= {$ \epsilon_r ~ [\%] $} ,
        axis y line*=left,
        axis x line*=bottom,
        title={$D=\dutyCycle\%~(\mu=\uDx)$ },
        title style = {
                at ={(0.5,1.1)},
                font=\footnotesize },
        legend style={
                legend columns = -1,
                at={(0.5,1)},
               anchor=south,
                draw=none,
                font=\tiny,
                column sep=1ex,
                legend cell align = left},
        ]


    \addplot [thin,smooth,black,mark=o,mark repeat=2]
      table [y=y1]{./3_modeling/err_rx_aprox_ORG_10het.dat};\label{pl_MDL}
    \addlegendentry{Org};
    \addplot [thin,smooth,mark=+,mark repeat=2]
      table [y=y1]{./3_modeling/err_rx_aprox_MAK_10het.dat};
    \addlegendentry{Mak};
    \addplot [thin,smooth,mark=x,mark repeat=2]
      table [y=y1]{./3_modeling/err_rx_aprox_RMAK_10het.dat};
    \addlegendentry{*Mak};


\end{semilogxaxis}

\end{tikzpicture}

        }
       \parbox[b]{.45\linewidth}{
            \raggedleft
            \newcommand\dutyCycle{23}
            \newcommand\uDx{2.12}
            \input{./3_modeling/rx_aprox_12_het.tex}
        }
    \end{subfigure}

    \begin{subfigure}{\textwidth}
       \parbox[b]{.45\linewidth}{
            \raggedright
            \newcommand\dutyCycle{36}
            \newcommand\uDx{2.43}
            % This file was created by matlab2tikz.
%
%The latest updates can be retrieved from
%  http://www.mathworks.com/matlabcentral/fileexchange/22022-matlab2tikz-matlab2tikz
%where you can also make suggestions and rate matlab2tikz.
%

\begin{tikzpicture}
\pgfplotsset{
    width=\pWidth,
    height=\pHeigh,
    scale only axis,
    ylabel near ticks,
    enlarge y limits={0.2},
    xlabel near ticks,
    ylabel near ticks,
    enlarge x limits={0.15},
    every tick label/.append style={font=\footnotesize},
    yticklabel style={text width=2em,align=right},
}

\begin{semilogxaxis}[
        xlabel= {$f_{sw}[Hz] $},
        ylabel= {$ \epsilon_r ~ [\%] $} ,
        axis y line*=left,
        axis x line*=bottom,
        %yticklabel style={xshift=0.5ex},
        enlarge y limits={0.1},
        title={$D=\dutyCycle\%~(\mu=\uDx)$ },
        title style = {
                at ={(0.5,1.1)},
                font=\footnotesize },
        legend style={
                legend columns = -1,
                at={(0.5,0.97)},
                anchor=south,
                draw=none,
                font=\tiny,
                column sep=1ex},
        ]


    \addplot [thin,smooth,black,mark=o,mark repeat=2]
      table [y=y3]{./3_modeling/err_rx_aprox_ORG_10het.dat};\label{pl_MDL}
    %%\addlegendentry{Model};
    \addplot [thin,smooth,mark=+,mark repeat=2]
      table [y=y3]{./3_modeling/err_rx_aprox_MAK_10het.dat};
    %\addlegendentry{Seeman};
    \addplot [thin,smooth,mark=x,mark repeat=2]
      table [y=y3]{./3_modeling/err_rx_aprox_RMAK_10het.dat};
    %\addlegendentry{Mak. rect. $u=\uDx$};


\end{semilogxaxis}

\end{tikzpicture}

        }
       \parbox[b]{.45\linewidth}{
            \raggedleft
            \newcommand\dutyCycle{50}
            \newcommand\uDx{2.54}
            \input{./3_modeling/rx_aprox_22_het.tex}
        }
    \end{subfigure}



\caption{Relative error of a single capacitor switch capacitor with heterogenous $\tau$ constants ($10 \tau_1 = \tau_2$)  between the closed form of $r_{scc}$ and the different approximations: \emph{Org} - Original, \emph{Mak} - Makowski and \emph{*Mak} - rectified Mackowski.  Solved for the circuit in Figure~\ref{fig:single_capacitor} with $c=1\mu F$ and $r_1=r_2=10\Omega$.}
\label{fig:rscc_aprox_hete}
\end{figure}

The accuracy of the different approximations was validated with the circuit of Figure~\ref{fig:single_capacitor} in two different scenarios, by measuring the relative error with respect to the analytical closed form solution of the circuit~\eqref{eq:r_scc_CF}. In the first case, Figure~\ref{fig:rscc_aprox_homo},  the circuit hast homogenous time constants ( $\tau_1 = \tau_2$). That is why the \emph{rectified Makowski (*Mak)} formulation presents the best results for all duty cycles, and matches with the \emph{Makowski (Mak)} approximation for $D=50\%$ since $\mu=2.54$. For smaller values of $D$ the \emph{Original (Org) } approximation is the second best, since $\mu$ trends to values closer to 2.

Nevertheless this improved accuracy of the \emph{rectified Makowski},  changes as the $\tau$ constants of the converter diverge from each other, as the second case of Figure~\ref{fig:rscc_aprox_hete} where $10\tau_1 = \tau_2$. In this scenario, the ~\emph{Original} approximation keeps $\epsilon_r$ below $\pm5\%$ for almost the full range of $D$, except for $D=10\%$ that it rises about a $-9\%$. \emph{Makowski} approximation has it best accuracy in the lowest range of $D$, but as  $D$ increases it rises above $5\%$.  \emph{Rectified Makowski} achieves its best at $D=23\%$, but it rises about $10\%$ for other values of $D$.

Considering the different published approximations, there is not a conclusive result that favours the use of an specific one. In addition, the use of an approximation for computing $r_{scc}$ from the two asymptotical limits has no other goal than provide a simple analytical expression for $r_{scc}$ in order to accelerate the optimization and design of the converters. Actually, this new proposed approximation obtains $\mu=2.54$ form an idealized and specific case of a converter, which the accuracy of it derates as the converter under study diverges from this idealized case. Therefore using the values of $\mu = 2.54$ or $\mu = f(D) $ becomes as  arbitrary as was in the initial proposed value of $\mu=2$. That is why this dissertation used still the original formulation of~\eqref{eq:r_scc}.


\afterpage{\clearpage}

\subsection{Conversion ratio}

The conversion ratio of the converter can be obtained with the source \emph{net} charge element from vector $\mathbf{a}$ as
\begin{equation}
m=\frac{{v_{trg}}}{v_{src}}=\sum_{j=1}^{phases}a_{in}^j.
\label{eq:r_ssl}
\end{equation}
where $a_{in}$  corresponds to the input voltage source term of the charge vector multiplier $\mathbf{a}$.

\afterpage{\clearpage}

\clearpage
\section{Multiple Output Converter}
Another advantage that SCC offers is to provide multiple outputs using a single SCC stage. In this multi-port configuration, the energy supply is connected to input port, and the converter provides multiple output ports with different conversion ratios. A clear application was presented by \citeauthor{2012Kumar} in~\cite{2012Kumar} with the Triple Output Fixed Ratio Converter (TOFRC); where a 2:1 Ladder converter combined with two inductors provides three fixed output voltages using a single SCC stage.
\begin{figure}[!h]
\centering
\ctikzset { bipoles/length=1cm}
\begin{circuitikz}[american,scale=0.65]
\draw
    (1,0) to[short,o-]
    (0,0) to[V = $V_{src}$]
    (0,3) to[short,-o]
    (1,3) ;

\draw
    (2,3) --
    (2.5,3)

    (2,0) --
    (2.5,0)

    node[ocirc]  (IC)  at (2,0) {}
    node[ocirc]  (I) at (2,3) {}
    (I) to[open,v=$v_{i}$] (IC);


\draw [thick]
    (2.5,-0.5) --
    (2.5,3.5)  --
    (5.5,3.5)  --
    (5.5,-0.5) --
    (2.5,-0.5);

\draw (4,2.5)node[]{$\frac{v_{1}}{v_{i}}=m_1$};
\draw (4,0.5)node[]{$\frac{v_{n}}{v_{i}}=m_n$};

\draw
    (5.5,1.25) to[short,-o](6,1.25)
    (5.5,-0.25)  to[short,-o] (6,-0.25)
    (6.25,1) to[open,v^<=$v_{n}$] (6.25,0);
\draw
    (7,-0.25) to[short,o-]
    (8,-0.25) to[/tikz/circuitikz/bipoles/length=0.5cm,R= Load $n$,mirror]
    (8,1.25) to[short,-o] (7,1.25) ;
    
\draw
    (5.5,3.25) to[short,-o](6,3.25)
    (5.5,1.75)  to[short,-o] (6,1.75)
    (6.25,3) to[open,v^<=$v_{1}$] (6.25,2);

\draw
    (7,1.75) to[short,o-]
    (8,1.75) to[/tikz/circuitikz/bipoles/length=0.5cm,R= Load 1,mirror]
    (8,3.25) to[short,-o] (7,3.25) ;
\end{circuitikz}
\label{fig:two_port}
\caption[Block diagram of a multi-output SCC]{Block diagram of a general multiple output port configuration of a Switched Capacitor Converter. }
\end{figure}

\subsection{The Output Trans-Resistance Model}
When considering a converter with multiple outputs, the load effects have to be taken into account for all the outputs. Actually, when the converter is loaded, it produces a voltage drop throughout outputs of the converter. Therefore, the output current of one output node has an influence to the other outputs. In order to model these effects a new model based on trans-resistance parameters is proposed.
\begin{figure}[!h]
\centering
\ctikzset { bipoles/length=1cm}
\begin{circuitikz}[american voltages, scale=0.65]
\draw
    (0,0) to[V = $ m_1  v_{src}  $] (0,3)
    (3,3) to[american controlled voltage source,l_=$i_1 z_{11} + i_2 z_{12} + \cdots + i_n z_{1n} $] (0,3)
    (3,3) to[short,i>_=$i_1$,-o] (4,3)
    (4,0) to[short,o-] (0,0)
    (4,3) to[open,v^=$v_1$] (4,0);

\draw
    (8,0) to[V = $ m_n  v_{src}  $] (8,3)
    (11,3) to[american controlled voltage source,l_=$i_1 z_{n1} + i_2 z_{n2} + \cdots + i_n z_{nn} $] (8,3)
    (11,3) to[short,i>_=$i_n$,-o] (12,3)
    (12,0) to[short,o-] (8,0)
    (12,3) to[open,v^=$v_n$] (12,0);

\end{circuitikz}
\caption{Output trans-resistance model of a switched capacitor converter.}
\label{fig:scc_model_tr}
\end{figure}

The proposed model is shown in Figure\ref{fig:scc_model_tr}; as it can be seen, each output is represented using two controlled voltage sources connected in anti-series. One source provides the \emph{target voltage}  associated with the output, taking the value from the input voltage, $v_{src}$, multiplied by the respective conversion ratio associated to that output, $m_x$.

The other source, produces a voltage droop associated with the losses in the converter. The current delivered by each loaded node adds a specific contribution to the converter losses. Therefore, this voltage source takes the value given by the linear combination of all the converter output currents weighted by their associated trans-resistance factor $z$.

The trans-resistance factor $z_{xy}$ produces a voltage drop at the output $x$ proportional to the charge (\emph{i.e}. current) delivered by the output $y$.  It can be seen that the trans-resistance factor $z_{xx}$ corresponds to the voltage drop of the same output where the current is delivered, thus this parameter is  the output impedance for that node. Since all the trans-resistance factors relate current to voltage, they are in \emph{Ohms}.

With the proposed model, the converter behavior can be described as
\begin{equation}
 \mathbf{v_o} = -\mathbf{Z} \cdot \mathbf{i_o} + \mathbf{m} \cdot v_{src},
 \label{eq:admit_sol}
\end{equation}
where $\mathbf{Z}$ is the \emph{trans-resistance matrix}. %$\mathbf{Z}$ is symmetric for two phase converters.


\subsection{Power losses and trans-resistance parameters}
\begin{figure}[!h]
\centering
\ctikzset { bipoles/length=1cm}
\begin{circuitikz}[american voltages, scale=0.65]
\draw
    (0,0) to[V = $ m_1  v_{src}  $] (0,3)
    (3,3) to[american controlled voltage source,l_=$i_1 z_{11} + i_2 z_{12} $] (0,3)
    (3,3) to[short,i>_=$i_1$,-o] (4,3)
    (4,0) to[short,o-] (0,0)
    (4,3) to[open,v^=$v_1$] (4,0);

\draw
    (8,0) to[V = $ m_2  v_{src}  $] (8,3)
    (11,3) to[american controlled voltage source,l_=$i_1 z_{21} + i_2 z_{22} $] (8,3)
    (11,3) to[short,i>_=$i_2$,-o] (12,3)
    (12,0) to[short,o-] (8,0)
    (12,3) to[open,v^=$v_2$] (12,0);

\end{circuitikz}
\caption{Two output converter.}
\label{fig:model_duality}
\end{figure}
Using the trans-resistance matrix $\mathbf{Z}$ the losses of the converter can be computed. For a two output converter, modeled as shown in Figure~\ref{fig:model_duality}, the losses associated to each output would be
\begin{equation}
 P_{o1} = i_1^2 ~ z_{11} + i_1 ~ i_2 ~ z_{12}
 \label{eq:ploss_1}
\end{equation}

\begin{equation}
 P_{o2} = i_1 ~ i_2 z_{21} + i_2^2 ~ z_{22},
 \label{eq:ploss_2}
\end{equation}
and the total converter losses are
\begin{equation}
 P_{total} = i_1^2 ~ z_{11} + i_2^2 ~ z_{22} +  i_1 ~ i_2 ~ z_{12} ~ z_{21}  .
 \label{eq:ploss_3}
\end{equation}
%\subsubsection{Slow Switching Limit}
Using the the charge flow analysis  described in the previous section, the total losses of a two output converter can be computed as well. In order to make the analysis less cumbersome, the phases are eluded and losses are computed in a single capacitor for the SSL region. The results can be extended for any converter with any number of phases and capacitors.


In the case of a multiple-output converter, each of the individual outputs produces a \emph{redistributed} charge flow through the capacitors that can be individually quantified, being $g_{i,1}$  the \emph{redistributed} cahrge flow multiplier associated to the first output, $g_{i,2}$ associated to the second output. The total \emph{redistributed} charge is the sum of each individual contributions as
\begin{equation}
 g_i =  (g_{i,1} ~ q_{o,1} +  g_{i,2} ~ q_{o,2}).
 \label{eq:g_i_total}
\end{equation}
Substituting~\eqref{eq:g_i_total} in~\eqref{eq:pwr_ssl} the losses produced in capacitor $c_i$ of the two output converter are
\begin{equation}
 P_{c_{i}} = f_{sw} \frac{1}{2 ~ c_i} (g_{i,1} ~ q_{o,1} +  g_{i,2} ~ q_{o,2})^2.
 \label{eq:ploss_c_2}
\end{equation}
expanding terms and substituting $q_{o,1}=i_1/f_{sw}$ and $q_{o,2}=i_2/f_{sw}$ into~\eqref{eq:ploss_c_2}  yelds
\begin{equation}
 P_{c_{i}} =  \frac{1}{2 ~ f_{sw} ~ c_i} (i_1^2 ~g_{i,1}^2  +  i_2^2 ~ g_{i,2}^2 + 2 ~ i_{1} ~ i_{2} ~ g_{i,1}~g_{i,2} ).
 \label{eq:ploss_c_3}
\end{equation}
It can be seen that the trans-resistance parameters of~\eqref{eq:ploss_3} can be directly matched with the \emph{redistributed charge flow multipliers} in ~\eqref{eq:ploss_c_3} as
\begin{center}
    \renewcommand{\arraystretch}{2}
    \begin{tabular} {l c c c }
	$z_{11}$ & = & $\raisebox{0.8ex}{$g_{i,1}^2$}\big/ \raisebox{-0.6ex}{$2 f_{sw} c_i$}$ & $[\Omega] $\\
	$z_{22}$ & = & $\raisebox{0.8ex}{$g_{i,2}^2$}\big/ \raisebox{-0.6ex}{$2 f_{sw} c_i$} $& $[\Omega]$\\
	$z_{12} + z_{21} $ & = & $\raisebox{0.8ex}{$g_{i,1}g_{i,2}$}\big/ \raisebox{-0.6ex}{$ f_{sw} c_i$} $& $ [\Omega]$
    \end{tabular}
\end{center}
Therefore the general expressions of the SSL trans-resistance parameters are given as a function of the \emph{redistributed charge multipliers} as
\begin{equation}
  z_{ssl,xx} =  \frac{1}{2 f_{sw}} \sum_{i=1}^{caps.} \sum_{j=1}^{phas.}
  \frac{ \left ( g_{i,x}^j \right )^2 } {c_i}.
 \label{eq:z_ssl_xx}
\end{equation}

\begin{equation}
  z_{ssl,xy} + z_{ssl,yx} =  \frac{1}{f_{sw}} \sum_{i=1}^{caps.} \sum_{j=1}^{phas.}
  \frac{g_{i,x}^j g_{i,y}^j}{c_i}.
 \label{eq:z_ssl_xy}
\end{equation}
%\subsubsection{Fast Switching Limit}
The same analysis can be done for the FSL, but in this case the losses are compute for a single resistor.
As in the SSL case of a multiple-output converter, each of the individual outputs produces a charge flow through the switches that can be individually quantified, being $ar_{i,1}$ associated to the first output, $ar_{i,2}$ associated to the second output, etc. The total \emph{switch} charge multiplier is the sum of each individual \emph{switch} multiplier as
\begin{equation}
 ar_i =  (ar_{i,1} ~ q_{o,1} +  ar_{i,2} ~ q_{o,2}).
 \label{eq:ar_i_total}
\end{equation}
Substituting~\eqref{eq:ar_i_total} in~\eqref{eq:pwr_ssl}, the power dissipated in $r_i$ of the two output converter results in
\begin{equation}
 P_{r_{i}} =  \frac{r_i}{D} (i_1^2 ~ar_{i,1}^2  +  i_2^2 ~ ar_{i,2}^2 + 2 ~ i_{1} ~ i_{2} ~ ar_{i,1}~ar_{i,2}),
 \label{eq:ploss_r_1}
\end{equation}
leading to a similar polynomial solution of the previous case. Hence the general expressions for the FSL trans-resistance parameters are
\begin{equation}
  z_{fsl,xx} =   \sum_{i=1}^{swts.} \sum_{j=1}^{phas.}
  \frac{r_{i}}{D^j} \left ( ar_{i,x}^j \right )^2,
 \label{eq:z_fsl_xx}
\end{equation}
\begin{equation}
  z_{fsl,xy} + z_{fsl,yx} =   \sum_{i=1}^{swts.} \sum_{j=1}^{phas.}
  \frac{r_{i}}{D^j} ar_{i,x}^j ar_{i,y}^j,
 \label{eq:z_fsl_xy}
\end{equation}

Notice that~\eqref{eq:z_ssl_xy} and ~\eqref{eq:z_fsl_xy} do not provide the individual expressions for the cross trans-resistance parameters $z_{xy}$ and $z_{yx}$. Actually, the individual quantification of these parameters is related to the sequence order of the different circuit modes for the converter, but this relation has not yet been founded\footnote{Converters with more than 2 phases are beyond the scope of the H-SCC, and so, this dissertation.}. Fortunately,  two-phase converters do not have cardinality  in the sequence of the switching modes, resulting in symmetry of these parameters , and making $\mathbf{Z}$ matrix to be symmetric. Consequently, the generic expressions of the trans-resistance parameters for two phase converters are reduced to two:
\begin{equation}
  z_{ssl,xy}  =  \frac{1}{2~f_{sw}} \sum_{i=1}^{caps.} \sum_{j=1}^{phas.}
  \frac{g_{i,x}^j g_{i,y}^j}{c_i}.
 \label{eq:z_ssl_xy_2ph}
\end{equation}
\begin{equation}
  z_{fsl,xy} =   \sum_{i=1}^{swts.} \sum_{j=1}^{phas.}
  \frac{r_{i}}{D^j} ar_{i,x}^j ar_{i,y}^j,
 \label{eq:z_fsl_xy_2ph}
\end{equation}

%As aforementioned,  for converters with more than two phases, the relation between sequentiality of the circuit modes and the cross trans-conductances has not yet been found, since converters with more that 2-phases are beyond the scope of this work and the H-SCC.

\subsection{Trans-resistance Parameters Methodology}
Based on the \emph{charge flow analysis} for current-loaded SCCs, each converter output has three associated sets of charge flow vectors per switching phase. Thus, for a given converter, the different vector types can be collected in a matrix, where each column corresponds to a converter output and each row corresponds to a circuit component.

Therefore the \emph{charge flow multipliers} are collected in a matrix as
\begin{equation}
 \mathbf{A}^j =
   \bordermatrix { ~ & out_1 & out_2 & ~ & out_n \cr
     v_{src} & a_{1,1}^j  & a_{1,2}^j & \cdots & a_{1,n}^j \cr
     c_1    & a_{2,1}^j  & a_{2,2}^j & \cdots & a_{2,n}^j \cr
      ~     & \vdots     & \vdots & \ddots & \vdots \cr
     c_p    & a_{p,1}^j  & a_{p,2}^j & \cdots & a_{p,n}^j \cr},
 \label{eq:A_matrix}
\end{equation}
where the elements of the first row  $a_{1,x}^j$ corresponds to the \emph{charge flow multiplier}  delivered by the input voltage source associated to the charge flow through the $x$\emph{-th} output. The remaining elements after the first row are associated with the charge flow in the capacitors. Therefore $a_{1,1}$ is the net charge flow in capacitor $c_1$ due to the charge delivered at the $1st$ output node of a converter with $p$ capacitors and $n$ outputs.

Likewise, the \emph{charge pumped multipliers} are collected in the following matrix
\begin{equation}
 \mathbf{B}^j =
   \bordermatrix { ~ & out_1 & out_2 & ~ & out_n \cr
     c_1  & b_{1,1}^j  & b_{1,2}^j & \cdots & b_{1,n}^j \cr
     c_2  & b_{2,1}^j  & b_{2,2}^j & \cdots & b_{2,n}^j \cr
      ~   & \vdots     & \vdots & \ddots & \vdots \cr
     c_p  & b_{p,1}^j  & b_{p,2}^j & \cdots & b_{p,n}^j \cr}
     \label{eq:A_matrix},
\end{equation}
where all the elements are associated with the converter capacitors.

On the other hand, the \emph{switch charge flow multipliers} lead to the following matrix
\begin{equation}
 \mathbf{Ar}^j =
   \bordermatrix { ~ & out_1 & out_2 & ~ & out_n \cr
     sw_1  & ar_{1,1}^j  & ar_{1,2}^j & \cdots & ar_{1,n}^j \cr
     sw_2  & ar_{2,1}^j  & ar_{2,2}^j & \cdots & ar_{2,n}^j \cr
      ~    & \vdots     & \vdots & \ddots & \vdots \cr
     sw_p  & ar_{p,1}^j  & ar_{p,2}^j & \cdots & ar_{p,n}^j \cr}.
 \label{eq:A_matrix}
\end{equation}
where all the elements are associated with the converter switches. This matrix can be extended with the Equivalent Series Resistance (ESR) of the capacitors, but for the sake of clarity they are not included in the present calculations yet.

 The converter is described with two trans-resistance matrix: one for the SSL, $\mathbf{Z_{ssl}}$, and another for the FSL, $\mathbf{Z_{fsl}}$.

\subsection{Slow Switching Limit Trans-resistance Matrix}

The \emph{redistributed} charge flow multipliers matrix can be obtained from the
matrices $\mathbf{A}$ and $\mathbf{B}$  as
\begin{equation}
 \mathbf{G}^j = \mathbf{A}_{(2:end,1:end)}^j - D^j \mathbf{B}^j,
 \label{eq:R_matrix}.
\end{equation}
The \emph{redistributed charge} corresponds to the charge that flows between capacitors; therefore it is the root cause of
losses associated with the SSL operation regime~\cite{Seeman:EECS-2009-78}.

The SSL trans-resistance factors can be individually obtained from the redistributed charge multipliers as described in \eqref{eq:z_ssl_xy_2ph}. In order to obtain directly the trans-resistance matrix, the operation in \eqref{eq:z_ssl_xy_2ph} is performed in  two steps. First, the outer product of  each row of $\mathbf{G}^j$ is taken with itself as
\begin{equation}
 \mathbf{K}_i^j =[\mathbf{G}_{(i,1:end)}^j ]^T \mathbf{G}_{(i,1:end)}^j ,
 \label{eq:K_matrix}
\end{equation}
where the matrix $\mathbf{K_i}$ contains all the possible products of the  $i^{th}$ row. Since each row in $\mathbf{G}$ is associated with a capacitor, there is a matrix $\mathbf{K_i}$ for each capacitor $C_i$.
Second, with the set of $\mathbf{K}$ matrices the trans-resistance matrix is obtained as
\begin{equation}
 \mathbf{Z_{ssl}} = \frac{1}{2 F_{sw}} \sum_{j=1}^{phas.} \sum_{i=1}^{caps.} \frac{1}{C_i} \mathbf{K}_i^j.
 \label{eq:G_ssl}
\end{equation}

\subsection{Fast Switching Limit trans-resistance Matrix}
For the FSL, the trans-resistance matrix is obtained using the switch charge multipliers
contained in matrix $\mathbf{Ar}$. The operation to obtain the trans-resistance matrix as described
in \eqref{eq:z_ssl_xy_2ph} is performed in two steps. First, a set of matrices are obtained by taking the outer product
of each row of $\mathbf{Ar}$ with itself as
\begin{equation}
 \mathbf{Kr}_i^j = \mathbf{Ar}_{(i,1:end)}^j [\mathbf{Ar}_{(i,1:end)}^j]^T,
 \label{eq:Kr_matrix}
\end{equation}
yielding a matrix for each row in $\mathbf{Ar}$ associated with a switch \emph{on}-resistance ($r_{i}$). Second, with the set of matrices $\mathbf{Kr}$ the FSL trans-resistance matrix is obtained as
\begin{equation}
 \mathbf{Z_{fsl}} =  \sum_{i=1}^{swts.} \sum_{j=1}^{phas.} \frac{_{i}}{D^j}
\mathbf{Kr}_i^j,
 \label{eq:G_fsl}
\end{equation}

\subsection{Converter trans-resistance Matrix}
The total trans-resistance values are approximated using~\eqref{eq:r_scc} as
\begin{equation}
 \mathbf{Z}_{(x,y)} \approx \sqrt{\mathbf{Z_{ssl,(x,y)}}^2 + \mathbf{Z_{fsl,(x,y)}}^2}.
 \label{eq:G_total}
\end{equation}

\subsection{Conversion Ratio Vector }

The conversion ratio vector is obtained as
\begin{equation}
 \mathbf{m} = \sum_{j=1}^{phas.}[\mathbf{A}^j_{(1,1:end)}]^T.
 \label{eq:m_equation}
\end{equation}



\section{Summary}
This chapter presented a new methodology to analyse SCC that compared with the previous enables to:
\begin{itemize}
  \item Compute the equivalent output resistance from any of the converter nodes.
  \item Compute the conversion ration form any of the converter nodes.
  \item Model converter with multiple outputs.
  \item Compute the coupling parameters between outputs for 2-phase converters.
  \item Include the effects of the output capacitor in $r_{scc}$.
  \item Include the effects of variations in duty cycle in the SSL region.
  \item Model both SCCs and H-SCCs.
\end{itemize}

In addition, a discussion about the different approximations of the $r_{scc}$ using the two asymptotical limits ($r_{ssl}$ and $r_{fsl}$) was provided. Concluding that the \emph{arbitrary} of the original approximation was not less accurate than the new proposed formulations, as the circuit under study diverges from the reference circuit used in these new formulations. Giving the rational, to consider the original formulation as the most appropriated.

\clearpage
\bibliographystyle{plainnat}
\bibliography{references} 