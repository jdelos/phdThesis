\chapter[Advanced Modeling of H-SCC]{Modeling of Hybrid Switched Capacitor Converters}

Switch capacitor converters are circuits composed by a large number of components, switches and capacitors, and require special methods to study them. At the same time, SCCs are lossy circuits by nature due to non adiabatic energy transfer between capacitors, which is not observed in inductive converters. Hence SCCs are lossy by default from the beginning of the design process    

\section{Single Output Converters}
Switched Capacitor Converters has been always treated as a two-port converter with single input and a single output as shown in Fig.\ref{fig:two_port}. The input port is connected to a voltage source and the output port feeds the load. The SCC provides between input, $v_i$, and output, $v_o$, a voltage conversion, $m$,  that  steps up, steps down or/and inverts the polarity of the input voltage. Up to present all the circuit theory devoted SCCs is valid only for the two-port configuration, therefore this section is dedicated to revisit the classical concepts of single output SCC and to introduce new ones that enable a broader use of such converters.

\begin{figure}[!h]
\centering
\ctikzset { bipoles/length=1cm}
\begin{circuitikz}[scale=0.65]
\draw
    (1,0) to[short,o-]
    (0,0) to[V = $V_{supply}$]
    (0,3) to[short,-o]
    (1,3) ;

\draw
    (2,3) --
    (2.5,3)

    (2,0) --
    (2.5,0)

    node[ocirc]  (IC)  at (2,0) {}
    node[ocirc]  (I) at (2,3) {}
    (I) to[open,v=$v_{i}$] (IC);


\draw [thick]
    (2.5,-0.5) --
    (2.5,3.5)  --
    (5.5,3.5)  --
    (5.5,-0.5) --
    (2.5,-0.5);

\draw (4,2)node[anchor=north]{$\frac{v_o}{v_{i}}=m$} ;
\draw
    (5.5,3) -- (6,3)
    (5.5,0) -- (6,0)
    node[ocirc]  (O)  at (6,3) {}
    node[ocirc]  (OC) at (6,0) {}
    (O) to[open,v^<=$v_{o}$] (OC);

\draw
    (7,0) to[short,o-]
    (8,0) to[ R= $Load$,mirror]
    (8,3) to[short,-o]
    (7,3) ;
\end{circuitikz}
\label{fig:two_port}
\caption{General two port configuration of a Switched Capacitor Converter. }
\end{figure}


\subsection{The Output Impedance Model}
\subsection{Identifying the source of losses in the charge transfer}
\subsection{Re-formulating the charge flow analysis}
\subsubsection[SSL Capacitor Charge Flow]{Slow Switching Limit: Re-defining the Capacitor Charge Flow Vectors}
\subsubsection[FSL Switch Charge Flow]{Fast Switching Limit: Re-defining the Switch Charge Flow Vectors}

\subsection{Load Model: Voltage Sink versus Current Sink}
\subsection{Sensitivity of the inductor current ripple}

\section{Multiple Output Converters}
\subsection{The Output Trans-Resistance Model}
\subsection{Obtaining the Trans-Resistance parameters with the charge flow analysis }



\chapter[Optimization and Design]{Optimization and Design of Hybrid-Switched Capacitor Converters}
\section{Introduction}
\section{Study in the correlation of the design parameters and the Output Impedance}
\section{Encapsulating the Switches and Capacitors area breakdown in an optimization procedure}
\section{Insights towards a complete optimization}

\chapter[Dynamic Study]{Dynamic Study of Hybrid-Switched Capacitor Converters}
\section{Small Signal Analysis}


\clearpage
\bibliographystyle{plainnat}
\bibliography{references} 